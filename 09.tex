\section{Volume 9}

\subsection{^{だい}{第}77^{わ}{話}\ ^{しお}{塩}}

\page[5]
\Alpha ^{うみ}{海}の^{みち}{道}

\Alpha ^{なみ}{波}から^{に}{逃}げるように^{うえ}{上}へ^{うえ}{上}へと、
つけかえられてきた^{みち}{道}の^{かせき}{化石}

\page[8]
\Alpha うちあげられた^{かいそう}{海藻}、^{てつ}{鉄}サビ、^{さかな}{魚}の^{あわ}{泡}

\Alpha ^{しお}{塩}のしみた^{いた}{板}、^{まつ}{松}やに、^{こうぶつ}{鉱物}^{ゆ}{油}

\page[9]
\Alpha ^{やま}{山}で^{わす}{忘}れていた、^{わたし}{私}のにおいだ


\subsection{第78話\ むらさきの^{ひとみ}{瞳}}

\page[15]
\Takahiro あ……
\ いらっしゃ……

\page[16]
\Alpha わっ!!

\Alpha あ〜〜
\ しっぱいした!

\Takahiro アルファ

\Alpha へへ

\Alpha ひさしぶり

\page[17]
\Takahiro おかえり

\Alpha うん
\ ただいま

\Alpha ふ〜ん
\ お^{みせ}{店}、^{てつだ}{手伝}ってるんだ

\Takahiro うん
\ じいちゃんと^{はんはん}{半々}くらいかなー

\Sign ^{だいこん}{大根}
\ ^{とうゆ}{灯油}
\ ガソリン

\Takahiro ほとんどダイコン^{や}{屋}のコゾーだよ

\Alpha へー

\page[18]
\Alpha えらいなあ

\Takahiro あっ、イス^{も}{持}ってくるよ
\ なんかのむ?

\Alpha いただきー

\page[19]
\Takahiro はい
\ お^{ちゃ}{茶}しかなかった

\Alpha ありがと

\page[21]
\Takahiro あ

\Takahiro はい、お^{ちゃ}{茶}

\Alpha あ
\ うん

\Alpha ^{せい}{背}たけ

\Alpha また、のびたね!
\ びっくりしちゃった

\Takahiro うん

\page[22]
\Takahiro ^{いちねん}{一年}か……

\Takahiro もっと^{なが}{長}い^{あいだ}{間}いなかった^{き}{気}がするけど

\Alpha そだね

\Takahiro でも、なんか、おとといくらいに^{あ}{会}ったみたいな^{かん}{感}じもする

\Alpha そだね

\Alpha なんかさー
\ ^{こえ}{声}
\ ^{ひく}{低}くなってない?

\Takahiro まーねー
\ なんかねー

\page[23]
\Takahiro ^{さいしょ}{最初}、へんなカスレた^{こえ}{声}になってきてね……

\Takahiro ^{すこ}{少}ししたら、^{たか}{高}い^{ところ}{所}の^{こえ}{声}が
\ スパッと^{で}{出}なくなって

\Alpha そっか

\Alpha なーんかさー
\ お^{ねえ}{姉}さん^{かぜ}{風}ふかせらんないなー、もう

\Takahiro そんなことないよ

\page[25]
\Alpha さあて
\ ^{ひ}{日}が^{お}{落}ちる^{まえ}{前}に、うち、^{あ}{開}けなきゃ
\ ごっそさま

\Takahiro うん

\Takahiro そだ

\Takahiro バイク、^{かえ}{返}さなきゃな……

\Alpha あ、いつでもいいよ

\Takahiro ^{きょう}{今日}、^{くるま}{車}なんだ
\ ^{おく}{送}ってこうか?

\Alpha ^{くるま}{車}!
\ へ〜〜
\ もう^{うんてん}{運転}できるんだー

\page[26]
\Alpha んーー、でも、やっぱ^{きょう}{今日}は^{ある}{歩}いてこうかなあ
\ ごめん、^{こんど}{今度}、^{おく}{送}ってね

\Takahiro そか
\ じゃ、また^{こんど}{今度}

\Alpha うん

\page[27]
\Takahiro じいちゃんにも^{い}{言}っとくよ

\Alpha ありがと!


\subsection{第79話\ ^{つち}{土}の^{よる}{夜}}

\page[30]
\Alpha なんか、わざとゆっくり^{かえ}{帰}った

\page[31]
\Alpha ^{ひ}{日}は、もうだいぶ^{まえ}{前}に^{お}{落}ちている

\page[32]
\Alpha ^{みせ}{店}の^{か}{欠}けた、^{うち}{家}の^{かたち}{形}には、まだ、^{な}{慣}れてない

\page[33]
\Alpha ブレーカーをあげる、ガスせんをひらく

\Alpha ただいまー

\page[34]
\Sign アルファさんへ
\ ^{たか}{鷹}^{つ}{津}ココネ

\page[35]
\Alpha ココネが^{にど}{二度}ほど^{き}{来}たようだ

\page[37]
\Alpha メイポロの^{げんえき}{原液}があった

\Alpha コゲる^{すんぜん}{寸前}まで^{に}{煮}つめて、お^{ゆ}{湯}で^{わ}{割}って

\Alpha うちのことあれこれを^{おも}{思}い^{だ}{出}してくる

\page[41]
\Alpha ^{いちねん}{一年}ぶりにねむったような^{き}{気}がした

\Alpha ^{からだ}{体}がふとんにとけていく

\page[42]
\Alpha ふとんごと^{じめん}{地面}にとけていく


\subsection{第80話\ ^{かざみ}{風見}^{さかな}{魚}}

\page[46]
\Alpha とりあえず

\page[47]
\Alpha こいつだけでも^{た}{立}てとこうと^{おも}{思}う

\Alpha えと

\Alpha どこだったっけ……


\subsection{第81話\ ^{いちねん}{一年}^{くうかん}{空間}}

\page[56]
\Ojisan よ

\Alpha だだいば〜〜!!!

\Ojisan あだだだだだだだだだ

\page[57]
\Alpha ごめんなさい

\Ojisan やー

\Ojisan まー
\ ^{ぶじ}{無事}けえれてよかった

\Ojisan ほ
\ あずかってたカギよー

\Alpha ありがとうございました

\page[58]
\Alpha るすの^{あいだ}{間}の^{はなし}{話}を^{き}{聞}いた

\Alpha ココネが4^{かい}{回}^{き}{来}ていたこと

\Alpha おじさんが^{あし}{足}にケガして、しばらく^{ある}{歩}けなかったこと

\Alpha ^{わたし}{私}の^{し}{知}らない^{まいにち}{毎日}が、ほんとにあった

\page[59]
\Ojisan へーよ
\ アルファさんよ

\Alpha あ、はい

\Ojisan へー
\ どうよ

\Ojisan ^{りょこう}{旅行}……
\ ^{い}{行}ってみて

\Alpha ^{い}{行}ってよかったです、やっぱ

\Alpha ^{わたし}{私}も、まあ、^{ちか}{近}くの^{くに}{国}のことくらいは……

\Alpha かなり^{し}{知}ってるつもりで、いたんですけど

\page[60]
\Alpha ^{ほん}{本}でわかることと^{げんち}{現地}で^{かん}{感}じることは……

\Alpha ^{べつ}{別}モノなんだなあって

\Ojisan そうな

\page[61]
\Alpha もっと^{とお}{遠}くまで^{い}{行}くつもりだったんですけどね

\Alpha ^{みず}{水}とかガスの^{りょうきん}{料金}とるとこがあったり……

\Alpha あと、^{いごこち}{居心地}よくて、^{ながい}{長居}しちゃったりとか

\Alpha ^{じかん}{時間}もお^{かね}{金}も、ぜんぜん^{た}{足}りませんよー

\page[62]
\Ojisan アルファさんよ

\Alpha はい?

\Ojisan ^{かせ}{稼}ぎの^{ほう}{方}は?

\Alpha かせぎ……
\ バイト……いっぱいやりましたよー

\Alpha ちょっとだけ^{だい}{大}^{あかじ}{赤字}かな……

\Ojisan まー
\ そんなだべなー

\page[63]
\Alpha そんななんで

\Alpha お^{みせ}{店}……
\ ^{かんぺき}{完璧}に^{もとどお}{元通}りにするのは、まだだーいぶ^{さき}{先}になりそうです

\Alpha だーーいぶ

\Ojisan まー
\ ゆっくしやんな

\Ojisan ^{いじ}{意地}でも、^{もとどお}{元通}りじゃねえとやだってんじゃなきゃー

\Ojisan ちっとぐれー
\ アルファさんの^{かって}{勝手}がはいっても、いいかもしんねーしなー

\page[64]
\Ojisan こんだ、タカがバイクもってくんって

\Alpha はい

\Ojisan あのやろ、エンジンものの^{しごと}{仕事}がしてえだと

\Alpha ありゃま

\Ojisan じゃ、またな

\Alpha あ
\ おじさん

\page[65]
\Alpha くび……
\ だいじょうぶ?

\Ojisan おお
\ そうよ

\Ojisan まー
\ いてえっちゃいつもどっかいてえしなー

\Ojisan じゃ

\page[67]
\Alpha まだ、お^{ひる}{昼}だ

\Alpha ^{たいよう}{太陽}がものすごく、のろい

\Alpha ああ

\Alpha ^{きょう}{今日}は、1キロも^{ある}{歩}いてないんだなあ

\page[68]
\Alpha きのうまでの^{いちねん}{一年}^{かん}{間}が、じわじわひとかたまりに

\Alpha ^{かこ}{過去}になっていく

\Alpha うまれてはじめて

\Alpha ひとつ^{とし}{齢}をとった^{き}{気}がしている


\subsection{第82話\ クロマツ^{どお}{通}り}

\page[73]
\Sensei いやー

\Sensei そうじゃないかと^{おも}{思}ったのよ

\Sensei このへんじゃ^{わか}{若}いコ、めったに^{み}{見}ないし

\Sensei なんか、^{ふんいき}{雰囲気}がいつもアルファさんが^{はな}{話}す^{かん}{感}じの、まんまだったから

\Kokone は
\ はじめまして

\page[74]
\Sensei で
\ ここで^{ま}{待}ち^{あ}{合}わせなの?

\Kokone はい

\Kokone ^{いえ}{家}にはいっちゃうとでかけるのが、おっくうになるから

\Kokone ^{きょう}{今日}は^{そと}{外}で^{ごうりゅう}{合流}しようってアルファさんが……

\Sensei なるほど

\Sensei アルファさんがおくれるなんてめずらしいわね

\Kokone あ、いえ、まだあと2^{じかん}{時間}あるんです

\Kokone ^{わたし}{私}が^{はや}{早}すぎたんです

\Sensei あらら

\page[75]
\Sensei ココネさん、よかったら、^{わたし}{私}の^{ところ}{所}で^{ま}{待}たない?

\Kokone は?

\Sensei すぐそこだから、バイク、^{もん}{門}のとこ^{お}{置}いとけば、アルファさん、わかるわよ

\Kokone え……
\ でもー

\Kokone お^{しごと}{仕事}のおじゃまに……

\Sensei あ
\ それは、ほぼ^{だいじょうぶ}{大丈夫}

\page[76]
\Sensei うち、^{いそが}{忙}しさは、カフェアルファなみだから

\Sensei お^{ちゃ}{茶}、たくさんあるわよ

\Kokone わっ

\page[77]
\Sensei どうぞ

\Kokone おじゃまします

\Kokone あ……
\ ひさしぶり^{びょういん}{病院}のにおい

\Sensei まー
\ ^{びょういん}{病院}っていってもね
\ ^{おうきゅうしょち}{応急処置}くらいがせいいっぱいなのよ、^{いま}{今}は

\Kokone はー

\Kokone でも、アルファさんはこちらで^{だいしゅじゅつ}{大手術}してもらったと

\Sensei ええ、まあね、そこはほれ

\Sensei まーー
\ あがっちゃって、あがっちゃって

\Kokone あ、はい

\page[78]
\Sensei で
\ まあ、^{おや}{親}の^{いさん}{遺産}^{く}{食}いつぶしてるのよ

\Kokone はー

\Kokone ^{せんせい}{先生}

\Sensei はい?

\Kokone ^{せんせい}{先生}は、その……

\Kokone アルファさんに^{いぜん}{以前}、^{き}{聞}いたことなんですけど……

\page[79]
\Kokone ^{わたし}{私}たち……
\ っていうか……
\ その……

\Kokone 「アルファタイプ」の^{むかし}{昔}のことに、くわしいと……

\Kokone あの……

\Kokone いきなりで^{しつれい}{失礼}かと^{おも}{思}いますけど、^{すこ}{少}し^{おし}{教}えていただきたいなと……

\Sensei ココネさんはそういう^{はなし}{話}が^{だい}{大}^{す}{好}きだって

\Sensei アルファさんが^{い}{言}ってたわ

\page[80]
\Sensei ^{かのじょ}{彼女}、あんまりそういう^{こと}{事}に^{きょうみ}{興味}^{しめ}{示}さないもんね

\Sensei そうでしょ

\Sensei ええ……そうね

\Sensei ^{わたし}{私}は、あなたたちの^{れきし}{歴史}のほんとに^{さいしょ}{最初}の^{ところ}{所}に、
ちょっとだけいたわ

\Sensei ^{つぎ}{次}に^{かか}{係}わった^{とき}{時}にはもう、
ココネさん^{たち}{達}の^{ちょっけい}{直系}のお^{ねえ}{姉}さんがいたから……

\page[81]
\Sensei えび^{ちゃいろ}{茶色}の^{かみ}{髪}と^{め}{目}をした

\Sensei そうね……
\ みんなより^{すこ}{少}し^{としうえ}{年上}に^{み}{見}える^{こ}{子}だったわね

\Kokone A7M1^{かた}{型}^{き}{機}……!
\ まぼろしの……

\Sensei そう、くわしいのね
\ へえ、「^{まぼろし}{幻}の」なの

\page[82]
\Sensei ^{かのじょ}{彼女}も「アルファ」って^{なまえ}{名前}なのよ、
\ ^{わたし}{私}の^{みょうじ}{名字}でね……

\Sensei ^{こ}{子}^{うみ}{海}^{いし}{石}アルファ


\subsection{第83話\ ^{あお}{青}い^{おと}{音}}

\page[84]
\Kokone ^{わたし}{私}とアルファさんの「お^{ねえ}{姉}さん」の^{はなし}{話}を^{せんせい}{先生}に^{き}{聞}きました

\Kokone ここに^{かのじょ}{彼女}がつれていられた^{とき}{時}のこと

\Kokone ここでしばらく^{せいかつ}{生活}していたこと

\page[85]
\Kokone まばたきくらいしかできなかった^{かのじょ}{彼女}がはじめて^{い}{言}った^{じょうだん}{冗談}のこと

\Kokone ^{あお}{青}い^{もの}{物}が^{す}{好}きなこと
\ ^{ふく}{服}がキライなこと

\Kokone いままで^{わたし}{私}がためこんできた^{ちしき}{知識}が^{かる}{軽}くふきとぶ、
^{ち}{血}の^{かよ}{通}ったエピソード

\Sensei その^{こ}{子}が、またつれていかれた

\Sensei それからだいぶ^{なが}{長}いことたって

\Sensei すぐ^{ちか}{近}くにロボットのコがいることを^{し}{知}ったの

\Sensei それが、^{いま}{今}^{く}{来}るアルファさん

\Kokone はー

\page[86]
\Kokone アルファさんが^{く}{来}るまでにはまだ^{すこ}{少}しある

\Kokone あの、^{せんせい}{先生}
\ A7の^{まえ}{前}って……あの
\ どんなだったんでしょう

\Sensei ^{わたし}{私}もそのあたり^{し}{知}らないのよ
\ A5とか、6とか

\Sensei どこかにちゃんと^{きろく}{記録}はあるんだろうけど

\Sensei でもね……たぶん、^{たの}{楽}しい^{はなし}{話}ばかりじゃないと^{おも}{思}うのよ

\Kokone ^{わたし}{私}も

\Kokone なんか、そんな^{き}{気}がしてました

\page[87]
\Sensei ^{わたし}{私}が^{し}{知}ってるのはA7の^{ひと}{人}たちだけなんだけど

\Sensei どこか^{かんかく}{感覚}^{てき}{的}なみんなを^{み}{見}てると、
とてもハードな^{かてい}{過程}があったことはわかるわ

\Sensei そうだ
\ ココネさん

\Kokone はい

\Sensei いい^{きかい}{機会}だから

\page[88]
\Sensei こうゆうのがあってね

\Sign Aー2(M1ーM4)

\Kokone そそそそ

\Sensei そう
\ 「Aー2」って^{か}{書}いてあるでしょ

\Sensei レコードなんだけどある^{いみ}{意味}、みんなのご^{せんぞ}{先祖}^{さま}{様}なのかもしれないわ

\page[89]
\Sensei あの^{たね}{種}の^{おと}{音}やリズムのサンプル^{しゅう}{集}で

\Sensei ^{ふつう}{普通}に^{き}{聞}いても、どうってことないものなんだけど

\Sensei ^{き}{聞}いてみる?

\page[91]
\Kokone ^{くうちゅう}{空中}に^{じゅうまん}{充満}する
\ ^{おと}{音}の^{つぶつぶ}{粒々}

\Kokone これは^{おんがく}{音楽}だろうか

\Kokone ^{すこ}{少}し^{ちが}{違}う^{き}{気}がする

\Kokone ^{あま}{甘}く、^{にが}{苦}く、^{はな}{鼻}の^{おく}{奥}がすずしいような

\Kokone ^{かわ}{乾}いた^{こえだ}{小枝}をポキポキ^{お}{折}るような^{はだ}{肌}ざわり

\Kokone ああ……そうか
\ これは……

\Kokone ^{わたし}{私}の^{し}{知}っている^{けしき}{景色}に^{に}{似}ている

\page[93]
\Kokone その^{じどう}{児童}^{かん}{館}で^{み}{見}てから、ずーっと^{き}{気}になってたんです

\Kokone いつかは^{き}{聞}かなきゃって

\Sensei へえ……
\ こんなレコードうちにしかないと^{おも}{思}ってたわ
\ びっくり

\Sensei ^{むかし}{昔}は、なんだこれって^{おも}{思}ってたのよね

\Sensei ^{いま}{今}、^{き}{聞}いてみたらこれはこれで、けっこういいかもしれないわ

\page[94]
\Alpha おじゃまします!!


\subsection{第84話\ ^{かいばつ}{海抜}70}

\page[96]
\Kokone どうも
\ おじゃましました

\Sensei いーえー

\page[97]
\Kokone お^{ひる}{昼}
\ ^{うみぞ}{海沿}いの^{すな}{砂}だらけの^{みち}{道}を^{なんか}{南下}

\Kokone アルファさんと^{いっしょ}{一緒}に^{はし}{走}るのは^{まえ}{前}に^{おく}{送}ってもらった^{とき}{時}、
^{いらい}{以来}です

\page[98]
\Kokone バイクでこのへんを^{はし}{走}ってみると

\Kokone アルファさんが^{わたし}{私}の^{す}{住}むムサシノを
「^{たい}{平}らな^{くに}{国}だ」と^{い}{言}ったワケがよくわかります

\page[99]
\Kokone フワッフワッと^{き}{気}まぐれに^{ま}{曲}がる、
アルファさんの^{えら}{選}ぶ^{みち}{道}は^{じょうげさゆう}{上下左右}、^{なみ}{波}のようにうねり

\Kokone ^{さき}{先}の^{み}{見}えないトンネルみたいな^{たに}{谷}の^{みち}{道}と、
^{うみ}{海}へジャンプするような^{おね}{尾根}^{みち}{道}をくりかえします

\page[100]
\Kokone ^{やま}{山}の^{なか}{中}に^{へん}{変}に^{あお}{青}く^{なみ}{波}のない^{すいめん}{水面}がありました

\Kokone ^{いけ}{池}のような、
それはくねくね^{ま}{曲}がった^{たに}{谷}の^{おく}{奥}まで^{はい}{入}りこんだ^{うみ}{海}なのだそうです

\Kokone ^{ひかり}{光}とにおい、^{いろ}{色}のパターンがどっとおしよせてくる^{みち}{道}

\page[101]
\Kokone ^{からだ}{体}のどこか

\Kokone ^{いま}{今}まで^{つか}{使}ってなかったところが、^{あ}{開}いていく^{かんかく}{感覚}

\page[103]
\Alpha ここはね
\ いつも^{ひとり}{一人}で^{く}{来}るんだ

\Alpha ^{やま}{山}ってほど^{たか}{高}くないけど、ここらじゃ^{いちばん}{一番}^{たか}{高}いし

\Alpha ^{わたし}{私}、^{たか}{高}いとこ^{だい}{大}^{す}{好}きだから

\Alpha ここでいろいろ^{かんが}{考}えたり、^{かんが}{考}えなかったり

\page[104]
\Kokone ハチジョウススキトベラ

\Kokone はじめて^{き}{聞}く^{くさき}{草木}の^{なまえ}{名前}です

\Kokone その^{なか}{中}を^{とお}{通}ってくるとろっとした^{こ}{濃}い

\Kokone ^{くうき}{空気}のにおい

\page[105]
\Kokone アルファさん

\Kokone ^{わたし}{私}たち
\ ^{おと}{音}やにおいでできてるんですよ……

\Kokone たとえばなしとかじゃなくて……

\Alpha ^{し}{知}ってるよー

\page[108]
\Alpha ごめんね〜〜
\ ^{ね}{寝}ちゃって

\Alpha ^{お}{起}こしてくれればよかったのに〜〜

\Kokone いえ

\page[109]
\Alpha じゃー
\ バイクはここにかためといてー
\ はい
\ あがってー

\Kokone おじゃまします

\Kokone ^{わたし}{私}

\Kokone ここに^{す}{住}みたいなあ

\page[110]
\Alpha ココネー

\Kokone あっ
\ はい!


\subsection{第85話\ かえる}

\page[115]
\Alpha ^{なみいた}{波板}が^{て}{手}にはいったので、ひさしをつけてみた

\Alpha ^{ごうせいじゅし}{合成樹脂}の^{なみいた}{波板}は^{はんとうめい}{半透明}の^{にゅうはくいろ}{乳白色}で

\Alpha ^{した}{下}からすかして^{み}{見}る、
^{たいよう}{太陽}の^{ひかり}{光}がだいだい^{いろ}{色}の^{わ}{輪}っかになる

\page[116]
\Alpha その^{ひ}{日}の^{よる}{夜}らか^{あめ}{雨}はもう^{いつか}{五日}^{あいだ}{間}^{ふ}{降}りつづけてる

\Alpha うすい^{やね}{屋根}に^{あ}{当}たる^{あめ}{雨}の^{おと}{音}はやたらとにぎやかで

\Alpha はじめのうちはやかましくてしょうがなかった

\Alpha でも、^{はんにち}{半日}、その^{なか}{中}にいたら^{な}{慣}れてしまった

\page[117]
\Alpha ^{あめ}{雨}はふきこんでくるけど

\Alpha ^{かぜ}{風}さえなければ、ここはぬれないですむ

\page[118]
\Alpha お^{きゃく}{客}さんが^{く}{来}るわけではないし

\Alpha お^{みせ}{店}の^{じゅんび}{準備}ができてるわけでもないから

\Alpha ^{べつ}{別}に^{いちにちじゅう}{一日中}、ここにいる^{ひつよう}{必要}はない

\Alpha でも、このごろは^{は}{晴}れてる^{とき}{時}よりも^{なが}{長}い^{じかん}{時間}、ここにいる

\page[119]
\Alpha ^{きゅう}{急}に^{くら}{暗}くなってきた

\Alpha まわりの^{けしき}{景色}が^{へん}{変}に^{きいろ}{黄色}くなった

\page[120]
\Alpha ^{とお}{遠}くの^{ほう}{方}の^{はやし}{林}からけむってきたと^{おも}{思}ったとたん

\Alpha としゃぶりになった

\Alpha ^{ちい}{小}さい^{みず}{水}の^{つぶ}{粒}が^{かお}{顔}にあたる

\page[121]
\Alpha ^{き}{木}でできた^{ところ}{所}が^{しっけ}{湿気}でぺとぺとになって

\Alpha ^{あめ}{雨}^{おと}{音}で、なにも^{き}{聞}こえなくなる

\page[123]
\Alpha ^{あめ}{雨}が^{すこ}{少}し^{よわ}{弱}くなって、^{ちか}{近}くの^{はやし}{林}からだんだん^{み}{見}えてくる

\Alpha まだ^{あか}{明}るいけど、^{でんちゅう}{電柱}のあかりがふたつついている

\Alpha ^{あめ}{雨}は^{ゆうがた}{夕方}にはあがると^{おも}{思}う

\page[124]
\Alpha ^{うみ}{海}のとなりだから、^{むし}{虫}はあまり^{で}{出}ないけど、それでも、やっぱり、いる

\page[125]
\Alpha ^{わたし}{私}のまわりにも^{か}{蚊}が^{と}{飛}んできた

\Alpha うでにとまって

\Alpha そして

\Alpha ^{すこ}{少}し^{こま}{困}ってるようだった

\Alpha ^{と}{飛}んでいった

\page[126]
\Alpha ^{か}{蚊}には^{わる}{悪}いけど、^{こんねん}{今年}はじめての^{かと}{蚊取}り^{せんこう}{線香}をつける

\page[127]
\Alpha ^{けむり}{煙}がたった^{しゅんかん}{瞬間}、
^{ご}{五}、^{ろく}{六}^{こ}{個}の^{ばめん}{場面}がパパッと^{はし}{走}って

\Alpha ^{どうじ}{同時}に、なにか、さみしくもなる

\page[128]
\Alpha ぱっと、^{なつ}{夏}になる


\subsection{第86話\ おつかれのイエー}

\page[131]
\Sign ^{こおり}{氷}

\Alpha お^{みせ}{店}をよしずで^{かこ}{囲}ってみた

\page[132]
\Alpha ^{やね}{屋根}の^{うえ}{上}にものせたので、^{ほら}{洞}くつみたいになったけど

\Alpha すずしい

\Alpha ^{むぎちゃ}{麦茶}とドクダミ^{ちゃ}{茶}とあとコーヒー^{まめ}{豆}も^{すこ}{少}し^{ようい}{用意}した

\Alpha まだ、お^{きゃく}{客}さんは^{こ}{来}ないだろうと^{おも}{思}ってたら

\Alpha ^{あさゆう}{朝夕}に^{つ}{釣}りの^{ひと}{人}やただ^{ある}{歩}いてる^{ひと}{人}がけっこういて、
^{なんにん}{何人}か^{よ}{寄}ってくれる

\page[133]
\Alpha お^{ちゃ}{茶}だけってのもなんだから、きゅうりの^{しお}{塩}もみとトマトを^{ひ}{冷}やしておく

\Alpha どこが「カフェ」かってメニュー

\Alpha カンビールが^{いっぽん}{一本}だけあるけど、これは^{だ}{出}さない

\Alpha ^{わたし}{私}のビール

\page[134]
\Alpha ^{こおり}{氷}

\Alpha ^{あさ}{朝}、^{ぎょきょう}{漁協}で、^{か}{買}ってきたのを、くだいて、シロップにつけてみる

\Alpha いまいち

\Alpha んーん

\page[135]
\Alpha ありがとございました

\Alpha さ!

\page[136]
\Alpha いえー

\Ojisan いえー


\subsection{第87話\ ^{いりえ}{入江}の^{もの}{者}たち}

\page[141]
\Ayase わりかった!

\page[144]
\Makki あっ
\ ごめん

\Makki おどろかした?

\page[145]
\Ayase あ〜〜
\ あせった!

\Ayase ここ^{なんねん}{何年}かで^{いちばん}{一番}あせった

\Makki ^{し}{知}ってるよ
\ アヤセ

\Makki ^{さかな}{魚}^{す}{好}きのミサゴマニアでしょ?

\Makki タカヒロにきいた

\Ayase ん〜〜
\ まあそう

\Ayase マニア・・

\page[146]
\Ayase タカヒロはよ
\ どうしてんよ

\Makki ^{きょう}{今日}はねー
\ バイトー

\Ayase そうか
\ もう、そんな^{とし}{歳}か

\Ayase はえーなー

\Makki アヤセさー

\Makki さっき、わたしのことミサゴって^{おも}{思}ったでしょ

\Ayase ん……まあ
\ ちょっとな……

\page[147]
\Makki ぱっと^{み}{見}
\ あんなかな?

\Ayase いや

\Ayase え?
\ 「あんな」って

\Ayase おめえも^{み}{見}たのか

\Makki ミサゴ?
\ ^{にかい}{2回}ぐらいね

\Ayase ^{にかい}{2回}……
\ いつごろよ

\Makki さ〜〜

\Makki もうずっと^{まえ}{前}だよ……
\ ^{わす}{忘}れちゃった

\page[148]
\Ayase おめえ
\ ^{いま}{今}いくつだ?

\Makki いくつみえる?

\Ayase さあね

\Makki じゅういち〜

\Ayase そうか……

\Ayase もうひとつ^{きかい}{機会}があったんだなー

\Ayase ^{いっぽ}{一歩}おそかったかな……

\page[149]
\Ayase おめえよ
\ ^{なまえ}{名前}は?

\Makki まつき!
”^{ま}{真}の^{つき}{月}”ってかいてね

\Makki みんなはマッキってゆうよ

\Ayase ほお

\Makki アヤセー
\ ミサゴってさ
\ やっぱ……

\Makki ロボットの^{ひと}{人}なのかなあ

\Ayase いや

\Ayase ^{じだい}{時代}、ちがうしな……
\ ミサゴの^{ほう}{方}が^{なんじゅうねん}{何十年}か^{ふる}{古}い

\page[150]
\Makki じゃ^{なに}{何}?

\Ayase うーん

\Ayase オレにもよくわかんねえんだけどな

\Ayase ^{ひと}{人}かなあと

\Makki なん^{じゅうねん}{十年}もおんなしで?

\Ayase そう
\ そうゆうのがな……
\ なんだかな

\Makki そっか

\page[151]
\Ayase マッキ、おめえ
\ そいつ、ひっぱり^{だ}{出}せたのか?

\Makki んーん

\Makki こいつから^{の}{乗}ってきたよ

\page[152]
\Makki わお


\subsection{第88話\ ミナミトビカマス}

\page[154]
\Ayase カマスの^{きげん}{機嫌}がよくなったんで、マッキに^{と}{飛}ぶところを^{み}{見}せてやる

\Makki かっこいー

\Ayase んー

\page[155]
\Ayase トビカマスはパートナーの^{て}{手}と^{しせん}{視線}を^{み}{見}ながら^{と}{飛}ぶ

\Ayase パートナーはそれを^{み}{見}て、^{さき}{先}のコースを^{あたま}{頭}の^{なか}{中}に^{えが}{描}く

\Makki そっか
\ で、こういってー

\Ayase そいで
\ こう……

\Ayase なんか、ズレるな

\page[157]
\Makki あ〜〜
\ あせった!

\Ayase そーなー

\Ayase マッキ
\ ^{いま}{今}、こいつ、おめえの^{め}{目}^{み}{見}て^{と}{飛}んでたぞ

\Makki えっ

\Ayase なあ

\Ayase こんだ、おめー
\ やってみねえか

\page[158]
\Makki んっ!

\page[159]
\Ayase カマスは^{どくとく}{独特}の^{うつく}{美}しい^{こ}{弧}を^{えが}{描}いて^{と}{飛}ぶ

\Ayase マッキの^{しせん}{視線}はその^{さき}{先}を^{しぜん}{自然}に^{よ}{読}んでいく

\page[160]
\Makki う

\Ayase マッキよう

\Ayase おれ、おめえぐれえできるようになるまで、^{なんねん}{何年}もかかったぞ

\page[161]
\Ayase じゃ
\ またくんわ

\Makki うん

\page[162]
\Makki またね

\Ayase じゃあな

\Makki おー

\page[164]
\Ayase ^{ゆうしょく}{夕食}ーー
\ カリントとお^{ちゃ}{茶}

\Ayase ^{く}{食}うぞ、てめえ


\subsection{みーやんのいつかまたあの^{こ}{子}と}
