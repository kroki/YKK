\section{Volume 9}

\subsection{^{だい}{第}77^{わ}{話}\ ^{しお}{塩}}

\page[5]
\A ^{うみ}{海}の^{みち}{道}

\A ^{なみ}{波}から^{に}{逃}げるように^{うえ}{上}へ^{うえ}{上}へと、
つけかえられてきた^{みち}{道}の^{かせき}{化石}

\page[8]
\A うちあげられた^{かいそう}{海藻}、^{てつ}{鉄}サビ、^{さかな}{魚}の^{あわ}{泡}

\A ^{しお}{塩}のしみた^{いた}{板}、^{まつ}{松}やに、^{こうぶつ}{鉱物}^{ゆ}{油}

\page
\A ^{やま}{山}で^{わす}{忘}れていた、^{わたし}{私}のにおいだ


\subsection{第78話\ むらさきの^{ひとみ}{瞳}}

\page[15]
\T あ……
\ いらっしゃ……

\page
\A わっ!!

\A あ〜〜
\ しっぱいした!

\T アルファ

\A へへ

\A ひさしぶり

\page
\T おかえり

\A うん
\ ただいま

\A ふ〜ん
\ お^{みせ}{店}、^{てつだ}{手伝}ってるんだ

\T うん
\ じいちゃんと^{はんはん}{半々}くらいかなー

\Sign ^{だいこん}{大根}
\ ^{とうゆ}{灯油}
\ ガソリン

\T ほとんどダイコン^{や}{屋}のコゾーだよ

\A へー

\page
\A えらいなあ

\T あっ、イス^{も}{持}ってくるよ
\ なんかのむ?

\A いただきー

\page
\T はい
\ お^{ちゃ}{茶}しかなかった

\A ありがと

\page[21]
\T あ

\T はい、お^{ちゃ}{茶}

\A あ
\ うん

\A ^{せい}{背}たけ

\A また、のびたね!
\ びっくりしちゃった

\T うん

\page
\T ^{いちねん}{一年}か……

\T もっと^{なが}{長}い^{あいだ}{間}いなかった^{き}{気}がするけど

\A そだね

\T でも、なんか、おとといくらいに^{あ}{会}ったみたいな^{かん}{感}じもする

\A そだね

\A なんかさー
\ ^{こえ}{声}
\ ^{ひく}{低}くなってない?

\T まーねー
\ なんかねー

\page
\T ^{さいしょ}{最初}、へんなカスレた^{こえ}{声}になってきてね……

\T ^{すこ}{少}ししたら、^{たか}{高}い^{ところ}{所}の^{こえ}{声}が
\ スパッと^{で}{出}なくなって

\A そっか

\A なーんかさー
\ お^{ねえ}{姉}さん^{かぜ}{風}ふかせらんないなー、もう

\T そんなことないよ

\page[25]
\A さあて
\ ^{ひ}{日}が^{お}{落}ちる^{まえ}{前}に、うち、^{あ}{開}けなきゃ
\ ごっそさま

\T うん

\T そだ

\T バイク、^{かえ}{返}さなきゃな……

\A あ、いつでもいいよ

\T ^{きょう}{今日}、^{くるま}{車}なんだ
\ ^{おく}{送}ってこうか?

\A ^{くるま}{車}!
\ へ〜〜
\ もう^{うんてん}{運転}できるんだー

\page
\A んーー、でも、やっぱ^{きょう}{今日}は^{ある}{歩}いてこうかなあ
\ ごめん、^{こんど}{今度}、^{おく}{送}ってね

\T そか
\ じゃ、また^{こんど}{今度}

\A うん

\page
\T じいちゃんにも^{い}{言}っとくよ

\A ありがと!


\subsection{第79話\ ^{つち}{土}の^{よる}{夜}}

\page[30]
\A なんか、わざとゆっくり^{かえ}{帰}った

\page
\A ^{ひ}{日}は、もうだいぶ^{まえ}{前}に^{お}{落}ちている

\page
\A ^{みせ}{店}の^{か}{欠}けた、^{うち}{家}の^{かたち}{形}には、まだ、^{な}{慣}れてない

\page
\A ブレーカーをあげる、ガスせんをひらく

\A ただいまー

\page
\Sign アルファさんへ
\ ^{たか}{鷹}^{つ}{津}ココネ

\page
\A ココネが^{にど}{二度}ほど^{き}{来}たようだ

\page[37]
\A メイポロの^{げんえき}{原液}があった

\A コゲる^{すんぜん}{寸前}まで^{に}{煮}つめて、お^{ゆ}{湯}で^{わ}{割}って

\A うちのことあれこれを^{おも}{思}い^{だ}{出}してくる

\page[41]
\A ^{いちねん}{一年}ぶりにねむったような^{き}{気}がした

\A ^{からだ}{体}がふとんにとけていく

\page
\A ふとんごと^{じめん}{地面}にとけていく


\subsection{第80話\ ^{かざみ}{風見}^{さかな}{魚}}

\page[46]
\A とりあえず

\page
\A こいつだけでも^{た}{立}てとこうと^{おも}{思}う

\A えと

\A どこだったっけ……


\subsection{第81話\ ^{いちねん}{一年}^{くうかん}{空間}}

\page[56]
\O よ

\A だだいば〜〜!!!

\O あだだだだだだだだだ

\page
\A ごめんなさい

\O やー

\O まー
\ ^{ぶじ}{無事}けえれてよかった

\O ほ
\ あずかってたカギよー

\A ありがとうございました

\page
\A るすの^{あいだ}{間}の^{はなし}{話}を^{き}{聞}いた

\A ココネが4^{かい}{回}^{き}{来}ていたこと

\A おじさんが^{あし}{足}にケガして、しばらく^{ある}{歩}けなかったこと

\A ^{わたし}{私}の^{し}{知}らない^{まいにち}{毎日}が、ほんとにあった

\page
\O へーよ
\ アルファさんよ

\A あ、はい

\O へー
\ どうよ

\O ^{りょこう}{旅行}……
\ ^{い}{行}ってみて

\A ^{い}{行}ってよかったです、やっぱ

\A ^{わたし}{私}も、まあ、^{ちか}{近}くの^{くに}{国}のことくらいは……

\A かなり^{し}{知}ってるつもりで、いたんですけど

\page
\A ^{ほん}{本}でわかることと^{げんち}{現地}で^{かん}{感}じることは……

\A ^{べつ}{別}モノなんだなあって

\O そうな

\page
\A もっと^{とお}{遠}くまで^{い}{行}くつもりだったんですけどね

\A ^{みず}{水}とかガスの^{りょうきん}{料金}とるとこがあったり……

\A あと、^{いごこち}{居心地}よくて、^{ながい}{長居}しちゃったりとか

\A ^{じかん}{時間}もお^{かね}{金}も、ぜんぜん^{た}{足}りませんよー

\page
\O アルファさんよ

\A はい?

\O ^{かせ}{稼}ぎの^{ほう}{方}は?

\A かせぎ……
\ バイト……いっぱいやりましたよー

\A ちょっとだけ^{だい}{大}^{あかじ}{赤字}かな……

\O まー
\ そんなだべなー

\page
\A そんななんで

\A お^{みせ}{店}……
\ ^{かんぺき}{完璧}に^{もとどお}{元通}りにするのは、まだだーいぶ^{さき}{先}になりそうです

\A だーーいぶ

\O まー
\ ゆっくしやんな

\O ^{いじ}{意地}でも、^{もとどお}{元通}りじゃねえとやだってんじゃなきゃー

\O ちっとぐれー
\ アルファさんの^{かって}{勝手}がはいっても、いいかもしんねーしなー

\page
\O こんだ、タカがバイクもってくんって

\A はい

\O あのやろ、エンジンものの^{しごと}{仕事}がしてえだと

\A ありゃま

\O じゃ、またな

\A あ
\ おじさん

\page
\A くび……
\ だいじょうぶ?

\O おお
\ そうよ

\O まー
\ いてえっちゃいつもどっかいてえしなー

\O じゃ

\page[67]
\A まだ、お^{ひる}{昼}だ

\A ^{たいよう}{太陽}がものすごく、のろい

\A ああ

\A ^{きょう}{今日}は、1キロも^{ある}{歩}いてないんだなあ

\page
\A きのうまでの^{いちねん}{一年}^{かん}{間}が、じわじわひとかたまりに

\A ^{かこ}{過去}になっていく

\A うまれてはじめて

\A ひとつ^{とし}{齢}をとった^{き}{気}がしている


\subsection{第82話\ クロマツ^{どお}{通}り}

\page[73]
\S いやー

\S そうじゃないかと^{おも}{思}ったのよ

\S このへんじゃ^{わか}{若}いコ、めったに^{み}{見}ないし

\S なんか、^{ふんいき}{雰囲気}がいつもアルファさんが^{はな}{話}す^{かん}{感}じの、まんまだったから

\K は
\ はじめまして

\page
\S で
\ ここで^{ま}{待}ち^{あ}{合}わせなの?

\K はい

\K ^{いえ}{家}にはいっちゃうとでかけるのが、おっくうになるから

\K ^{きょう}{今日}は^{そと}{外}で^{ごうりゅう}{合流}しようってアルファさんが……

\S なるほど

\S アルファさんがおくれるなんてめずらしいわね

\K あ、いえ、まだあと2^{じかん}{時間}あるんです

\K ^{わたし}{私}が^{はや}{早}すぎたんです

\S あらら

\page
\S ココネさん、よかったら、^{わたし}{私}の^{ところ}{所}で^{ま}{待}たない?

\K は?

\S すぐそこだから、バイク、^{もん}{門}のとこ^{お}{置}いとけば、アルファさん、わかるわよ

\K え……
\ でもー

\K お^{しごと}{仕事}のおじゃまに……

\S あ
\ それは、ほぼ^{だいじょうぶ}{大丈夫}

\page
\S うち、^{いそが}{忙}しさは、カフェアルファなみだから

\S お^{ちゃ}{茶}、たくさんあるわよ

\K わっ

\page
\S どうぞ

\K おじゃまします

\K あ……
\ ひさしぶり^{びょういん}{病院}のにおい

\S まー
\ ^{びょういん}{病院}っていってもね
\ ^{おうきゅうしょち}{応急処置}くらいがせいいっぱいなのよ、^{いま}{今}は

\K はー

\K でも、アルファさんはこちらで^{だいしゅじゅつ}{大手術}してもらったと

\S ええ、まあね、そこはほれ

\S まーー
\ あがっちゃって、あがっちゃって

\K あ、はい

\page
\S で
\ まあ、^{おや}{親}の^{いさん}{遺産}^{く}{食}いつぶしてるのよ

\K はー

\K ^{せんせい}{先生}

\S はい?

\K ^{せんせい}{先生}は、その……

\K アルファさんに^{いぜん}{以前}、^{き}{聞}いたことなんですけど……

\page
\K ^{わたし}{私}たち……
\ っていうか……
\ その……

\K 「アルファタイプ」の^{むかし}{昔}のことに、くわしいと……

\K あの……

\K いきなりで^{しつれい}{失礼}かと^{おも}{思}いますけど、^{すこ}{少}し^{おし}{教}えていただきたいなと……

\S ココネさんはそういう^{はなし}{話}が^{だい}{大}^{す}{好}きだって

\S アルファさんが^{い}{言}ってたわ

\page
\S ^{かのじょ}{彼女}、あんまりそういう^{こと}{事}に^{きょうみ}{興味}^{しめ}{示}さないもんね

\S そうでしょ

\S ええ……そうね

\S ^{わたし}{私}は、あなたたちの^{れきし}{歴史}のほんとに^{さいしょ}{最初}の^{ところ}{所}に、
ちょっとだけいたわ

\S ^{つぎ}{次}に^{かか}{係}わった^{とき}{時}にはもう、
ココネさん^{たち}{達}の^{ちょっけい}{直系}のお^{ねえ}{姉}さんがいたから……

\page
\S えび^{ちゃいろ}{茶色}の^{かみ}{髪}と^{め}{目}をした

\S そうね……
\ みんなより^{すこ}{少}し^{としうえ}{年上}に^{み}{見}える^{こ}{子}だったわね

\K A7M1^{かた}{型}^{き}{機}……!
\ まぼろしの……

\S そう、くわしいのね
\ へえ、「^{まぼろし}{幻}の」なの

\page
\S ^{かのじょ}{彼女}も「アルファ」って^{なまえ}{名前}なのよ、
\ ^{わたし}{私}の^{みょうじ}{名字}でね……

\S ^{こ}{子}^{うみ}{海}^{いし}{石}アルファ


\subsection{第83話\ ^{あお}{青}い^{おと}{音}}

\page[84]
\K ^{わたし}{私}とアルファさんの「お^{ねえ}{姉}さん」の^{はなし}{話}を^{せんせい}{先生}に^{き}{聞}きました

\K ここに^{かのじょ}{彼女}がつれていられた^{とき}{時}のこと

\K ここでしばらく^{せいかつ}{生活}していたこと

\page
\K まばたきくらいしかできなかった^{かのじょ}{彼女}がはじめて^{い}{言}った^{じょうだん}{冗談}のこと

\K ^{あお}{青}い^{もの}{物}が^{す}{好}きなこと
\ ^{ふく}{服}がキライなこと

\K いままで^{わたし}{私}がためこんできた^{ちしき}{知識}が^{かる}{軽}くふきとぶ、
^{ち}{血}の^{かよ}{通}ったエピソード

\S その^{こ}{子}が、またつれていかれた

\S それからだいぶ^{なが}{長}いことたって

\S すぐ^{ちか}{近}くにロボットのコがいることを^{し}{知}ったの

\S それが、^{いま}{今}^{く}{来}るアルファさん

\K はー

\page
\K アルファさんが^{く}{来}るまでにはまだ^{すこ}{少}しある

\K あの、^{せんせい}{先生}
\ A7の^{まえ}{前}って……あの
\ どんなだったんでしょう

\S ^{わたし}{私}もそのあたり^{し}{知}らないのよ
\ A5とか、6とか

\S どこかにちゃんと^{きろく}{記録}はあるんだろうけど

\S でもね……たぶん、^{たの}{楽}しい^{はなし}{話}ばかりじゃないと^{おも}{思}うのよ

\K ^{わたし}{私}も

\K なんか、そんな^{き}{気}がしてました

\page
\S ^{わたし}{私}が^{し}{知}ってるのはA7の^{ひと}{人}たちだけなんだけど

\S どこか^{かんかく}{感覚}^{てき}{的}なみんなを^{み}{見}てると、
とてもハードな^{かてい}{過程}があったことはわかるわ

\S そうだ
\ ココネさん

\K はい

\S いい^{きかい}{機会}だから

\page
\S こうゆうのがあってね

\Sign Aー2(M1ーM4)

\K そそそそ

\S そう
\ 「Aー2」って^{か}{書}いてあるでしょ

\S レコードなんだけどある^{いみ}{意味}、みんなのご^{せんぞ}{先祖}^{さま}{様}なのかもしれないわ

\page
\S あの^{たね}{種}の^{おと}{音}やリズムのサンプル^{しゅう}{集}で

\S ^{ふつう}{普通}に^{き}{聞}いても、どうってことないものなんだけど

\S ^{き}{聞}いてみる?

\page[91]
\K ^{くうちゅう}{空中}に^{じゅうまん}{充満}する
\ ^{おと}{音}の^{つぶつぶ}{粒々}

\K これは^{おんがく}{音楽}だろうか

\K ^{すこ}{少}し^{ちが}{違}う^{き}{気}がする

\K ^{あま}{甘}く、^{にが}{苦}く、^{はな}{鼻}の^{おく}{奥}がすずしいような

\K ^{かわ}{乾}いた^{こえだ}{小枝}をポキポキ^{お}{折}るような^{はだ}{肌}ざわり

\K ああ……そうか
\ これは……

\K ^{わたし}{私}の^{し}{知}っている^{けしき}{景色}に^{に}{似}ている

\page[93]
\K その^{じどう}{児童}^{かん}{館}で^{み}{見}てから、ずーっと^{き}{気}になってたんです

\K いつかは^{き}{聞}かなきゃって

\S へえ……
\ こんなレコードうちにしかないと^{おも}{思}ってたわ
\ びっくり

\S ^{むかし}{昔}は、なんだこれって^{おも}{思}ってたのよね

\S ^{いま}{今}、^{き}{聞}いてみたらこれはこれで、けっこういいかもしれないわ

\page
\A おじゃまします!!


\subsection{第84話\ ^{かいばつ}{海抜}70}

\page[96]
\K どうも
\ おじゃましました

\S いーえー

\page
\K お^{ひる}{昼}
\ ^{うみぞ}{海沿}いの^{すな}{砂}だらけの^{みち}{道}を^{なんか}{南下}

\K アルファさんと^{いっしょ}{一緒}に^{はし}{走}るのは^{まえ}{前}に^{おく}{送}ってもらった^{とき}{時}、
^{いらい}{以来}です

\page
\K バイクでこのへんを^{はし}{走}ってみると

\K アルファさんが^{わたし}{私}の^{す}{住}むムサシノを
「^{たい}{平}らな^{くに}{国}だ」と^{い}{言}ったワケがよくわかります

\page
\K フワッフワッと^{き}{気}まぐれに^{ま}{曲}がる、
アルファさんの^{えら}{選}ぶ^{みち}{道}は^{じょうげさゆう}{上下左右}、^{なみ}{波}のようにうねり

\K ^{さき}{先}の^{み}{見}えないトンネルみたいな^{たに}{谷}の^{みち}{道}と、
^{うみ}{海}へジャンプするような^{おね}{尾根}^{みち}{道}をくりかえします

\page
\K ^{やま}{山}の^{なか}{中}に^{へん}{変}に^{あお}{青}く^{なみ}{波}のない^{すいめん}{水面}がありました

\K ^{いけ}{池}のような、
それはくねくね^{ま}{曲}がった^{たに}{谷}の^{おく}{奥}まで^{はい}{入}りこんだ^{うみ}{海}なのだそうです

\K ^{ひかり}{光}とにおい、^{いろ}{色}のパターンがどっとおしよせてくる^{みち}{道}

\page
\K ^{からだ}{体}のどこか

\K ^{いま}{今}まで^{つか}{使}ってなかったところが、^{あ}{開}いていく^{かんかく}{感覚}

\page[103]
\A ここはね
\ いつも^{ひとり}{一人}で^{く}{来}るんだ

\A ^{やま}{山}ってほど^{たか}{高}くないけど、ここらじゃ^{いちばん}{一番}^{たか}{高}いし

\A ^{わたし}{私}、^{たか}{高}いとこ^{だい}{大}^{す}{好}きだから

\A ここでいろいろ^{かんが}{考}えたり、^{かんが}{考}えなかったり

\page
\K ハチジョウススキトベラ

\K はじめて^{き}{聞}く^{くさき}{草木}の^{なまえ}{名前}です

\K その^{なか}{中}を^{とお}{通}ってくるとろっとした^{こ}{濃}い

\K ^{くうき}{空気}のにおい

\page
\K アルファさん

\K ^{わたし}{私}たち
\ ^{おと}{音}やにおいでできてるんですよ……

\K たとえばなしとかじゃなくて……

\A ^{し}{知}ってるよー

\page[108]
\A ごめんね〜〜
\ ^{ね}{寝}ちゃって

\A ^{お}{起}こしてくれればよかったのに〜〜

\K いえ

\page
\A じゃー
\ バイクはここにかためといてー
\ はい
\ あがってー

\K おじゃまします

\K ^{わたし}{私}

\K ここに^{す}{住}みたいなあ

\page
\A ココネー

\K あっ
\ はい!


\subsection{第85話\ かえる}

\page[115]
\A ^{なみいた}{波板}が^{て}{手}にはいったので、ひさしをつけてみた

\A ^{ごうせいじゅし}{合成樹脂}の^{なみいた}{波板}は^{はんとうめい}{半透明}の^{にゅうはくいろ}{乳白色}で

\A ^{した}{下}からすかして^{み}{見}る、
^{たいよう}{太陽}の^{ひかり}{光}がだいだい^{いろ}{色}の^{わ}{輪}っかになる

\page
\A その^{ひ}{日}の^{よる}{夜}らか^{あめ}{雨}はもう^{いつか}{五日}^{あいだ}{間}^{ふ}{降}りつづけてる

\A うすい^{やね}{屋根}に^{あ}{当}たる^{あめ}{雨}の^{おと}{音}はやたらとにぎやかで

\A はじめのうちはやかましくてしょうがなかった

\A でも、^{はんにち}{半日}、その^{なか}{中}にいたら^{な}{慣}れてしまった

\page
\A ^{あめ}{雨}はふきこんでくるけど

\A ^{かぜ}{風}さえなければ、ここはぬれないですむ

\page
\A お^{きゃく}{客}さんが^{く}{来}るわけではないし

\A お^{みせ}{店}の^{じゅんび}{準備}ができてるわけでもないから

\A ^{べつ}{別}に^{いちにちじゅう}{一日中}、ここにいる^{ひつよう}{必要}はない

\A でも、このごろは^{は}{晴}れてる^{とき}{時}よりも^{なが}{長}い^{じかん}{時間}、ここにいる

\page
\A ^{きゅう}{急}に^{くら}{暗}くなってきた

\A まわりの^{けしき}{景色}が^{へん}{変}に^{きいろ}{黄色}くなった

\page
\A ^{とお}{遠}くの^{ほう}{方}の^{はやし}{林}からけむってきたと^{おも}{思}ったとたん

\A としゃぶりになった

\A ^{ちい}{小}さい^{みず}{水}の^{つぶ}{粒}が^{かお}{顔}にあたる

\page
\A ^{き}{木}でできた^{ところ}{所}が^{しっけ}{湿気}でぺとぺとになって

\A ^{あめ}{雨}^{おと}{音}で、なにも^{き}{聞}こえなくなる

\page[123]
\A ^{あめ}{雨}が^{すこ}{少}し^{よわ}{弱}くなって、^{ちか}{近}くの^{はやし}{林}からだんだん^{み}{見}えてくる

\A まだ^{あか}{明}るいけど、^{でんちゅう}{電柱}のあかりがふたつついている

\A ^{あめ}{雨}は^{ゆうがた}{夕方}にはあがると^{おも}{思}う

\page
\A ^{うみ}{海}のとなりだから、^{むし}{虫}はあまり^{で}{出}ないけど、それでも、やっぱり、いる

\page
\A ^{わたし}{私}のまわりにも^{か}{蚊}が^{と}{飛}んできた

\A うでにとまって

\A そして

\A ^{すこ}{少}し^{こま}{困}ってるようだった

\A ^{と}{飛}んでいった

\page
\A ^{か}{蚊}には^{わる}{悪}いけど、^{こんねん}{今年}はじめての^{かと}{蚊取}り^{せんこう}{線香}をつける

\page
\A ^{けむり}{煙}がたった^{しゅんかん}{瞬間}、
^{ご}{五}、^{ろく}{六}^{こ}{個}の^{ばめん}{場面}がパパッと^{はし}{走}って

\A ^{どうじ}{同時}に、なにか、さみしくもなる

\page
\A ぱっと、^{なつ}{夏}になる


\subsection{第86話\ おつかれのイエー}

\page[131]
\Sign ^{こおり}{氷}

\A お^{みせ}{店}をよしずで^{かこ}{囲}ってみた

\page
\A ^{やね}{屋根}の^{うえ}{上}にものせたので、^{ほら}{洞}くつみたいになったけど

\A すずしい

\A ^{むぎちゃ}{麦茶}とドクダミ^{ちゃ}{茶}とあとコーヒー^{まめ}{豆}も^{すこ}{少}し^{ようい}{用意}した

\A まだ、お^{きゃく}{客}さんは^{こ}{来}ないだろうと^{おも}{思}ってたら

\A ^{あさゆう}{朝夕}に^{つ}{釣}りの^{ひと}{人}やただ^{ある}{歩}いてる^{ひと}{人}がけっこういて、
^{なんにん}{何人}か^{よ}{寄}ってくれる

\page
\A お^{ちゃ}{茶}だけってのもなんだから、きゅうりの^{しお}{塩}もみとトマトを^{ひ}{冷}やしておく

\A どこが「カフェ」かってメニュー

\A カンビールが^{いっぽん}{一本}だけあるけど、これは^{だ}{出}さない

\A ^{わたし}{私}のビール

\page
\A ^{こおり}{氷}

\A ^{あさ}{朝}、^{ぎょきょう}{漁協}で、^{か}{買}ってきたのを、くだいて、シロップにつけてみる

\A いまいち

\A んーん

\page
\A ありがとございました

\A さ!

\page
\A いえー

\O いえー


\subsection{第87話\ ^{いりえ}{入江}の^{もの}{者}たち}

\page[141]
\Y わりかった!

\page[144]
\M あっ
\ ごめん

\M おどろかした?

\page
\Y あ〜〜
\ あせった!

\Y ここ^{なんねん}{何年}かで^{いちばん}{一番}あせった

\M ^{し}{知}ってるよ
\ アヤセ

\M ^{さかな}{魚}^{す}{好}きのミサゴマニアでしょ?

\M タカヒロにきいた

\Y ん〜〜
\ まあそう

\Y マニア・・

\page
\Y タカヒロはよ
\ どうしてんよ

\M ^{きょう}{今日}はねー
\ バイトー

\Y そうか
\ もう、そんな^{とし}{歳}か

\Y はえーなー

\M アヤセさー

\M さっき、わたしのことミサゴって^{おも}{思}ったでしょ

\Y ん……まあ
\ ちょっとな……

\page
\M ぱっと^{み}{見}
\ あんなかな?

\Y いや

\Y え?
\ 「あんな」って

\Y おめえも^{み}{見}たのか

\M ミサゴ?
\ ^{にかい}{2回}ぐらいね

\Y ^{にかい}{2回}……
\ いつごろよ

\M さ〜〜

\M もうずっと^{まえ}{前}だよ……
\ ^{わす}{忘}れちゃった

\page
\Y おめえ
\ ^{いま}{今}いくつだ?

\M いくつみえる?

\Y さあね

\M じゅういち〜

\Y そうか……

\Y もうひとつ^{きかい}{機会}があったんだなー

\Y ^{いっぽ}{一歩}おそかったかな……

\page
\Y おめえよ
\ ^{なまえ}{名前}は?

\M まつき!
”^{ま}{真}の^{つき}{月}”ってかいてね

\M みんなはマッキってゆうよ

\Y ほお

\M アヤセー
\ ミサゴってさ
\ やっぱ……

\M ロボットの^{ひと}{人}なのかなあ

\Y いや

\Y ^{じだい}{時代}、ちがうしな……
\ ミサゴの^{ほう}{方}が^{なんじゅうねん}{何十年}か^{ふる}{古}い

\page
\M じゃ^{なに}{何}?

\Y うーん

\Y オレにもよくわかんねえんだけどな

\Y ^{ひと}{人}かなあと

\M なん^{じゅうねん}{十年}もおんなしで?

\Y そう
\ そうゆうのがな……
\ なんだかな

\M そっか

\page
\Y マッキ、おめえ
\ そいつ、ひっぱり^{だ}{出}せたのか?

\M んーん

\M こいつから^{の}{乗}ってきたよ

\page
\M わお


\subsection{第88話\ ミナミトビカマス}

\page[154]
\Y カマスの^{きげん}{機嫌}がよくなったんで、マッキに^{と}{飛}ぶところを^{み}{見}せてやる

\M かっこいー

\Y んー

\page
\Y トビカマスはパートナーの^{て}{手}と^{しせん}{視線}を^{み}{見}ながら^{と}{飛}ぶ

\Y パートナーはそれを^{み}{見}て、^{さき}{先}のコースを^{あたま}{頭}の^{なか}{中}に^{えが}{描}く

\M そっか
\ で、こういってー

\Y そいで
\ こう……

\Y なんか、ズレるな

\page[157]
\M あ〜〜
\ あせった!

\Y そーなー

\Y マッキ
\ ^{いま}{今}、こいつ、おめえの^{め}{目}^{み}{見}て^{と}{飛}んでたぞ

\M えっ

\Y なあ

\Y こんだ、おめー
\ やってみねえか

\page
\M んっ!

\page
\Y カマスは^{どくとく}{独特}の^{うつく}{美}しい^{こ}{弧}を^{えが}{描}いて^{と}{飛}ぶ

\Y マッキの^{しせん}{視線}はその^{さき}{先}を^{しぜん}{自然}に^{よ}{読}んでいく

\page
\M う

\Y マッキよう

\Y おれ、おめえぐれえできるようになるまで、^{なんねん}{何年}もかかったぞ

\page
\Y じゃ
\ またくんわ

\M うん

\page
\M またね

\Y じゃあな

\M おー

\page[164]
\Y ^{ゆうしょく}{夕食}ーー
\ カリントとお^{ちゃ}{茶}

\Y ^{く}{食}うぞ、てめえ


\subsection{みーやんのいつかまたあの^{こ}{子}と}
