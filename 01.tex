\section{Volume 1}

\subsection{ヨコハマ^{か}{買}い^{だ}{出}し^{きこう}{紀行}}

\page[9]
\Sign Closed ^{もく}{木}ようまで
\ ^{か}{買}いだしです
\ Cafe アルファ

\page[10]
\Alpha すいませーん

\Alpha もしもーし
\ おーーい

\Sign ガソリン
\ あり^{ます}{〼}

\page[11]
\Ojisan あー
\ どうもね

\Alpha ^{まん}{満}タンお^{ねが}{願}いしまず

\Ojisan お^{きゃく}{客}なんて
\ ^{いっしゅうかん}{1週間}ぶりだから
\ どうもね

\Ojisan おや

\page[12]
\Ojisan お^{きゃく}{客}さん^{にし}{西}の^{みさき}{岬}のコーヒー^{や}{屋}さんだべ
\ アルファさん

\Alpha えっ?
\ ええ
\ そうですけど

\Ojisan あんた^{ゆうめいじん}{有名人}だよファン^{おお}{多}いよ

\Ojisan かわいいってね

\Alpha そんな
\ え〜〜〜

\Alpha でっ
\ でも
\ うち
\ めったにお^{きゃく}{客}さん^{き}{来}ませんよ

\Ojisan たまに^{とお}{通}んの^{み}{見}るだけでいいだとよ
\ てれてんだよ

\page[13]
\Ojisan ほい
\ ^{まん}{満}タン
\ ^{とおで}{遠出}かい

\Alpha ええ
\ ^{よこはま}{横浜}まで^{まめ}{豆}^{か}{買}いに

\Alpha いくら?

\Ojisan ^{よこはま}{横浜}!!
\ そりゃ^{とお}{遠}いや!!

\Ojisan いいよ
\ ^{きょう}{今日}はサービスだ!!

\Alpha そりゃわるいっすよ

\Ojisan ま
\ いいからよ
\ ^{き}{気}いつけなよ
\ ^{みち}{道}がわりいからよ

\Ojisan いまなし^{みち}{道}ィ
\ ^{な}{無}くなんしよー
\ たぶん
\ まっつぐは^{い}{行}けめえ

\page[14]
\Ojisan ^{かえ}{帰}ってきたらまた^{よ}{寄}りなよ
\ したら^{あと}{後}でコーヒーごちになんから
\ ^{だいきん}{代金}として

\Alpha ありがとうおじさん

\Ojisan あー
\ あんたロボットだって?

\Ojisan いいねー
\ ^{けんこう}{健康}そうで
\ おじさんなんて^{からだ}{体}
\ ガッタガタ

\Alpha ふふ
\ じゃ
\ とりかえます?

\page[15]
\Ojisan へへ
\ お^{たが}{互}い
\ ^{かみ}{神}さまがくれた^{からだ}{体}はまっとうしなきゃな

\Ojisan ^{き}{気}ィつけな

\Alpha ありがとー

\Ojisan じつはわしもファン

\page[16]
\Alpha ^{なんねん}{何年}か^{まえ}{前}^{わたし}{私}のオーナーは

\Alpha ^{わたし}{私}に^{みせ}{店}をあずけたいきなり
\ どこかへ^{い}{行}ってしまった

\Alpha どこにいるやら
\ ^{なに}{何}やってるやら

\Alpha いつかは^{かえ}{帰}ってくるのかしらね

\Alpha ^{わたし}{私}はロボットでよかったと^{おも}{思}う

\Alpha いくらでも^{ま}{待}っていられるからね

\page[18]
\Alpha ありゃま

\Alpha うーん
\ ^{ねんねん}{年々}^{うみ}{海}が^{あ}{上}がってくるみたい

\page[19]
\Alpha だから^{いま}{今}はこういった^{おね}{尾根}^{みち}{道}ばかり

\page[20]
\Alpha ^{よこはま}{横浜}も^{うみ}{海}の^{した}{下}で

\Alpha ^{おか}{丘}の^{うえ}{上}が^{いま}{今}の^{よこはま}{横浜}です

\page[22]
\Alpha あのー
\ コーヒー^{まめ}{豆}ください

\Person あい

\Alpha ^{すう}{数}^{ねんまえ}{年前}までの「^{だいとし}{大都市}ヨコハマ」が^{ゆめ}{夢}みたい

\Alpha ^{いま}{今}はゆったり^{とき}{時}の^{なが}{流}れる「^{ひと}{人}の^{まち}{街}」

\page[23]
\Alpha この^{すう}{数}^{ねん}{年}で^{よ}{世}の^{なか}{中}も^{ずいぶん}{随分}^{か}{変}わったわ

\Alpha ^{じだい}{時代}の^{たそがれ}{黄昏}がこんなにゆったりのんびりと^{く}{来}る
\ ものだったなんて

\Alpha ^{わたし}{私}は^{たぶん}{多分}この^{たそがれ}{黄昏}の^{よ}{世}を
\ ずっと^{み}{見}ていくんだと^{おも}{思}う

\page[24]
\Sign アルファへ
\ ^{げんき}{元気}なようで
\ ^{あんしん}{安心}しました

\Alpha ^{わたし}{私}には^{じかん}{時間}はいくらでもあるからね

\Alpha おじさん
\ ただいまー

\Ojisan おー


\subsection{^{だい}{弟}1^{わ}{話}\ ^{はがね}{鋼}の^{かお}{香}る^{よる}{夜}}

\page[29]
\Alpha ふう

\Alpha なにか
\ もっと^{でんごん}{伝言}ほしかったよなー

\Sign アルファへ
\ ^{げんき}{元気}なようで
\ ^{あんしん}{安心}しました

\Alpha あっ
\ いらっしゃいませ!

\Ojisan よう
\ また^{き}{来}たよ

\Alpha ブレンドね!

\page[30]
\Ojisan ヒマそうだな
\ え?

\Alpha ^{きょう}{今日}はおじさんだけよ

\Ojisan !おや

\Alpha あ

\Ojisan へえ!
\ アルファさんも^{てっぽう}{鉄砲}^{も}{持}ちかよ

\Alpha ええ
\ あ
\ ^{だいじょうぶ}{大丈夫}ですよ
\ タマ^{い}{入}れてませんから

\Ojisan へえ
\ この^{かた}{型}はめずらしいな

\Ojisan ふつう^{も}{持}つならあれとか

\page[31]
\Ojisan ええ
\ その^{こうせき}{鉱石}みたいな^{かたち}{形}と^{いろ}{色}がいいでしょ

\Alpha オーナーの

\Alpha プレゼントなんです

\Ojisan ほお

\Alpha ^{ごしん}{護身}^{よう}{用}にって^{わた}{渡}されたんです
\ けど

\Alpha ^{わたし}{私}にとっては^{ぶき}{武器}っていうか

\page[32]
\Alpha う〜〜〜ん
\ そういうものじゃないんです

\Ojisan よく^{み}{見}ると
\ ^{じゅう}{銃}はカドがとれてピカピカしていた

\Ojisan たぶんいつもなでていたんだろう

\Ojisan ^{たから}{宝}もんだな

\Alpha え

\Ojisan ^{たから}{宝}もんだ

\Alpha えへへ

\page[33]
\Alpha おかわり
\ サービスでネ

\Ojisan おおーー
\ わりいな

\page[34]
\Ojisan ここは^{ほんとう}{本当}におちつく

\Alpha ちゃんとした^{みせ}{店}なら
\ ^{かまくら}{鎌倉}や^{はやま}{葉山}あたりにゃまだあんけどよ

\Alpha うちは
\ ほとんどコーヒー^{つ}{付}き
\ ^{てんぼう}{展望}^{だい}{台}ですから

\Alpha わたし
\ ここが^{す}{好}きです

\Alpha こうやっておじさんとかと^{はな}{話}したり
\ ^{うみ}{海}^{み}{見}たり

\Alpha にぎやかな^{とき}{時}も
\ ひとりの^{とき}{時}も

\Alpha みんな^{す}{好}き

\page[35]
\Alpha じゃん

\Ojisan あにそれ

\Alpha オーナーの^{しゅみ}{趣味}です
\ ^{げっきん}{月琴}っていうの

\Alpha ふだん
\ ^{ひとまえ}{人前}では^{ひ}{弾}かないんです
\ けど

\page[36]
\Alpha なんか
\ ^{きょう}{今日}だけね

\Ojisan お
\ おう

\Takahiro あっ
\ こっちにいたの

\Alpha あら!

\Ojisan ちっ

\Alpha いらっしゃい

\Ojisan おう

\Takahiro 「おう」じゃないよじいちゃん
\ ひとりで^{き}{来}てんたもんなーー

\Ojisan おう

\Alpha タカヒロなんか^{の}{飲}む?

\page[37]
\Takahiro じゃ
\ メイポロ
\ ね
\ なにそれやってよ

\Alpha なんか
\ にぎやかんなっちゃったね

\Alpha いい?
\ タカヒロ
\ ^{きょう}{今日}だけの^{とくべつ}{特別}だからネ

\Alpha ^{ちゅうごく}{中国}の
\ ^{よる}{夜}の^{きょく}{曲}です

\page[39]
\Ojisan とつとつと
\ ひかえめにつまびく^{げっきん}{月琴}の^{おと}{音}は
\ やがてハミングと
\ ^{かさ}{重}なって
\ なめらかに^{なが}{流}れはじめた

\page[40]
\Ojisan ^{ときどき}{時々}
\ わしにはアルファさんが^{ほんとう}{本当}の^{にんげん}{人間}なんじゃないかと

\Ojisan そんなふうに^{おも}{思}えてくる

\page[42]
\Alpha その^{ひ}{日}の^{へいてん}{閉店}^{じかん}{時間}は

\Alpha ちょっとだけおそくなりました


\subsection{第2話\ ^{いりえ}{入江}のミサゴ}

\page[44]
\Takahiro その^{ひ}{日}はアルファの^{みせ}{店}にあそびにきていた

\Alpha タカヒロが^{ひとり}{一人}で^{く}{来}るなんてめずらしいね
\ はい

\Takahiro ありがと

\Alpha おじいさんは^{きょう}{今日}は?スタンド?

\Takahiro きのうから^{ちょうないかい}{町内会}(^{の}{飲}み^{かい}{会})

\Takahiro ねえ
\ アルファ

\Alpha ん?

\page[45]
\Takahiro アルファはここに^{き}{来}てもう^{なが}{長}いんでしょ?

\Alpha うん

\Takahiro アルファなら^{し}{知}ってるかと^{おも}{思}って

\Takahiro きのう^{へん}{変}なもの^{み}{見}ちゃったんだ

\Alpha こわい^{はなし}{話}はいやだよ

\Takahiro すげーこわかった

\Alpha ひ〜〜ん

\Takahiro アルファはこわがったけど
\ ぼくは^{はなし}{話}を^{き}{聞}いてもらった

\page[46]
\Takahiro きのう^{ゆうがた}{夕方}^{ひがた}{干潟}でアサリとってたんだ

\Takahiro ぼく^{ひとり}{一人}しかいなくてすんごいとれた

\Takahiro ちょっと^{さき}{先}で^{さかな}{魚}の^{む}{群}れが
\ ワラッと^{うご}{動}いた

\Takahiro そのとき

\page[49]
\Takahiro なんだこの^{ひと}{人}はすっぱだかで

\Takahiro でも^{わら}{笑}ったから
\ ちょっとほっとしたんだ

\Takahiro けど!!

\page[51]
\Takahiro ^{き}{気}がついた^{とき}{時}は^{じんか}{人家}の^{ちか}{近}くまで^{はし}{走}っていた

\Takahiro アサリも^{お}{置}きっぱなしだ

\Takahiro でも
\ とても^{と}{取}りになんか^{い}{行}けない

\Takahiro うちに^{かえ}{帰}ってもじいちゃんは^{ちょうないかい}{町内会}でいないし

\Takahiro ^{あさ}{朝}まで^{ねむ}{眠}れなかったよ
\ こわくて

\Takahiro これ^{ぜんぶ}{全部}ほんとだよ

\Takahiro アルファなんだと^{おも}{思}う?

\page[52]
\Alpha ミサゴだ

\Takahiro み?

\Alpha ミサゴっていうの^{かのじょ}{彼女}
\ ^{むかし}{昔}
\ オーナーに^{き}{聞}いたわ
\ ^{ほんとう}{本当}の^{はなし}{話}だったのね

\Takahiro みさご

\Alpha うん
\ ^{むかし}{昔}から^{いりえ}{入江}の^{もり}{森}にいるって

\page[53]
\Alpha ^{こ}{小}^{あじろ}{網代}の^{いりえ}{入江}にはミサゴっていう^{ふしぎ}{不思議}な^{ひと}{人}が
\ ^{さかな}{魚}とって^{く}{暮}らしてたって

\Alpha ^{おとな}{大人}を^{きら}{嫌}うけど^{こども}{子供}は^{す}{好}きで
\ ときどき^{あそ}{遊}びに^{く}{来}るって
\ でもキバがこわいから
\ みんな^{に}{逃}げちゃうの

\Alpha オーナーは^{たに}{谷}がもっと^{ふ}{深}かった^{ころ}{頃}^{み}{見}たっていうから
\ ずっと^{わか}{若}い^{すがた}{姿}のままなのね

\page[54]
\Takahiro そういえば

\Takahiro キバを^{み}{見}るまではあんましこわいって^{かん}{感}じなかったな

\Alpha タキヒロが^{ひとり}{一人}でいるから^{あそ}{遊}びのきたのかもね

\Takahiro うん

\Takahiro あっ

\Alpha こりゃ
\ ばーーっとくるね

\page[55]
\Alpha ^{とちゅう}{途中}でどしゃぶりになるよ

\Takahiro だーーっと^{かえ}{帰}るよ

\Alpha ほんとに^{と}{泊}まってけばいいのに

\Takahiro だいじょぶだって

\Takahiro ごっそさんでした
\ またねーー

\Alpha うん
\ ^{き}{気}をつけてね

\page[56]
\Takahiro ^{あま}{甘}かった!

\Takahiro ^{ほんや}{本屋}なんか^{よ}{寄}んじゃなかった!

\page[57]
\Takahiro やべいな

\Takahiro ぞくぞくしてきた

\Takahiro くそ
\ ^{き}{気}も^{とお}{遠}くなってきた

\Takahiro じいちゃん
\ アルファ

\Takahiro ア

\page[59]
\Takahiro みさごはキバまるだしでにっと^{わら}{笑}った

\Takahiro でも^{こんど}{今度}はそんなにこわくなかった

\Takahiro それからあったかくなってきて

\Takahiro ぼーーっとしてきて

\page[60]
\Takahiro お

\Takahiro あ
\ あれ

\Alpha ああ
\ ^{お}{起}きた?
\ よかったー
\ ^{よる}{夜}みんなで^{さが}{探}したのよ

\Takahiro みさごは?

\Alpha え?

\page[61]
\Takahiro みさごが
\ あっためてくれた

\Alpha ありゃま!

\Alpha それで^{ふく}{服}^{かわ}{乾}いてたのかな

\Takahiro あれ

\Alpha あ

\Alpha ほんと!

\page[62]
\Takahiro 「みさごなんで^{かえ}{帰}っちゃったのかな」

\Alpha 「タカヒロも^{かわ}{乾}いたし
\ ^{ひとめ}{人目}につくのもヤだったのね
\ きっと」

\Alpha こんど^{あ}{会}えたらお^{れい}{礼}^{い}{言}わなきゃね

\Takahiro うん

\Takahiro おとなになる^{まえ}{前}に


\subsection{第3話\ しましまのごろごろ}

\page[65]
\Alpha お

\Alpha ふう

\page[67]
\Alpha んっ

\Ojisan よう

\Alpha おはようございます

\page[68]
\Alpha あ〜〜っと
\ お^{みせ}{店}もうちょっとおくれちゃいますけど

\Ojisan いや
\ ^{きょう}{今日}はちがうんだよ

\Ojisan こないだタカ^{ぼう}{坊}が^{せわ}{世話}んなったからよう
\ わりいっつうんで

\Ojisan これ
\ やんよ

\Alpha ありゃま!

\Alpha ちょっとまってね

\Alpha うれしい!
\ スイカ^{だい}{大}スキなんです!

\Ojisan そりゃあよかった

\page[69]
\Alpha えへへ

\Ojisan あーー
\ いやよ

\Ojisan ^{ぜんぶ}{全部}だけんど

\Alpha えっ!!

\Ojisan ^{ことし}{今年}は^{かず}{数}もできもよくてよ

\Ojisan ^{こ}{肥}やしにすんのもあんだしよー

\Alpha う〜〜
\ で
\ でも
\ こんなにはいらないわ

\Alpha 3コじゃだめ?

\page[70]
\Ojisan ん

\Ojisan じゃ10コがノルマだ

\Alpha ノルマ?

\Ojisan で
\ こうやって^{くば}{配}ってまわんだよ^{きょう}{今日}

\Ojisan あー
\ そうよ

\page[71]
\Ojisan これ^{く}{食}いな

\Ojisan ^{けさ}{今朝}あがったやつ

\Alpha わ!

\Alpha んーー
\ うれしいけど

\Alpha ^{わたし }{私}^{どうぶつ}{動物}^{せい}{性}のタンパク^{しつ}{質}ダメなんです

\Ojisan か〜〜
\ あんだえ〜〜

\Alpha ^{ぎゅうにゅう}{牛乳}や^{たまご}{卵}で^{れんしゅう}{練習}してんですけど

\Alpha すぐぶったおれちゃって
\ ^{きこう}{機構}^{てき}{的}にダメみたい

\page[72]
\Ojisan あによーー

\Ojisan だめじゃんかよ^{す}{好}き^{きら}{嫌}いはよ

\Alpha そうじゃなくて!

\Ojisan ま
\ しゃあねえや

\Ojisan かわりにもう2コか?

\Alpha えっ!!

\Ojisan ごっそさん
\ じゃまたな

\page[73]
\Alpha う〜〜ん
\ ^{れいぞうこ}{冷藏庫}は2コでいっぱいだし

\page[74]
\Alpha そうだ!
\ お^{きゃく}{客}さんがきたらおめやげに

\Alpha そうね
\ リボンなんかもつけちゃったりなんかして!

\Alpha 「リボンのスイカの^{みせ}{店}」なんて^{い}{言}われたりして

\Alpha でも

\Alpha まず
\ お^{きゃく}{客}さんだよな〜〜

\Alpha ^{き}{来}た!!

\page[75]
\Alpha いらっしゃいませーー!

\page[76]
\Person いいとこですね!

\Alpha ありがとございます

\Person ^{ほうさく}{豊作}みたいっすね!

\Alpha え?

\Alpha ええ!
\ ^{こん}{今}もれなく

\Person いや〜〜!!

\Person ^{とちゅう}{途中}のスタンドでたくさんもらっちゃいましてね!

\Person あんなに

\Alpha そっ
\ そうすか

\Person リッ
\ リボンかわいいですね!

\Alpha どうも

\page[77]
\Sign ^{ざいこ}{在庫}^{ほうふ}{豊富}
\ スイカつき
\ ガソリン
\ あり^{ます}{〼}

\Alpha ふう

\Alpha ぷっ


\subsection{第4話\ ^{あめ}{雨}とその^{あと}{後}}

\page[80]
\Alpha ほんのちょっとの^{よう}{用}だったけど
\ ^{てんき}{天気}はつきあってくれなかった

\Alpha んっ

\page[81]
\Alpha うちまでもたないなこりゃ

\page[83]
\Ojisan ^{ちけ}{近}えな

\page[85]
\Alpha すみませんでした

\Ojisan ^{あんしん}{安心}しな
\ いい^{びょういん}{病院}があんから

\Ojisan ^{せんせい}{先生}が^{し}{知}り^{あ}{合}いでよ
\ ^{とお}{遠}くのでけえ^{びょういん}{病院}よか^{たよ}{頼}りにならあ

\page[88]
\Ojisan ^{せんせい}{先生}

\Sensei もう^{だいじょうぶ}{大丈夫}
\ ^{かみなり}{雷}は^{からだ}{体}の^{そと}{外}を^{とお}{通}ってたし

\Sensei ^{かみ}{髪}と^{ひふ}{皮膚}がコゲてるから
\ ^{に}{2} ^{さんにち}{3日}かからけどね

\Sensei ロボットの^{かんじゃ}{患者}さんなんて^{ひさ}{久}しぶりだわ

\Ojisan ^{に}{2} ^{さんにち}{3日}
\ ^{じゅうしょう}{重傷}に^{み}{見}えましたが

\Sensei もう
\ お^{はなし}{話}もできるわよ

\page[89]
\Ojisan よ

\Alpha おじさん

\Ojisan こりゃあ
\ ^{さんにち}{3日}で^{なお}{治}るって^{言}{せ}ってたけんど

\Alpha うん
\ ^{おお}{大}ゲサに^{み}{見}えるけど
\ ^{ひふ}{皮膚}のコーティングだけだから

\Alpha おじさん
\ あの

\Ojisan あん?

\Alpha タカヒロにはないしょにしてね

\page[90]
\Ojisan おう
\ でも^{むか}{迎}えに^{く}{来}ん^{とき}{時}くらいはいいべ?

\Alpha うん

\Ojisan ^{あした}{明日}また
\ ^{かお}{顔}^{み}{見}に^{く}{来}んからよ
\ なんかいる?

\Alpha あ
\ できれば^{き}{着}がえを
\ ^{かって}{勝手}^{ぐち}{口}があいてますんで

\Ojisan あーー
\ じゃ
\ ^{てきとう}{適当}になんか^{も}{持}ってくるわ
\ ま
\ のんびりしなよ

\Alpha すいません

\Ojisan じゃ
\ よろしくお^{ねが}{願}いします

\Sensei はい

\page[91]
\Sensei ^{きぶん}{気分}はどう?

\Alpha おかげさまで
\ ありがとうございます

\Sensei お^{れい}{礼}はあのおっさんに^{い}{言}ってね

\Sensei ^{あした}{明日}お^{ひる}{昼}から^{ひふ}{皮膚}のクリーニングとコーティングをしましょう

\Sensei それまで^{うご}{動}きにくくしたけど
\ ほかはなんでもないわ

\Sensei ^{きょう}{今日}はゆっくり^{やす}{休}みなさい

\page[92]
\Narrator ^{よくじつ}{翌日}1^{じ}{次}コーティング^{しゅうりょう}{終了}

\Sensei あらあら

\Alpha あっ

\Sensei できが^{あら}{粗}いんで
\ ^{しんぱい}{心配}なのね

\Alpha はあ

\Sensei ^{だいじょうぶ}{大丈夫}
\ ^{こん}{今}は^{したじ}{下地}だから
\ ケバだってるのよ

\page[93]
\Sensei 2^{じ}{次}コートと^{しあ}{仕上}げでしっとりすべすべになるわ

\Alpha はあ
\ そうなんですか

\Sensei ^{かみ}{髪}も^{あたら}{新}しくしたの
\ コゲてたからね

\Alpha えっ!

\Sensei ^{ぜん}{前}と^{おな}{同}じ^{て}{手}^{かみ}{髪}よ
\ やっぱり^{いろ}{色}が^{か}{変}わっちゃイヤよね

\Alpha ^{あんしん}{安心}しました
\ ふう

\Alpha ほんとになんてお^{れい}{礼}を^{い}{言}ったらいいのか

\Sensei だからお^{れい}{礼}はあのおっさんにわ

\Sensei あした^{いちにち}{1日}で^{お}{終}わるから

\Sensei あさってには^{かえ}{帰}れるわ

\page[94]
\Alpha あ

\Alpha なんだか^{なみだ}{涙}がでてきた

\Alpha ^{ほんらい}{本来}
\ ^{わたし}{私}の^{るいせん}{涙腺}は
\ ^{め}{目}を^{うるお}{潤}すためだけのものなんだけど

\Alpha こんな^{こと}{事}は
\ ^{ひとり}{一人}で^{げっきん}{月琴}^{ひ}{弾}いてる^{とき}{時}なんかには
\ よくあった
\ それとは^{べつ}{別}の^{かん}{感}じだけど

\Alpha こういう^{きも}{気持}ちも^{す}{好}きだな

\page[95]
\Narrator ^{かえ}{帰}りの^{ひ}{日}

\Ojisan アルファさんは^{ようい}{用意}できてましたか?

\Sensei ちょっと^{み}{身}だしなみで^{なや}{悩}んでたみたいよ

\Ojisan あんだかなー
\ おら
\ ^{はし}{走}んじゃねえ

\Ojisan ありがとうございました
\ ツケの^{ほう}{方}はいずれ

\Sensei あら
\ こちらこそ^{たの}{楽}しかったわ
\ ^{むすめ}{娘}ができたみたいで

\Sensei それに
\ ガソリンや^{やさい}{野菜}のツケもたまりまくってたし

\Takahiro アルファ^{かえ}{帰}るよ!

\page[96]
\Takahiro ア

\Alpha あ

\Takahiro かっ
\ ^{かみがた}{髪型}^{か}{変}えたの!?

\Alpha ちっ
\ ^{ちが}{違}うっ!!

\Ojisan うわっ!
\ あんだこりゃ
\ わはは

\Alpha ひ〜〜ん!

\Sensei ^{しんぴん}{新品}の^{かみ}{髪}だから
\ ピンピンよなじむまで

\Alpha どっ
\ どのくらいでなじみます?

\page[97]
\Sensei まあ
\ ^{い}{1}^{しゅうかん}{週間}かしら

\Alpha いっ
\ ^{い}{1}^{しゅうかん}{週間}!!

\Alpha まだ^{なみだ}{涙}がでてきた
\ ^{きも}{気持}ちはぜんぜん^{ちが}{違}う

\page[98]
\Alpha おじさん
\ あの
\ ありがとうございました

\Ojisan あに
\ ^{言}{せ}^{う}{え}だかよー

\Ojisan ^{れい}{礼}は^{せんせい}{先生}にで^{じゅうぶん}{十分}だよ

\Alpha あの
\ ^{わたし}{私}
\ お^{れい}{礼}する^{もの}{物}がないんで
\ お^{ふたり}{二人}はずっとコーヒーでもなんでもただで

\Ojisan あんだかなあ
\ いいじゃんかよー
\ ^{べつ}{別}によー

\Ojisan こういうときや
\ あんた^{かぞく}{家族}みてえなもんだしよー

\page[99]
\Alpha うわ〜〜ん

\Ojisan あんだ
\ あんだ?

\Ojisan あに^{な}{泣}くかよ
\ ええ?

\Alpha う
\ うれしいんです

\Alpha ありが

\Takahiro ぶっ

\page[100]
\Takahiro わはははははは

\Alpha ひ〜〜ん


\subsection{第5話\ エンドレス^{ちょうない}{町内}^{かい}{会}}

\page[102]
\Narrator アルファさんを^{ちょうない}{町内}^{かい}{会}にさそった
\ ^{かのじょ}{彼女}の^{ぜんかい}{全快}^{いわ}{祝}いも^{かね}{兼}てただけんど

\page[103]
\Narrator アルファさんが^{つ}{着}いたころには
\ もうできあがってたな

\Person アルファさんにはせっかく^{き}{来}てもらっただけんどよ
\ あんた
\ ^{さしみ}{刺身}とか^{く}{食}えねえだっつうからよ

\Person うちンのだけんどミカンとかは^{く}{食}う?

\Alpha ああっ
\ ^{く}{食}います!!

\page[104]
\Ojisan アルファさんよ
\ ひとり
\ シラフでミカン^{く}{食}っててもつまらねえべ
\ ちっとだけでも^{の}{飲}めねえのかい

\Alpha いえ
\ お^{さけ}{酒}は^{かお}{香}りだけで
\ ^{じゅうぶん}{十分}^{たの}{楽}しんでます

\Alpha ^{の}{飲}むのは^{つよ}{強}くないんですよ

\page[105]
\Person あ〜〜によ〜〜
\ のめんなら^{い}{言}ってけえなよ
\ アルちゃんよ〜〜

\Alpha え?!
\ アルちゃん?

\Person そうだよ〜〜
\ はい
\ これね

\Person ま ま ま ま

\Alpha わかりました

\Alpha でも
\ おちょこに1^{はい}{杯}だけですよ
\ それが^{げんかい}{限界}ですから

\page[106]
\Alpha ふう

\page[107]
\Person じゃー
\ ^{つぎ}{次}アルちゃんもなんかやってよ

\Alpha ん〜〜
\ よし
\ じゃあねえ

\Alpha ^{てびょうし}{手拍子}して

\Alpha こんなかんじで

\Person こう?

\Alpha そうそう

\page[109]
\Narrator ^{とつぜん}{突然}^{はじ}{始}まった
\ ^{み}{見}たこともない^{まい}{舞}に
\ ^{ぜんいん}{全員}^{どぎも}{度肝}をめかれた

\page[110]
\Narrator おどりは
\ だんだん^{てびょうし}{手拍子}の^{たんじゅん}{単純}なリズムにとけて
\ ^{なが}{流}れるような^{うご}{動}きへと^{うつ}{移}っていく

\page[111]
\Narrator ^{みな}{皆}
\ その^{とき}{時}は^{ゆめ}{夢}のようだったと^{い}{言}う
\ アルファさん^{じしん}{自身}おぼえていなかった

\Narrator ^{なが}{長}かったのか
\ ^{みじか}{短}かったのか

\page[112]
\Narrator やがて
\ ^{しず}{静}かに^{や}{止}んだ

\Alpha おしまい

\Person いや〜〜
\ なんか
\ すごかったねー
\ あっけにとられちゃったよ

\Alpha えへへ

\page[113]
\Person お

\Ojisan 2^{はい}{杯}も^{の}{飲}ますからよ〜〜

\Person いや
\ アルちゃんはえらかった

\page[114]
\Alpha ばっ

\Narrator その^{ひ}{日}は^{けっきょく}{結局}
\ ^{ぜんいん}{全員}
\ ^{きゅうじつ}{休日}に
\ ^{ごご}{午後}は
\ ^{みな}{皆}でアルファさんの^{みせ}{店}におしかけたな

\Alpha ぐー


\subsection{第6話\ プレ^{ねしょうがつ}{寝正月}}

\page[116]
\Alpha う〜〜ん

\Alpha またおフロで^{としこ}{年越}ししちゃったよ

\Alpha ^{がんじつ}{元日}^{まいとし}{毎年}^{はつひ}{初日}を^{み}{見}に^{い}{行}く

\Alpha ^{あいぼう}{相棒}のスクーターはおじさんが^{なお}{直}してくれた

\page[117]
\Alpha ^{ことし}{今年}はタカヒロもいっしょだ

\Takahiro アルファ!

\Alpha うわっ!

\page[118]
\Alpha ありゃまー
\ タカヒロ

\Takahiro ^{ことし}{今年}もよろしく

\Alpha よろしくね
\ まあ
\ はいってよ

\Takahiro おじゃまします

\Takahiro まだ^{はや}{早}かった?

\Alpha ちょっとね
\ だいぶね

\Alpha やっぱりおじさんは^{い}{行}かないの?

\Takahiro ^{としこ}{年越}し^{ちょうない}{町内}^{かい}{会}だって

\page[119]
\Alpha ん〜〜
\ それじゃ^{さむ}{寒}いでしょ

\Alpha ^{わたし}{私}のジャケット^{か}{貸}してあげるよ

\Takahiro ^{だいじょうぶ}{大丈夫}だよ
\ さっき^{あつ}{暑}いくらいだったし

\Alpha ^{じてんしゃ}{自轉車}だからねーー
\ バイクだとすごい^{さむ}{寒}いよ

\Takahiro アルファで^{かぜ}{風}よけるから^{へいき}{平気}だよ

\page[120]
\Alpha このやろ!

\Alpha うしろはとっかからないからね
\ ちゃんと^{うで}{腕}まわしてないと
\ ^{お}{落}としてっちゃうよ

\Takahiro うん

\Takahiro じゃごーー!

\page[122]
\Alpha ここらでコーヒーブレイクかな?

\Alpha はい

\Takahiro ありがと
\ ^{かん}{缶}のコーヒーなんて^{ひさ}{久}しぶりだよ

\page[123]
\Takahiro なんか
\ うまい

\Alpha でしょ!

\Alpha ^{さむ}{寒}い^{よる}{夜}にバイクで^{はし}{走}った^{とき}{時}の
\ ^{かん}{缶}コーヒーって^{いじょう}{異常}においしいのよ

\Takahiro これミルクはいってるよ

\Alpha このくらいなら^{へいき}{平気}よ
\ カフェオレはつらいけどね

\page[124]
\Sign みそ^{しる}{汁}
\ ^{あまざけ}{甘酒}

\page[125]
\Alpha たくさんじゃないけど

\Alpha ここには^{まいとし}{毎年}けっこう^{ひと}{人}があつまる

\Alpha ^{ひ}{冷}えてきた?

\Takahiro えっ?

\Takahiro まだ^{だいじょうぶ}{大丈夫}だよ!

\Alpha ほーお
\ ふーん

\Alpha ^{わたし}{私}ねー
\ ちょっと^{まえ}{前}が^{さむ}{寒}くて

\page[126]
\Alpha しばらくこうしててもいいかな?

\Takahiro うん
\ いいよ

\page[127]
\Alpha ふだんがあんまり^{ふゆ}{冬}っぽくないぶん
\ やっぱり
\ こうして^{さむ}{寒}い^{ちゅう}{中}
\ ^{はつひ}{初日}を^{み}{見}なきゃ
\ 「^{しんねん}{新年}」って^{き}{気}がしないね

\page[128]
\Alpha ^{みな}{皆}
\ ^{じぶん}{自分}のやり^{かた}{方}で^{み}{見}ている

\Person ^{くも}{雲}あんからへー
\ ^{み}{見}えないや

\Person んー

\Alpha ^{いま}{今}は^{むかし}{昔}ほど^{きせつ}{季節}がはっきりしないけど

\Alpha みんな^{まえ}{前}よりも
\ ^{ものごと}{物事}に^{かん}{感}じ^{はい}{入}る^{こと}{事}が
\ ^{おお}{多}くなったと^{おも}{思}う

\page[129]
\Takahiro もう
\ ^{まえ}{前}は^{さむ}{寒}くなくなった?

\Alpha えっ?
\ ああ
\ うん

\Alpha もうちょっとここにいよっか

\Alpha うん

\page[130]
\Takahiro なんか
\ ねむくなってきたよ

\Alpha そうね
\ やっぱ
\ ^{はや}{早}いとこ^{かえ}{帰}ってねちゃおっか


\subsection{第7話\ ^{ごぜん}{午前}2/2}

\page[132]
\Narrator AM7:20
\ ^{だいさん}{第三}^{けいひん}{京浜}^{せん}{線}^{しゅうてん}{終点}

\page[138]
\Alpha コーヒー

\Alpha 8^{わり}{割}がた^{わたし}{私}がのんでるかなーー

\page[139]
\Person ^{かえ}{帰}りの^{びん}{便}は^{てはい}{手配}してあんの?

\Kokone はい

\Person ^{さいご}{最後}まで^{おく}{送}ってやりてえけどせっぱつまっててよ
\ わりいな

\Kokone いえ
\ ^{たす}{助}かりました

\page[141]
\Ojisan で
\ その^{みち}{道}のどんづまりだよ

\Kokone わかりました

\Ojisan つれてってやりてえけど
\ じきガソリン^{や}{屋}くんだよな

\Kokone いえ
\ すぐそこみたいですし

\Ojisan わりいな

\Ojisan ^{に}{似}てんな

\page[142]
\Narrator AM11:05

\page[143]
\Kokone ^{はつせの}{初瀬野}

\page[144]
\Alpha ^{おもや}{母屋}に?

\Alpha はーーい

\Alpha どちらさま?

\page[145]
\Kokone ^{はつせの}{初瀬野}アルファさんですね

\Alpha はい

\Alpha あっ
\ いや〜〜
\ フルネームなんか^{ひさ}{久}しぶりだから
\ ^{いっしゅん}{一瞬}^{かんが}{考}えちゃった

\Kokone はあ

\Kokone ^{わたし}{私}
\ ムサシノの^{たくはい}{宅配}^{びん}{便}ですが

\Alpha ^{たくはい}{宅配}
\ ふわ〜〜
\ たいへんだったでしょーー
\ ムサシノからじゃ

\Kokone いえ

\page[146]
\Alpha ^{とど}{届}けものなんて^{ひさ}{久}しぶりよ
\ どちらから?

\Kokone ええ

\Kokone ^{はつせの}{初瀬野}^{さま}{様}から

\Alpha オーナーから


\subsection{みーやんのやっぱこれっす!}

\Sign ^{こうきゅう}{高級}^{まめ}{豆}
\ ^{こくさん}{国産}^{だいず}{大豆}

\Sign がめてきた^{ふくろ}{袋}

\Sign イライラ

\Sign ポイだ
