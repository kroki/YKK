\section{Volume 1}

\subsection{ヨコハマ^{か}{買}い^{だ}{出}し^{きこう}{紀行}}

\page[9]
\Sign Closed ^{もく}{木}ようまで
\ 買いだしです
\ Cafe アルファ

\page
\A すいませーん

\A もしもーし
\ おーーい

\Sign ガソリン
\ あり^{ます}{〼}

\page
\O あー
\ どうもね

\A ^{まん}{満}タンお^{ねが}{願}いしまず

\O お^{きゃく}{客}なんて
\ ^{いしゅうかん}{1週間}ぶりだから
\ どうもね

\O おや

\page
\O お客さん^{にし}{西}の^{みさき}{岬}のコーヒー^{や}{屋}さんだべ
\ アルファさん

\A えっ?
\ ええ
\ そうですけど

\O あんた^{ゆうめいじん}{有名人}だよファン^{おお}{多}いよ

\O かわいいってね

\A そんな
\ え〜〜〜

\A でっ
\ でも
\ うち
\ めったにお客さん^{き}{来}ませんよ

\O たまに^{とお}{通}んの^{み}{見}るだけでいいだとよ
\ てれてんだよ

\page
\O ほい
\ 満タン
\ ^{とおで}{遠出}かい

\A ええ
\ ^{よこはま}{横浜}まで^{まめ}{豆}買いに

\A いくら?

\O 横浜!!
\ そりゃ遠いや!!

\O いいよ
\ ^{きょう}{今日}はサービスだ!!

\A そりゃわるいっすよ

\O ま
\ いいからよ
\ ^{き}{気}いつけなよ
\ ^{みち}{道}がわりいからよ

\O いまなし道ィ
\ ^{な}{無}くなんしよー
\ たぶん
\ まっつぐは^{い}{行}けめえ

\page
\O ^{かえ}{帰}ってきたらまた^{よ}{寄}りなよ
\ したら^{あと}{後}でコーヒーごちになんから
\ ^{だいきん}{代金}として

\A ありがとうおじさん

\O あー
\ あんたロボットだって?

\O いいねー
\ ^{けんこう}{健康}そうで
\ おじさんなんて^{からだ}{体}
\ ガッタガタ

\A ふふ
\ じゃ
\ とりかえます?

\page
\O へへ
\ お^{たが}{互}い
\ ^{かみ}{神}さまがくれた体はまっとうしなきゃな

\O 気ィつけな

\A ありがとー

\O じつはわしもファン

\page
\A ^{なんねん}{何年}か^{まえ}{前}^{わたし}{私}のオーナーは

\A 私に^{みせ}{店}をあずけたいきなり
\ どこかへ行ってしまった

\A どこにいるやら
\ ^{なに}{何}やってるやら

\A いつかは帰ってくるのかしらね

\A 私はロボットでよかったと^{おも}{思}う

\A いくらでも^{ま}{待}っていられるからね

\page[18]
\A ありゃま

\A うーん
\ ^{ねんねん}{年々}^{うみ}{海}が^{あ}{上}がってくるみたい

\page
\A だから^{いま}{今}はこういった^{おわ}{尾根}道ばかり

\page
\A 横浜も海の^{した}{下}で

\A ^{おか}{丘}の^{うえ}{上}が今の横浜です

\page[22]
\A あのー
\ コーヒー豆ください

\P あい

\A ^{かず}{数}^{ねんまえ}{年前}までの「^{だいとし}{大都市}ヨコハマ」が^{ゆめ}{夢}みたい

\A ^{いま}{今}はゆったり^{とき}{時}の^{なが}{流}れる「^{ひと}{人}の^{まち}{街}」

\page
\A この^{かず}{数}^{ねん}{年}で^{よ}{世}の^{なか}{中}も^{ずいぶん}{随分}^{か}{変}わったわ

\A ^{じだい}{時代}の^{たそがれ}{黄昏}がこんなにゆったりのんびりと^{く}{来}る
\ ものだったなんて

\A ^{わたし}{私}は^{たぶん}{多分}この^{たそがれ}{黄昏}の^{よ}{世}を
\ ずっと^{み}{見}ていくんだと^{おも}{思}う

\page
\Sign アルファへ
\ ^{げんき}{元気}なようで
\ ^{あんしん}{安心}しました

\A ^{わたし}{私}には^{じかん}{時間}はいくらでもあるからね

\A おじさん
\ ただいまー

\O おー


\subsection{^{だい}{弟}1^{わ}{話}\ ^{はがね}{鋼}の^{かお}{香}る^{よる}{夜}}

\page[29]
\A ふう

\A なにか
\ もっと^{でんごん}{伝言}ほしかったよなー

\Sign アルファへ
\ ^{げんき}{元気}なようで
\ ^{あんしん}{安心}しました

\A あっ
\ いらっしゃいませ!

\O よう
\ また^{き}{来}たよ

\A ブレンドね!

\page
\O ヒマそうだな
\ え?

\A 今日はおじさんだけよ

\O !おや

\A あ

\O へえ!
\ アルファさんも^{てっぽう}{鉄砲}^{も}{持}ちかよ

\A ええ
\ あ
\ ^{だいじょうぶ}{大丈夫}ですよ
\ タマ^{い}{入}れてませんから

\O へえ
\ この^{かた}{型}はめずらしいな

\O ふつう^{も}{持}つならあれとか

\page
\O ええ
\ その^{こうせき}{鉱石}みたいな^{かたち}{形}と^{いろ}{色}がいいでしょ

\A オーナーの

\A プレゼントなんです

\O ほお

\A ^{ごしん}{護身}^{よう}{用}にって^{わた}{渡}されたんです
\ けど

\A 私にとっては^{ぶき}{武器}っていうか

\page
\A う〜〜〜ん
\ そういうものじゃないんです

\O よく見ると
\ ^{じゅう}{銃}はカドがとれてピカピカしていた

\O たぶんいつもなでていたんだろう

\O ^{たから}{宝}もんだな

\A え

\O 宝もんだ

\A えへへ

\page
\A おかわり
\ サービスでネ

\O おおーー
\ わりいな

\page
\O ここは^{ほんと}{本当}におちつく

\A ちゃんとした店なら
\ ^{かまくら}{鎌倉}や^{はやま}{葉山}あたりにゃまだあんけどよ

\A うちは
\ ほとんどコーヒー^{つ}{付}き
\ ^{てんぼう}{展望}^{だい}{台}ですから

\A わたし
\ ここが^{す}{好}きです

\A こうやっておじさんとかと^{はな}{話}したり
\ 海見たり

\A にぎやかな時も
\ ひとりの時も

\A みんな好き

\page
\A じゃん

\O あにそれ

\A オーナーの^{しゅみ}{趣味}です
\ ^{げっきん}{月琴}っていうの

\A ふだん
\ ^{ひとまえ}{人前}では^{ひ}{弾}かないんです
\ けど

\page
\A なんか
\ 今日だけね

\O お
\ おう

\T あっ
\ こっちにいたの

\A あら!

\O ちっ

\A いらっしゃい

\O おう

\T 「おう」じゃないよじいちゃん
\ ひとりで^{き}{来}てんたもんなーー

\O おう

\A タカヒロなんか^{の}{飲}む?

\page
\T じゃ
\ メイポロ
\ ね
\ なにそれやってよ

\A なんか
\ にぎやかんなっちゃったね

\A いい?
\ タカヒロ
\ 今日だけの^{とくべつ}{特別}だからネ

\A ^{ちゅうごく}{中国}の
\ 夜の^{きょく}{曲}です

\page[39]
\O とつとつと
\ ひかえめにつまびく^{げっきん}{月琴}の^{おと}{音}は
\ やがてハミングと
\ ^{かさ}{重}なって
\ なめらかに^{なが}{流}れはじめた

\page
\O ^{ときどき}{時々}
\ わしにはアルファさんが本当の^{にんげん}{人間}なんじゃないかと

\O そんなふうに^{おも}{思}えてくる

\page[42]
\A その^{ひ}{日}の^{へいてん}{閉店}^{じかん}{時間}は

\A ちょっとだけおそくなりました


\subsection{第2話\ ^{いりえ}{入江}のミサゴ}

\page[44]
\T その^{ひ}{日}はアルファの^{みせ}{店}にあそびにきていた

\A タカヒロが^{ひとり}{一人}で^{く}{来}るなんてめずらしいね
\ はい

\T ありがと

\A おじいさんは^{きょう}{今日}は?スタンド?

\T きのうから^{ちょうないかい}{町内会}(^{の}{飲}み^{かい}{会})

\T ねえ
\ アルファ

\A ん?

\page
\T アルファはここに^{き}{来}てもう^{なが}{長}いんでしょ?

\A うん

\T アルファなら^{し}{知}ってるかと^{おも}{思}って

\T きのう^{へん}{変}なもの^{み}{見}ちゃったんだ

\A こわい^{はなし}{話}はいやだよ

\T すげーこわかった

\A ひ〜〜ん

\T アルファはこわがったけど
\ ぼくは^{はなし}{話}を^{き}{聞}いてもらった

\page
\T きのう^{ゆうがた}{夕方}^{ひがた}{干潟}でアサリとってたんだ

\T ぼく^{ひとり}{一人}しかいなくてすんごいとれた

\T ちょっと^{さき}{先}で^{さかな}{魚}の^{む}{群}れが
\ ワラッと^{うご}{動}いた

\T そのとき

\page[49]
\T なんだこの^{ひと}{人}はすっぱだかで

\T でも^{わら}{笑}ったから
\ ちょっとほっとしたんだ

\T けど!!

\page[51]
\T ^{き}{気}がついた^{とき}{時}は^{じんか}{人家}の^{ちか}{近}くまで^{はし}{走}っていた

\T アサリも^{お}{置}きっぱなしだ

\T でも
\ とても^{と}{取}りになんか^{い}{行}けない

\T うちに^{かえ}{帰}ってもじいちゃんは^{ちょうないかい}{町内会}でいないし

\T ^{あさ}{朝}まで^{ねむ}{眠}れなかったよ
\ こわくて

\T これ^{ぜんぶ}{全部}ほんとだよ

\T アルファなんだと^{おも}{思}う?

\page
\A ミサゴだ

\T み?

\A ミサゴっていうの^{かのじょ}{彼女}
\ ^{むかし}{昔}
\ オーナーに^{き}{聞}いたわ
\ ^{ほんと}{本当}の^{はなし}{話}だったのね

\T みさご

\A うん
\ ^{むかし}{昔}から^{いりえ}{入江}の^{もり}{森}にいるって

\page
\A ^{こ}{小}^{あじろ}{網代}の^{いりえ}{入江}にはミサゴっていう^{ふしぎ}{不思議}な^{ひと}{人}が
\ ^{さかな}{魚}とって^{く}{暮}らしてたって

\A ^{おとな}{大人}を^{きら}{嫌}うけど^{こども}{子供}は^{す}{好}きで
\ ときどき^{あそ}{遊}びに^{く}{来}るって
\ でもキバがこわいから
\ みんな^{に}{逃}げちゃうの

\A オーナーは^{たに}{谷}がもっと^{ふ}{深}かった^{ころ}{頃}^{み}{見}たっていうから
\ ずっと^{わか}{若}い^{すがた}{姿}のままなのね

\page
\T そういえば

\T キバを^{み}{見}るまではあんましこわいって^{かん}{感}じなかったな

\A タキヒロが^{ひとり}{一人}でいるから^{あそ}{遊}びのきたのかもね

\T うん

\T あっ

\A こりゃ
\ ばーーっとくるね

\page
\A ^{とちゅう}{途中}でどしゃぶりになるよ

\T だーーっと^{かえ}{帰}るよ

\A ほんとに^{と}{泊}まってけばいいのに

\T だいじょぶだって

\T ごっそさんでした
\ またねーー

\A うん
\ ^{き}{気}をつけてね

\page
\T ^{あま}{甘}かった!

\T ^{ほんや}{本屋}なんか^{よ}{寄}んじゃなかった!

\page
\T やべいな

\T ぞくぞくしてきた

\T くそ
\ ^{き}{気}も^{とお}{遠}くなってきた

\T じいちゃん
\ アルファ

\T ア

\page[59]
\T みさごはキバまるだしでにっと^{わら}{笑}った

\T でも^{こんど}{今度}はそんなにこわくなかった

\T それからあったかくなってきて

\T ぼーーっとしてきて

\page
\T お

\T あ
\ あれ

\A ああ
\ ^{お}{起}きた?
\ よかったー
\ ^{よる}{夜}みんなで^{さが}{探}したのよ

\T みさごは?

\A え?

\page
\T みさごが
\ あっためてくれた

\A ありゃま!

\A それで^{ふく}{服}^{かわ}{乾}いてたのかな

\T あれ

\A あ

\A ほんと!

\page
\T 「みさごなんで^{かえ}{帰}っちゃったのかな」

\A 「タカヒロも^{かわ}{乾}いたし
\ ^{ひとめ}{人目}につくのもヤだったのね
\ きっと」

\A こんど^{あ}{会}えたらお^{れい}{礼}^{い}{言}わなきゃね

\T うん

\T おとなになる^{まえ}{前}に


\subsection{第3話\ しましまのごろごろ}

\page[65]
\A お

\A ふう

\page[67]
\A んっ

\O よう

\A おはようございます

\page
\A あ〜〜っと
\ お^{みせ}{店}もうちょっとおくれちゃいますけど

\O いや
\ ^{きょう}{今日}はちがうんだよ

\O こないだタカ^{ぼう}{坊}が^{せわ}{世話}んなったからよう
\ わりいっつうんで

\O これ
\ やんよ

\A ありゃま!

\A ちょっとまってね

\A うれしい!
\ スイカ^{だい}{大}スキなんです!

\O そりゃあよかった

\page
\A えへへ

\O あーー
\ いやよ

\O ^{ぜんぶ}{全部}だけんど

\A えっ!!

\O ^{ことし}{今年}は^{かず}{数}もできもよくてよ

\O ^{こ}{肥}やしにすんのもあんだしよー

\A う〜〜
\ で
\ でも
\ こんなにはいらないわ

\A 3コじゃだめ?

\page
\O ん

\O じゃ10コがノルマだ

\A ノルマ?

\O で
\ こうやって^{くば}{配}ってまわんだよ^{きょう}{今日}

\O あー
\ そうよ

\page
\O これ^{く}{食}いな

\O ^{けさ}{今朝}あがったやつ

\A わ!

\A んーー
\ うれしいけど

\A ^{わたし}{私}^{どうぶつ}{動物}^{せい}{性}のタンパク^{しつ}{質}ダメなんです

\O か〜〜
\ あんだえ〜〜

\A ^{ぎゅうにゅう}{牛乳}や^{たまご}{卵}で^{れんしゅう}{練習}してんですけど

\A すぐぶったおれちゃって
\ ^{きこう}{機構}^{てき}{的}にダメみたい

\page
\O あによーー

\O だめじゃんかよ^{す}{好}き^{きら}{嫌}いはよ

\A そうじゃなくて!

\O ま
\ しゃあねえや

\O かわりにもう2コか?

\A えっ!!

\O ごっそさん
\ じゃまたな

\page
\A う〜〜ん
\ ^{れいぞうこ}{冷藏庫}は2コでいっぱいだし

\page
\A そうだ!
\ お^{きゃく}{客}さんがきたらおめやげに

\A そうね
\ リボンなんかもつけちゃったりなんかして!

\A 「リボンのスイカの^{みせ}{店}」なんて^{い}{言}われたりして

\A でも

\A まず
\ お^{きゃく}{客}さんだよな〜〜

\A ^{き}{来}た!!

\page
\A いらっしゃいませーー!

\page
\P いいとこですね!

\A ありがとございます

\P ^{ほうさく}{豊作}みたいっすね!

\A え?

\A ええ!
\ ^{こん}{今}もれなく

\P いや〜〜!!

\P ^{とちゅう}{途中}のスタンドでたくさんもらっちゃいましてね!

\P あんなに

\A そっ
\ そうすか

\P リッ
\ リボンかわいいですね!

\A どうも

\page
\Sign ^{ざいこ}{在庫}^{ほうふ}{豊富}
\ スイカつき
\ ガソリン
\ あり^{ます}{〼}

\A ふう

\A ぷっ


\subsection{第4話\ ^{あめ}{雨}とその^{あと}{後}}

\page[80]
\A ほんのちょっとの^{よう}{用}だったけど
\ ^{てんき}{天気}はつきあってくれなかった

\A んっ

\page
\A うちまでもたないなこりゃ

\page[83]
\O ^{ちけ}{近}えな

\page[85]
\A すみませんでした

\O ^{あんしん}{安心}しな
\ いい^{びょういん}{病院}があんから

\O ^{せんせい}{先生}が^{し}{知}り^{あ}{合}いでよ
\ ^{とお}{遠}くのでけえ^{びょういん}{病院}よか^{たよ}{頼}りにならあ

\page[88]
\O ^{せんせい}{先生}

\S もう^{だいじょうぶ}{大丈夫}
\ ^{かみなり}{雷}は^{からだ}{体}の^{そと}{外}を^{とお}{通}ってたし

\S ^{かみ}{髪}と^{ひふ}{皮膚}がコゲてるから
\ ^{に}{2} ^{さんにち}{3日}かからけどね

\S ロボットの^{かんじゃ}{患者}さんなんて^{ひさ}{久}しぶりだわ

\O ^{に}{2} ^{さんにち}{3日}
\ ^{じゅうしょう}{重傷}に^{み}{見}えましたが

\S もう
\ お^{はなし}{話}もできるわよ

\page
\O よ

\A おじさん

\O こりゃあ
\ ^{さんにち}{3日}で^{なお}{治}るって^{言}{せ}ってたけんど

\A うん
\ ^{おお}{大}ゲサに^{み}{見}えるけど
\ ^{ひふ}{皮膚}のコーティングだけだから

\A おじさん
\ あの

\O あん?

\A タカヒロにはないしょにしてね

\page
\O おう
\ でも^{むか}{迎}えに^{く}{来}ん^{とき}{時}くらいはいいべ?

\A うん

\O ^{あした}{明日}また
\ ^{かお}{顔}^{み}{見}に^{く}{来}んからよ
\ なんかいる?

\A あ
\ できれば^{き}{着}がえを
\ ^{かって}{勝手}^{ぐち}{口}があいてますんで

\O あーー
\ じゃ
\ ^{てきとう}{適当}になんか^{も}{持}ってくるわ
\ ま
\ のんびりしなよ

\A すいません

\O じゃ
\ よろしくお^{ねが}{願}いします

\S はい

\page
\S ^{きぶん}{気分}はどう?

\A おかげさまで
\ ありがとうございます

\S お^{れい}{礼}はあのおっさんに^{い}{言}ってね

\S ^{あした}{明日}お^{ひる}{昼}から^{ひふ}{皮膚}のクリーニングとコーティングをしましょう

\S それまで^{うご}{動}きにくくしたけど
\ ほかはなんでもないわ

\S ^{きょう}{今日}はゆっくり^{やす}{休}みなさい

\page
\N ^{よくじつ}{翌日}1^{じ}{次}コーティング^{しゅうりょう}{終了}

\S あらあら

\A あっ

\S できが^{あら}{粗}いんで
\ ^{しんぱい}{心配}なのね

\A はあ

\S ^{だいじょうぶ}{大丈夫}
\ ^{こん}{今}は^{したじ}{下地}だから
\ ケバだってるのよ

\page
\S 2^{じ}{次}コートと^{しあ}{仕上}げでしっとりすべすべになるわ

\A はあ
\ そうなんですか

\S ^{かみ}{髪}も^{あたら}{新}しくしたの
\ コゲてたからね

\A えっ!

\S ^{ぜん}{前}と^{おな}{同}じ^{て}{手}^{かみ}{髪}よ
\ やっぱり^{いろ}{色}が^{か}{変}わっちゃイヤよね

\A ^{あんしん}{安心}しました
\ ふう

\A ほんとになんてお^{れい}{礼}を^{い}{言}ったらいいのか

\S だからお^{れい}{礼}はあのおっさんにわ

\S あした^{いちにち}{1日}で^{お}{終}わるから

\S あさってには^{かえ}{帰}れるわ

\page
\A あ

\A なんだか^{なみだ}{涙}がでてきた

\A ^{ほんらい}{本来}
\ ^{わたし}{私}の^{るいせん}{涙腺}は
\ ^{め}{目}を^{うるお}{潤}すためだけのものなんだけど

\A こんな^{こと}{事}は
\ ^{ひとり}{一人}で^{げっきん}{月琴}^{ひ}{弾}いてる^{とき}{時}なんかには
\ よくあった
\ それとは^{べつ}{別}の^{かん}{感}じだけど

\A こういう^{きも}{気持}ちも^{す}{好}きだな

\page
\N ^{かえ}{帰}りの^{ひ}{日}

\O アルファさんは^{ようい}{用意}できてましたか?

\S ちょっと^{み}{身}だしなみで^{なや}{悩}んでたみたいよ

\O あんだかなー
\ おら
\ ^{はし}{走}んじゃねえ

\O ありがとうございました
\ ツケの^{ほう}{方}はいずれ

\S あら
\ こちらこそ^{たの}{楽}しかったわ
\ ^{むすめ}{娘}ができたみたいで

\S それに
\ ガソリンや^{やさい}{野菜}のツケもたまりまくってたし

\T アルファ^{かえ}{帰}るよ!

\page
\T ア

\A あ

\T かっ
\ ^{かみがた}{髪型}^{か}{変}えたの!?

\A ちっ
\ ^{ちが}{違}うっ!!

\O うわっ!
\ あんだこりゃ
\ わはは

\A ひ〜〜ん!

\S ^{しんぴん}{新品}の^{かみ}{髪}だから
\ ピンピンよなじむまで

\A どっ
\ どのくらいでなじみます?

\page
\S まあ
\ ^{い}{1}^{しゅうかん}{週間}かしら

\A いっ
\ ^{い}{1}^{しゅうかん}{週間}!!

\A まだ^{なみだ}{涙}がでてきた
\ ^{きも}{気持}ちはぜんぜん^{ちが}{違}う

\page
\A おじさん
\ あの
\ ありがとうございました

\O あに
\ ^{言}{せ}^{う}{え}だかよー

\O ^{れい}{礼}は^{せんせい}{先生}にで^{じゅうぶん}{十分}だよ

\A あの
\ ^{わたし}{私}
\ お^{れい}{礼}する^{もの}{物}がないんで
\ お^{ふたり}{二人}はずっとコーヒーでもなんでもただで

\O あんだかなあ
\ いいじゃんかよー
\ ^{べつ}{別}によー

\O こういうときや
\ あんた^{かぞく}{家族}みてえなもんだしよー

\page
\A うわ〜〜ん

\O あんだ
\ あんだ?

\O あに^{な}{泣}くかよ
\ ええ?

\A う
\ うれしいんです

\A ありが

\T ぶっ

\page
\T わはははははは

\A ひ〜〜ん


\subsection{第5話\ エンドレス^{ちょうない}{町内}^{かい}{会}}

\page[102]
\N アルファさんを^{ちょうない}{町内}^{かい}{会}にさそった
\ ^{かのじょ}{彼女}の^{ぜんかい}{全快}^{いわ}{祝}いも^{かね}{兼}てただけんど

\page
\N アルファさんが^{つ}{着}いたころには
\ もうできあがってたな

\P アルファさんにはせっかく^{き}{来}てもらっただけんどよ
\ あんた
\ ^{さしみ}{刺身}とか^{く}{食}えねえだっつうからよ

\P うちンのだけんどミカンとかは^{く}{食}う?

\A ああっ
\ ^{く}{食}います!!

\page
\O アルファさんよ
\ ひとり
\ シラフでミカン^{く}{食}っててもつまらねえべ
\ ちっとだけでも^{の}{飲}めねえのかい

\A いえ
\ お^{さけ}{酒}は^{かお}{香}りだけで
\ ^{じゅうぶん}{十分}^{たの}{楽}しんでます

\A ^{の}{飲}むのは^{つよ}{強}くないんですよ

\page
\P あ〜〜によ〜〜
\ のめんなら^{い}{言}ってけえなよ
\ アルちゃんよ〜〜

\A え?!
\ アルちゃん?

\P そうだよ〜〜
\ はい
\ これね

\P ま ま ま ま

\A わかりました

\A でも
\ おちょこに1^{はい}{杯}だけですよ
\ それが^{げんかい}{限界}ですから

\page
\A ふう

\page
\P じゃー
\ ^{つぎ}{次}アルちゃんもなんかやってよ

\A ん〜〜
\ よし
\ じゃあねえ

\A ^{てびょうし}{手拍子}して

\A こんなかんじで

\P こう?

\A そうそう

\page[109]
\N ^{とつぜん}{突然}^{はじ}{始}まった
\ ^{み}{見}たこともない^{まい}{舞}に
\ ^{ぜんいん}{全員}^{どぎも}{度肝}をめかれた

\page
\N おどりは
\ だんだん^{てびょうし}{手拍子}の^{たんじゅん}{単純}なリズムにとけて
\ ^{なが}{流}れるような^{うご}{動}きへと^{うつ}{移}っていく

\page
\N ^{みな}{皆}
\ その^{とき}{時}は^{ゆめ}{夢}のようだったと^{い}{言}う
\ アルファさん^{じしん}{自身}おぼえていなかった

\N ^{なが}{長}かったのか
\ ^{みじか}{短}かったのか

\page
\N やがて
\ ^{しず}{静}かに^{や}{止}んだ

\A おしまい

\P いや〜〜
\ なんか
\ すごかったねー
\ あっけにとられちゃったよ

\A えへへ

\page
\P お

\O 2^{はい}{杯}も^{の}{飲}ますからよ〜〜

\P いや
\ アルちゃんはえらかった

\page
\A ばっ

\N その^{ひ}{日}は^{けっきょく}{結局}
\ ^{ぜんいん}{全員}
\ ^{きゅうじつ}{休日}に
\ ^{ごご}{午後}は
\ ^{みな}{皆}でアルファさんの^{みせ}{店}におしかけたな

\A ぐー


\subsection{第6話\ プレ^{ねしょうがつ}{寝正月}}

\page[116]
\A う〜〜ん

\A またおフロで^{としこ}{年越}ししちゃったよ

\A ^{がんじつ}{元日}^{まいとし}{毎年}^{はつひ}{初日}を^{み}{見}に^{い}{行}く

\A ^{あいぼう}{相棒}のスクーターはおじさんが^{なお}{直}してくれた

\page
\A ^{ことし}{今年}はタカヒロもいっしょだ

\T アルファ!

\A うわっ!

\page
\A ありゃまー
\ タカヒロ

\T ^{ことし}{今年}もよろしく

\A よろしくね
\ まあ
\ はいってよ

\T おじゃまします

\T まだ^{はや}{早}かった?

\A ちょっとね
\ だいぶね

\A やっぱりおじさんは^{い}{行}かないの?

\T ^{としこ}{年越}し^{ちょうない}{町内}^{かい}{会}だって

\page
\A ん〜〜
\ それじゃ^{さむ}{寒}いでしょ

\A ^{わたし}{私}のジャケット^{か}{貸}してあげるよ

\T ^{だいじょうぶ}{大丈夫}だよ
\ さっき^{あつ}{暑}いくらいだったし

\A ^{じてんしゃ}{自轉車}だからねーー
\ バイクだとすごい^{さむ}{寒}いよ

\T アルファで^{かぜ}{風}よけるから^{へいき}{平気}だよ

\page
\A このやろ!

\A うしろはとっかからないからね
\ ちゃんと^{うで}{腕}まわしてないと
\ ^{お}{落}としてっちゃうよ

\T うん

\T じゃごーー!

\page[122]
\A ここらでコーヒーブレイクかな?

\A はい

\T ありがと
\ ^{かん}{缶}のコーヒーなんて^{ひさ}{久}しぶりだよ

\page
\T なんか
\ うまい

\A でしょ!

\A ^{さむ}{寒}い^{よる}{夜}にバイクで^{はし}{走}った^{とき}{時}の
\ ^{かん}{缶}コーヒーって^{いじょう}{異常}においしいのよ

\T これミルクはいってるよ

\A このくらいなら^{へいき}{平気}よ
\ カフェオレはつらいけどね

\page
\Sign みそ^{しる}{汁}
\ ^{あまざけ}{甘酒}

\page
\A たくさんじゃないけど

\A ここには^{まいとし}{毎年}けっこう^{ひと}{人}があつまる

\A ^{ひ}{冷}えてきた?

\T えっ?

\T まだ^{だいじょうぶ}{大丈夫}だよ!

\A ほーお
\ ふーん

\A ^{わたし}{私}ねー
\ ちょっと^{まえ}{前}が^{さむ}{寒}くて

\page
\A しばらくこうしててもいいかな?

\T うん
\ いいよ

\page
\A ふだんがあんまり^{ふゆ}{冬}っぽくないぶん
\ やっぱり
\ こうして^{さむ}{寒}い^{ちゅう}{中}
\ ^{はつひ}{初日}を^{み}{見}なきゃ
\ 「^{しんねん}{新年}」って^{き}{気}がしないね

\page
\A ^{みな}{皆}
\ ^{じぶん}{自分}のやり^{かた}{方}で^{み}{見}ている

\P ^{くも}{雲}あんからへー
\ ^{み}{見}えないや

\P んー

\A ^{いま}{今}は^{むかし}{昔}ほど^{きせつ}{季節}がはっきりしないけど

\A みんな^{まえ}{前}よりも
\ ^{ものごと}{物事}に^{かん}{感}じ^{はい}{入}る^{こと}{事}が
\ ^{おお}{多}くなったと^{おも}{思}う

\page
\T もう
\ ^{まえ}{前}は^{さむ}{寒}くなくなった?

\A えっ?
\ ああ
\ うん

\A もうちょっとここにいよっか

\A うん

\page
\T なんか
\ ねむくなってきたよ

\A そうね
\ やっぱ
\ ^{はや}{早}いとこ^{かえ}{帰}ってねちゃおっか


\subsection{第7話\ ^{ごぜん}{午前}2/2}

\page[132]
\N AM7:20
\ ^{だいさん}{第三}^{けいひん}{京浜}^{せん}{線}^{しゅうてん}{終点}

\page[138]
\A コーヒー

\A 8^{わり}{割}がた^{わたし}{私}がのんでるかなーー

\page
\P ^{かえ}{帰}りの^{びん}{便}は^{てはい}{手配}してあんの?

\K はい

\P ^{さいご}{最後}まで^{おく}{送}ってやりてえけどせっぱつまっててよ
\ わりいな

\K いえ
\ ^{たす}{助}かりました

\page[141]
\O で
\ その^{みち}{道}のどんづまりだよ

\K わかりました

\O つれてってやりてえけど
\ じきガソリン^{や}{屋}くんだよな

\K いえ
\ すぐそこみたいですし

\O わりいな

\O ^{に}{似}てんな

\page
\N AM11:05

\page
\K ^{はつせの}{初瀬野}

\page
\A ^{おもや}{母屋}に?

\A はーーい

\A どちらさま?

\page
\K ^{はつせの}{初瀬野}アルファさんですね

\A はい

\A あっ
\ いや〜〜
\ フルネームなんか^{ひさ}{久}しぶりだから
\ ^{いっしゅん}{一瞬}^{かんが}{考}えちゃった

\K はあ

\K ^{わたし}{私}
\ ムサシノの^{たくはい}{宅配}^{びん}{便}ですが

\A ^{たくはい}{宅配}
\ ふわ〜〜
\ たいへんだったでしょーー
\ ムサシノからじゃ

\K いえ

\page
\A ^{とど}{届}けものなんて^{ひさ}{久}しぶりよ
\ どちらから?

\K ええ

\K ^{はつせの}{初瀬野}^{さま}{様}から

\A オーナーから


\subsection{みーやんのやっぱこれっす!}

\Sign ^{こうきゅう}{高級}^{まめ}{豆}
\ ^{こくさん}{国産}^{だいず}{大豆}

\Sign がめてきた^{ふくろ}{袋}

\Sign イライラ

\Sign ポイだ
