\section{Volume 11}

\subsection{^{だい}{第}100^{わ}{話}\ ふたり}

\page[3]
\Sign ^{つか}{使}えるドア

\Sign ドアノブ


\subsection{第101話\ ^{かいてん}{開店}の^{に}{二}^{はい}{杯}}

\page[10]
\Alpha お^{みせ}{店}ができました

\Alpha ^{しょうめん}{正面}を^{しろ}{白}くぬりました

\page
\Alpha ほとんど^{ごうはん}{合板}だし、^{そと}{外}のデッキも、
まだスノコだけど、とりあえず「^{しつない}{室内}」には、なりました

\Alpha ここのところコーヒー^{まめ}{豆}がないので

\Alpha しばらくは、ムギ^{ちゃ}{茶}とソバ^{ちゃ}{茶}とインスタントコーヒーのニセ^{きっさ}{喫茶}^{てん}{店}です

\page[14]
\Alpha いらっしゃいませー

\Maruko ほほお……

\Alpha あ!

\Alpha お^{きゃく}{客}さん!
\ あ〜〜〜と
\ ロボットの!!

\Maruko んー

\Maruko へえ……
\ で、これが^{れい}{例}のお^{みせ}{店}

\Alpha そうなんです

\page
\Maruko なんかこう……
\ ぼろっちくない?

\Alpha ははは〜〜
\ なにせ、^{てづく}{手作}りなもんで

\Alpha ほっといてください

\Alpha お^{きゃく}{客}さんが^{しんそう}{新装}^{かいてん}{開店}^{ひとり}{一人}^{め}{目}のお^{きゃく}{客}さんです

\Maruko おー
\ そりゃめでたいね

\Maruko ^{きょう}{今日}は、もう^{ひとり}{一人}^{き}{来}てんだけど

\Maruko あれ?

\Maruko どしての^{こ}{来}なよ

\page
\Kokone あ……どうも……

\Alpha わー!!

\Alpha ココネ!
\ ……え?

\Kokone ごっ……
\ ごぶさたしてます

\Alpha こちらお^{し}{知}り^{あ}{合}い?

\Kokone は?

\Kokone ^{まる}{丸}^{こ}{子}さんです

\Alpha えっ

\page
\Maruko どおも

\Alpha はーー

\Alpha お^{きゃく}{客}さんが「^{まる}{丸}^{こ}{子}さん」でしたか

\Alpha なーんだ
\ この^{まえ}{前}、^{い}{言}ってくれればー

\Maruko ほんとに^{ぜんぜん}{全然}^{き}{気}づかなかったの?

\Maruko にぶいぞ

\page
\Maruko この^{ちか}{近}くまで^{く}{来}る^{よう}{用}があったね

\Maruko で、ついでだから、^{かのじょ}{彼女}にもつきあってもらったんだ

\Alpha なるほど

\Kokone あ…あの……
\ ^{まる}{丸}^{こ}{子}さん……

\Kokone あ……
\ すみません
\ ^{きゅう}{急}な^{はなし}{話}で

\Alpha んーん

\page
\Maruko ふつう、いきなり^{く}{来}るよねえ

\Kokone でも、^{わたし}{私}、もうちょっとちゃんと……
\ のんびり^{き}{来}てみたかったかな……

\Alpha ありがと

\Alpha ひさしぶりだね

\Alpha まー
\ すわっちゃって、すわっちゃって

\page
\Kokone あの……
\ おめでとうございます
\ お^{みせ}{店}、^{あたら}{新}しくなって

\Alpha へへ
\ しばらく、これでやってこっかなってね

\Alpha ^{まえ}{前}よか^{やす}{安}っぽくなっちゃったけど、ワイルドさが^{ま}{増}したってことでー

\Kokone あはは

\page
\Alpha ふたりはさ
\ よく^{いっしょ}{一緒}に^{あそ}{遊}んでんの?

\Kokone あ……
\ ってゆうか

\Maruko まあねー
\ ^{いえ}{家}もわりと^{ちか}{近}くでさー

\Alpha あ、い〜な〜

\Maruko あ・・いえ
\ あそぶってゆうか
\ あの〜・・

\page
\Alpha はい!

\Alpha ^{とうてん}{当店}、^{きょう}{今日}^{さいしょ}{最初}で^{さいご}{最後}のコーヒーでございますよ

\Kokone え?

\Alpha ちょうどよかった

\Alpha ^{かいてん}{開店}^{きねん}{記念}

\Kokone いただきます


\subsection{第102話\ A7M3^{まる}{丸}^{こ}{子}マルコ}

\page[27]
\Maruko アルファさん

\Alpha え?
\ はい!

\page
\Maruko あんた、まじめに、お^{みせ}{店}やろうって、^{おも}{思}ったことある?

\Alpha え?

\Maruko どうやったらコーヒー^{まめ}{豆}を^{き}{切}らさずに、いけるかとか

\Maruko もっと、お^{きゃく}{客}さんに^{き}{来}てもらうにはとか、そうゆうの^{かんが}{考}えてる?

\page
\Maruko ひとのやり^{かた}{方}は^{べつ}{別}に^{じゆう}{自由}だけどさ

\Maruko ^{しょくひ}{食費}が^{ひく}{低}い^{ごと}{事}に、あぐらかいて、だらーっと^{い}{生}きてるロボット^{み}{見}ると

\Maruko ^{わたし}{私}ちょっと、ムッとくるんだよね

\Maruko こんなイナカにいると、わかんないだろうけど

\Maruko ほかの^{ひと}{人}は…
\ ロボットは

\Maruko ^{じぶん}{自分}の^{しごと}{仕事}、もっと^{しんけん}{真剣}にやってるよ

\Kokone ま…
\ ま…

\page
\Maruko ここはいいよね
\ あったかいし

\Maruko まわりのご^{ろうじん}{老人}はやさしいし

\Maruko たまに^{く}{来}る、お^{きゃく}{客}さんにニコニコ^{えがお}{笑顔}^{み}{見}せてれば

\Maruko それで、みんなにかわいがられるんだから

\page
\Alpha あ〜〜と

\Alpha コーヒー

\Alpha だめですかねえ……

\Maruko だめだね

\Alpha はあ・・

\page
\Alpha ごめんなさい
\ もっと^{べんきょう}{勉強}します

\page
\Alpha あの〜〜
\ ^{まる}{丸}^{こ}{子}さん

\Maruko ん

\Alpha ^{まる}{丸}^{こ}{子}さんとはお^{ともだち}{友達}としてお^{はなし}{話}とかしたいんですけど……

\Alpha いいですか?

\Maruko はー
\ もう「お^{ともだち}{友達}」ですか

\Maruko いいよ

\Alpha よかった
\ じゃ

\page
\Alpha あのさあ

\Alpha うそでニコニコしてやっていけるわけないじゃんよ

\page
\Alpha たしかに^{わたし}{私}ダラーってしてんけどさ

\Alpha あそびで、お^{みせ}{店}やったり、お^{ちゃ}{茶}^{だ}{出}したりしてる
\ つまりないよ

\Alpha まあ、^{じょうれん}{常連}さんが^{き}{来}てくれるから
\ やっとこ、お^{みせ}{店}の^{かたち}{形}してるけど

\Alpha ^{からだ}{体}がロボットじゃなかったら、^{た}{食}べてけないかもね

\Alpha でもさあ

\page
\Alpha ^{しょくひ}{食費}がかかんない^{からだ}{体}だとか……

\Alpha ラッキーって^{おも}{思}ったりはするけど

\Alpha ^{じぶん}{自分}がロボットだってこと、いつも^{き}{気}にしてなきゃなんて
\ ^{わたし}{私}、^{おも}{思}ってないから

\Alpha それだけ^{い}{言}っとくよ

\page
\Maruko わかってるよ

\Maruko コーヒーも、けっこうのめる

\Alpha へ?

\page
\Maruko ひやかしだよ

\Maruko おめでとう
\ ^{かいてん}{開店}

\Alpha え……
\ ああ
\ ありがと…

\page
\Maruko また
\ くるかも

\Alpha うん
\ きてよ

\page
\Maruko ココネさん

\Maruko ^{きょう}{今日}は^{わる}{悪}かったね

\Kokone いいえ


\subsection{第103話\ ^{がけ}{崖}の^{みず}{水}}

\page[43]
\Alpha ^{ほんとう}{本当}は
\ ^{まる}{丸}^{こ}{子}さんの^{い}{言}うとおりだ

\Alpha ^{しごと}{仕事}

\Alpha ^{しょうばい}{商売}としてのお^{みせ}{店}のこと

\Alpha ^{わたし}{私}は、そのへんのこと^{ぜんぜん}{全然}だめだ

\page
\Alpha コーヒー^{まめ}{豆}のことはどうしても、^{げんかい}{限界}がある

\Alpha で……
\ ^{みず}{水}さがしをやろうと^{おも}{思}った

\Alpha ^{いど}{井戸}の^{みず}{水}も、^{すいどう}{水道}の^{みず}{水}も、まずくはないけど

\Alpha ときどき、^{しお}{塩}や^{てつ}{鉄}の^{あじ}{味}がまじる^{き}{気}がする

\page
\Alpha ^{むかし}{昔}からの^{すいげん}{水源}^{ち}{地}はうちからは、すごく^{とお}{遠}い

\Alpha ^{ちか}{近}くで^{みず}{水}が、^{わ}{湧}いていそうな^{ところ}{所}といえば

\Alpha ^{たに}{谷}のどんづまりとか、ガケの^{した}{下}とかが、^{ゆうりょく}{有力}なんだけど

\page
\Alpha どこも^{ひかげ}{日陰}で、ジメっと^{どろ}{泥}っぽくて

\Alpha ^{き}{気}が^{めい}{滅入}ってくる

\Alpha ^{とうめい}{透明}な^{みず}{水}が、こうポワポワって^{わ}{湧}いてるのが、^{りそう}{理想}なんだけど

\Alpha これなんか

\Alpha カニ^{あな}{穴}からジクジク…

\page
\Alpha ^{うみ}{海}のガケに^{き}{来}た

\Alpha んん

\Alpha お

\page[49]
\Alpha コンクリートの^{ちい}{小}さなマスの^{なか}{中}に、^{す}{澄}んだ^{みず}{水}が、たまっている

\Alpha ^{わ}{湧}き^{みず}{水}のない、このあたりでは、こうやって、
しみ^{だ}{出}した^{みず}{水}を^{あつ}{集}めて^{つか}{使}ってたんだろう

\page
\Alpha あふれた^{みず}{水}も^{いわ}{岩}にあけた^{あな}{穴}をいくつもつなげて^{つか}{使}えるようにしてある

\Alpha ^{だいこん}{大根}^{あら}{洗}い^{よう}{用}かな

\page
\Alpha ん〜〜

\Alpha ちょっとコケの^{あじ}{味}するけど

\Alpha もらってこ……

\page[53]
\Alpha ^{しろ}{白}い……
\ キノコ?

\Alpha みたいなのが^{た}{立}っている

\page[55]
\Alpha お^{ちゃ}{茶}でのんでみた

\Alpha うん

\page
\Alpha ^{う}{売}り^{もの}{物}にしてはいけない^{みず}{水}のような^{き}{気}がした

\Alpha ^{みず}{水}は、^{きなが}{気長}にさがそうと^{おも}{思}う


\subsection{第104話\ ^{わた}{渡}り}

\page[59]
\Person さてー

\Person ここでひとつ^{じょうほう}{情報}があります

\page
\Person ^{そら}{空}をよく^{み}{見}るようなヒマな^{かた}{方}なら、ご^{ぞん}{存}じでしょう

\Person ^{すうねんまえ}{数年前}から、
「ターポン」という^{しろ}{白}い^{おお}{大}きなものが^{ときどき}{時々}^{とお}{通}ってますね

\Person まだ^{み}{見}たことないと^{い}{言}う^{かた}{方}

\Person ^{きょう}{今日}が、チャンスですよ

\Person これは、ここ^{ろく}{六}^{ねん}{年}ほどたまに^{そら}{空}を^{とお}{通}っている

\Person まあ、でっかいヒコーキの^{たぐい}{類}なんですけどね

\page
\Person ^{きょう}{今日}から^{ろく}{六}^{ねん}{年}^{かん}{間}は^{みなみはんきゅう}{南半球}に
^{い}{行}ってしまうとされています

\Person はあ、そんなサイクルがあったんですねえ

\Person ^{つぎ}{次}にまた、もどって^{く}{来}るという^{ほしょう}{保証}はないので

\Person ^{きょう}{今日}の…えーと
\ お^{ひる}{昼}から^{いち}{一}^{じかん}{時間}^{くらい}{位}の、あいだーー

\Person ちょっと^{そら}{空}を^{み}{見}てみましょう

\Person ^{さいわ}{幸}い、
^{きょう}{今日}の^{しま}{島}の^{なんがん}{南岸}^{ちほう}{地方}はよく^{は}{晴}れてますね

\page[63]
\Person あ……

\Person ^{いま}{今}、ちょうどスタジオの^{まうえ}{真上}です……

\page[65]
\Person あそこには、まだ、^{ひと}{人}がとじこめられているという^{はなし}{話}です

\Person ^{いま}{今}ごろ^{ちじょう}{地上}を^{み}{見}てるかも、しれませんね……

\page[69]
\Person ^{いちばん}{一番}・^{さんばん}{三番}の^{けつばさ}{毛翼}、^{せいせい}{生成}を^{かくにん}{確認}しました……

\Person アルファー^{しつちょう}{室長}、おつかれさまでした

\ASevenMOne いえ

\Person ^{しん}{新}^{こうろ}{航路}へ^{いこう}{移行}します……

\page
\ASevenMOne ^{した}{下}の^{しま}{島}には、まだ、^{ひと}{人}の^{せいかつ}{生活}が^{み}{見}える

\ASevenMOne ^{ろく}{六}^{ねんご}{年後}にも、まだ、^{ひと}{人}のにおいが、しっかりありますように

\ASevenMOne ^{げんき}{元気}で……
\ さようなら

\page
\Person ^{い}{行}ってしまいましたねえ

\page
\Person ^{ろく}{六}^{ねんご}{年後}にまた、^{げんき}{元気}にもどってくることが、できますように

\Person さようなら


\subsection{第105話\ 「^{ちょうこく}{超黒}」}

\page[74]
\Alpha 「^{ちょうこくとう}{超黒糖}」

\Alpha ^{じょうれん}{常連}さんからいただいた、^{みなみ}{南}の^{くに}{国}のおみやげです

\Sign ^{がんそ}{元祖}
\ ^{ちょうこく}{超黒}

\Alpha これは、タダ^{もの}{者}じゃないですよ

\Alpha ^{さとう}{砂糖}ファンの^{わたし}{私}ですが^{ま}{負}けました

\page
\Alpha はじめはちょっとクセのある^{くろざとう}{黒砂糖}かな〜、くらいなんですけど

\Alpha それが

\page
\Alpha くう

\Alpha うまーい
\ うまーい

\Alpha すさまじく^{ふくざつ}{複雑}な^{あま}{甘}さの^{なみ}{波}が、^{からだ}{体}を^{はし}{走}ります

\page
\Alpha ふう…

\Alpha すん
\ んん
\ すばらしい

\Alpha これは、あれですね

\Alpha ココネの^{い}{言}ってた、あやしげなレコード

\Alpha あれに^{ちか}{近}いんじゃないかなあ

\Alpha ヘタすると、^{けしき}{景色}くらい、^{み}{見}えそうですもん

\Alpha こう……

\Alpha なんて^{い}{言}うか

\Alpha こう……

\page
\Takahiro ごぶさたー

\Alpha あ

\Takahiro どしたの

\Alpha あ、いや……
\ なんでもない

\page
\Takahiro へえ……
\ これがねえ

\Alpha へへへ

\Alpha うん
\ すごいよ
\ タカヒロも^{た}{食}べてみ

\Takahiro ん〜〜
\ たしかに、^{こ}{濃}い^{あじ}{味}だとは^{おも}{思}うけど……

\Takahiro でも、^{くろざとう}{黒砂糖}だなあ

\Alpha ありゃ?

\page
\Alpha ^{わたし}{私}だけなのかなあ

\Takahiro アルファ、もっかい^{た}{食}べてみてよ

\Alpha え〜〜
\ もったいないなあ

\Alpha じゃあ、もう^{いっこ}{1個}だけ

\Takahiro うん

\Alpha これ、ゆっくりなめるんだよ
\ タカヒロ、かじったでしょ

\Takahiro あー
\ かもしんない

\page[82]
\Takahiro だっ
\ ^{だいじょうぶ}{大丈夫}?

\Alpha うますぎる!

\page
\Alpha ^{あば}{暴}れまわるおいしさ^{かん}{感}

\Alpha きっと、どこか^{ひと}{人}と^{ちが}{違}うツボにはまるんでしょう

\Takahiro ^{しごと}{仕事}^{ば}{場}の^{はなし}{話}とかしようと^{おも}{思}ったんだけど

\Takahiro ^{くるま }{車}^{かえ}{返}さなきゃ

\Takahiro またくるよ

\Alpha うん

\Alpha ^{ぎゅうにゅう}{牛乳}の^{とき}{時}もそうだっけ

\Alpha ^{みち}{未知}の^{ツボ}{弱点}もまだありそうです

\page
\Alpha タカヒロは、「アルファのマタタビ」と^{い}{言}ってました

\Alpha ピッタリすぎる^{ひょうげん}{表現}です

\Alpha んー
\ まーねー


\subsection{第106話\ ^{みち}{道}と^{まち}{町}と^{じゅうにん}{住人}}

\page[87]
\Alpha また
\ ^{すこ}{少}し、お^{みず}{水}ください

\page
\Ayase ^{れい}{例}の「^{がいとう}{街灯}の^{き}{木}」がたくさんある^{やま}{山}あいの^{みち}{道}

\Ayase ここは、^{ふじ}{富士}^{さん}{山}^{ちか}{近}くのあの^{みち}{道}とは^{ちが}{違}って、
^{ひと}{人}は^{とお}{通}らない

\page
\Ayase ^{き}{木}は、なぜか、やっぱり^{むかし}{昔}の^{おおどお}{大通}りの^{いち}{位置}をなぞるように、
^{は}{生}えている

\Ayase ここにも、^{もと}{元}は^{ちゅう}{中}くらいの^{まち}{町}があった

\Ayase ^{まえ}{前}からここに、^{く}{来}るように、すすめられていたのは

\Ayase この^{き}{木}がめあてではない

\Ayase あれだ

\page
\Ayase ^{いっけん}{一見}、ビルのようだが^{まど}{窓}はなく、^{びみょう}{微妙}にいびつだ

\Ayase あそこへ^{む}{向}かう、^{みち}{道}などはなく、^{かこ}{過去}にも^{なに}{何}かが、
^{た}{建}てられた^{きろく}{記録}はないという

\Ayase ヤブこいで^{い}{行}ってみるか

\Ayase ^{いちにち}{一日}かかるよなあ

\page
\Person あらー
\ なに、こんなとこで

\Person ^{まよ}{迷}ったの?
\ ^{りょこう}{旅行}じゃないよね

\Ayase おばちゃん、ここの^{ひと}{人}?

\Ayase ども

\Person ^{おく}{奥}にちょっと^{はたけ}{畑}が、あんの

\Person ^{にい}{兄}ちゃん、この^{さき}{先}^{ある}{歩}ってたら、^{よる}{夜}んなるよ

\Ayase ああ
\ そうすかやっぱ

\Ayase いや、なんか、^{へん}{変}なとこがあるって^{き}{聞}いたんで……

\Person うん
\ ^{へん}{変}なとこだよ、ここは

\Ayase ここ……

\Ayase ^{もと}{元}は^{まち}{町}だったんですよね

\Person うん
\ ^{えき}{駅}もあったよ

\page
\Ayase で
\ あの、^{やま}{山}の^{うえ}{上}の^{しろ}{白}いビルはなんすか?

\Person あれねー

\Person でも、ビルじゃないよ

\Ayase え?

\Person ^{むかし}{昔}ねえ、みんなで^{まる}{丸}^{いちにち}{一日}かけて、
^{み}{見}に^{い}{行}ったんだよ

\Person ^{や}{柔}っこいんだ

\Person キノコみたいな
\ コルクみたいな

\Person ^{なんねん}{何年}^{まえ}{前}だったか、それっきり^{い}{行}ってない

\page
\Person あのビルもそうだけど
\ なんか^{へん}{変}だよ、ここは

\Person いつも^{なに}{何}か^{けはい}{気配}がするし

\Person あー
\ そうだ

\Person この^{さき}{先}に、^{こうえん}{公園}のあとがあるの

\Person ^{い}{行}ってごらん、あんたには^{おもしろ}{面白}い^{ところ}{所}かもしんない

\Ayase ほお

\Person ここで^{ま}{待}ってるよ

\Person ふもとまで^{おく}{送}ってあげる

\Ayase どうも
\ ちょっと^{み}{見}てきます

\page
\Ayase よくある^{はいきょ}{廃墟}だが

\Ayase ^{とお}{通}りだけが^{みょう}{妙}にくっきりと、きれいに^{のこ}{残}っている

\Ayase ^{とちゅう}{途中}、^{せっかい}{石灰}の^{やま}{山}のようなものをいくつか^{み}{見}た

\page
\Ayase ^{むかし}{昔}はぼうぼうでもたしかに^{もと}{元}は、
^{たかだい}{高台}の^{こうえん}{公園}だったという^{かん}{感}じだ

\Ayase ^{した}{下}の^{ぼんち}{盆地}には、わりと^{おお}{大}きな^{かわ}{川}
と、^{いま}{今}も^{い}{生}きている^{まち}{町}が^{にしび}{西日}に^{ひか}{光}っている

\Ayase ^{まち}{町}には、^{やま}{山}のアレに^{に}{似}たビルがある

\Ayase ^{ゆうがた}{夕方}のきれいな^{こうえん}{公園}だったんだろう

\page[97]
\Ayase ^{にんぎょう}{人形}
\ いや、キノコか?

\Ayase あっちの^{しろ}{白}い^{つち}{土}^{も}{盛}りは

\Takahiro あの^{かたち}{形}は

\page
\Ayase ども……
\ お^{ま}{待}たせしました

\Person どう?
\ おもしろかった?

\Ayase はい

\Person じゃ、^{かえ}{帰}ろうか
\ ^{よる}{夜}は^{よる}{夜}で^{きみ}{気味}わるいからで

\Ayase はは


\subsection{第107話\ ^{じょうれん}{常連}さん}

\page[103]
\Alpha いらっしゃいませ!

\page
\Maruko よ

\Alpha ひさしぶり

\Maruko あのさあ

\Maruko そこまで、あからさまにイヤそうな^{かお}{顔}するかなあ

\Alpha はははははは

\Alpha ぜんぜん、そんな^{かお}{顔}してないですよ

\page
\Alpha きょうは
\ ^{ひとり}{一人}で?

\Maruko そうだよ

\Alpha そうすか

\Alpha まーまー
\ どうぞ、どうぞ

\Maruko とりあえず
\ コーヒー

\Alpha はい

\page[107]
\Alpha おまたせ
\ コーヒーです

\Alpha あいかわらずですけど

\page
\Maruko ふふ

\Alpha なっ……
\ なんすか

\Maruko やっぱり^{まえ}{前}にすわっちゃうんだねえ

\Alpha へ?

\Alpha あ…うん
\ もう、クセかなあ

\Maruko クセだね

\page
\Alpha ^{まる}{丸}^{こ}{子}さん……
\ ^{くるま}{車}の^{おと}{音}しなかったけど

\Alpha ^{ある}{歩}きで^{き}{来}たとか?

\Maruko ^{くるま}{車}だけ^{した}{下}の^{みち}{道}に^{お}{置}いてきたんだよ

\Alpha なんでまた、わざわざ

\Maruko ふふん

\Maruko おかげで^{きたい}{期待}^{どお}{通}りのモノ^{み}{見}られた

\Maruko 「うわっヤなやつが^{き}{来}た」ってのが
\ モロに^{で}{出}ちゃってるあんたの^{かお}{顔}ー

\Alpha うわっヤなやつ

\page
\Maruko このあいだはさあ……

\Maruko ココネさんに^{わる}{悪}いことしたなって^{おも}{思}ったよ

\page
\Alpha はい
\ ^{こんど}{今度}は^{みず}{水}がちょっと^{ちが}{違}うんですよ

\Maruko へー

\Maruko あ
\ ほんと……

\Alpha とっときの^{みず}{水}なんです

\Alpha これ、おつまい

\Maruko ほー

\Maruko たしかに、ちょっと^{ちが}{違}うね

\Alpha でしょー

\page
\Maruko なにこれー

\Alpha あははははは

\Alpha だ……だだだいじょうぶ!!

\Alpha とっときの^{くろざとう}{黒砂糖}!

\Alpha ひ〜ひ〜

\Alpha おどろいた?
\ ごめんね〜〜

\Maruko いや……
\ いいけどさ
\ おいしいし

\page
\Maruko さあて

\Maruko ^{ちず}{地図}……
\ ^{かえ}{返}してもらおうかな

\Alpha えっ?!

\Maruko ずっと^{まえ}{前}^{わす}{忘}れてったやつ

\Alpha えっ?!

\Maruko ^{なに}{何}?
\ ないの?

\Alpha いえ……
\ あの〜〜
\ ^{も}{持}ってきます

\page
\Alpha あの〜〜

\Maruko なんだ
\ あるんじゃん
\ ^{なに}{何}?

\Maruko ^{あお}{青}い、うねうねとか、ピンクの^{さんかく}{三角}とか

\Maruko なに?
\ この^{か}{書}きこみ

\Maruko ^{ぜん}{全}ページ

\Alpha あ〜〜と
\ いろいろ……
\ シミョレーションしてて

\page
\Alpha あの……
\ ^{べんしょう}{弁償}します

\Maruko やるよ

\Alpha え……

\Maruko ^{わたし}{私}のは、もう^{あたら}{新}しいのがあるんだよ

\Maruko お^{とくいさき}{得意先}とか^{こま}{細}かいところ^{うつ}{写}さしてもらおうと^{おも}{思}ってね

\Maruko あんたも、この^{か}{書}きこみが^{だいじ}{大事}なんでしょ?

\Alpha あ……
\ それじゃ
\ いただきます

\page
\Maruko じゃ

\Alpha また


\subsection{第108話\ しずく}

\page[118]
\Alpha お^{みせ}{店}のドアと^{まど}{窓}をはずしてしまった

\page
\Alpha このごろ、^{てんない}{店内}の^{くうき}{空気}がムレムレになっていく
\ ^{いっぽう}{一方}で……

\Alpha ^{わたし}{私}が^{つく}{作}ったベニヤ^{いた}{板}のカベに^{もんだい}{問題}があるのは^{たし}{確}かだった

\Alpha ためしに、ドアと^{まど}{窓}をはずしてみたら、^{かぜ}{風}が、どどっと^{はい}{入}ってきて、
^{いっしゅん}{一瞬}で^{くうき}{空気}が^{い}{入}れかわった

\Alpha まあ、^{きょねん}{去年}の^{なつ}{夏}、あずま^{や}{屋}^{ふう}{風}だった、お^{みせ}{店}に

\Alpha ^{からだ}{体}の^{ほう}{方}も^{な}{慣}れちゃったのかもしれない

\page
\Person ども……
\ やってる?

\Alpha あっ、はい!

\Alpha いらっしゃいませ!

\Alpha やっと、なんとか、^{かたち}{形}になったカベではあるんだけど

\Alpha ^{おも}{思}いきって、^{と}{取}っぱらってしまいたい^{ゆうわく}{誘惑}が…
\ むくむくと…

\page
\Alpha あの〜〜
\ ^{くら}{暗}いようでしたら、^{あ}{明}かりを…
\ そのヒモでどうぞ

\Person え?
\ あっ
\ はーい…

\page
\Alpha ^{とお}{遠}くから、しめっぽいにおいがぷ〜んとしてくる

\Alpha やがて、サーッとこまかい^{あめ}{雨}が^{ふ}{降}ってきた

\Alpha お^{きゃく}{客}さんは、^{あ}{明}かりをつけず、^{ふたり}{二人}して、だまって^{そと}{外}を^{み}{見}ている

\page[124]
\Alpha ありがとうございました

\page
\Alpha ぬれた^{いえ}{家}のにおいがする

\Alpha ずっと^{はいおく}{廃屋}の^{なか}{中}で^{わす}{忘}れられていた、^{き}{木}の^{いた}{板}だ

\Alpha また^{あめ}{雨}に^{ぬ}{濡}れるようになって、いろんなことを^{おも}{思}い^{だ}{出}したんだろう

\page
\Alpha ^{いた}{板}をもらってきた^{いえ}{家}にあった^{せいかつ}{生活}

\Alpha よそのうちのにおいがする

\page
\Alpha あたりに、^{みず}{水}なんかないのに

\Alpha どこにいたのか、カエルが、すぐ^{ちか}{近}くで^{な}{鳴}いている

\page
\Alpha ^{いっぴき}{一匹}で、ずっと^{な}{鳴}いている

\page
\Alpha ^{いま}{今}だったら

\page
\Alpha ^{いま}{今}の^{わたし}{私}だったら、「いってらっしゃい」なんてただ^{おく}{送}り^{だ}{出}したり、するだろうか

\page[132]
\Alpha ^{こんねん}{今年}も、あずま^{や}{屋}にしちゃおう


\subsection{第109話\ ^{しょん }{潮}^{ばた}{端}の^{こ}{子}}

\page[137]
\Alpha あー
\ やっぱ
\ ここは^{すず}{凉}しいなあ

\Alpha ^{ふしぎ}{不思議}だよねー
\ ^{く}{来}るたんびに^{おも}{思}うけど

\Takahiro うん

\Alpha とくに^{かぜ}{風}^{どお}{通}りがいいわけでもないのにねえ

\Alpha ^{ひ}{日}なただし

\Takahiro うん

\page
\Alpha ^{まえ}{前}、^{き}{来}たのって、あれ^{にねん}{二年}くらい^{まえ}{前}だっけ

\Alpha もっと^{まえ}{前}だっけ

\Alpha スクーターの^{はなし}{話}したっけね

\Takahiro うん

\page
\Alpha なんか、あっとゆうまだなあ

\Alpha タカヒロ、アクセルあけるのも、あんなに、こわがってたのに

\Takahiro うん

\Alpha それが、もう、ドライブにさそってくれる

\Takahiro うん

\page
\Takahiro アルファ

\Alpha ん?

\Takahiro あれさあ

\Takahiro ^{こんど}{今度}、^{にし}{西}の^{ほう}{方}の^{くに}{国}に^{けんしゅう }{研修}^{うけ}{受}に^{い}{行}くんだ

\Alpha ありゃま

\Takahiro このへんじゃ^{しごと}{仕事}の^{ちしき}{知識}ってゆうか……
\ ^{げんかい}{限界}あるでしょ

\Takahiro いい^{きかい}{機会}だと^{おも}{思}って

\Alpha そっか!
\ すごいね〜〜
\ おとなだね〜〜
\ そっか〜〜〜

\page
\Takahiro いちおう^{すこ}{少}し^{なや}{悩}んだんだけど

\Takahiro ^{き}{決}めちゃったよ

\Alpha うん
\ いいと^{おも}{思}う

\Alpha そっか……

\Alpha じゃあ
\ ^{かえ}{帰}ってきた^{とき}{時}は^{いちにんまえ}{一人前}だ

\Takahiro うん

\page
\Takahiro でも、^{かえ}{帰}ってこないと^{おも}{思}う

\Alpha え?

\Takahiro たぶん……
\ あっちの^{にんげん}{人間}になっちゃうと^{おも}{思}う

\page
\Alpha ^{にし}{西}の^{ほう}{方}って
\ どこ?

\Takahiro ^{ふじ}{富士}^{さん}{山}よかむこうだよ

\Alpha ^{とお}{遠}いね

\Takahiro うん

\page
\Alpha いつ
\ ^{い}{行}くの?

\Takahiro あさって
\ きぬがさに、あっちのバスが^{く}{来}るんだ

\Alpha ^{きゅう}{急}だね

\Takahiro ^{き}{決}めたら、すぐ^{うご}{動}かないと

\Takahiro たぶん、ずっと^{うご}{動}かないと^{おも}{思}うから……

\Alpha うん……
\ ^{わたし}{私}も、そう^{おも}{思}う

\Alpha そっか!

\page
\Alpha さびしくなるなあ

\Takahiro できるだけ……
\ ヒマみつけて^{く}{来}るようにするよ

\Alpha うん

\page
\Takahiro ぐ

\Alpha なんてゆうかさ……

\Alpha ^{ま}{待}つことだけな^{とくい}{得意}ワザだから
\ ねえちゃん

\Alpha ^{ほんき}{本気}で、^{ま}{待}っちゃうよ

\page
\Takahiro うん


\subsection{第110話\ ふたりの^{ふね}{船}}

\page[153]
\Makki ば・

\Makki はあ

\Makki およがないの?

\Takahiro んー
\ あした、^{あさ}{朝}^{はえ}{早}えからよ

\Takahiro ^{たいりょく }{体力}^{おんぞん}{温存}だな

\Makki じじくせー!

\page
\Makki どかな?
\ こんどの^{みずぎ}{水着}

\Takahiro んー?
\ いいんじゃねえの?

\Makki そっか

\Takahiro まだ^{じかん}{時間}あんぞ
\ ^{およ}{泳}いできな

\Makki うん

\page
\Makki はあ

\Takahiro て^{め}{べ}え

\page[157]
\Takahiro もう
\ いいのか?

\Takahiro そろそろ、^{なみ}{波}^{で}{出}てくんぞ

\Makki うん
\ もういい

\Makki こんどさあ、いつ、こっちくんの?

\Takahiro ん?
\ ん〜〜……
\ わかんねえな

\page
\Makki ^{かえ}{帰}って
\ こない?

\Takahiro わかんねえな

\Makki わたしもさ!
\ いっしょについてってやろうか!

\Takahiro ばーか

\Takahiro ^{ときどき}{時々}もどってくんよ

\page
\Makki わたしも

\Makki そのうち、どっか^{い}{行}っちゃうかもしんない

\Takahiro あに
\ やりてえ^{こと}{事}あんのか?

\Makki うん……
\ いくつか

\Takahiro そうか

\Makki いつか^{あそ}{遊}びに^{い}{行}くよ

\Takahiro おお^{こ}{来}い
\ ^{お}{落}ち^{つ}{着}いたらな

\page
\Takahiro そろそろ^{かえ}{帰}んかー

\Makki やだ

\page
\Makki ちっちゃいとき

\Makki おぼえてる?

\Takahiro あん?

\Makki そこに、すわりたい

\Takahiro ^{あか}{赤}んぼかよ

\page
\Takahiro こうやってみんと、よくわかんな

\Takahiro マッキは^{てあし}{手足}が^{なが}{長}くなった

\Makki 13^{さい}{歳}だもん

\page
\Takahiro ねちゃってもいいぞ

\Makki うん


\subsection{たまに^{み}{見}るきみたち}
\Person ^{つき}{月}イチくらいで^{みせ}{店}に^{く}{来}る

\Person ^{ふじいろ}{藤色}の^{かみ}{髪}のおとなしい^{かん}{感}じのあのこ

\Person ^{かのじょ}{彼女}は^{ぼく}{僕}のこと…

\Person 3^{ねん}{年}に^{いちど}{一度}くらい^{みせ}{店}に^{く}{来}る

\Person タマ^{むし}{虫}^{いろ}{色}の^{かみ}{髪}の^{わら}{笑}^{かお}{顔}のかわいいあのひと

\Person ^{かのじょ}{彼女}はオレのこと

\Person どう^{おも}{思}ってんのかなあ

\Shiba あんたら、だれだっけ

\Shiba えーと
