\section{Volume 14}

\subsection{^{だい}{第}131^{わ}{話}\ air}

\page[6]
\M あいっ
\ ごぶさた

\A おわっ!!

\A マッキちゃん!

\M あ
\ ドアのベル^{か}{替}えたんだね

\page
\M はい、おみやげ

\M ^{まめ}{豆}!
\ コロンビアとコナとフォルモサ

\A うお!

\A ありがと

\A や〜〜
\ ^{くるま}{車}の^{おと}{音}でさあ
\ ^{まる}{丸}^{こ}{子}さんかと^{おも}{思}ったよ

\M あの^{くるま}{車}ね、^{けっきょく}{結局}、^{わたし}{私}が^{か}{買}っちゃったんだ

\M マルコさん^{か}{買}ってもあんまし^{の}{乗}んなかったし

\A そっか

\A ^{げんき}{元気}?

\M うん

\page
\A それ、^{せいふく}{制服}だよね

\M うん、^{き}{着}て^{き}{来}ちゃった

\M ^{はいたつ}{配達}^{よう}{用}の^{ふく}{服}……

\M ココネさんともよくペア^{く}{組}むよ

\A そっか

\page
\M マルコさんはねえ

\M ^{いま}{今}、なんか、イラストのバイトしてる

\A ありゃ
\ ココネの^{かいしゃ}{会社}には、^{けっきょく}{結局}、^{い}{行}かなかったんだ

\M うん、あの^{ひと}{人}なりのプライドが、なんかあるみたい

\A あー

\M ^{かいしゃ}{会社}にシバさんってゆうすごい^{ひと}{人}がいてねー

\A あっ
\ ^{し}{知}ってるー

\page[12]
\M ^{きょう}{今日}はうちで^{ね}{寝}て

\M ^{あした}{明日}の^{あさ}{朝}、^{ちょく}{直}で^{かいしゃ}{会社}に^{い}{行}かなきゃ

\A ありゃま

\A また^{き}{来}なね

\M うん
\ ガソリン^{だい}{代}^{かいしゃ}{会社}^{も}{持}ちなんだ

\A あ、いいな

\page
\A タカヒロから^{れんらく}{連絡}ある?

\M ん?
\ うん、たまにね……

\M ^{こんど}{今度}、^{やす}{休}みとって^{くるま}{車}で^{はままつ}{浜松}まで^{い}{行}ってみるんだ

\A おわ

\page
\M じゃ

\A いってらっしゃい

\page
\A さて
\ コーヒー^{まめ}{豆}で^{あそ}{遊}ぶか!


\subsection{第132話\ ^{み}{見}て、^{ある}{歩}き、よろこぶ^{もの}{者}}

\page[20]
\A お^{ま}{待}たせしました!

\S ううん

\S そんなに^{ま}{待}ってないわよ

\A どうも
\ さきほどは

\S わるいわね、^{とお}{遠}くまで^{こ}{来}させちゃって

\A こんな^{とちゅう}{途中}じゃなくても、^{せんせい}{先生}の^{ところ}{所}まで^{い}{行}きますよぉ

\S たまにはね、^{ひろ}{広}い^{ところ}{所}で……

\page
\A そですね

\A あっ、これ
\ さっき^{い}{言}ってた、うちの^{くさか}{草刈}り^{き}{機}です

\A しょぼいですよ

\S ううん
\ このくらいがちょうどいいわ

\page
\S あの、おっさんは^{げんき}{元気}?

\S スタンド、^{むじん}{無人}^{しき}{式}になったでしょ

\A あいかわらずですよ

\A もっぱら^{はたけ}{畑}やってます

\A トマトとか、オクラなんか、^{う}{売}るほど^{つく}{作}って

\S あら、やっぱり

\S ^{わたし}{私}もね

\S ^{さいきん}{最近}、^{うご}{動}くようにしてるのよ

\page
\S うちの^{なか}{中}で、お^{ちゃ}{茶}ばっかり^{の}{飲}んでたから

\S ^{ちゃいろ}{茶色}くなってきちゃったわ

\A はは〜

\S ^{くるま}{車}は……
\ もうちょっと^{か}{借}りとこうかな

\S ^{べつ}{別}に、いいわよね

\A いいと^{おも}{思}います

\page[25]
\S アルファさん

\A はい?

\page
\S あなたがいて、よかったわ

\S ^{ほんと}{本当}に

\A は?

\page
\A なんですかー
\ いきなり!

\A ^{わたし}{私}の^{ほう}{方}こそ、^{せんせい}{先生}がいなきゃ、もう^{し}{死}んでますよ!

\A ^{かみなり}{雷}とか・

\S あらら

\S そうじゃないのよ

\S ^{わたし}{私}はね、^{きも}{気持}ちがいいの

\S ^{わたし}{私}が^{いちばん}{一番}^{たの}{楽}しんでるのよ

\page
\S これ、あげるわ

\A え

\page
\A そんな

\A これ、^{せんせい}{先生}が^{がくせい}{学生}の^{とき}{時}に

\S うん

\S もらってくれたら、うれしいなあ

\S ^{おおむかし}{大昔}に、^{あそ}{遊}びで^{つく}{作}ったモノ、もらっても^{こま}{困}るかもしれないけどね

\S ^{わたし}{私}の^{わか}{若}い^{ころ}{頃}の

\S ^{め}{目}と^{あし}{足}なの

\page
\S ^{みらい}{未来}へ

\S ^{つ}{連}れて^{い}{行}って

\S ^{わたし}{私}は…

\S ^{いま}{今}は、そばにある^{もの}{物}をちゃんと^{み}{見}なきゃ

\page
\S ^{あし}{足}も、あるし

\A わかりました

\A ^{わたし}{私}が^{い}{行}ける^{ところ}{所}までいっしょに……

\S ありがと


\subsection{第133話\ ^{うみ}{海}の^{しゅう}{衆}}

\page[36]
\K ふたりとも、^{ふつう}{普通}じゃないですね……

\R ^{にんげん}{人間}じゃないね

\page
\A はあ

\A ^{からだ}{体}おもい・・

\M ふたりは^{およ}{泳}がないんですか?

\K ん〜〜
\ ^{わたし}{私}、このくらいの^{ところ}{所}で、ぱちゃぱちゃしてる^{ほう}{方}がいいかな

\M ココネさん^{けっこう}{結構}^{およ}{泳}ぐじゃないですかー

\K ううん、^{ぜんぜん}{全然}……

\M いや〜〜
\ アルファさんが^{ぜんりょく}{全力}で^{およ}{泳}ぐとこ^{はじ}{初}めて^{み}{見}たけど

\M ^{ぎょらい}{魚雷}だねー!

\page
\M またはアザラシ!

\K う〜〜

\M ^{まる}{丸}^{こ}{子}さーん

\M ^{はや}{早}く^{およ}{泳}がないと、もう^{なみ}{波}^{で}{出}てきますよー

\R あー
\ うん
\ おかまいなくー

\page
\M ^{およ}{泳}がないんですか?

\R これで^{じゅうぶん}{充分}……

\page
\R また、^{き}{来}たいねえ

\M ^{き}{来}ましょうよ

\M いつかまた


\subsection{第134話\ ラジオ}

\page[46]
\AM もう、^{なんどめ}{何度目}の^{きた}{北}^{こうろ}{航路}になるだろう

\AM ^{ちじょう}{地上}はますます^{しず}{静}かになっている

\page
\AM また、いくつかの^{まち}{街}の^{ひ}{灯}が^{き}{消}えてしまった

\AM ^{ひと}{人}の^{あかり}{灯}が^{き}{消}えたあとしばらくすると

\AM ^{まち}{町}や^{みち}{道}をなぞるように、^{あお}{青}い^{ひ}{灯}が^{あらわ}{現}れることがある

\page
\AM ^{よる}{夜}の^{ちじょう}{地上}は、もう、まっ^{あお}{青}に^{ひか}{光}っている

\AM ^{した}{下}では、みんな、どんな^{め}{目}にあっているのだろうか

\page
\AM ^{こうだい}{広大}な^{くに}{国}ほど^{はや}{早}く^{ちんもく}{沈黙}してしまった

\AM ^{わたし}{私}の^{そだ}{育}ったあの^{くに}{国}も…

\AM でも

\AM コンパクトにまとまった^{ちいき}{地域}からは、まだ

\AM わりと、^{ひと}{人}の^{けはい}{気配}が^{つた}{伝}わってくる

\AM ^{わたし}{私}が^{う}{生}まれたあの^{しま}{島}からも

\AM ^{のうてんき}{脳天気}なラジオの^{おんがく}{音楽}がよく^{なが}{流}れてくる

\page[54]
\AM ^{あめ}{雨}のにおい、^{しお}{潮}のにおい

\AM ^{たきび}{焚火}のにおい……

\AM アスファルトのにおい、^{さかな}{魚}のにおい

\AM ^{はいき}{排気}ガスのにおい……!

\page
\AM もし、^{ちじょう}{地上}が^{あお}{青}い^{ひかり}{光}だけになってしまったら

\AM ^{わたし}{私}は

\AM あそこに^{い}{行}ってもいいだろうか

\AM そうなってほしいような

\AM ほしくないような


\subsection{第135話\ CAFE\ ALPHA}

\page[58]
\N あの^{みせ}{店}へ^{い}{行}く

\page
\N ^{くさ}{草}のこんもりしげったガソリンスタンドを^{よこ}{横}に^{み}{見}て

\Sign セルフ

\N ^{いぜん}{以前}よりも、さらにガタガタにせまくなった^{みち}{道}を^{ある}{歩}く

\page
\N ^{とお}{遠}くに^{ちい}{小}さく、^{しろ}{白}く^{ひか}{光}る^{いえ}{家}が、「カフェアルファ」だ

\N ^{ぜんたい}{全体}^{てき}{的}に、^{くさ}{草}がもりあがった^{こと}{事}^{いがい}{以外}は、
^{いぜん}{以前}と^{まった}{全}く^{おな}{同}じようだが

\N なんとなく、^{かん}{感}じが^{ちが}{違}うような^{き}{気}も…

\page
\N ああ、^{かざみ}{風見}の^{さかな}{魚}が^{あたら}{新}しくなっている

\N ベルも…

\page
\A あっ
\ いらっしゃいませ

\A おわっ!

\A お^{ひさ}{久}しぶり!

\N ^{じゅう}{十}なん^{ねん}{年}ぶりで、^{ようぼう}{容貌}も^{か}{変}わった^{わたし}{私}だが、
^{かのじょ}{彼女}はすぐにわかったようだ

\A ごぶさたです

\page
\A ^{いま}{今}、^{ぎゅうにゅう}{牛乳}がないんですよ

\A ^{きょう}{今日}は^{とうにゅう}{豆乳}^{い}{入}りなんですけど…

\N と^{い}{言}いながら、^{かのじょ}{彼女}はコーヒーを^{ふた}{2}つ^{も}{持}ってきて

\N いつも、^{とうぜん}{当然}のように^{まえ}{前}の^{せき}{席}にすわる

\page[65]
\N ^{はなし}{話}が^{はじ}{始}まると、^{かのじょ}{彼女}は、しばらく^{と}{止}まらない

\N ^{ぜんりょく}{全力}で^{はな}{話}す

\N もう

\N しゃべる、しゃべる

\page
\N ^{わら}{笑}って、おこって

\N なにやら、ぶつくさ^{い}{言}う

\page
\N ^{て}{手}が、とまり

\page
\N ^{なに}{何}か、じいっと^{かんが}{考}えこみ

\page
\N また、^{わら}{笑}う

\page
\A ありがとうございましたー

\A また、いらしてくださいね

\page
\N ^{ひ}{日}が^{みじか}{短}くなった…
\ ^{そら}{空}は、まだ^{あか}{明}るいが、あたりは^{かげ}{影}の^{いろ}{色}だ

\N また10^{ねん}{年}^{ご}{後}、^{わたし}{私}は、あの^{みせ}{店}に^{い}{行}くと^{おも}{思}う

\N その^{とき}{時}、まだ^{わたし}{私}がいればだが…

\N たとえ20^{ねん}{年}^{ご}{後}になっても、たぶん^{かのじょ}{彼女}は、^{わたし}{私}をおぼえているだろう

\page
\N そういう^{みせ}{店}だ


\subsection{第136話\ ^{たか}{鷹}^{つ}{津}ココネ}

\page[74]
\K おまたせー

\page
\K なんだ
\ けっきょく^{ろちゅう}{路駐}だあ

\SH ^{ちゅうしゃじょう }{駐車場}^{し}{閉}まってたよ
\ いいよ、ここで、どうせ^{くるま}{車}^{こ}{来}ないし

\K そうだね

\SH うん
\ おお、やっぱ、カドの^{みせ}{店}、やってたんだ

\page
\K うん、コロッケパンとヤキソバパン
\ どっちがいい?

\SH ん〜〜
\ どっちでもいいなあ

\K じゃ、わたしコロッケー

\K じゃ、はい
\ ヤキソバ

\K あと、これ…
\ コーヒー^{ぎゅうにゅう}{牛乳}しかなかった

\SH おー
\ あっただけ^{じょうとう}{上等}

\K ^{すわ}{座}っちゃおよ

\SH あ、うん

\page[78]
\SH マッキちゃんさ
\ よくやってるよね

\K うん

\K もう^{いちにんまえ}{一人前}だよねー

\SH ちょっとガサツだけどねー

\SH ^{せいかく}{性格}がいい

\K うん

\page
\SH ココネ
\ ^{わたし}{私}さあ

\SH ^{らいげつ}{来月}から、^{かんり}{管理}^{ぶ}{部}の^{かた}{方}に^{いどう}{異動}になった

\K えーーっ!!

\SH とゆうワケだから…

\SH これからは「おてもと^{びん}{便}」マッキちゃんと^{ふたり}{二人}でお^{ねが}{願}い

\K え〜〜

\SH ^{うつ}{移}るったって、200メートル^{くらい}{位}しか^{はな}{離}れてないし

\SH ^{きゅうじつ}{休日}とか^{おな}{同}じなんだから、いつでも^{あ}{会}えるよ

\K そうだけどさ〜〜

\page
\SH ^{わたしたち}{私達}さあ、^{なが}{長}いよねー
\ ^{なんねん}{何年}になる?

\K ^{じゅう}{十}…
\ ^{ながい}{長い}よね……

\page
\SH ^{よ}{良}くも^{わる}{悪}くも、ココネはずうっと^{にゅうしゃ}{入社}^{とうじ}{当時}のまんまだなあ

\SH ふんい^{き}{気}

\K あ〜〜…
\ そこは、やっぱ、^{たいしつ}{体質}の^{げんかい}{限界}ってのも、あってー…

\SH まあ
\ はっきり^{い}{言}ってうらやましー

\K う〜〜ん

\page
\SH ごめんね!
\ イジりたくなるんだよ、ココネカタいから

\K うー

\page
\K シバちゃん
\ ほんとは、^{わたし}{私}、ずうっとこうして^{い}{生}きていきたい

\SH うん

\page
\K シバちゃん
\ これからも、^{ときどき}{時々}こうやって^{あ}{会}おうね

\SH うん

\SH ^{だいじょうぶ}{大丈夫}!
\ ^{ひるやす}{昼休}みだって^{あ}{会}えるよ

\SH ^{べつ}{別}に^{とお}{遠}い^{ところ}{所}へ^{い}{行}っちゃうわけじゃなし

\K うん

\page[86]
\SH さっ
\ ^{ごご}{午後}の^{ぶん}{分}、やっちゃおっか!

\SH ん〜

\K うん

\SH あとは、^{すく}{少}ないから、バッとやって

\SH 4^{じ}{時}ごろまでどっかで、お^{ちゃ}{茶}^{の}{飲}んでよう

\K うん


\subsection{第137話\ みんなのふね}

\page[92]
\M アルファさんへ

\page
\M バタバタしてたけど、とりあえず、おちつきました

\M タカヒロはいまだに^{わたし}{私}をジャマ^{もの}{者}あつかいしてるけど

\M ^{お}{追}い^{だ}{出}す^{ようす}{様子}もないみたいなので、やっかいになろうと^{おも}{思}います

\page
\M まあ「^{き}{来}てみろ」と^{い}{言}ったのはタカヒロの^{ほう}{方}だから

\M しばらく^{いすわ}{居座}るつもりです

\M ただのいそうろうはイヤなので

\M こっちで^{しごと}{仕事}をさがしはじめました

\page
\M ^{はままつ}{浜松}は、とんでもなくでかい^{まち}{町}です
\ ^{ひと}{人}がうじゃうじゃいます

\M こっちの^{ひと}{人}によれば、^{とうごく}{東国}エリアは^{のはら}{野原}のイメージしかないそうです

\M ^{わたし}{私}たち、イナカ^{もの}{者}だったんですよ!

\page
\M タカヒロは、ご^{きんじょ}{近所}に^{わたし}{私}のことを^{いもうと}{妹}だって^{い}{言}ってるみたいです

\M ^{わら}{笑}わせてくれますね

\page
\M このあいだは、^{そうべつ}{送別}^{かい}{会}に^{き}{来}てくれてありがとう

\M ムサシノ^{うんそう}{運送}は、まあ、^{きゅうりょう}{給料}は^{やす}{安}いけど、
いい^{しごと}{仕事}^{ば}{場}でした

\M 5^{ねんかん}{年間}つとめて、やっと^{ちゅうけん}{中堅}って^{かん}{感}じになってきたとこだけど

\M ^{まよ}{迷}ったけど

\page
\M アルファさんにも^{お}{押}っぺされて、^{はままつ}{浜松}に^{き}{来}てみて

\M まずは^{よ}{良}かったと^{おも}{思}います

\M アルファさんも、ヒマ^{とき}{時}(いつもか)こっちに^{あそ}{遊}びに^{き}{来}てください

\M ^{ま}{待}ってますよ

\page
\M それじゃ、またね

\M マッキ

\page
\M ^{ついしん}{追伸}

\M ^{せんじつ}{先日}、タカヒロ^{たち}{達}が^{ひこう}{飛行}^{き}{機}を^{いっき}{一機}、
^{ご}{五}、^{ろく}{六}^{き}{機}^{ぶん}{分}の^{ぶひん}{部品}からでっちあげました

\M ^{こんど}{今度}、^{かながわ}{神奈川}まで、^{と}{飛}ぶんだけど、たぶん、タカヒロが^{い}{行}くことになります

\page
\M そしたら、^{わたし}{私}も、^{いっしょ}{一緒}に^{い}{行}けると^{おも}{思}います

\M ^{とうじ}{冬至}の^{ひ}{日}のお^{ひる}{昼}すぎに

\M ^{だい}{台}の^{はら}{原}のお^{やしろ}{社}に^{き}{来}てください

\M そこを^{めじるし}{目印}にして

\page
\M ^{ふたり}{二人}で、^{と}{飛}んで^{い}{行}きます


\subsection{第138話\ ^{めざ}{目覚}める^{ひと}{人}}

\page[108]
\M ^{なつ}{夏}はさあ、なにかと、ここに^{き}{来}たんだよ

\A ふ〜ん

\page
\A ^{わたし}{私}は、めったに^{こ}{来}ないなあ、ここ……

\M だろうね

\A ^{そと}{外}でゴハン^{た}{食}べるんなら、^{すなはま}{砂浜}の^{ほう}{方}がよかったんじゃない?

\M まあね

\M でも、^{いま}{今}、^{す}{住}んでる^{ところ}{所}がひらけた^{うみ}{海}ばっかだし

\M ^{わたし}{私}がぱっと^{おも}{思}う、^{うみ}{海}っていったら、やっぱここかな

\A そか

\page
\M アルファさんはさあ

\M ずっとここにいんの?

\A ん?

\M どっか、ほかの^{くに}{国}とか……
\ ^{まち}{街}でくらしたいって^{おも}{思}ったりしない?

\A ん〜〜

\A ときどき、どっか^{い}{行}きたくなるけどね……

\A でも、もどってくると^{おも}{思}うよ

\M ふうん

\page
\M ^{しょうじき}{正直}、お^{みせ}{店}きついでしょ

\A すでに、やってけません

\A でも、やめないよ

\A ^{かたち}{形}だけになってもね

\A あのね、けっこうバカにしたもんじゃないよ

\A ^{いま}{今}だってお^{きゃく}{客}さん^{ふつか}{二日}にひとりは^{く}{来}るんだから〜

\M それは^{はんじょう}{繁盛}?

\page
\A サエちゃんは?
\ まだトイレ?

\M うん

\M そのへん、まわってくるって

\M こうゆう、ゴチャッとした^{はやし}{林}^{す}{好}きなんだ、あの^{こ}{子}

\page
\M ”^{ち}{血}”かなあ

\A ^{ひとり}{一人}で^{へいき}{平気}?
\ ^{かのじょ}{彼女}

\M ^{だいじょうぶ}{大丈夫}!
\ サルと^{おな}{同}じ

\A サエッタってかわいい^{なまえ}{名前}だね
\ どうゆう^{いみ}{意味}?

\M しらない
\ タカヒロがね、なんか^{おも}{思}いつきで

\page
\A サエちゃん、^{いま}{今}ってみたらさあ

\A ^{むかし}{昔}のあんた^{たち}{達}とまるっきり^{おな}{同}じ!

\A だして2で^{わ}{割}らないかんじ

\M かーもね〜

\page
\M さっ
\ そろそろ、^{つ}{連}れ^{もど}{戻}すかあ!

\M アルファさん、ここで^{ま}{待}っててね

\A うん

\page[117]
\M サエ〜〜

\page
\M あ,サエ
\ ^{かえ}{帰}るよ

\SA おかあちゃん

\M どうした?

\SA はだかの^{おんな}{女}のひとがね…

\SA ニーって、わらって、それから

\SA ^{うえ}{上}に^{と}{飛}んでっちゃった

\page
\M そっか
\ ^{だいじょうぶ}{大丈夫}

\M こわい^{ひと}{人}じゃないよ

\page
\SA おねえちゃんも^{あ}{会}ったの?!

\A ん〜〜
\ まだなんだ


\subsection{第139話\ ^{ゆうなぎ}{夕凪}^{つうしん}{通信}}

\page[123]
\A ^{そと}{外}^{ある}{歩}きをしています

\A まわりには^{だれ}{誰}もいません

\page
\A うちまでつながる^{ほそ}{細}い^{みち}{道}は、^{くさ}{草}の^{なか}{中}に、^{いま}{今}にも、
^{き}{消}えそうになりながらもしぶとく^{かたち}{形}をとどめて

\A くっきりと

\A ^{まえ}{前}にも^{ま}{増}して、^{しろ}{白}く^{ひか}{光}るようになりました

\page
\A たまに

\A タカヒロ^{たち}{達}から^{てがみ}{手紙}が^{き}{来}ます

\A サエちゃんは、ますますサル^{ど}{度}にみがきがかかってるらしいです

\page[135]
\A ^{わたし}{私}は

\A ^{いま}{今}も

\A いっぱい^{み}{見}て、いっぱい^{ある}{歩}いてます

\page
\A オーナー

\A ^{み}{見}てくれてますか


\subsection{第140話\ ^{さいしゅうかい}{最終回}\ ヨコハマ^{か}{買}い^{だ}{出}し^{きこう}{紀行}}

\page[140]
\A コーヒー^{まめ}{豆}を^{か}{買}いに^{い}{行}きます

\A めったに^{ひと}{人}が^{とお}{通}らない^{うみぞ}{海沿}いの^{みち}{道}は、
ほとんど^{すな}{砂}だらけで、あっちこっち、^{なみ}{波}にかじられてるけど

\A できる^{かぎ}{限}り^{うみ}{海}の^{ちか}{近}くを^{はし}{走}ります

\page
\A ん〜

\page
\A ^{よこはま}{横浜}まで^{い}{行}く^{みち}{道}は、もう^{かんぜん}{完全}に、
^{やまみち}{山道}のルートだけになってしまいました

\A ^{じゅうたくがい}{住宅街}の^{だんだん}{段々}が^{つづ}{続}く^{みち}{道}、
^{もと}{元}ハイキングコースの^{おね}{尾根}の^{みち}{道}

\page
\A ^{とお}{通}る^{ひと}{人}もほとんどいないのは、^{ほか}{他}の^{みち}{道}も^{おな}{同}じなのに

\A このルートだけは、なぜか^{き}{消}える^{けはい}{気配}があいません

\page[145]
\A あの〜〜

\A コーヒー^{まめ}{豆}ください

\P あい

\A ごぶさたです
\ お^{げんき}{元気}ですか?

\A えへへ

\P あい

\R あ!
\ じいさん

\R いいよー
\ ^{わたし}{私}が^{で}{出}るから!

\A あ

\P あい

\R ^{すわ}{座}ってなよー

\page
\R よ!

\Sign ^{まめ}{豆}
\ ^{よこはま}{横浜}^{こうよう}{紅葉}^{やま}{山}

\R ひさしぶりー

\A うん

\A ^{よこはま}{横浜}もさすがに^{ひとで}{人出}が^{すく}{少}なくなったよね

\R まあねえ

\R でも、^{しなもの}{品物}はまだ^{りゅうつう}{流通}してるからね、ここ

\R ^{とうぶん}{当分}の^{あいだ}{間}は^{いすわ}{居座}れそうかな〜

\R はい!
\ かなりおまけしといた

\A うお!
\ ありがと

\page
\R ^{こんど}{今度}、^{あそ}{遊}びで^{き}{来}なよ

\A うん!
\ ^{まる}{丸}^{こ}{子}さんもうち^{き}{来}てよね

\Sign ^{まめ}{豆}\ たばこ\ ^{しお}{塩}

\page
\A ^{ゆうがた}{夕方}を^{かん}{感}じて、
^{ひ}{日}かげの”^{がいとう}{街灯}の^{き}{木}”が^{あお}{青}く^{ひか}{光}りはじめる

\A ^{ひと}{人}の^{けはい}{気配}にふり^{かえ}{返}る^{とき}{時}

\A そこには、よく^{がいとう}{街灯}の^{き}{木}や、^{しろ}{白}い^{ひと}{人}^{かた}{型}キノコがあります

\A そこは、かつての^{みち}{道}や^{ひろば}{広場}の^{あと}{跡}

\A ^{むかし}{昔}から^{ひとびと}{人々}がたたずんだ^{ばしょ}{場所}

\A ^{じめん}{地面}がおぼえてる
\ 「^{ひと}{人}の^{きおく}{記憶}」です

\page
\A ^{わたし}{私}の^{ばしょ}{場所}はカフェアルファ

\A ^{わたし}{私}の^{み}{見}てきたこと

\A みんなのこと

\page
\A ずっと

\A ^{わす}{忘}れないよ

\A お^{まつ}{祭}りのようだった^{よ}{世}の^{なか}{中}が

\A ゆっくりと、おちついてきた

\A あのころのこと

\page
\A のちに、^{ゆうなぎ}{夕凪}の^{じだい}{時代}と^{よ}{呼}ばれる
\ てろてろの^{じかん}{時間}

\A つかの^{ま}{間}のひととき

\A ご^{あんない}{案内}しましょう

\A ^{よる}{夜}が^{く}{来}る^{まえ}{前}に

\A まだあったかいコンクリートにすわって

\K おかえりなさい

\page
\A ただいま!

\A ^{ひと}{人}の^{よる}{夜}が

\A やすらかな^{じだい}{時代}でありますように
