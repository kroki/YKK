\section{Volume 14}

\subsection{^{だい}{第}131^{わ}{話}\ air}

\page[6]
\Makki あいっ
\ ごぶさた

\Alpha おわっ!!

\Alpha マッキちゃん!

\Makki あ
\ ドアのベル^{か}{替}えたんだね

\page[7]
\Makki はい、おみやげ

\Makki ^{まめ}{豆}!
\ コロンビアとコナとフォルモサ

\Alpha うお!

\Alpha ありがと

\Alpha や〜〜
\ ^{くるま}{車}の^{おと}{音}でさあ
\ ^{まる}{丸}^{こ}{子}さんかと^{おも}{思}ったよ

\Makki あの^{くるま}{車}ね、^{けっきょく}{結局}、^{わたし}{私}が^{か}{買}っちゃったんだ

\Makki マルコさん^{か}{買}ってもあんまし^{の}{乗}んなかったし

\Alpha そっか

\Alpha ^{げんき}{元気}?

\Makki うん

\page[8]
\Alpha それ、^{せいふく}{制服}だよね

\Makki うん、^{き}{着}て^{き}{来}ちゃった

\Makki ^{はいたつ}{配達}^{よう}{用}の^{ふく}{服}……

\Makki ココネさんともよくペア^{く}{組}むよ

\Alpha そっか

\page[9]
\Makki マルコさんはねえ

\Makki ^{いま}{今}、なんか、イラストのバイトしてる

\Alpha ありゃ
\ ココネの^{かいしゃ}{会社}には、^{けっきょく}{結局}、^{い}{行}かなかったんだ

\Makki うん、あの^{ひと}{人}なりのプライドが、なんかあるみたい

\Alpha あー

\Makki ^{かいしゃ}{会社}にシバさんってゆうすごい^{ひと}{人}がいてねー

\Alpha あっ
\ ^{し}{知}ってるー

\page[12]
\Makki ^{きょう}{今日}はうちで^{ね}{寝}て

\Makki ^{あした}{明日}の^{あさ}{朝}、^{ちょく}{直}で^{かいしゃ}{会社}に^{い}{行}かなきゃ

\Alpha ありゃま

\Alpha また^{き}{来}なね

\Makki うん
\ ガソリン^{だい}{代}^{かいしゃ}{会社}^{も}{持}ちなんだ

\Alpha あ、いいな

\page[13]
\Alpha タカヒロから^{れんらく}{連絡}ある?

\Makki ん?
\ うん、たまにね……

\Makki ^{こんど}{今度}、^{やす}{休}みとって^{くるま}{車}で^{はままつ}{浜松}まで^{い}{行}ってみるんだ

\Alpha おわ

\page[14]
\Makki じゃ

\Alpha いってらっしゃい

\page[15]
\Alpha さて
\ コーヒー^{まめ}{豆}で^{あそ}{遊}ぶか!


\subsection{第132話\ ^{み}{見}て、^{ある}{歩}き、よろこぶ^{もの}{者}}

\page[20]
\Alpha お^{ま}{待}たせしました!

\Sensei ううん

\Sensei そんなに^{ま}{待}ってないわよ

\Alpha どうも
\ さきほどは

\Sensei わるいわね、^{とお}{遠}くまで^{こ}{来}させちゃって

\Alpha こんな^{とちゅう}{途中}じゃなくても、^{せんせい}{先生}の^{ところ}{所}まで^{い}{行}きますよぉ

\Sensei たまにはね、^{ひろ}{広}い^{ところ}{所}で……

\page[21]
\Alpha そですね

\Alpha あっ、これ
\ さっき^{い}{言}ってた、うちの^{くさか}{草刈}り^{き}{機}です

\Alpha しょぼいですよ

\Sensei ううん
\ このくらいがちょうどいいわ

\page[22]
\Sensei あの、おっさんは^{げんき}{元気}?

\Sensei スタンド、^{むじん}{無人}^{しき}{式}になったでしょ

\Alpha あいかわらずですよ

\Alpha もっぱら^{はたけ}{畑}やってます

\Alpha トマトとか、オクラなんか、^{う}{売}るほど^{つく}{作}って

\Sensei あら、やっぱり

\Sensei ^{わたし}{私}もね

\Sensei ^{さいきん}{最近}、^{うご}{動}くようにしてるのよ

\page[23]
\Sensei うちの^{なか}{中}で、お^{ちゃ}{茶}ばっかり^{の}{飲}んでたから

\Sensei ^{ちゃいろ}{茶色}くなってきちゃったわ

\Alpha はは〜

\Sensei ^{くるま}{車}は……
\ もうちょっと^{か}{借}りとこうかな

\Sensei ^{べつ}{別}に、いいわよね

\Alpha いいと^{おも}{思}います

\page[25]
\Sensei アルファさん

\Alpha はい?

\page[26]
\Sensei あなたがいて、よかったわ

\Sensei ^{ほんとう}{本当}に

\Alpha は?

\page[27]
\Alpha なんですかー
\ いきなり!

\Alpha ^{わたし}{私}の^{ほう}{方}こそ、^{せんせい}{先生}がいなきゃ、もう^{し}{死}んでますよ!

\Alpha ^{かみなり}{雷}とか・

\Sensei あらら

\Sensei そうじゃないのよ

\Sensei ^{わたし}{私}はね、^{きも}{気持}ちがいいの

\Sensei ^{わたし}{私}が^{いちばん}{一番}^{たの}{楽}しんでるのよ

\page[28]
\Sensei これ、あげるわ

\Alpha え

\page[29]
\Alpha そんな

\Alpha これ、^{せんせい}{先生}が^{がくせい}{学生}の^{とき}{時}に

\Sensei うん

\Sensei もらってくれたら、うれしいなあ

\Sensei ^{おおむかし}{大昔}に、^{あそ}{遊}びで^{つく}{作}ったモノ、もらっても^{こま}{困}るかもしれないけどね

\Sensei ^{わたし}{私}の^{わか}{若}い^{ころ}{頃}の

\Sensei ^{め}{目}と^{あし}{足}なの

\page[30]
\Sensei ^{みらい}{未来}へ

\Sensei ^{つ}{連}れて^{い}{行}って

\Sensei ^{わたし}{私}は…

\Sensei ^{いま}{今}は、そばにある^{もの}{物}をちゃんと^{み}{見}なきゃ

\page[31]
\Sensei ^{あし}{足}も、あるし

\Alpha わかりました

\Alpha ^{わたし}{私}が^{い}{行}ける^{ところ}{所}までいっしょに……

\Sensei ありがと


\subsection{第133話\ ^{うみ}{海}の^{しゅう}{衆}}

\page[36]
\Kokone ふたりとも、^{ふつう}{普通}じゃないですね……

\Maruko ^{にんげん}{人間}じゃないね

\page[37]
\Alpha はあ

\Alpha ^{からだ}{体}おもい・・

\Makki ふたりは^{およ}{泳}がないんですか?

\Kokone ん〜〜
\ ^{わたし}{私}、このくらいの^{ところ}{所}で、ぱちゃぱちゃしてる^{ほう}{方}がいいかな

\Makki ココネさん^{けっこう}{結構}^{およ}{泳}ぐじゃないですかー

\Kokone ううん、^{ぜんぜん}{全然}……

\Makki いや〜〜
\ アルファさんが^{ぜんりょく}{全力}で^{およ}{泳}ぐとこ^{はじ}{初}めて^{み}{見}たけど

\Makki ^{ぎょらい}{魚雷}だねー!

\page[38]
\Makki またはアザラシ!

\Kokone う〜〜

\Makki ^{まる}{丸}^{こ}{子}さーん

\Makki ^{はや}{早}く^{およ}{泳}がないと、もう^{なみ}{波}^{で}{出}てきますよー

\Maruko あー
\ うん
\ おかまいなくー

\page[39]
\Makki ^{およ}{泳}がないんですか?

\Maruko これで^{じゅうぶん}{充分}……

\page[40]
\Maruko また、^{き}{来}たいねえ

\Makki ^{き}{来}ましょうよ

\Makki いつかまた


\subsection{第134話\ ラジオ}

\page[46]
\ASevenMOne もう、^{なんどめ}{何度目}の^{きた}{北}^{こうろ}{航路}になるだろう

\ASevenMOne ^{ちじょう}{地上}はますます^{しず}{静}かになっている

\page[47]
\ASevenMOne また、いくつかの^{まち}{街}の^{ひ}{灯}が^{き}{消}えてしまった

\ASevenMOne ^{ひと}{人}の^{あかり}{灯}が^{き}{消}えたあとしばらくすると

\ASevenMOne ^{まち}{町}や^{みち}{道}をなぞるように、^{あお}{青}い^{ひ}{灯}が^{あらわ}{現}れることがある

\page[48]
\ASevenMOne ^{よる}{夜}の^{ちじょう}{地上}は、もう、まっ^{あお}{青}に^{ひか}{光}っている

\ASevenMOne ^{した}{下}では、みんな、どんな^{め}{目}にあっているのだろうか

\page[49]
\ASevenMOne ^{こうだい}{広大}な^{くに}{国}ほど^{はや}{早}く^{ちんもく}{沈黙}してしまった

\ASevenMOne ^{わたし}{私}の^{そだ}{育}ったあの^{くに}{国}も…

\ASevenMOne でも

\ASevenMOne コンパクトにまとまった^{ちいき}{地域}からは、まだ

\ASevenMOne わりと、^{ひと}{人}の^{けはい}{気配}が^{つた}{伝}わってくる

\ASevenMOne ^{わたし}{私}が^{う}{生}まれたあの^{しま}{島}からも

\ASevenMOne ^{のうてんき}{脳天気}なラジオの^{おんがく}{音楽}がよく^{なが}{流}れてくる

\page[54]
\ASevenMOne ^{あめ}{雨}のにおい、^{しお}{潮}のにおい

\ASevenMOne ^{たきび}{焚火}のにおい……

\ASevenMOne アスファルトのにおい、^{さかな}{魚}のにおい

\ASevenMOne ^{はいき}{排気}ガスのにおい……!

\page[55]
\ASevenMOne もし、^{ちじょう}{地上}が^{あお}{青}い^{ひかり}{光}だけになってしまったら

\ASevenMOne ^{わたし}{私}は

\ASevenMOne あそこに^{い}{行}ってもいいだろうか

\ASevenMOne そうなってほしいような

\ASevenMOne ほしくないような


\subsection{第135話\ CAFE\ ALPHA}

\page[58]
\Narrator あの^{みせ}{店}へ^{い}{行}く

\page[59]
\Narrator ^{くさ}{草}のこんもりしげったガソリンスタンドを^{よこ}{横}に^{み}{見}て

\Sign セルフ

\Narrator ^{いぜん}{以前}よりも、さらにガタガタにせまくなった^{みち}{道}を^{ある}{歩}く

\page[60]
\Narrator ^{とお}{遠}くに^{ちい}{小}さく、^{しろ}{白}く^{ひか}{光}る^{いえ}{家}が、「カフェアルファ」だ

\Narrator ^{ぜんたい}{全体}^{てき}{的}に、^{くさ}{草}がもりあがった^{こと}{事}^{いがい}{以外}は、
^{いぜん}{以前}と^{まった}{全}く^{おな}{同}じようだが

\Narrator なんとなく、^{かん}{感}じが^{ちが}{違}うような^{き}{気}も…

\page[61]
\Narrator ああ、^{かざみ}{風見}の^{さかな}{魚}が^{あたら}{新}しくなっている

\Narrator ベルも…

\page[62]
\Alpha あっ
\ いらっしゃいませ

\Alpha おわっ!

\Alpha お^{ひさ}{久}しぶり!

\Narrator ^{じゅう}{十}なん^{ねん}{年}ぶりで、^{ようぼう}{容貌}も^{か}{変}わった^{わたし}{私}だが、
^{かのじょ}{彼女}はすぐにわかったようだ

\Alpha ごぶさたです

\page[63]
\Alpha ^{いま}{今}、^{ぎゅうにゅう}{牛乳}がないんですよ

\Alpha ^{きょう}{今日}は^{とうにゅう}{豆乳}^{い}{入}りなんですけど…

\Narrator と^{い}{言}いながら、^{かのじょ}{彼女}はコーヒーを^{ふた}{2}つ^{も}{持}ってきて

\Narrator いつも、^{とうぜん}{当然}のように^{まえ}{前}の^{せき}{席}にすわる

\page[65]
\Narrator ^{はなし}{話}が^{はじ}{始}まると、^{かのじょ}{彼女}は、しばらく^{と}{止}まらない

\Narrator ^{ぜんりょく}{全力}で^{はな}{話}す

\Narrator もう

\Narrator しゃべる、しゃべる

\page[66]
\Narrator ^{わら}{笑}って、おこって

\Narrator なにやら、ぶつくさ^{い}{言}う

\page[67]
\Narrator ^{て}{手}が、とまり

\page[68]
\Narrator ^{なに}{何}か、じいっと^{かんが}{考}えこみ

\page[69]
\Narrator また、^{わら}{笑}う

\page[70]
\Alpha ありがとうございましたー

\Alpha また、いらしてくださいね

\page[71]
\Narrator ^{ひ}{日}が^{みじか}{短}くなった…
\ ^{そら}{空}は、まだ^{あか}{明}るいが、あたりは^{かげ}{影}の^{いろ}{色}だ

\Narrator また10^{ねん}{年}^{ご}{後}、^{わたし}{私}は、あの^{みせ}{店}に^{い}{行}くと^{おも}{思}う

\Narrator その^{とき}{時}、まだ^{わたし}{私}がいればだが…

\Narrator たとえ20^{ねん}{年}^{ご}{後}になっても、たぶん^{かのじょ}{彼女}は、^{わたし}{私}をおぼえているだろう

\page[72]
\Narrator そういう^{みせ}{店}だ


\subsection{第136話\ ^{たか}{鷹}^{つ}{津}ココネ}

\page[74]
\Kokone おまたせー

\page[75]
\Kokone なんだ
\ けっきょく^{ろちゅう}{路駐}だあ

\Shiba ^{ちゅうしゃじょう }{駐車場}^{し}{閉}まってたよ
\ いいよ、ここで、どうせ^{くるま}{車}^{こ}{来}ないし

\Kokone そうだね

\Shiba うん
\ おお、やっぱ、カドの^{みせ}{店}、やってたんだ

\page[76]
\Kokone うん、コロッケパンとヤキソバパン
\ どっちがいい?

\Shiba ん〜〜
\ どっちでもいいなあ

\Kokone じゃ、わたしコロッケー

\Kokone じゃ、はい
\ ヤキソバ

\Kokone あと、これ…
\ コーヒー^{ぎゅうにゅう}{牛乳}しかなかった

\Shiba おー
\ あっただけ^{じょうとう}{上等}

\Kokone ^{すわ}{座}っちゃおよ

\Shiba あ、うん

\page[78]
\Shiba マッキちゃんさ
\ よくやってるよね

\Kokone うん

\Kokone もう^{いちにんまえ}{一人前}だよねー

\Shiba ちょっとガサツだけどねー

\Shiba ^{せいかく}{性格}がいい

\Kokone うん

\page[79]
\Shiba ココネ
\ ^{わたし}{私}さあ

\Shiba ^{らいげつ}{来月}から、^{かんり}{管理}^{ぶ}{部}の^{かた}{方}に^{いどう}{異動}になった

\Kokone えーーっ!!

\Shiba とゆうワケだから…

\Shiba これからは「おてもと^{びん}{便}」マッキちゃんと^{ふたり}{二人}でお^{ねが}{願}い

\Kokone え〜〜

\Shiba ^{うつ}{移}るったって、200メートル^{くらい}{位}しか^{はな}{離}れてないし

\Shiba ^{きゅうじつ}{休日}とか^{おな}{同}じなんだから、いつでも^{あ}{会}えるよ

\Kokone そうだけどさ〜〜

\page[80]
\Shiba ^{わたしたち}{私達}さあ、^{なが}{長}いよねー
\ ^{なんねん}{何年}になる?

\Kokone ^{じゅう}{十}…
\ ^{ながい}{長い}よね……

\page[81]
\Shiba ^{よ}{良}くも^{わる}{悪}くも、ココネはずうっと^{にゅうしゃ}{入社}^{とうじ}{当時}のまんまだなあ

\Shiba ふんい^{き}{気}

\Kokone あ〜〜…
\ そこは、やっぱ、^{たいしつ}{体質}の^{げんかい}{限界}ってのも、あってー…

\Shiba まあ
\ はっきり^{い}{言}ってうらやましー

\Kokone う〜〜ん

\page[82]
\Shiba ごめんね!
\ イジりたくなるんだよ、ココネカタいから

\Kokone うー

\page[83]
\Kokone シバちゃん
\ ほんとは、^{わたし}{私}、ずうっとこうして^{い}{生}きていきたい

\Shiba うん

\page[84]
\Kokone シバちゃん
\ これからも、^{ときどき}{時々}こうやって^{あ}{会}おうね

\Shiba うん

\Shiba ^{だいじょうぶ}{大丈夫}!
\ ^{ひるやす}{昼休}みだって^{あ}{会}えるよ

\Shiba ^{べつ}{別}に^{とお}{遠}い^{ところ}{所}へ^{い}{行}っちゃうわけじゃなし

\Kokone うん

\page[86]
\Shiba さっ
\ ^{ごご}{午後}の^{ぶん}{分}、やっちゃおっか!

\Shiba ん〜

\Kokone うん

\Shiba あとは、^{すく}{少}ないから、バッとやって

\Shiba 4^{じ}{時}ごろまでどっかで、お^{ちゃ}{茶}^{の}{飲}んでよう

\Kokone うん


\subsection{第137話\ みんなのふね}

\page[92]
\Makki アルファさんへ

\page[93]
\Makki バタバタしてたけど、とりあえず、おちつきました

\Makki タカヒロはいまだに^{わたし}{私}をジャマ^{もの}{者}あつかいしてるけど

\Makki ^{お}{追}い^{だ}{出}す^{ようす}{様子}もないみたいなので、やっかいになろうと^{おも}{思}います

\page[94]
\Makki まあ「^{き}{来}てみろ」と^{い}{言}ったのはタカヒロの^{ほう}{方}だから

\Makki しばらく^{いすわ}{居座}るつもりです

\Makki ただのいそうろうはイヤなので

\Makki こっちで^{しごと}{仕事}をさがしはじめました

\page[95]
\Makki ^{はままつ}{浜松}は、とんでもなくでかい^{まち}{町}です
\ ^{ひと}{人}がうじゃうじゃいます

\Makki こっちの^{ひと}{人}によれば、^{とうごく}{東国}エリアは^{のはら}{野原}のイメージしかないそうです

\Makki ^{わたし}{私}たち、イナカ^{もの}{者}だったんですよ!

\page[96]
\Makki タカヒロは、ご^{きんじょ}{近所}に^{わたし}{私}のことを^{いもうと}{妹}だって^{い}{言}ってるみたいです

\Makki ^{わら}{笑}わせてくれますね

\page[97]
\Makki このあいだは、^{そうべつ}{送別}^{かい}{会}に^{き}{来}てくれてありがとう

\Makki ムサシノ^{うんそう}{運送}は、まあ、^{きゅうりょう}{給料}は^{やす}{安}いけど、
いい^{しごと}{仕事}^{ば}{場}でした

\Makki 5^{ねんかん}{年間}つとめて、やっと^{ちゅうけん}{中堅}って^{かん}{感}じになってきたとこだけど

\Makki ^{まよ}{迷}ったけど

\page[98]
\Makki アルファさんにも^{お}{押}っぺされて、^{はままつ}{浜松}に^{き}{来}てみて

\Makki まずは^{よ}{良}かったと^{おも}{思}います

\Makki アルファさんも、ヒマ^{とき}{時}(いつもか)こっちに^{あそ}{遊}びに^{き}{来}てください

\Makki ^{ま}{待}ってますよ

\page[99]
\Makki それじゃ、またね

\Makki マッキ

\page[100]
\Makki ^{ついしん}{追伸}

\Makki ^{せんじつ}{先日}、タカヒロ^{たち}{達}が^{ひこう}{飛行}^{き}{機}を^{いっき}{一機}、
^{ご}{五}、^{ろく}{六}^{き}{機}^{ぶん}{分}の^{ぶひん}{部品}からでっちあげました

\Makki ^{こんど}{今度}、^{かながわ}{神奈川}まで、^{と}{飛}ぶんだけど、たぶん、タカヒロが^{い}{行}くことになります

\page[101]
\Makki そしたら、^{わたし}{私}も、^{いっしょ}{一緒}に^{い}{行}けると^{おも}{思}います

\Makki ^{とうじ}{冬至}の^{ひ}{日}のお^{ひる}{昼}すぎに

\Makki ^{だい}{台}の^{はら}{原}のお^{やしろ}{社}に^{き}{来}てください

\Makki そこを^{めじるし}{目印}にして

\page[102]
\Makki ^{ふたり}{二人}で、^{と}{飛}んで^{い}{行}きます


\subsection{第138話\ ^{めざ}{目覚}める^{ひと}{人}}

\page[108]
\Makki ^{なつ}{夏}はさあ、なにかと、ここに^{き}{来}たんだよ

\Alpha ふ〜ん

\page[109]
\Alpha ^{わたし}{私}は、めったに^{こ}{来}ないなあ、ここ……

\Makki だろうね

\Alpha ^{そと}{外}でゴハン^{た}{食}べるんなら、^{すなはま}{砂浜}の^{ほう}{方}がよかったんじゃない?

\Makki まあね

\Makki でも、^{いま}{今}、^{す}{住}んでる^{ところ}{所}がひらけた^{うみ}{海}ばっかだし

\Makki ^{わたし}{私}がぱっと^{おも}{思}う、^{うみ}{海}っていったら、やっぱここかな

\Alpha そか

\page[110]
\Makki アルファさんはさあ

\Makki ずっとここにいんの?

\Alpha ん?

\Makki どっか、ほかの^{くに}{国}とか……
\ ^{まち}{街}でくらしたいって^{おも}{思}ったりしない?

\Alpha ん〜〜

\Alpha ときどき、どっか^{い}{行}きたくなるけどね……

\Alpha でも、もどってくると^{おも}{思}うよ

\Makki ふうん

\page[111]
\Makki ^{しょうじき}{正直}、お^{みせ}{店}きついでしょ

\Alpha すでに、やってけません

\Alpha でも、やめないよ

\Alpha ^{かたち}{形}だけになってもね

\Alpha あのね、けっこうバカにしたもんじゃないよ

\Alpha ^{いま}{今}だってお^{きゃく}{客}さん^{ふつか}{二日}にひとりは^{く}{来}るんだから〜

\Makki それは^{はんじょう}{繁盛}?

\page[112]
\Alpha サエちゃんは?
\ まだトイレ?

\Makki うん

\Makki そのへん、まわってくるって

\Makki こうゆう、ゴチャッとした^{はやし}{林}^{す}{好}きなんだ、あの^{こ}{子}

\page[113]
\Makki ”^{ち}{血}”かなあ

\Alpha ^{ひとり}{一人}で^{へいき}{平気}?
\ ^{かのじょ}{彼女}

\Makki ^{だいじょうぶ}{大丈夫}!
\ サルと^{おな}{同}じ

\Alpha サエッタってかわいい^{なまえ}{名前}だね
\ どうゆう^{いみ}{意味}?

\Makki しらない
\ タカヒロがね、なんか^{おも}{思}いつきで

\page[114]
\Alpha サエちゃん、^{いま}{今}ってみたらさあ

\Alpha ^{むかし}{昔}のあんた^{たち}{達}とまるっきり^{おな}{同}じ!

\Alpha だして2で^{わ}{割}らないかんじ

\Makki かーもね〜

\page[115]
\Makki さっ
\ そろそろ、^{つ}{連}れ^{もど}{戻}すかあ!

\Makki アルファさん、ここで^{ま}{待}っててね

\Alpha うん

\page[117]
\Makki サエ〜〜

\page[118]
\Makki あ,サエ
\ ^{かえ}{帰}るよ

\Sae おかあちゃん

\Makki どうした?

\Sae はだかの^{おんな}{女}のひとがね…

\Sae ニーって、わらって、それから

\Sae ^{うえ}{上}に^{と}{飛}んでっちゃった

\page[119]
\Makki そっか
\ ^{だいじょうぶ}{大丈夫}

\Makki こわい^{ひと}{人}じゃないよ

\page[120]
\Sae おねえちゃんも^{あ}{会}ったの?!

\Alpha ん〜〜
\ まだなんだ


\subsection{第139話\ ^{ゆうなぎ}{夕凪}^{つうしん}{通信}}

\page[123]
\Alpha ^{そと}{外}^{ある}{歩}きをしています

\Alpha まわりには^{だれ}{誰}もいません

\page[124]
\Alpha うちまでつながる^{ほそ}{細}い^{みち}{道}は、^{くさ}{草}の^{なか}{中}に、^{いま}{今}にも、
^{き}{消}えそうになりながらもしぶとく^{かたち}{形}をとどめて

\Alpha くっきりと

\Alpha ^{まえ}{前}にも^{ま}{増}して、^{しろ}{白}く^{ひか}{光}るようになりました

\page[125]
\Alpha たまに

\Alpha タカヒロ^{たち}{達}から^{てがみ}{手紙}が^{き}{来}ます

\Alpha サエちゃんは、ますますサル^{ど}{度}にみがきがかかってるらしいです

\page[135]
\Alpha ^{わたし}{私}は

\Alpha ^{いま}{今}も

\Alpha いっぱい^{み}{見}て、いっぱい^{ある}{歩}いてます

\page[136]
\Alpha オーナー

\Alpha ^{み}{見}てくれてますか


\subsection{第140話\ ^{さいしゅうかい}{最終回}\ ヨコハマ^{か}{買}い^{だ}{出}し^{きこう}{紀行}}

\page[140]
\Alpha コーヒー^{まめ}{豆}を^{か}{買}いに^{い}{行}きます

\Alpha めったに^{ひと}{人}が^{とお}{通}らない^{うみぞ}{海沿}いの^{みち}{道}は、
ほとんど^{すな}{砂}だらけで、あっちこっち、^{なみ}{波}にかじられてるけど

\Alpha できる^{かぎ}{限}り^{うみ}{海}の^{ちか}{近}くを^{はし}{走}ります

\page[141]
\Alpha ん〜

\page[142]
\Alpha ^{よこはま}{横浜}まで^{い}{行}く^{みち}{道}は、もう^{かんぜん}{完全}に、
^{やまみち}{山道}のルートだけになってしまいました

\Alpha ^{じゅうたくがい}{住宅街}の^{だんだん}{段々}が^{つづ}{続}く^{みち}{道}、
^{もと}{元}ハイキングコースの^{おね}{尾根}の^{みち}{道}

\page[143]
\Alpha ^{とお}{通}る^{ひと}{人}もほとんどいないのは、^{ほか}{他}の^{みち}{道}も^{おな}{同}じなのに

\Alpha このルートだけは、なぜか^{き}{消}える^{けはい}{気配}があいません

\page[145]
\Alpha あの〜〜

\Alpha コーヒー^{まめ}{豆}ください

\Person あい

\Alpha ごぶさたです
\ お^{げんき}{元気}ですか?

\Alpha えへへ

\Person あい

\Maruko あ!
\ じいさん

\Maruko いいよー
\ ^{わたし}{私}が^{で}{出}るから!

\Alpha あ

\Person あい

\Maruko ^{すわ}{座}ってなよー

\page[146]
\Maruko よ!

\Sign ^{まめ}{豆}
\ ^{よこはま}{横浜}^{こうよう}{紅葉}^{やま}{山}

\Maruko ひさしぶりー

\Alpha うん

\Alpha ^{よこはま}{横浜}もさすがに^{ひとで}{人出}が^{すく}{少}なくなったよね

\Maruko まあねえ

\Maruko でも、^{しなもの}{品物}はまだ^{りゅうつう}{流通}してるからね、ここ

\Maruko ^{とうぶん}{当分}の^{あいだ}{間}は^{いすわ}{居座}れそうかな〜

\Maruko はい!
\ かなりおまけしといた

\Alpha うお!
\ ありがと

\page[147]
\Maruko ^{こんど}{今度}、^{あそ}{遊}びで^{き}{来}なよ

\Alpha うん!
\ ^{まる}{丸}^{こ}{子}さんもうち^{き}{来}てよね

\Sign ^{まめ}{豆}\ たばこ\ ^{しお}{塩}

\page[148]
\Alpha ^{ゆうがた}{夕方}を^{かん}{感}じて、
^{ひ}{日}かげの”^{がいとう}{街灯}の^{き}{木}”が^{あお}{青}く^{ひか}{光}りはじめる

\Alpha ^{ひと}{人}の^{けはい}{気配}にふり^{かえ}{返}る^{とき}{時}

\Alpha そこには、よく^{がいとう}{街灯}の^{き}{木}や、^{しろ}{白}い^{ひと}{人}^{かた}{型}キノコがあります

\Alpha そこは、かつての^{みち}{道}や^{ひろば}{広場}の^{あと}{跡}

\Alpha ^{むかし}{昔}から^{ひとびと}{人々}がたたずんだ^{ばしょ}{場所}

\Alpha ^{じめん}{地面}がおぼえてる
\ 「^{ひと}{人}の^{きおく}{記憶}」です

\page[149]
\Alpha ^{わたし}{私}の^{ばしょ}{場所}はカフェアルファ

\Alpha ^{わたし}{私}の^{み}{見}てきたこと

\Alpha みんなのこと

\page[150]
\Alpha ずっと

\Alpha ^{わす}{忘}れないよ

\Alpha お^{まつ}{祭}りのようだった^{よ}{世}の^{なか}{中}が

\Alpha ゆっくりと、おちついてきた

\Alpha あのころのこと

\page[151]
\Alpha のちに、^{ゆうなぎ}{夕凪}の^{じだい}{時代}と^{よ}{呼}ばれる
\ てろてろの^{じかん}{時間}

\Alpha つかの^{ま}{間}のひととき

\Alpha ご^{あんない}{案内}しましょう

\Alpha ^{よる}{夜}が^{く}{来}る^{まえ}{前}に

\Alpha まだあったかいコンクリートにすわって

\Kokone おかえりなさい

\page[152]
\Alpha ただいま!

\Alpha ^{ひと}{人}の^{よる}{夜}が

\Alpha やすらかな^{じだい}{時代}でありますように
