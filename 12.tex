\section{Volume 12}

\subsection{^{だい}{第}111^{わ}{話}\ プラグ}

\page[4]
\A ふおあ〜

\A おとといは、いそがしかった

\page
\A も〜〜〜〜
\ ^{ごぜんちゅう}{午前中}から、^{ゆうがた}{夕方}すぎまで、
お^{きゃく}{客}さんのいない^{しゅんかん}{瞬間}がなかった

\A うちの^{みせ}{店}にとっては^{とし}{年}に^{いっかい}{一回}、
あるかないかの^{だい}{大}いそがしだった

\A ^{へん}{変}なカンとパターンが^{み}{身}についてしまった

\A ^{たしょう}{多少}の^{れいがい}{例外}はあるけど

\A お^{きゃく}{客}さんの^{く}{来}る^{ひ}{日}と^{こ}{来}ない^{ひ}{日}がほぼわかる

\page
\A そんなわけで、^{きょう}{今日}は、お^{きゃく}{客}さんの^{こ}{来}ない^{ひ}{日}だ

\A わかる

\A 100パーセントまず^{こ}{来}ない!

\A ^{だんげん}{断言}できる!

\page
\A ^{わたし}{私}がバイクについてできることはオイルを^{こうかん}{交換}することと

\A あとはみがくだけだ

\A なんかエンジンがぐずれば、プラグをみがく

\A いいのか、わるいのか、わかんなくても、プラグをみがく

\A まあ、^{きかい}{機械}オンチだ

\A ^{じつ}{実}は^{ひだり}{左}の^{ものおき}{物置}に^{うご}{動}かない^{くるま}{車}もあるんだけど

\A ^{いま}{今}は^{かんぜん}{完全}に^{もの}{物}の^{した}{下}に^{う}{埋}まっている

\page[10]
\A まあ、なんとか

\A まわってくれるようになったかな

\page
\A ^{みず}{水}の^{なか}{中}の^{でんちゅう}{電柱}、^{いしがき}{石垣}、^{やね}{屋根}

\A まだ^{なま}{生}の^{きおく}{記憶}だけど

\A ^{なが}{流}れてくる^{どしゃ}{土砂}がどんどん^{まち}{町}を^{う}{埋}めていく

\A ^{あたら}{新}しい^{じめん}{地面}ができていく

\page
\A ^{かえ}{帰}る

\page
\A プラグをみがいた

\A かからない

\A かからない

\A ふう

\page
\A ん〜ん

\page[16]
\A エンジンは^{なにごと}{何事}もなかったようにかかった

\A かかる^{き}{気}もした

\A なぜだろう

\page
\A ^{きょう}{今日}はお^{きゃく}{客}さんの100パーセント^{こ}{来}ない^{ひ}{日}

\page
\A ^{あした}{明日}もたぶん^{こ}{来}ないと^{おも}{思}う

\A あさっては^{く}{来}るかな


\subsection{第112話\ ^{むさしの}{武蔵野}^{はら}{原}}

\page[22]
\Sign かんぱち
\ ^{つじ}{辻}の^{ちゃ}{茶}

\K あの

\P は…

\K これに…
\ ムギ^{ちゃ}{茶}をいただきたいんですけど

\P あっ…
\ はい

\page
\P あ……これ
\ ^{いま}{今}、サービス^{ちゅう}{中}なんで……
\ クッキー

\Sign ^{つじ}{辻}の^{ちゃ}{茶}
\ くるみ

\K わっ!

\page
\K ^{こうしゅうかいどう}{甲州街道}をちょっと^{はな}{離}れるともうヤブの^{なか}{中}

\K ^{じもと}{地元}の^{ひと}{人}でなければ、
この^{のはら}{野原}を^{ぶじ}{無事}に^{とお}{通}りぬけることはできません

\page
\K ^{くさ}{草}の^{うみ}{海}

\K ムサシノに^{みやこ}{都}があったころ、ここは、^{いえ}{家}で、^{う}{埋}まっていました

\K ^{しん}{信}じられないけど

\page
\Sign ^{かき}{火気}^{げんきん}{厳禁}

\SH ココネー

\page
\K ごめん
\ ^{ま}{待}った?

\SH んー
\ そんなでもない

\K ^{わたし}{私}とシバちゃんのお^{き}{気}に^{い}{入}りの^{たてもの}{建物}です

\page
\SH ^{うみ}{海}だねー

\K ^{ふね}{船}だねー


\subsection{第113話\ さかな}

\page[32]
\P アルファ

\page
\P ほら、^{み}{見}てごらん

\A あかい

\P そうだね

\page[35]
\P どう?

\A あかい

\P うん……
\ ^{ふく}{服}も^{にあ}{似合}ってる

\A ひひ

\page[37]
\P アルファ

\P リボンでね

\page[39]
\P アルファ

\page
\P ^{いっしょ}{一緒}に^{い}{行}くか?

\A いえ
\ わたしは、うちにいます

\page
\A やりたいことがいっぱいあるんですよー

\A あそこの^{は}{葉}っぱのうらの^{いろ}{色}も、みてないし

\A うちを^{ひだり}{左}まわりで^{いっしゅう}{一周}してみたいし

\A いま、^{まいにち}{毎日}、おもしろいんです

\P ^{たいせつ}{大切}なことだね

\A はい

\page[43]
\A ん〜〜

\A は

\page
\A のど、かわいたな


\subsection{第114話\ はこにわ}

\page[47]
\O おう

\M よ

\A よ

\page
\O おう

\O あんだえー
\ お^{じょう}{嬢}さまじゃんかよ

\A かーわいー

\M ふう

\M どうも

\M でもなんで、こんなかっこすんだろ
\ ふだん^{ぎ}{着}でもいいのに

\A ねー
\ まーいいけど

\page
\O へー
\ ^{おく}{遅}^{れ}{え}てまーかと^{おも}{思}っただけんどよー

\O へー
\ ^{?}{来}やしねえ

\P まあてーげー
\ そんなんだべ

\P  アワくったってしゃーあんもんか

\P まあ
\ ^{は}{晴}れでよかった

\P そーなー

\page[53]
\M タカヒロ

\T うちの^{かぞく}{家族}です

\NA おお

\NA みなさん^{せいそう}{正装}?

\T ……ですね
\ なんでまた

\page
\A タカヒロからの^{てがみ}{手紙}がみんなに^{き}{来}たのは^{みっか}{三日}^{まえ}{前}

\A 「ナイにくっついて^{かお}{顔}を^{み}{見}せに^{い}{行}く」というものでした

\M もう、^{かえ}{帰}っちゃうのかな

\A たぶんまた^{もど}{戻}ってくるよ

\A タカヒロが^{はたら}{働}きに^{い}{行}った「エンジンの^{みやこ}{都}」^{はままつ}{浜松}は、
ナイの^{しごと}{仕事}^{ば}{場}でもあった

\A ^{なんど}{何度}もナイの^{ひこう}{飛行}^{き}{機}を^{み}{見}かけたタカヒロは、
^{ひこう}{飛行}^{じょう}{場}まで^{み}{見}に^{かよ}{通}うようになったそうです

\page[56]
\O おう
\ ハネ^{ふ}{振}ってんわ

\P けえんだな

\page
\P や〜〜〜〜
\ タカもたいした^{おとな}{大人}になったもんだ

\O カッコつけてなー

\A みなさん、やっぱカン^{ちが}{違}いしてます

\page
\NA どうだった?

\NA ^{う}{生}まれた^{ところ }{所}^{そら}{空}から^{み}{見}て

\T はい

\T きれいでした
\ ^{ちい}{小}さくて

\NA そうか

\T はい

\page
\A なんだった?

\M うん

\page
\M いや……
\ ただ^{てがみ}{手紙}とか

\A おお!


\subsection{第115話\ ^{はつひ}{初日}の^{で}{出}}

\page[64]
\A ^{はつひ}{初日}の^{で}{出}を^{み}{見}に^{き}{来}ました

\page[66]
\A タカヒロと^{き}{来}た^{とき}{時}^{いらい}{以来}です

\A ^{こんねん}{今年}の^{はつひ}{初日}は^{こけら}{柿}のような^{いろ}{色}で

\A あの^{とき}{時}の^{しろ}{白}い^{こうせん}{光線}とはまた^{べつ}{別}の^{かん}{感}じです

\page
\M アルファさん

\page
\M よ

\A ありゃ!

\A マッキちゃん
\ いつからいたの?

\M ずっといたよ

\M ^{き}{気}づいたのさっきだけど

\page
\M あけましておめでとう

\A あ
\ おめでとう

\A なに、マッキちゃん
\ ^{ひとり}{一人}で^{き}{来}たの?

\M おじさんとおばさんと

\M ^{さき}{先}に^{かえ}{帰}ったよ
\ アルファさんと^{かえ}{帰}るって^{い}{言}っといた

\A おっ!

\page
\A ^{わたし}{私}、バイクだから

\A スカートだと、めちゃくちゃ^{さむ}{寒}いと^{おも}{思}うけど

\A ^{かくご}{覚悟}するようにー

\M わかった

\page
\M このあいだはどうもね

\A どうもねー

\A あ、^{てがみ}{手紙}
\ なんだって^{い}{言}ってた?
\ タカヒロ

\M いや……
\ なんかあっちの^{まち}{街}のこととか、^{しごと}{仕事}のこととか

\M アルファさんは?

\A おんなじ、おんなじ!
\ ^{きんきょう}{近況}とか……

\page
\M あの
\ アルファさんさあ

\A え?

\M お^{みせ}{店}……
\ ^{ひとで}{人手}^{ぶそく}{不足}とかない?

\A え?
\ ^{ひとで}{人手}?
\ うちの^{みせ}{店}?

\A ^{わたし}{私}ひとりでもう^{あま}{余}ってるくらいだなあ

\M そっか

\page
\A なに?
\ アルバイト?

\M うん

\A おこづかいたんないの?

\M ってゆうか

\M なんか、やりたい

\A おりょ

\A いいね!

\A まー「^{しごと}{仕事}」は15^{さい}{歳}くらいになってからでもいいと^{おも}{思}うけど

\page
\M うん

\A あ、でも、うちで^{すこ}{少}し^{あそ}{遊}んでみる?

\M やとってくれる?!

\A えっ?

\A やとうって
\ ^{きゅうりょう}{給料}は^{で}{出}ません

\M そっか

\A ^{わたし}{私}の^{ぶん}{分}もありません

\M そっか

\page
\A まあ……
\ ^{しごと}{仕事}じゃなくても
\ ^{みせ}{店}、^{み}{見}に^{こ}{来}ない?

\A なんか、おもしろいかもよ

\M うん

\A じゃ、^{けんがく}{見学}に

\M うん

\page
\A まあ、お^{きゃく}{客}さんも^{こ}{来}ないと^{おも}{思}うけどね!

\M それはお^{みせ}{店}?


\subsection{第116話\ ^{まつ}{松}の^{き}{木}の^{した}{下}}

\page[79]
\A こら、こら

\page
\S ^{き}{来}たんなら、ひと^{こと}{言}、^{こえ}{声}かけてよ

\O はあ

\O いや、^{い}{一}ぷきしたら、^{い}{行}ってみんかなって^{おも}{思}ってたんすよ

\S あ、そう

\S ^{いっぽん}{一本}、ちょうだい

\O あ、はい

\page
\O ^{ひとりぐ}{一人暮}らしになってみんとあれっすよ

\O ま〜〜〜
\ しゃべんねえ、しゃべんねえ

\page
\O ま〜〜〜
\ ^{いえ}{家}がまた^{ひれ}{広}え^{ひれ}{広}え

\S ふふん

\S なーに^{い}{言}ってんの

\S ^{わたし}{私}なんてもう、ずーっと^{まえ}{前}から^{まいにち}{毎日}そんな^{じょうたい}{状態}よ

\O あ
\ ああ……
\ そうすね

\page
\S ま
\ ^{な}{慣}れるもんよ

\O はあ

\S ^{な}{慣}れて、どうすんだってのも、あるけどね

\O なんなんすか

\page[85]
\O ^{せんせい}{先生}

\S ん?

\O ^{ひとり}{一人}でいんと、こう……
「やりのこした^{かん}{感}」つうか

\O どっとくることねえすか

\S しょっちゆうよ

\O しょっちゆう

\page
\S やること、^{おお}{多}すぎてさ

\S ^{いっしょう}{一生}って、^{いっかい}{一回}だけじゃ^{た}{足}んないもんなのよ、たぶん

\S だから、せめて

\S ひとつふたつでも、なにか……

\page
\O 「ひとり^{もの}{者}のプロ」って^{かん}{感}じのセリフっすね

\S んー?

\S なんかすごくカチンときたね

\O ほ
\ ほめてんすよ

\page
\S べつにまだ、ひとりっきりになったつもりもないし

\O そうすね

\S ^{なか}{中}でお^{ちゃ}{茶}すすってく?

\O いいすね


\subsection{第117話\ ^{こうど}{高度}1m}

\page[91]
\A はい、いったんストーップ

\M ふ〜

\A ^{しんちょう}{慎重}だねえ

\M つかれる

\page
\A あのね
\ もうちょっとアクセルガバッとひねった^{ほう}{方}が^{らく}{楽}に^{はし}{走}るよ
\ こう……

\M わかるけどさ

\M できれば^{ひとり}{一人}で^{の}{乗}りたいんだけど

\A いちおう、^{はじ}{初}めて^{こうどう}{公道}^{はし}{走}るわけだし

\A まあ、「^{ちょうきょり}{長距離}^{おお}{大}^{にもつ}{荷物}^{はこ}{運}び」の^{れんしゅう}{練習}ってことで

\M それ^{ひつよう}{必要}な^{れんしゅう}{練習}?

\page
\A うん
\ たまに……
\ まれに……

\M ^{あそ}{遊}びたいだけなんでしょ

\A そんなことはないです

\A じゃ、^{い}{行}ってみよっか!

\A こっからは^{いっぽん}{一本}^{みち}{道}だからラクだよ

\M うん

\page
\A なれてきたね

\M うん……
\ いい^{かん}{感}じ

\page
\A マッキちゃんは^{そと}{外}の^{ほう}{方}がいいや、やっぱ

\M え?

\A ^{みせ}{店}じゃなんか、どんよりしてるもんねー

\M ヒマなの!

\M お^{きゃく}{客}さん^{みっか}{三日}で^{ひとり}{一人}じゃん……

\A んー
\ でも^{さんばい}{三杯}のんでた

\page
\A アクセルをあおり^{たいじゅう}{体重}を^{さゆう}{左右}に^{と}{飛}ぶように^{はし}{走}る

\A ^{すこ}{少}しずつ、スクーターとマッキちゃんがまざってくる

\page
\M エンジンの^{かいてん}{回転}と^{ふうあつ}{風圧}と^{けしき}{景色}の
^{なが}{流}れ^{ほう}{方}のちょうどいい^{はや}{速}さを^{み}{見}つけると

\M ハンドルのこともアクセルのことも^{わす}{忘}れてくる

\M だんだんバイクがなくなって

\M まるで、^{じぶん}{自分}の^{からだ}{体}で、
^{ちじょう}{地上}1mの^{くうちゅう}{空中}を^{と}{飛}んでる^{かん}{感}じになってくる

\page
\M ^{はや}{速}すぎず、おそすぎず

\M ^{たいじゅう}{体重}を^{むいしき}{無意識}に^{さゆう}{左右}にかけて、^{みち}{道}をなぞっていく

\M ^{と}{飛}んでいるのに、^{じめん}{地面}の^{て}{手}ざわりさえわかる

\M バイクになってしまう

\page
\M アルファさん

\A ん?

\M わたしもバイク^{か}{買}う

\A おおっ!

\A いつでも^{わたし}{私}の^{か}{貸}すけど

\M うん……
\ ううん

\M ^{じぶん}{自分}のがほしい

\A おおっ

\page
\M あ、でも、やっぱ^{か}{借}りる

\A お


\subsection{第118話\ ^{まち}{町}で}

\page[102]
\A ありがと

\P あい

\page
\R また〜〜

\R やだも〜〜

\R ^{てんちょう}{店長}さん

\R それじゃ、またよろしくお^{ねが}{願}いしますね♡

\P とりあえず3セットな

\P まあ、しようがねえや、マルちゃんの^{たの}{頼}みじゃ

\R やったあ!

\page
\P えーと、ハンコ……

\P はいよ
\ ん?
\ どした?

\R あっ……
\ いえ

\page
\R あ……
\ ありがとうございました♡

\P ん

\R じゃ、^{しつれい}{失礼}しまあす

\P はいよ、おつかれー

\page
\R ふ〜〜

\A こんにちは!

\R よ

\A か〜〜わい〜〜

\R あ〜〜
\ もお〜〜!!

\page
\Sign ところてん

\A ^{いがい}{意外}な^{ところ}{所}で^{あ}{会}ったね

\R ^{さいあく}{最悪}のとこ^{み}{見}られた……

\A んー?

\R なかなか、^{わら}{笑}えたでしょ

\A ^{わら}{笑}いはしないけど

\page
\A ^{べつ}{別}の^{ひと}{人}かと^{おも}{思}った
\ ^{さいしょ}{最初}

\R だよな〜〜

\R あの^{えいぎょう }{営業}^{かお}{顔}でさあ……

\R ^{こうひょう}{好評}なんだー、^{わたし}{私}

\page
\R あっ
\ ココネさんには

\R し〜〜

\R なんか
\ なんとなく

\A ^{い}{言}わないけどさ

\A ^{べつ}{別}に、^{き}{気}にしないで^{おも}{思}うよ、ココネは

\R だよねー

\page
\A ^{まる}{丸}^{こ}{子}さんて

\R ん?

\A ^{まる}{丸}^{こ}{子}さんて、^{なまえ}{名前}はなんて^{い}{言}うの?

\R え?

\A 「^{まる}{丸}^{こ}{子}」って^{みょうじ}{名字}でしょ?

\A まだ^{なまえ}{名前}^{き}{聞}いてないや

\R あ

\R マルコ

\A え?

\page
\R 「^{まる}{丸}^{こ}{子}マルコ」

\R ^{なまえ}{名前}カタカナで

\A ありゃま!

\R 「マルコ」なんて^{おとこ}{男}の^{こ}{子}の^{なまえ}{名前}みたいだけど

\R ^{もと}{元}はひらがなの「まるこ」ってイメージで……

\R ^{ひび}{響}きが^{す}{好}きだったんだ^{とうじ}{当時}

\A ほー

\R だから^{みょうじ}{名字}も「^{まる}{丸}^{こ}{子}」にしちゃったんだねえ

\A え?

\page
\R ^{わたし}{私}ね……
\ ^{もと}{元}のオーナーの^{みょうじ}{名字}なのがイヤで

\R だから^{じぶん}{自分}で^{か}{変}えちゃったんだ

\A え゛っ!

\A え……だっ……
\ そんなこと……

\R できるよ

\R やりたければね

\page
\A ^{まる}{丸}^{こ}{子}さんは
\ オーナーがきらいだったの?

\R きらいってゆうか……
\ まあ、^{じぶん}{自分}なりのケジメかなあ

\R いろいろあったね……

\A ふ〜〜ん……
\ そっか……
\ へ〜〜

\A ^{わたし}{私}のはね
\ オーナーの^{みょうじ}{名字}
\ 「^{はつ}{初}^{せ}{瀬}^{の}{野}」

\R うん
\ ^{し}{知}ってる

\page
\Sign そばぜんざいと^{こ}{小}^{うめ}{梅}

\R ココネさんはねえ……
\ アルファさんのことが、も〜〜
\ ^{だい}{大}^{す}{好}き……

\A ^{わたし}{私}もココネ^{す}{好}き

\page
\R ^{わたし}{私}のココネさん^{す}{好}き

\A うん

\page
\R ナイも^{す}{好}き

\A え゛?!


\subsection{第119話\ ソバゼンザイ}

\page[119]
\A マ……

\page[121]
\A ^{さいきん}{最近}、^{き}{気}がついたこと

\page
\A マッキちゃんは^{まる}{丸}^{こ}{子}さんとちょっと^{に}{似}ている

\A ^{へん}{変}なとこで、やけにつっかかってくる^{ところ}{所}とか

\A そのわりに^{みょう}{妙}にモロいこととか

\page
\M むぇ〜〜
\ ……うぇ?

\page[128]
\A ^{はや}{早}いなあ

\page
\A ^{はや}{早}すぎるよ

\page
\M ごめん
\ ねちゃったよ……

\A あ、んーん

\A ^{だれ}{誰}も^{こ}{来}なかったし……

\page
\M そか

\M ^{べんきょう}{勉強}になんない^{みせ}{店}だなー
\ ここは……

\A ^{こま}{困}ったもんだ

\M なに^{つく}{作}ってんの?

\A そばぜんざい

\page
\M ソバゼンザイ

\M ^{み}{見}ていい?

\A いいよ


\subsection{^{アールアンドピー}{R&P}}

\page[133]
\K ゴール、ここですよー

\A おー

\Sign アルファ
\ 「ウサギー^{ごう}{号}」
\ 8^{ばりき}{馬力}^{きゅう}{級}^{でんどう}{電動}

\Sign ^{まる}{丸}^{こ}{子}
\ 「はと^{まる}{丸}」
\ 8^{ばりき}{馬力}^{きゅう}{級}^{でんどう}{電動}

\page[136]
\R どっち^{か}{勝}った?!

\A わかんない!!

\K うー


\subsection{第120話\ ^{こえ}{声}}

\page[138]
\A きぬがさまで^{か}{買}い^{もの}{物}に^{い}{行}く

\A きぬがさはこの^{へん}{辺}で、^{ゆいいつ}{唯一}の^{まち}{町}っぽい^{まち}{町}だ

\page[140]
\A ぬるい^{くうき}{空気}

\A いろんなにおいのする
\ コンソメ^{いろ}{色}の^{ゆうがた}{夕方}

\A ^{ちょく}{直}に^{かえ}{帰}ったりはしない

\A 「^{きた}{北}の^{まち}{町}」に^{い}{行}ってみようと^{おも}{思}う

\page
\A ^{きた}{北}の^{まち}{町}への^{みち}{道}は^{むかし}{昔}からの^{おおどお}{大通}り

\A いつも、^{ひと}{人}の^{けはい}{気配}はあるのに、^{すがた}{姿}はあまり^{み}{見}ない

\A ^{むじん}{無人}の^{のはら}{野原}よりも^{しず}{静}かだ

\page[143]
\A ^{きた}{北}の^{まち}{町}

\A いつか^{せんせい}{先生}と^{き}{来}た

\page
\A ^{きた}{北}の^{まち}{町}はもともと^{よこすか}{横須賀}と^{よ}{呼}ばれていた

\A ^{さか}{坂}の^{した}{下}の^{まち}{街}が、
^{うみ}{海}にひたった^{すがた}{姿}は^{よこはま}{横浜}と^{に}{似}ている

\A でも、「^{よこすか}{横須賀}」の^{なまえ}{名前}は、なぜか^{やま}{山}の^{うえ}{上}まで、
^{も}{持}って^{く}{来}られることはなかった

\page
\A ヨコスカの^{まち}{街}は^{なまえ}{名前}といっしょに^{うみ}{海}の^{そこ}{底}にある

\page[147]
\A ^{からだ}{体}の^{おく}{奥}から

\page
\A ^{こえ}{声}が^{で}{出}てくるので

\A ^{おも}{思}いきり、^{おお}{大}きな^{こえ}{声}を^{だ}{出}した



\subsection{アルファのかお\ 10^{ねん}{年}アルバム}

\A ^{じゅうねん}{十年}ひとむかしー

\A ^{じゅうねんまえ}{十年前}は、^{だいたい}{大体}こんな^{かん}{感}じでした

\A ^{いま}{今}は、こんなです

\A ^{さいきん}{最近}じゃ、こんなです
