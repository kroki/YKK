\section{Volume 12}

\subsection{^{だい}{第}111^{わ}{話}\ プラグ}

\page[4]
\Alpha ふおあ〜

\Alpha おとといは、いそがしかった

\page
\Alpha も〜〜〜〜
\ ^{ごぜんちゅう}{午前中}から、^{ゆうがた}{夕方}すぎまで、
お^{きゃく}{客}さんのいない^{しゅんかん}{瞬間}がなかった

\Alpha うちの^{みせ}{店}にとっては^{とし}{年}に^{いっかい}{一回}、
あるかないかの^{だい}{大}いそがしだった

\Alpha ^{へん}{変}なカンとパターンが^{み}{身}についてしまった

\Alpha ^{たしょう}{多少}の^{れいがい}{例外}はあるけど

\Alpha お^{きゃく}{客}さんの^{く}{来}る^{ひ}{日}と^{こ}{来}ない^{ひ}{日}がほぼわかる

\page
\Alpha そんなわけで、^{きょう}{今日}は、お^{きゃく}{客}さんの^{こ}{来}ない^{ひ}{日}だ

\Alpha わかる

\Alpha 100パーセントまず^{こ}{来}ない!

\Alpha ^{だんげん}{断言}できる!

\page
\Alpha ^{わたし}{私}がバイクについてできることはオイルを^{こうかん}{交換}することと

\Alpha あとはみがくだけだ

\Alpha なんかエンジンがぐずれば、プラグをみがく

\Alpha いいのか、わるいのか、わかんなくても、プラグをみがく

\Alpha まあ、^{きかい}{機械}オンチだ

\Alpha ^{じつ}{実}は^{ひだり}{左}の^{ものおき}{物置}に^{うご}{動}かない^{くるま}{車}もあるんだけど

\Alpha ^{いま}{今}は^{かんぜん}{完全}に^{もの}{物}の^{した}{下}に^{う}{埋}まっている

\page[10]
\Alpha まあ、なんとか

\Alpha まわってくれるようになったかな

\page
\Alpha ^{みず}{水}の^{なか}{中}の^{でんちゅう}{電柱}、^{いしがき}{石垣}、^{やね}{屋根}

\Alpha まだ^{なま}{生}の^{きおく}{記憶}だけど

\Alpha ^{なが}{流}れてくる^{どしゃ}{土砂}がどんどん^{まち}{町}を^{う}{埋}めていく

\Alpha ^{あたら}{新}しい^{じめん}{地面}ができていく

\page
\Alpha ^{かえ}{帰}る

\page
\Alpha プラグをみがいた

\Alpha かからない

\Alpha かからない

\Alpha ふう

\page
\Alpha ん〜ん

\page[16]
\Alpha エンジンは^{なにごと}{何事}もなかったようにかかった

\Alpha かかる^{き}{気}もした

\Alpha なぜだろう

\page
\Alpha ^{きょう}{今日}はお^{きゃく}{客}さんの100パーセント^{こ}{来}ない^{ひ}{日}

\page
\Alpha ^{あした}{明日}もたぶん^{こ}{来}ないと^{おも}{思}う

\Alpha あさっては^{く}{来}るかな


\subsection{第112話\ ^{むさしの}{武蔵野}^{はら}{原}}

\page[22]
\Sign かんぱち
\ ^{つじ}{辻}の^{ちゃ}{茶}

\Kokone あの

\Person は…

\Kokone これに…
\ ムギ^{ちゃ}{茶}をいただきたいんですけど

\Person あっ…
\ はい

\page
\Person あ……これ
\ ^{いま}{今}、サービス^{ちゅう}{中}なんで……
\ クッキー

\Sign ^{つじ}{辻}の^{ちゃ}{茶}
\ くるみ

\Kokone わっ!

\page
\Kokone ^{こうしゅうかいどう}{甲州街道}をちょっと^{はな}{離}れるともうヤブの^{なか}{中}

\Kokone ^{じもと}{地元}の^{ひと}{人}でなければ、
この^{のはら}{野原}を^{ぶじ}{無事}に^{とお}{通}りぬけることはできません

\page
\Kokone ^{くさ}{草}の^{うみ}{海}

\Kokone ムサシノに^{みやこ}{都}があったころ、ここは、^{いえ}{家}で、^{う}{埋}まっていました

\Kokone ^{しん}{信}じられないけど

\page
\Sign ^{かき}{火気}^{げんきん}{厳禁}

\Shiba ココネー

\page
\Kokone ごめん
\ ^{ま}{待}った?

\Shiba んー
\ そんなでもない

\Kokone ^{わたし}{私}とシバちゃんのお^{き}{気}に^{い}{入}りの^{たてもの}{建物}です

\page
\Shiba ^{うみ}{海}だねー

\Kokone ^{ふね}{船}だねー


\subsection{第113話\ さかな}

\page[32]
\Person アルファ

\page
\Person ほら、^{み}{見}てごらん

\Alpha あかい

\Person そうだね

\page[35]
\Person どう?

\Alpha あかい

\Person うん……
\ ^{ふく}{服}も^{にあ}{似合}ってる

\Alpha ひひ

\page[37]
\Person アルファ

\Person リボンでね

\page[39]
\Person アルファ

\page
\Person ^{いっしょ}{一緒}に^{い}{行}くか?

\Alpha いえ
\ わたしは、うちにいます

\page
\Alpha やりたいことがいっぱいあるんですよー

\Alpha あそこの^{は}{葉}っぱのうらの^{いろ}{色}も、みてないし

\Alpha うちを^{ひだり}{左}まわりで^{いっしゅう}{一周}してみたいし

\Alpha いま、^{まいにち}{毎日}、おもしろいんです

\Person ^{たいせつ}{大切}なことだね

\Alpha はい

\page[43]
\Alpha ん〜〜

\Alpha は

\page
\Alpha のど、かわいたな


\subsection{第114話\ はこにわ}

\page[47]
\Ojisan おう

\Makki よ

\Alpha よ

\page
\Ojisan おう

\Ojisan あんだえー
\ お^{じょう}{嬢}さまじゃんかよ

\Alpha かーわいー

\Makki ふう

\Makki どうも

\Makki でもなんで、こんなかっこすんだろ
\ ふだん^{ぎ}{着}でもいいのに

\Alpha ねー
\ まーいいけど

\page
\Ojisan へー
\ ^{おく}{遅}^{れ}{え}てまーかと^{おも}{思}っただけんどよー

\Ojisan へー
\ ^{?}{来}やしねえ

\Person まあてーげー
\ そんなんだべ

\Person  アワくったってしゃーあんもんか

\Person まあ
\ ^{は}{晴}れでよかった

\Person そーなー

\page[53]
\Makki タカヒロ

\Takahiro うちの^{かぞく}{家族}です

\Nai おお

\Nai みなさん^{せいそう}{正装}?

\Takahiro ……ですね
\ なんでまた

\page
\Alpha タカヒロからの^{てがみ}{手紙}がみんなに^{き}{来}たのは^{みっか}{三日}^{まえ}{前}

\Alpha 「ナイにくっついて^{かお}{顔}を^{み}{見}せに^{い}{行}く」というものでした

\Makki もう、^{かえ}{帰}っちゃうのかな

\Alpha たぶんまた^{もど}{戻}ってくるよ

\Alpha タカヒロが^{はたら}{働}きに^{い}{行}った「エンジンの^{みやこ}{都}」^{はままつ}{浜松}は、
ナイの^{しごと}{仕事}^{ば}{場}でもあった

\Alpha ^{なんど}{何度}もナイの^{ひこう}{飛行}^{き}{機}を^{み}{見}かけたタカヒロは、
^{ひこう}{飛行}^{じょう}{場}まで^{み}{見}に^{かよ}{通}うようになったそうです

\page[56]
\Ojisan おう
\ ハネ^{ふ}{振}ってんわ

\Person けえんだな

\page
\Person や〜〜〜〜
\ タカもたいした^{おとな}{大人}になったもんだ

\Ojisan カッコつけてなー

\Alpha みなさん、やっぱカン^{ちが}{違}いしてます

\page
\Nai どうだった?

\Nai ^{う}{生}まれた^{ところ }{所}^{そら}{空}から^{み}{見}て

\Takahiro はい

\Takahiro きれいでした
\ ^{ちい}{小}さくて

\Nai そうか

\Takahiro はい

\page
\Alpha なんだった?

\Makki うん

\page
\Makki いや……
\ ただ^{てがみ}{手紙}とか

\Alpha おお!


\subsection{第115話\ ^{はつひ}{初日}の^{で}{出}}

\page[64]
\Alpha ^{はつひ}{初日}の^{で}{出}を^{み}{見}に^{き}{来}ました

\page[66]
\Alpha タカヒロと^{き}{来}た^{とき}{時}^{いらい}{以来}です

\Alpha ^{こんねん}{今年}の^{はつひ}{初日}は^{こけら}{柿}のような^{いろ}{色}で

\Alpha あの^{とき}{時}の^{しろ}{白}い^{こうせん}{光線}とはまた^{べつ}{別}の^{かん}{感}じです

\page
\Makki アルファさん

\page
\Makki よ

\Alpha ありゃ!

\Alpha マッキちゃん
\ いつからいたの?

\Makki ずっといたよ

\Makki ^{き}{気}づいたのさっきだけど

\page
\Makki あけましておめでとう

\Alpha あ
\ おめでとう

\Alpha なに、マッキちゃん
\ ^{ひとり}{一人}で^{き}{来}たの?

\Makki おじさんとおばさんと

\Makki ^{さき}{先}に^{かえ}{帰}ったよ
\ アルファさんと^{かえ}{帰}るって^{い}{言}っといた

\Alpha おっ!

\page
\Alpha ^{わたし}{私}、バイクだから

\Alpha スカートだと、めちゃくちゃ^{さむ}{寒}いと^{おも}{思}うけど

\Alpha ^{かくご}{覚悟}するようにー

\Makki わかった

\page
\Makki このあいだはどうもね

\Alpha どうもねー

\Alpha あ、^{てがみ}{手紙}
\ なんだって^{い}{言}ってた?
\ タカヒロ

\Makki いや……
\ なんかあっちの^{まち}{街}のこととか、^{しごと}{仕事}のこととか

\Makki アルファさんは?

\Alpha おんなじ、おんなじ!
\ ^{きんきょう}{近況}とか……

\page
\Makki あの
\ アルファさんさあ

\Alpha え?

\Makki お^{みせ}{店}……
\ ^{ひとで}{人手}^{ぶそく}{不足}とかない?

\Alpha え?
\ ^{ひとで}{人手}?
\ うちの^{みせ}{店}?

\Alpha ^{わたし}{私}ひとりでもう^{あま}{余}ってるくらいだなあ

\Makki そっか

\page
\Alpha なに?
\ アルバイト?

\Makki うん

\Alpha おこづかいたんないの?

\Makki ってゆうか

\Makki なんか、やりたい

\Alpha おりょ

\Alpha いいね!

\Alpha まー「^{しごと}{仕事}」は15^{さい}{歳}くらいになってからでもいいと^{おも}{思}うけど

\page
\Makki うん

\Alpha あ、でも、うちで^{すこ}{少}し^{あそ}{遊}んでみる?

\Makki やとってくれる?!

\Alpha えっ?

\Alpha やとうって
\ ^{きゅうりょう}{給料}は^{で}{出}ません

\Makki そっか

\Alpha ^{わたし}{私}の^{ぶん}{分}もありません

\Makki そっか

\page
\Alpha まあ……
\ ^{しごと}{仕事}じゃなくても
\ ^{みせ}{店}、^{み}{見}に^{こ}{来}ない?

\Alpha なんか、おもしろいかもよ

\Makki うん

\Alpha じゃ、^{けんがく}{見学}に

\Makki うん

\page
\Alpha まあ、お^{きゃく}{客}さんも^{こ}{来}ないと^{おも}{思}うけどね!

\Makki それはお^{みせ}{店}?


\subsection{第116話\ ^{まつ}{松}の^{き}{木}の^{した}{下}}

\page[79]
\Alpha こら、こら

\page
\Sensei ^{き}{来}たんなら、ひと^{こと}{言}、^{こえ}{声}かけてよ

\Ojisan はあ

\Ojisan いや、^{い}{一}ぷきしたら、^{い}{行}ってみんかなって^{おも}{思}ってたんすよ

\Sensei あ、そう

\Sensei ^{いっぽん}{一本}、ちょうだい

\Ojisan あ、はい

\page
\Ojisan ^{ひとりぐ}{一人暮}らしになってみんとあれっすよ

\Ojisan ま〜〜〜
\ しゃべんねえ、しゃべんねえ

\page
\Ojisan ま〜〜〜
\ ^{いえ}{家}がまた^{ひれ}{広}え^{ひれ}{広}え

\Sensei ふふん

\Sensei なーに^{い}{言}ってんの

\Sensei ^{わたし}{私}なんてもう、ずーっと^{まえ}{前}から^{まいにち}{毎日}そんな^{じょうたい}{状態}よ

\Ojisan あ
\ ああ……
\ そうすね

\page
\Sensei ま
\ ^{な}{慣}れるもんよ

\Ojisan はあ

\Sensei ^{な}{慣}れて、どうすんだってのも、あるけどね

\Ojisan なんなんすか

\page[85]
\Ojisan ^{せんせい}{先生}

\Sensei ん?

\Ojisan ^{ひとり}{一人}でいんと、こう……
「やりのこした^{かん}{感}」つうか

\Ojisan どっとくることねえすか

\Sensei しょっちゆうよ

\Ojisan しょっちゆう

\page
\Sensei やること、^{おお}{多}すぎてさ

\Sensei ^{いっしょう}{一生}って、^{いっかい}{一回}だけじゃ^{た}{足}んないもんなのよ、たぶん

\Sensei だから、せめて

\Sensei ひとつふたつでも、なにか……

\page
\Ojisan 「ひとり^{もの}{者}のプロ」って^{かん}{感}じのセリフっすね

\Sensei んー?

\Sensei なんかすごくカチンときたね

\Ojisan ほ
\ ほめてんすよ

\page
\Sensei べつにまだ、ひとりっきりになったつもりもないし

\Ojisan そうすね

\Sensei ^{なか}{中}でお^{ちゃ}{茶}すすってく?

\Ojisan いいすね


\subsection{第117話\ ^{こうど}{高度}1m}

\page[91]
\Alpha はい、いったんストーップ

\Makki ふ〜

\Alpha ^{しんちょう}{慎重}だねえ

\Makki つかれる

\page
\Alpha あのね
\ もうちょっとアクセルガバッとひねった^{ほう}{方}が^{らく}{楽}に^{はし}{走}るよ
\ こう……

\Makki わかるけどさ

\Makki できれば^{ひとり}{一人}で^{の}{乗}りたいんだけど

\Alpha いちおう、^{はじ}{初}めて^{こうどう}{公道}^{はし}{走}るわけだし

\Alpha まあ、「^{ちょうきょり}{長距離}^{おお}{大}^{にもつ}{荷物}^{はこ}{運}び」の^{れんしゅう}{練習}ってことで

\Makki それ^{ひつよう}{必要}な^{れんしゅう}{練習}?

\page
\Alpha うん
\ たまに……
\ まれに……

\Makki ^{あそ}{遊}びたいだけなんでしょ

\Alpha そんなことはないです

\Alpha じゃ、^{い}{行}ってみよっか!

\Alpha こっからは^{いっぽん}{一本}^{みち}{道}だからラクだよ

\Makki うん

\page
\Alpha なれてきたね

\Makki うん……
\ いい^{かん}{感}じ

\page
\Alpha マッキちゃんは^{そと}{外}の^{ほう}{方}がいいや、やっぱ

\Makki え?

\Alpha ^{みせ}{店}じゃなんか、どんよりしてるもんねー

\Makki ヒマなの!

\Makki お^{きゃく}{客}さん^{みっか}{三日}で^{ひとり}{一人}じゃん……

\Alpha んー
\ でも^{さんばい}{三杯}のんでた

\page
\Alpha アクセルをあおり^{たいじゅう}{体重}を^{さゆう}{左右}に^{と}{飛}ぶように^{はし}{走}る

\Alpha ^{すこ}{少}しずつ、スクーターとマッキちゃんがまざってくる

\page
\Makki エンジンの^{かいてん}{回転}と^{ふうあつ}{風圧}と^{けしき}{景色}の
^{なが}{流}れ^{ほう}{方}のちょうどいい^{はや}{速}さを^{み}{見}つけると

\Makki ハンドルのこともアクセルのことも^{わす}{忘}れてくる

\Makki だんだんバイクがなくなって

\Makki まるで、^{じぶん}{自分}の^{からだ}{体}で、
^{ちじょう}{地上}1mの^{くうちゅう}{空中}を^{と}{飛}んでる^{かん}{感}じになってくる

\page
\Makki ^{はや}{速}すぎず、おそすぎず

\Makki ^{たいじゅう}{体重}を^{むいしき}{無意識}に^{さゆう}{左右}にかけて、^{みち}{道}をなぞっていく

\Makki ^{と}{飛}んでいるのに、^{じめん}{地面}の^{て}{手}ざわりさえわかる

\Makki バイクになってしまう

\page
\Makki アルファさん

\Alpha ん?

\Makki わたしもバイク^{か}{買}う

\Alpha おおっ!

\Alpha いつでも^{わたし}{私}の^{か}{貸}すけど

\Makki うん……
\ ううん

\Makki ^{じぶん}{自分}のがほしい

\Alpha おおっ

\page
\Makki あ、でも、やっぱ^{か}{借}りる

\Alpha お


\subsection{第118話\ ^{まち}{町}で}

\page[102]
\Alpha ありがと

\Person あい

\page
\Maruko また〜〜

\Maruko やだも〜〜

\Maruko ^{てんちょう}{店長}さん

\Maruko それじゃ、またよろしくお^{ねが}{願}いしますね♡

\Person とりあえず3セットな

\Person まあ、しようがねえや、マルちゃんの^{たの}{頼}みじゃ

\Maruko やったあ!

\page
\Person えーと、ハンコ……

\Person はいよ
\ ん?
\ どした?

\Maruko あっ……
\ いえ

\page
\Maruko あ……
\ ありがとうございました♡

\Person ん

\Maruko じゃ、^{しつれい}{失礼}しまあす

\Person はいよ、おつかれー

\page
\Maruko ふ〜〜

\Alpha こんにちは!

\Maruko よ

\Alpha か〜〜わい〜〜

\Maruko あ〜〜
\ もお〜〜!!

\page
\Sign ところてん

\Alpha ^{いがい}{意外}な^{ところ}{所}で^{あ}{会}ったね

\Maruko ^{さいあく}{最悪}のとこ^{み}{見}られた……

\Alpha んー?

\Maruko なかなか、^{わら}{笑}えたでしょ

\Alpha ^{わら}{笑}いはしないけど

\page
\Alpha ^{べつ}{別}の^{ひと}{人}かと^{おも}{思}った
\ ^{さいしょ}{最初}

\Maruko だよな〜〜

\Maruko あの^{えいぎょう }{営業}^{かお}{顔}でさあ……

\Maruko ^{こうひょう}{好評}なんだー、^{わたし}{私}

\page
\Maruko あっ
\ ココネさんには

\Maruko し〜〜

\Maruko なんか
\ なんとなく

\Alpha ^{い}{言}わないけどさ

\Alpha ^{べつ}{別}に、^{き}{気}にしないで^{おも}{思}うよ、ココネは

\Maruko だよねー

\page
\Alpha ^{まる}{丸}^{こ}{子}さんて

\Maruko ん?

\Alpha ^{まる}{丸}^{こ}{子}さんて、^{なまえ}{名前}はなんて^{い}{言}うの?

\Maruko え?

\Alpha 「^{まる}{丸}^{こ}{子}」って^{みょうじ}{名字}でしょ?

\Alpha まだ^{なまえ}{名前}^{き}{聞}いてないや

\Maruko あ

\Maruko マルコ

\Alpha え?

\page
\Maruko 「^{まる}{丸}^{こ}{子}マルコ」

\Maruko ^{なまえ}{名前}カタカナで

\Alpha ありゃま!

\Maruko 「マルコ」なんて^{おとこ}{男}の^{こ}{子}の^{なまえ}{名前}みたいだけど

\Maruko ^{もと}{元}はひらがなの「まるこ」ってイメージで……

\Maruko ^{ひび}{響}きが^{す}{好}きだったんだ^{とうじ}{当時}

\Alpha ほー

\Maruko だから^{みょうじ}{名字}も「^{まる}{丸}^{こ}{子}」にしちゃったんだねえ

\Alpha え?

\page
\Maruko ^{わたし}{私}ね……
\ ^{もと}{元}のオーナーの^{みょうじ}{名字}なのがイヤで

\Maruko だから^{じぶん}{自分}で^{か}{変}えちゃったんだ

\Alpha え゛っ!

\Alpha え……だっ……
\ そんなこと……

\Maruko できるよ

\Maruko やりたければね

\page
\Alpha ^{まる}{丸}^{こ}{子}さんは
\ オーナーがきらいだったの?

\Maruko きらいってゆうか……
\ まあ、^{じぶん}{自分}なりのケジメかなあ

\Maruko いろいろあったね……

\Alpha ふ〜〜ん……
\ そっか……
\ へ〜〜

\Alpha ^{わたし}{私}のはね
\ オーナーの^{みょうじ}{名字}
\ 「^{はつ}{初}^{せ}{瀬}^{の}{野}」

\Maruko うん
\ ^{し}{知}ってる

\page
\Sign そばぜんざいと^{こ}{小}^{うめ}{梅}

\Maruko ココネさんはねえ……
\ アルファさんのことが、も〜〜
\ ^{だい}{大}^{す}{好}き……

\Alpha ^{わたし}{私}もココネ^{す}{好}き

\page
\Maruko ^{わたし}{私}のココネさん^{す}{好}き

\Alpha うん

\page
\Maruko ナイも^{す}{好}き

\Alpha え゛?!


\subsection{第119話\ ソバゼンザイ}

\page[119]
\Alpha マ……

\page[121]
\Alpha ^{さいきん}{最近}、^{き}{気}がついたこと

\page
\Alpha マッキちゃんは^{まる}{丸}^{こ}{子}さんとちょっと^{に}{似}ている

\Alpha ^{へん}{変}なとこで、やけにつっかかってくる^{ところ}{所}とか

\Alpha そのわりに^{みょう}{妙}にモロいこととか

\page
\Makki むぇ〜〜
\ ……うぇ?

\page[128]
\Alpha ^{はや}{早}いなあ

\page
\Alpha ^{はや}{早}すぎるよ

\page
\Makki ごめん
\ ねちゃったよ……

\Alpha あ、んーん

\Alpha ^{だれ}{誰}も^{こ}{来}なかったし……

\page
\Makki そか

\Makki ^{べんきょう}{勉強}になんない^{みせ}{店}だなー
\ ここは……

\Alpha ^{こま}{困}ったもんだ

\Makki なに^{つく}{作}ってんの?

\Alpha そばぜんざい

\page
\Makki ソバゼンザイ

\Makki ^{み}{見}ていい?

\Alpha いいよ


\subsection{^{アールアンドピー}{R&P}}

\page[133]
\Kokone ゴール、ここですよー

\Alpha おー

\Sign アルファ
\ 「ウサギー^{ごう}{号}」
\ 8^{ばりき}{馬力}^{きゅう}{級}^{でんどう}{電動}

\Sign ^{まる}{丸}^{こ}{子}
\ 「はと^{まる}{丸}」
\ 8^{ばりき}{馬力}^{きゅう}{級}^{でんどう}{電動}

\page[136]
\Maruko どっち^{か}{勝}った?!

\Alpha わかんない!!

\Kokone うー


\subsection{第120話\ ^{こえ}{声}}

\page[138]
\Alpha きぬがさまで^{か}{買}い^{もの}{物}に^{い}{行}く

\Alpha きぬがさはこの^{へん}{辺}で、^{ゆいいつ}{唯一}の^{まち}{町}っぽい^{まち}{町}だ

\page[140]
\Alpha ぬるい^{くうき}{空気}

\Alpha いろんなにおいのする
\ コンソメ^{いろ}{色}の^{ゆうがた}{夕方}

\Alpha ^{ちょく}{直}に^{かえ}{帰}ったりはしない

\Alpha 「^{きた}{北}の^{まち}{町}」に^{い}{行}ってみようと^{おも}{思}う

\page
\Alpha ^{きた}{北}の^{まち}{町}への^{みち}{道}は^{むかし}{昔}からの^{おおどお}{大通}り

\Alpha いつも、^{ひと}{人}の^{けはい}{気配}はあるのに、^{すがた}{姿}はあまり^{み}{見}ない

\Alpha ^{むじん}{無人}の^{のはら}{野原}よりも^{しず}{静}かだ

\page[143]
\Alpha ^{きた}{北}の^{まち}{町}

\Alpha いつか^{せんせい}{先生}と^{き}{来}た

\page
\Alpha ^{きた}{北}の^{まち}{町}はもともと^{よこすか}{横須賀}と^{よ}{呼}ばれていた

\Alpha ^{さか}{坂}の^{した}{下}の^{まち}{街}が、
^{うみ}{海}にひたった^{すがた}{姿}は^{よこはま}{横浜}と^{に}{似}ている

\Alpha でも、「^{よこすか}{横須賀}」の^{なまえ}{名前}は、なぜか^{やま}{山}の^{うえ}{上}まで、
^{も}{持}って^{く}{来}られることはなかった

\page
\Alpha ヨコスカの^{まち}{街}は^{なまえ}{名前}といっしょに^{うみ}{海}の^{そこ}{底}にある

\page[147]
\Alpha ^{からだ}{体}の^{おく}{奥}から

\page
\Alpha ^{こえ}{声}が^{で}{出}てくるので

\Alpha ^{おも}{思}いきり、^{おお}{大}きな^{こえ}{声}を^{だ}{出}した



\subsection{アルファのかお\ 10^{ねん}{年}アルバム}

\Alpha ^{じゅうねん}{十年}ひとむかしー

\Alpha ^{じゅうねんまえ}{十年前}は、^{だいたい}{大体}こんな^{かん}{感}じでした

\Alpha ^{いま}{今}は、こんなです

\Alpha ^{さいきん}{最近}じゃ、こんなです
