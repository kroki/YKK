\section{Volume 8}

\subsection{^{だい}{第}66^{わ}{話}\ ^{かき}{柿}}

\page[4]
\A ^{かまくら}{鎌倉}には^{に}{二}ヵ^{げつ}{月}いた

\A おせわさまでした

\page
\Sign ^{ていしょく}{定食}
\ ゆきの
\ ^{そう}{總}^{ほんてん}{本店}

\A ^{す}{住}みこみで^{はたら}{働}かせてもらった^{しょくどう}{食堂}

\A ^{たいふう}{台風}^{ご}{後}の^{こうじちゅう}{工事中}かき^{い}{入}れ^{とき}{時}だけの
^{たんき}{短期}アルバイトだったけど

\A はじめてにしてはめぐまれていたと^{おも}{思}う

\page[7]
\A ^{いえ}{家}からもってきたカートはでこぼこ^{みち}{道}でじゃまなので

\A ^{しょくどう}{食堂}にあずけた

\page
\A ^{え}{江}の^{しま}{島}は^{りく}{陸}からはなれ^{いま}{今}は^{むじん}{無人}^{しま}{島}

\A このあたりは^{ひと}{人}の^{けはい}{気配}がない

\page
\A 「^{しゅと}{首都}」へ^{い}{行}ってみようと^{おも}{思}った

\A ^{ひ}{日}が^{く}{暮}れる^{まえ}{前}に^{つ}{着}きたいけど

\A けっこう^{ある}{歩}くな……

\A かながわの^{くに}{国}^{しゅと}{首都}は^{しょうなん}{湘南}^{こ}{湖}の^{ちか}{近}く
\ 「^{ち}{茅}^{が}{ヶ}^{さき}{崎}」の^{やま}{山}の^{なか}{中}で

\A ^{みち}{道}はくねくねと^{なが}{長}い

\page
\Sign ^{かながわ}{神奈川}^{くに}{国}^{こっかい}{国会}^{かん}{館}

\Sign ^{かき}{柿}あり〼

\page
\P ん?
\ きょうはもう^{し}{閉}めちゃったよ

\A あ〜〜
\ はい……

\A あの〜〜
\ ^{まち}{町}はどっちですか

\P ^{まち}{町}?
\ ここには^{こっかい}{国会}しかないなあ……

\P ^{いちばん}{一番}^{ちか}{近}い^{まち}{町}?
\ そうだなー
\ ^{よこはま}{横浜}と^{あつぎ}{厚木}……

\P あとはムサシノの^{まちだ}{町田}あたりかなあ……

\P ここも^{こっかい}{国会}の^{ひ}{日}は^{やたい}{屋台}なんか^{で}{出}て
\ にぎやかなんだけどねー
\ どしたの?

\A いえ……

\P ^{やど}{宿}?
\ ^{きた}{北}に^{い}{行}く^{みち}{道}に^{いっけん}{一軒}あるよ

\P ^{かき}{柿}あげるから、^{げんき}{元気}だしな
\ しゃごんでないで

\A どうも

\page
\A ^{かき}{柿}が^{おも}{重}いが

\A このへんは^{みず}{水}がないので、^{たす}{助}かる

\A ふう

\page
\A ^{しょうなん}{湘南}^{こ}{湖}

\A かつての^{さがみ}{相模}^{へいや}{平野}は
\ ^{ひろ}{広}く^{あさ}{浅}い^{きすい}{汽水}^{こ}{湖}になっている

\A あ

\A おいしい

\page[15]
\A ^{みずうみ}{湖}を^{ひだり}{左}に^{み}{見}て
\ ^{だいち}{台地}の^{うえ}{上}をめぐる
\ ^{おな}{同}じような^{けしき}{景色}の^{なか}{中}を^{ひと}{人}ひとり^{ぶん}{分}のはばしかない
\ ^{みち}{道}が^{えんえん}{延々}とつづく

\A こんなに^{ほそ}{細}く^{ひとどお}{人通}りも^{な}{無}い^{みち}{道}なのに、くっきりと^{いっぽん}{一本}

\A ^{こ}{濃}い^{くうき}{空気}

\A ^{じぶん}{自分}の^{あしおと}{足音}だけが^{き}{聞}こえる

\page
\A ^{ひ}{日}が^{お}{落}ちて、^{きゅう}{急}に^{ひ}{冷}えてくる

\A ^{かき}{柿}は^{おも}{重}いし

\A ^{やど}{宿}なんてないし

\page
\A どうしようかな

\A ^{あさ}{朝}まで^{ほ}{歩}こうかな

\page
\Sign うなぎや
\ ^{えび}{海老}^{めいてん}{名店}

\A ぽっと^{で}{出}た
\ うなぎ^{や}{屋}さんが^{やど}{宿}もやっていた

\A おかみさんは^{かき}{柿}を^{み}{見}て、^{おおわら}{大笑}いしていた


\subsection{第67話\ ^{みなと}{港}}

\page[20]
\A あ

\A 16^{ごう}{号}^{どお}{通}りを^{め}{目}ざして、^{きた}{北}に^{む}{向}かう^{とちゅう}{途中}

\A ^{まよ}{迷}った

\page
\A ^{みずべ}{水辺}をはなれて^{ある}{歩}きだしてからわかったことがひとつ……

\A どうも、^{わたし}{私}には「^{うみ}{海}の^{ほう}{方}・^{やま}{山}の^{ほう}{方}」で
^{じぶん}{自分}の^{いち}{位置}を^{かくにん}{確認}するクセがあるらしい

\A バラバラにちらかる^{ぞうきばやし}{雑木林}の^{みち}{道}は^{ほうこう}{方向}^{かん}{感}をマヒさせる

\page
\A あり?

\A なんか、さっき^{とお}{通}ったような……

\A すなおに^{みず}{水}っぺりの^{みち}{道}^{ある}{歩}ってった^{ほう}{方}がよかったかな

\page
\A たぶん……
\ ^{ある}{歩}く^{たび}{旅}をどこか^{あま}{甘}くみてたんだと^{おも}{思}う

\A ^{いちにち}{一日}^{ぶん}{分}の^{みち}{道}のりもバイクでなら、^{いちじかん}{時間}ってところだろう

\page
\A と^{い}{言}っても、バイクで^{とお}{通}れる^{ところ}{所}もほとんどなくて

\A ^{さき}{先}の^{み}{見}えない^{こみち}{小道}ばかりがつづく

\page[29]
\A いきなり^{ひろ}{広}い^{ところ}{所}に^{で}{出}た

\page
\Sign ^{うけつけ}{受付}

\Sign ^{はままつ}{浜松}^{びん}{便}1:30h

\Sign ^{ふくしま}{福島}^{びん}{便}2:00h

\Sign ^{ながおか}{長岡}^{びん}{便}2:00h

\P いらっしゃい
\ なにか?

\A あの……えと
\ ここ
\ どこですか?

\P は?

\page
\P ここったら、あんた
\ ^{あつぎ}{厚木}^{くうこう}{空港}だよ

\A え?!

\A アツギ?
\ あれ?

\A そっちには^{い}{行}ってないはずだけど……

\P そりゃ、あんた

\P ここは^{なまえ}{名前}だけだからね

\P ^{ほんと}{本当}の^{あつぎ}{厚木}の^{まち}{町}はずっと^{にし}{西}だよ

\A あ、なんだ

\A ……え?
\ ちょっと
\ クーコーって
\ ……それあの
\ ^{ひこうじょう}{飛行場}の^{くうこう}{空港}?

\P ^{くうこう}{空港}ったら^{ひこうじょう}{飛行場}だよ、ふつう

\page
\A え……じゃ
\ ^{ひこうき}{飛行機}とかくるんですか?!

\P そりゃまあ
\ ^{きょう}{今日}もこれから^{い}{一}^{びん}{便}くるよ

\P なにあんた
\ ^{ひこう}{飛行}^{き}{機}、^{み}{見}たことないの?

\A ^{ちか}{近}くではまだ……

\P あらそ

\P あーー
\ あのさ

\page
\P あんた、ロボットの^{ひと}{人}でしょ?

\A ええ

\P こんどの^{びん}{便}の^{うん}{運}ちゃんさ
\ うちのコなんだけどね

\P ロボットなんだよ

\A こっ

\A こんどの^{びん}{便}に?!

\P うん

\page
\A こ……いいいい
\ いつ^{く}{来}るんですか?!

\P んーー
\ ^{よてい}{予定}じゃ、まだ^{いち}{一}^{じかん}{時間}くらいあるけどね

\P どうする?
\ ^{ま}{待}つ?

\A えぇえ!!

\A もう^{なん}{何}^{じかん}{時間}でも!!

\page[36]
\P おそいね
\ きょうはこないかもね

\A うえ〜〜〜?!
\ そんな〜〜!

\P まー
\ よくあることだし
\ お^{ちゃ}{茶}のむ?


\subsection{第68話\ ^{ひこう}{飛行}^{き}{機}}

\page[38]
\A ^{はじ}{初}めて^{ちか}{近}くで^{み}{見}る^{ひこう}{飛行}^{き}{機}は、
^{あさがた}{朝方}の^{ほし}{星}に^{に}{似}ていた

\A ゆっくりと^{うご}{動}く^{ひか}{光}る^{てん}{点}が、
^{くろ}{黒}い^{てん}{点}に^{か}{変}わりながら^{おお}{大}きくなってくる

\page
\A まわりの^{くうき}{空気}がふるえた

\page[42]
\NA お^{きゃく}{客}さん?

\A あ、いえ
\ ^{けんがく}{見学}です

\A すごいですね!!

\NA そうですか?

\A ええ!!
\ もう……

\page
\NA なにか?

\A あの……
\ ロボットの^{かた}{方}ですとか

\NA まあ
\ あんたもそうみたいですね

\A まあ

\page
\A あの……
\ ^{おとこ}{男}のかた?

\NA え?
\ ああ
\ まあ^{いちおう}{一応}

\NA あんただって^{おんな}{女}の^{ひと}{人}でしょう

\A まっ
\ まあ^{いちおう}{一応}


\subsection{第69話\ ^{たきび}{焚火}}

\page[46]
\A ^{かぜ}{風}がなくてよかったー

\NA んー

\NA はい、これドクダミ^{ちゃ}{茶}

\A あ、どうも

\A いただきます

\page
\A あのー
\ まだ^{なまえ}{名前}^{き}{聞}いてなかったっけねえ

\NA ああ
\ そういや……

\A わたしアルファっていいます

\NA へえ……
\ アルファって、あの「アルファ^{かた}{型}」から?

\NA またストレートーな

\A へへへ……
\ あなたは?

\page
\NA おれはナイ

\A へ?

\NA だから、ナイ
\ ^{なまえ}{名前}が

\A え?!

\A あ……
\ そうなんだ……
\ それは、あれ?
\ ポリシーかなんかで?

\A ……じゃあ
\ とりあえず……
\ どう^{よ}{呼}ぼう……

\NA あーー

\page
\NA そうじゃなくて
\ 「ナイ」って^{い}{言}うんだ
\ ^{なまえ}{名前}が

\A わかったって
\ だから、なんて^{よ}{呼}べばいい……

\A ああ!!
\ ナイさん!

\NA はい

\A ^{なまえ}{名前}の^{ゆらい}{由来}……
\ わかるよ、なんとなく

\NA やっぱり?

\page
\A あー
\ そうだ
\ さっきはごめんなさい
\ ^{おお}{大}さわぎして……

\NA さっきって?

\A わたし、なんか^{かって}{勝手}に「ロボットは^{おんな}{女}の^{ひと}{人}」って^{おも}{思}ってたんで

\A その……
\ びっくりしちゃった

\NA ああ

\NA いや、オレそうゆうのなれてるし

\page
\NA おれもロボットの^{ひと}{人}^{なんにん}{何人}が^{し}{知}ってるけど

\NA まだ^{おとこ}{男}に^{あ}{会}ったことない……

\NA ほかの^{ひと}{人}もみんなそうみたいでさ

\NA めずらしがられるよどこでも

\A ふーん

\A じゃ、わたしすごいラッキーなのかな

\NA そうかもね

\page
\NA なんか、^{き}{気}になる?
\ やっぱ

\A あ!
\ あはは!
\ ごめんね
\ いや〜〜

\A ^{しょうじき}{正直}^{い}{言}って^{すこ}{少}し

\NA アルファが「^{おんな}{女}」だってくらいには、「^{おとこ}{男}」だよ

\NA ^{み}{見}ためはふつうだと^{おも}{思}う……

\A あ……
\ はい……
\ わかりました
\ なるほど

\page
\NA なんか……
\ ^{おとこ}{男}のロボットは^{よわ}{弱}いんだってさ

\NA ^{き}{聞}いた^{はなし}{話}じゃ、^{はや}{早}いうちにみんな^{し}{死}
んじゃったって

\NA なにが^{ちが}{違}うのかわかんないけど

\A そうなんだ

\NA うん

\page
\A でも^{こんや}{今夜}は^{たす}{助}かっちゃったよ

\A ^{のじゅく}{野宿}^{かくご}{覚悟}だったもん

\NA おばちゃんまぜて^{かわ}{川}の^{じ}{字}だけど

\A うん
\ ありがと
\ でも、この^{へん}{辺}ほんとに^{いえ}{家}ないね

\NA いちばん^{ちか}{近}い^{やど}{宿}でも、ずっとあっちでね
\ うなぎ^{や}{屋}で……

\A ^{きょう}{今日}そっちから^{き}{来}たの

\page
\A あのさ

\A あとで、^{ひこう}{飛行}^{き}{機}^{み}{見}せてほしいんだけど
\ いいかなあ

\NA いいよ

\NA あ、そうだ
\ あした^{はいたつ}{配達}なきゃ
\ ちっと^{ぶひん}{部品}かえてためし^{ひこう}{飛行}するけど……
\ ^{のる}{乗る}?

\A え〜〜っ?!
\ ほんと?!
\ いいの?!
\ のる?!

\A やった!!
\ ほんと!?
\ わ〜〜〜〜〜〜!!

\page[57]
\NA ねむくなった?

\A ん?
\ まだ^{へいき}{平気}

\A ナイ……
\ あのさあ

\NA うん

\page
\A なんか、^{しごと}{仕事}ない?

\NA ないなあ


\subsection{第70話\ ^{みず}{水}}

\page[61]
\A ナイー!!

\page
\A ほら、^{わたし}{私}も!
\ おんなじ!

\NA えっ
\ あ、ほんとだ
\ めずらしい

\A ^{わたし}{私}、なんとなく^{いっぴん}{一品}^{もの}{物}のつもりでいたよ

\NA いや……あれ?
\ ちょっと^{み}{見}せて

\NA わ……
\ これ、^{なみ}{並}の^{しよう}{仕様}じゃないよ

\A え?

\NA たぶん、^{ほんと}{本当}に^{いっぴん}{一品}^{もの}{物}だと^{おも}{思}う

\NA すごいの^{も}{持}ってるね

\page
\A えっ……あ
\ そうなの?

\A これもらったものなんだ

\NA ふーーん

\NA いいのもらったなあ

\A うん

\NA ^{しゃしん}{写真}ほしがる^{ともだち}{友達}いてね

\A へー

\page
\A ナイ〜〜
\ あの〜〜

\A なんか、^{まえ}{前}^{み}{見}えないんだけど……

\NA ああ
\ ザブトン^{つかう}{使う}?
\ あんまり^{か}{変}わんないと^{おも}{思}うけど

\A ありがと

\page
\NA いくよ

\A うん!

\page
\A ぐ

\page[68]
\A ほんとに^{と}{飛}んでる

\page
\NA あ、そうだ
\ アルファ、カメラのコード^{いま}{今}、^{も}{持}ってる?

\A うん

\NA それ、くわえて^{まえ}{前}のボードの^{あな}{穴}にさしてみな、おもしろいから

\A え?
\ うん……
\ これかな

\A うぇ!!

\NA なんかベロが^{かぜ}{風}に^{お}{押}される^{かん}{感}じしないか

\NA これ、^{な}{慣}れるとベロで^{そくど}{速度}とかわかるんだ

\A ベロ?

\page[71]
\A ^{ひこう}{飛行}^{き}{機}がないよ!!

\NA え?

\page
\A いきなり、はだかで^{そと}{外}に^{ほう}{放}り^{だ}{出}されたのかと^{おも}{思}った

\A いろんな^{ちから}{力}が^{からだ}{体}じゅうにドツとおしよせる

\A ナイも^{わたし}{私}も、しばらくわけがわからなかったけど

\A ^{すこ}{少}しして^{ふたり}{2人}ともどういうことなのか、わかってきた

\page
\NA アルファ、あんた

\page
\A あー

\A ^{みず}{水}の^{なか}{中}みたい

\NA ^{みず}{水}?

\A うん

\page
\Sign ノースアメリカンATー6「てきさん」

\Sign ^{きち}{基地}のある^{まち}{町}にはよくころかこてますね

\Sign これはナイのKNー021「あっぎ2^{ごう}{号}」

\page
\Sign ^{みず}{水}とお^{ちゃ}{茶}
\ かめ
\ 16^{ごう}{号}^{どお}{通}り


\subsection{第71話\ ^{たに}{谷}の^{みち}{道}}

\page[78]
\A ^{ひこう}{飛行}^{き}{機}は^{きょうれつ}{強烈}だった

\A あの^{ひ}{日}、^{みち}{道}に^{まよ}{迷}って、^{くうこう}{空港}に^{で}{出}てなかったら

\A おばちゃんにも、ナイにも、^{あ}{会}えなかっただろう

\page
\A あれからしばらく16^{ごう}{号}^{どお}{通}りにある^{みず}{水}^{や}{屋}さんの^{てつだ}{手伝}いをしていた

\A 16^{ごう}{号}^{どお}{通}りは^{たび}{旅}の^{ひと}{人}が^{おお}{多}い

\A ^{まいにち}{毎日}いろんな^{はなし}{話}が^{き}{聞}ける

\A あったかくなって、また、^{うご}{動}くことにした

\A ^{ほんらい}{本来}なら、^{はちおうじ}{八王子}あたりで^{つぎ}{次}の^{しごと}{仕事}をさがすところだけど

\A ひとつ、^{き}{気}になる^{はなし}{話}を^{き}{聞}いたので、そっちに^{む}{向}かって^{い}{行}く

\page
\A ^{おおどお}{大通}りをはずれて^{たに}{谷}の^{みち}{道}にはいる

\A ^{いわ}{岩}や^{とうぼく}{倒木}だらけの^{やま}{山}の^{みち}{道}は、
^{むかし}{昔}、ひとつの^{まち}{町}のメインストリートだった

\A もう、^{くるま}{車}の^{とお}{通}れない、アスファルトの^{うえ}{上}を^{こけ}{苔}がおおっている

\page
\A かつての^{しゃどう}{車道}を、ぬうようにつくられたせまい^{みち}{道}

\A この^{みち}{道}にあるという^{へん}{変}な^{しょくぶつ}{植物}のうわさ

\A ^{たび}{旅}の^{ひと}{人}が^{はな}{話}していた、^{しょくぶつ}{植物}というのは、この^{き}{木}のことだ

\page
\A どう^{み}{見}ても、^{こうえん}{公園}の^{がいろ}{街路}^{とう}{灯}にしか^{み}{見}えない

\A でも、^{おお}{大}きさはバラバラだし、^{ね}{根}っこもあるみたいだし

\A ^{たし}{確}かに、^{しぜん}{自然}のものではあるようだ

\page
\A まわりをよく^{み}{見}ると、あっちにもこっちにも^{は}{生}えている

\A ^{がけ}{崖}の^{とちゅう}{途中}とか、やぶの^{なか}{中}とか

\A ^{みち}{道}を^{い}{行}く^{ひとびと}{人々}は^{き}{木}のことを^{き}{気}にもとめない

\A ^{ほそ}{細}くくねる^{みち}{道}

\A この^{みち}{道}は^{ふじ}{富士}^{さん}{山}への^{ちかみち}{近道}なので、
^{ひとどお}{人通}りはそれなりにある

\page[85]
\A ^{ゆうがた}{夕方}

\A ^{くろ}{黒}い^{やま}{山}の^{かげ}{影}にだいだい^{いろ}{色}に^{ひか}{光}る
^{たに}{谷}の^{みち}{道}が^{う}{浮}かびあがる

\page
\A あの^{き}{木}が^{ひか}{光}りはじめた

\page
\A ^{き}{木}の^{ひかり}{光}は、^{みち}{道}を^{て}{照}らすちょうちんの
^{ひかり}{光}を^{むし}{無視}して、^{べつ}{別}のラインをつくりはじめる

\A ^{たび}{旅}の^{ひと}{人}に^{き}{聞}いた^{はなし}{話}

\page
\A ^{き}{木}の^{ひかり}{光}が^{しめ}{示}すのは、かつての^{まち}{町}のあとをなぞるライン

\A ^{みち}{道}の^{きおく}{記憶}

\A ^{ひと}{人}が^{わす}{忘}れてしまっても


\subsection{第72話\ ササゲ}

\page[95]
\T おめーよー
\ ^{おめ}{重}えよ!

\M ^{しつれい}{失礼}な!

\T でー?
\ ^{きょう}{今日}はどこつれてけって?

\M きぬがさまでつれてってよ

\T きぬがさ!
\ ちっと^{とお}{遠}いな

\M どーせ^{きょう}{今日}ヒマなんでしょが

\T ^{しつれい}{失礼}な!

\page
\M おまたへ

\T ^{なに}{何}、^{か}{買}ったのよ

\M これ?
\ ササゲ

\T ^{なに}{何}、ササゲって

\M ^{まめ}{豆}だよ

\T なんだ
\ ほかには?
\ もういいのか?

\M うん

\page
\M ねえ
\ すなはまの^{ほう}{方}^{とお}{通}ってこうよ

\T いいけどよ
\ ちっと^{じかん}{時間}おせえから、^{およ}{泳}げねえぞ

\M いいよ
\ ^{よ}{寄}ってくだけでも

\T そか

\page
\T ^{すなはま}{砂浜}には、ひさしぶりに^{き}{来}た

\T ^{しお}{潮}は、もう^{あ}{上}げていて、^{なみ}{波}も^{で}{出}ている

\page[102]
\M ん?

\T んーん


\subsection{第73話\ チョコレートケーキ}

\page[104]
\Sign ハニー
\ とうもろこし

\P ^{いっぽん}{一本}ちょうだい

\A あっ
\ はーい!

\page
\A あのね
\ ちょっと^{ま}{待}ってもらえれば、^{や}{焼}きなおしますよ

\P んーー
\ じゃ^{や}{焼}いて

\A はい

\A うちのは^{さんち}{産地}^{ちょくそう}{直送}だからね、
^{れいとう}{冷凍}なんかと^{ちが}{違}っておいしいですよ〜〜

\P あー
\ ^{れいとう}{冷凍}のは^{く}{食}えたもんじゃねえもんな

\P ふい〜

\P ここよ!
\ あれだよなー

\A はいー?

\page
\P ^{め}{目}が^{よ}{良}くなるよなー

\A そうですねー

\page[108]
\A いつも^{うみ}{海}のむこうに^{き}{切}り^{かみ}{紙}みたいに
^{う}{浮}かんでいた^{ふじ}{富士}^{さん}{山}が^{いま}{今}、
^{め}{目}の^{まえ}{前}にドカンとある

\A はーーい
\ おまたせ

\P うい

\P あ、そうだ
\ なんか^{の}{飲}み^{もの}{物}ある?

\A ええ

\page
\A お^{ちゃ}{茶}とコーヒーとゆずジュースがありますけど

\P んー、じゃあ、お^{ちゃ}{茶}

\A はい

\Sign カラス
\ ^{むぎ}{麦}

\P なにそれ……
\ あー、^{むぎちゃ}{麦茶}ね

\A ありあとあしたー

\page
\A ^{ちか}{近}くで^{み}{見}る^{ふじ}{富士}^{さん}{山}はなんか^{えんきん}{遠近}^{かん}{感}がおかしくて

\A ^{め}{目}が^{お}{追}いつかないほど^{おお}{大}きいのに、ホイと^{て}{手}でさわれそうに、^{み}{見}える

\A なんか、^{や}{柔}っこそうで^{いろ}{色}も^{かん}{感}じも、ちょうどココアパウダーがふってあるみたいで……

\A うまそうっす……

\page
\A ^{ふじ}{富士}^{さん}{山}の^{みなみがわ}{南側}が^{とお}{通}れないので、
こんなせまい^{みち}{道}でも^{くるま}{車}がよく^{とお}{通}る

\P あの〜〜
\ もっと^{ふじ}{富士}^{さん}{山}の^{ちか}{近}くに^{い}{行}ける^{みち}{道}ないですかねえ

\A ああ〜〜
\ ないみたいですよ

\P そーすか〜〜
\ ども

\A ^{だいはんじょう}{大繁盛}だねー

\page[113]
\A ^{きいろ}{黄色}
\ プロペラひとつ

\A おーい!!
\ おーい!!

\page
\A お〜い!!

\page[118]
\A ん


\subsection{第74話\ ^{もうまく}{網膜}}

\page[120]
\Sign アトリエ
\ ^{まる}{丸}^{こ}{子}

\page[123]
\K ^{まる}{丸}^{こ}{子}さん
\ あの〜〜
\ もっと^{かる}{軽}〜くしてもらわないとー……

\K こまります

\R いや〜〜

\R ひさしぶりだったからねえ

\R もりあがっちゃったよ
\ おいちゃん

\K おいちゃんなんですか

\page
\R じゃー
\ チェックね

\K はい

\R ひさしぶりに^{とお}{遠}くの^{ともだち}{友達}から^{けしき}{景色}が^{おく}{送}られてきた

\R ^{おく}{送}られてくる^{けしき}{景色}はそいつ^{この}{好}みのコントラ
ストの^{つよ}{強}い^{いろ}{色}あいで

\R ^{わたし}{私}は^{す}{好}きだ

\R はじめて、ココネさんに、^{じか}{直}に^{わた}{渡}してもらった^{とき}{時}はびっくりした

\R ^{かのじょ}{彼女}を^{とお}{通}して^{み}{見}る^{ふうけい}{風景}は、リアル^{かん}{感}がちがう

\page[127]
\R ひさしぶりに^{み}{見}る^{ともだち}{友達}の^{ひこう}{飛行}^{き}{機}

\R その^{まえ}{前}で、じっと、^{わたし}{私}を^{み}{見}ている、あずき^{いろ}{色}の^{ひとみ}{瞳}の^{ひと}{人}

\R ただでさえリアルなまわりの^{ふうけい}{風景}より、^{なんばい}{何倍}もくっきりとそこにいる

\R ^{みどりいろ}{緑色}の^{かみ}{髪}のこの^{ひと}{人}

\page
\K な……なにか^{へん}{変}な^{ところ}{所}ありました?

\R んーん
\ そうじゃなくてね

\R ココネさんやっぱ、すごいなって……

\R はい!
\ サインするよ

\K あっ
\ はい
\ どうも

\page
\R また、よろしくね

\K はい

\R じゃ
\ ^{こんど}{今度}、^{あそ}{遊}ぼうよ

\K あ、はい
\ ありがとうございました

\page
\R アルファさんか

\R ぱっと^{み}{見}て、なんとなくわかった

\R ^{いま}{今}、^{たび}{旅}に^{で}{出}ていると^{き}{聞}いてたけど

\page
\R ^{ふたり}{二人}の^{ともだち}{友達}をいっぺんに^{と}{取}られたような

\R ^{きも}{気持}ちにもなった
\ けどね

\page
\R ナイ
\ さあ
\ ^{てがみ}{手紙}くらいはそえてよね


\subsection{第75話\ ^{のび}{野火}}

\page[134]
\A ^{おおたるみ}{大垂水}^{とうげ}{峠}をこえて、むさしのの^{くに}{国}にはいる

\A ^{やま}{山}をおりるのは^{かんが}{考}えてみれば、ひさしぶりだ

\page
\Sign やきそば
\ …ステラ

\A ^{ひの}{日野}という^{だいち}{台地}にさしかかると、^{きゅう}{急}に、にぎやかになってきた

\A ^{だいち}{台地}が^{とぎ}{途切}れる^{がけ}{崖}の^{ところ}{所}に、^{ひと}{人}が^{あつ}{集}まっている

\page
\A ^{め}{目}の^{まえ}{前}にひろがるムサシノの^{はら}{原}を^{おお}{大}きな^{くも}{雲}がおおっている

\page
\A ^{けむり}{煙}だ

\A ムサシノのすすきの^{のはら}{野原}が、みんな、^{も}{燃}えている

\A こういった^{かじ}{火事}は^{なんねん}{何年}かに^{いちど}{一度}はあるそうだ

\A ^{ひ}{火}は、^{しぜん}{自然}に^{き}{消}えるまで^{ほう}{放}っておくしかないという

\A それまで、ここから^{ひがし}{東}へは^{い}{行}けない

\page[139]
\Sign …でん
\ あま^{さけ}{酒}

\A ^{かじ}{火事}は、あと^{みっか}{三日}は^{つづ}{続}くという

\A あの^{ひ}{火}のむこう^{がわ}{側}に^{い}{行}きたかったけど……

\page
\A ^{あした}{明日}、^{みなみ}{南}の^{ほう}{方}に^{い}{行}くことにした

\A ^{きょう}{今日}は、ここで、^{ひ}{火}を^{み}{見}ていようと^{おも}{思}う


\subsection{第76話\ ^{くり}{栗}}

\page[142]
\A ^{みなみ}{南}へ^{む}{向}かう^{ちょくせん}{直線}の^{みち}{道}

\A もう、ずうっとこんな^{ふうけい}{風景}が^{つづ}{続}いている

\A たき^{ひ}{火}の^{けむり}{煙}のにおい^{いがい}{以外}に、^{ひと}{人}の^{けはい}{気配}はない

\page
\A どっちを^{むか}{向}いても^{やま}{山}の^{なか}{中}みたいな^{みち}{道}

\A でも、ときどき、^{こ}{濃}い^{しお}{潮}のにおいがまじる

\A す〜

\A 「^{かえ}{帰}ってきている」と^{おも}{思}った

\A ほんとに、^{いちねん}{一年}も^{である}{出歩}いていたんだろうか

\page
\A この^{しま}{島}は、^{ちず}{地図}を^{み}{見}て^{そうぞう}{想像}するのより、ずっとずっと^{ひろ}{広}い

\A ^{いちねん}{一年}もあれば、もっと^{にし}{西}やもっと^{きた}{北}の「やまと」とか「みちのく」とかの
エリアにだって^{い}{行}けると^{おも}{思}ってた

\A でも、まー
\ ^{よ}{良}しとしよう

\page
\A ^{みち}{道}はただただ^{みなみ}{南}へのびている

\page
\A そういえば、きのうからなにも^{た}{食}べてなかったっけ

\A なんか^{ひと}{人}も、ひさしぶりに^{み}{見}た

\page
\A こ
\ こんにちは

\A いい〜〜い^{てんき}{天気}ですね〜〜

\P あ?
\ あーー
\ そうなー

\A じゃ

\P んー

\A あの〜〜
\ この^{さき}{先}なんか、^{た}{食}べさせてくれるお^{みせ}{店}とか、
あります?

\P あ?

\page
\P この^{さき}{先}?
\ ずーっと^{うみ}{海}までこんな^{はやし}{林}ん^{なか}{中}だよ

\A ふっ

\A そ
\ そですか
\ ども

\P あー
\ ねえさんよ

\P ^{くり}{栗}あんけど、^{く}{食}う?

\A あああ
\ ありがとうございます

\page
\P ^{くり}{栗}^{さんこう}{山行}った^{かえ}{帰}りでよー
\ よかったな

\P ほら
\ これにすんか?

\P ^{なんこ}{何個}くう?

\A あ
\ ^{いっこ}{1個}で……

\page
\P じゃ
\ おれも^{いっこ}{一個}いってみんかな

\P 30^{っぷん}{分}くらい^{ま}{待}ちな

\A はい

\A ^{わたし}{私}、^{くり}{栗}^{や}{焼}くのはじめて^{み}{見}ます

\A たいてい、^{に}{煮}ちゃいますよねー

\P そうか?

\page
\P よっ

\P ^{いちおう}{一応}ね

\page[154]
\P どこ^{と}{飛}んでくかわかんねえからよ

\P あぶねえよな

\P あちちちちち

\P ほい
\ あちいよ

\A おー

\A はお〜

\page[157]
\Sign たべきれないっス


\subsection{アルファの^{さくもつ}{作物}の^{あらし}{嵐}!}
\A 「^{もも}{桃}^{くり}{栗}^{さんねん}{三年}、^{かき}{柿}^{はちねん}{八年}、
  ^{なし}{梨}のバカのが^{じゅうはちねん}{十八年}」そか^{い}{言}いますネ

\Sign ^{さんねん}{三年}^{くり}{栗}

\A なでごこちがたまらん

\Sign ^{はちねん}{八年}^{かき}{柿}

\Sign ^{さんねん}{三年}^{もも}{桃}

\A 4〜5^{こ}{個}がまとまってつく。^{たま}{玉}の^{おお}{大}きさはふつう

\A いちばん^{うえ}{上}の。

\A ひとつがとてつもなくうまい。

\A まんなかの^{あじ}{味}は^{なみ}{並}。

\A ^{さいこう}{最高}にまずい。

\Sign ^{じゅうはちねん}{十八年}^{なし}{梨}は・・

\A くくくく・・
