\section{Volume 8}

\subsection{^{だい}{第}66^{わ}{話}\ ^{かき}{柿}}

\page[4]
\Alpha ^{かまくら}{鎌倉}には^{に}{二}ヵ^{げつ}{月}いた

\Alpha おせわさまでした

\page[5]
\Sign ^{ていしょく}{定食}
\ ゆきの
\ ^{そう}{總}^{ほんてん}{本店}

\Alpha ^{す}{住}みこみで^{はたら}{働}かせてもらった^{しょくどう}{食堂}

\Alpha ^{たいふう}{台風}^{ご}{後}の^{こうじちゅう}{工事中}かき^{い}{入}れ^{とき}{時}だけの
^{たんき}{短期}アルバイトだったけど

\Alpha はじめてにしてはめぐまれていたと^{おも}{思}う

\page[7]
\Alpha ^{いえ}{家}からもってきたカートはでこぼこ^{みち}{道}でじゃまなので

\Alpha ^{しょくどう}{食堂}にあずけた

\page[8]
\Alpha ^{え}{江}の^{しま}{島}は^{りく}{陸}からはなれ^{いま}{今}は^{むじん}{無人}^{しま}{島}

\Alpha このあたりは^{ひと}{人}の^{けはい}{気配}がない

\page[9]
\Alpha 「^{しゅと}{首都}」へ^{い}{行}ってみようと^{おも}{思}った

\Alpha ^{ひ}{日}が^{く}{暮}れる^{まえ}{前}に^{つ}{着}きたいけど

\Alpha けっこう^{ある}{歩}くな……

\Alpha かながわの^{くに}{国}^{しゅと}{首都}は^{しょうなん}{湘南}^{こ}{湖}の^{ちか}{近}く
\ 「^{ち}{茅}^{が}{ヶ}^{さき}{崎}」の^{やま}{山}の^{なか}{中}で

\Alpha ^{みち}{道}はくねくねと^{なが}{長}い

\page[10]
\Sign ^{かながわ}{神奈川}^{くに}{国}^{こっかい}{国会}^{かん}{館}

\Sign ^{かき}{柿}あり^{ます}{〼}

\page[11]
\Person ん?
\ きょうはもう^{し}{閉}めちゃったよ

\Alpha あ〜〜
\ はい……

\Alpha あの〜〜
\ ^{まち}{町}はどっちですか

\Person ^{まち}{町}?
\ ここには^{こっかい}{国会}しかないなあ……

\Person ^{いちばん}{一番}^{ちか}{近}い^{まち}{町}?
\ そうだなー
\ ^{よこはま}{横浜}と^{あつぎ}{厚木}……

\Person あとはムサシノの^{まちだ}{町田}あたりかなあ……

\Person ここも^{こっかい}{国会}の^{ひ}{日}は^{やたい}{屋台}なんか^{で}{出}て
\ にぎやかなんだけどねー
\ どしたの?

\Alpha いえ……

\Person ^{やど}{宿}?
\ ^{きた}{北}に^{い}{行}く^{みち}{道}に^{いっけん}{一軒}あるよ

\Person ^{かき}{柿}あげるから、^{げんき}{元気}だしな
\ しゃごんでないで

\Alpha どうも

\page[12]
\Alpha ^{かき}{柿}が^{おも}{重}いが

\Alpha このへんは^{みず}{水}がないので、^{たす}{助}かる

\Alpha ふう

\page[13]
\Alpha ^{しょうなん}{湘南}^{こ}{湖}

\Alpha かつての^{さがみ}{相模}^{へいや}{平野}は
\ ^{ひろ}{広}く^{あさ}{浅}い^{きすい}{汽水}^{こ}{湖}になっている

\Alpha あ

\Alpha おいしい

\page[15]
\Alpha ^{みずうみ}{湖}を^{ひだり}{左}に^{み}{見}て
\ ^{だいち}{台地}の^{うえ}{上}をめぐる
\ ^{おな}{同}じような^{けしき}{景色}の^{なか}{中}を^{ひと}{人}ひとり^{ぶん}{分}のはばしかない
\ ^{みち}{道}が^{えんえん}{延々}とつづく

\Alpha こんなに^{ほそ}{細}く^{ひとどお}{人通}りも^{な}{無}い^{みち}{道}なのに、くっきりと^{いっぽん}{一本}

\Alpha ^{こ}{濃}い^{くうき}{空気}

\Alpha ^{じぶん}{自分}の^{あしおと}{足音}だけが^{き}{聞}こえる

\page[16]
\Alpha ^{ひ}{日}が^{お}{落}ちて、^{きゅう}{急}に^{ひ}{冷}えてくる

\Alpha ^{かき}{柿}は^{おも}{重}いし

\Alpha ^{やど}{宿}なんてないし

\page[17]
\Alpha どうしようかな

\Alpha ^{あさ}{朝}まで^{ほ}{歩}こうかな

\page[18]
\Sign うなぎや
\ ^{えび}{海老}^{めいてん}{名店}

\Alpha ぽっと^{で}{出}た
\ うなぎ^{や}{屋}さんが^{やど}{宿}もやっていた

\Alpha おかみさんは^{かき}{柿}を^{み}{見}て、^{おおわら}{大笑}いしていた


\subsection{第67話\ ^{みなと}{港}}

\page[20]
\Alpha あ

\Alpha 16^{ごう}{号}^{どお}{通}りを^{め}{目}ざして、^{きた}{北}に^{む}{向}かう^{とちゅう}{途中}

\Alpha ^{まよ}{迷}った

\page[21]
\Alpha ^{みずべ}{水辺}をはなれて^{ある}{歩}きだしてからわかったことがひとつ……

\Alpha どうも、^{わたし}{私}には「^{うみ}{海}の^{ほう}{方}・^{やま}{山}の^{ほう}{方}」で
^{じぶん}{自分}の^{いち}{位置}を^{かくにん}{確認}するクセがあるらしい

\Alpha バラバラにちらかる^{ぞうきばやし}{雑木林}の^{みち}{道}は^{ほうこう}{方向}^{かん}{感}をマヒさせる

\page[22]
\Alpha あり?

\Alpha なんか、さっき^{とお}{通}ったような……

\Alpha すなおに^{みず}{水}っぺりの^{みち}{道}^{ある}{歩}ってった^{ほう}{方}がよかったかな

\page[23]
\Alpha たぶん……
\ ^{ある}{歩}く^{たび}{旅}をどこか^{あま}{甘}くみてたんだと^{おも}{思}う

\Alpha ^{いちにち}{一日}^{ぶん}{分}の^{みち}{道}のりもバイクでなら、^{いちじかん}{時間}ってところだろう

\page[24]
\Alpha と^{い}{言}っても、バイクで^{とお}{通}れる^{ところ}{所}もほとんどなくて

\Alpha ^{さき}{先}の^{み}{見}えない^{こみち}{小道}ばかりがつづく

\page[29]
\Alpha いきなり^{ひろ}{広}い^{ところ}{所}に^{で}{出}た

\page[30]
\Sign ^{うけつけ}{受付}

\Sign ^{はままつ}{浜松}^{びん}{便}1:30h

\Sign ^{ふくしま}{福島}^{びん}{便}2:00h

\Sign ^{ながおか}{長岡}^{びん}{便}2:00h

\Person いらっしゃい
\ なにか?

\Alpha あの……えと
\ ここ
\ どこですか?

\Person は?

\page[31]
\Person ここったら、あんた
\ ^{あつぎ}{厚木}^{くうこう}{空港}だよ

\Alpha え?!

\Alpha アツギ?
\ あれ?

\Alpha そっちには^{い}{行}ってないはずだけど……

\Person そりゃ、あんた

\Person ここは^{なまえ}{名前}だけだからね

\Person ^{ほんとう}{本当}の^{あつぎ}{厚木}の^{まち}{町}はずっと^{にし}{西}だよ

\Alpha あ、なんだ

\Alpha ……え?
\ ちょっと
\ クーコーって
\ ……それあの
\ ^{ひこうじょう}{飛行場}の^{くうこう}{空港}?

\Person ^{くうこう}{空港}ったら^{ひこうじょう}{飛行場}だよ、ふつう

\page[32]
\Alpha え……じゃ
\ ^{ひこうき}{飛行機}とかくるんですか?!

\Person そりゃまあ
\ ^{きょう}{今日}もこれから^{い}{一}^{びん}{便}くるよ

\Person なにあんた
\ ^{ひこう}{飛行}^{き}{機}、^{み}{見}たことないの?

\Alpha ^{ちか}{近}くではまだ……

\Person あらそ

\Person あーー
\ あのさ

\page[33]
\Person あんた、ロボットの^{ひと}{人}でしょ?

\Alpha ええ

\Person こんどの^{びん}{便}の^{うん}{運}ちゃんさ
\ うちのコなんだけどね

\Person ロボットなんだよ

\Alpha こっ

\Alpha こんどの^{びん}{便}に?!

\Person うん

\page[34]
\Alpha こ……いいいい
\ いつ^{く}{来}るんですか?!

\Person んーー
\ ^{よてい}{予定}じゃ、まだ^{いち}{一}^{じかん}{時間}くらいあるけどね

\Person どうする?
\ ^{ま}{待}つ?

\Alpha えぇえ!!

\Alpha もう^{なん}{何}^{じかん}{時間}でも!!

\page[36]
\Person おそいね
\ きょうはこないかもね

\Alpha うえ〜〜〜?!
\ そんな〜〜!

\Person まー
\ よくあることだし
\ お^{ちゃ}{茶}のむ?


\subsection{第68話\ ^{ひこう}{飛行}^{き}{機}}

\page[38]
\Alpha ^{はじ}{初}めて^{ちか}{近}くで^{み}{見}る^{ひこう}{飛行}^{き}{機}は、
^{あさがた}{朝方}の^{ほし}{星}に^{に}{似}ていた

\Alpha ゆっくりと^{うご}{動}く^{ひか}{光}る^{てん}{点}が、
^{くろ}{黒}い^{てん}{点}に^{か}{変}わりながら^{おお}{大}きくなってくる

\page[39]
\Alpha まわりの^{くうき}{空気}がふるえた

\page[42]
\Nai お^{きゃく}{客}さん?

\Alpha あ、いえ
\ ^{けんがく}{見学}です

\Alpha すごいですね!!

\Nai そうですか?

\Alpha ええ!!
\ もう……

\page[43]
\Nai なにか?

\Alpha あの……
\ ロボットの^{かた}{方}ですとか

\Nai まあ
\ あんたもそうみたいですね

\Alpha まあ

\page[44]
\Alpha あの……
\ ^{おとこ}{男}のかた?

\Nai え?
\ ああ
\ まあ^{いちおう}{一応}

\Nai あんただって^{おんな}{女}の^{ひと}{人}でしょう

\Alpha まっ
\ まあ^{いちおう}{一応}


\subsection{第69話\ ^{たきび}{焚火}}

\page[46]
\Alpha ^{かぜ}{風}がなくてよかったー

\Nai んー

\Nai はい、これドクダミ^{ちゃ}{茶}

\Alpha あ、どうも

\Alpha いただきます

\page[47]
\Alpha あのー
\ まだ^{なまえ}{名前}^{き}{聞}いてなかったっけねえ

\Nai ああ
\ そういや……

\Alpha わたしアルファっていいます

\Nai へえ……
\ アルファって、あの「アルファ^{かた}{型}」から?

\Nai またストレートーな

\Alpha へへへ……
\ あなたは?

\page[48]
\Nai おれはナイ

\Alpha へ?

\Nai だから、ナイ
\ ^{なまえ}{名前}が

\Alpha え?!

\Alpha あ……
\ そうなんだ……
\ それは、あれ?
\ ポリシーかなんかで?

\Alpha ……じゃあ
\ とりあえず……
\ どう^{よ}{呼}ぼう……

\Nai あーー

\page[49]
\Nai そうじゃなくて
\ 「ナイ」って^{い}{言}うんだ
\ ^{なまえ}{名前}が

\Alpha わかったって
\ だから、なんて^{よ}{呼}べばいい……

\Alpha ああ!!
\ ナイさん!

\Nai はい

\Alpha ^{なまえ}{名前}の^{ゆらい}{由来}……
\ わかるよ、なんとなく

\Nai やっぱり?

\page[50]
\Alpha あー
\ そうだ
\ さっきはごめんなさい
\ ^{おお}{大}さわぎして……

\Nai さっきって?

\Alpha わたし、なんか^{かって}{勝手}に「ロボットは^{おんな}{女}の^{ひと}{人}」って^{おも}{思}ってたんで

\Alpha その……
\ びっくりしちゃった

\Nai ああ

\Nai いや、オレそうゆうのなれてるし

\page[51]
\Nai おれもロボットの^{ひと}{人}^{なんにん}{何人}が^{し}{知}ってるけど

\Nai まだ^{おとこ}{男}に^{あ}{会}ったことない……

\Nai ほかの^{ひと}{人}もみんなそうみたいでさ

\Nai めずらしがられるよどこでも

\Alpha ふーん

\Alpha じゃ、わたしすごいラッキーなのかな

\Nai そうかもね

\page[52]
\Nai なんか、^{き}{気}になる?
\ やっぱ

\Alpha あ!
\ あはは!
\ ごめんね
\ いや〜〜

\Alpha ^{しょうじき}{正直}^{い}{言}って^{すこ}{少}し

\Nai アルファが「^{おんな}{女}」だってくらいには、「^{おとこ}{男}」だよ

\Nai ^{み}{見}ためはふつうだと^{おも}{思}う……

\Alpha あ……
\ はい……
\ わかりました
\ なるほど

\page[53]
\Nai なんか……
\ ^{おとこ}{男}のロボットは^{よわ}{弱}いんだってさ

\Nai ^{き}{聞}いた^{はなし}{話}じゃ、^{はや}{早}いうちにみんな^{し}{死}
んじゃったって

\Nai なにが^{ちが}{違}うのかわかんないけど

\Alpha そうなんだ

\Nai うん

\page[54]
\Alpha でも^{こんや}{今夜}は^{たす}{助}かっちゃったよ

\Alpha ^{のじゅく}{野宿}^{かくご}{覚悟}だったもん

\Nai おばちゃんまぜて^{かわ}{川}の^{じ}{字}だけど

\Alpha うん
\ ありがと
\ でも、この^{へん}{辺}ほんとに^{いえ}{家}ないね

\Nai いちばん^{ちか}{近}い^{やど}{宿}でも、ずっとあっちでね
\ うなぎ^{や}{屋}で……

\Alpha ^{きょう}{今日}そっちから^{き}{来}たの

\page[55]
\Alpha あのさ

\Alpha あとで、^{ひこう}{飛行}^{き}{機}^{み}{見}せてほしいんだけど
\ いいかなあ

\Nai いいよ

\Nai あ、そうだ
\ あした^{はいたつ}{配達}なきゃ
\ ちっと^{ぶひん}{部品}かえてためし^{ひこう}{飛行}するけど……
\ ^{のる}{乗る}?

\Alpha え〜〜っ?!
\ ほんと?!
\ いいの?!
\ のる?!

\Alpha やった!!
\ ほんと!?
\ わ〜〜〜〜〜〜!!

\page[57]
\Nai ねむくなった?

\Alpha ん?
\ まだ^{へいき}{平気}

\Alpha ナイ……
\ あのさあ

\Nai うん

\page[58]
\Alpha なんか、^{しごと}{仕事}ない?

\Nai ないなあ


\subsection{第70話\ ^{みず}{水}}

\page[61]
\Alpha ナイー!!

\page[62]
\Alpha ほら、^{わたし}{私}も!
\ おんなじ!

\Nai えっ
\ あ、ほんとだ
\ めずらしい

\Alpha ^{わたし}{私}、なんとなく^{いっぴん}{一品}^{もの}{物}のつもりでいたよ

\Nai いや……あれ?
\ ちょっと^{み}{見}せて

\Nai わ……
\ これ、^{なみ}{並}の^{しよう}{仕様}じゃないよ

\Alpha え?

\Nai たぶん、^{ほんとう}{本当}に^{いっぴん}{一品}^{もの}{物}だと^{おも}{思}う

\Nai すごいの^{も}{持}ってるね

\page[63]
\Alpha えっ……あ
\ そうなの?

\Alpha これもらったものなんだ

\Nai ふーーん

\Nai いいのもらったなあ

\Alpha うん

\Nai ^{しゃしん}{写真}ほしがる^{ともだち}{友達}いてね

\Alpha へー

\page[64]
\Alpha ナイ〜〜
\ あの〜〜

\Alpha なんか、^{まえ}{前}^{み}{見}えないんだけど……

\Nai ああ
\ ザブトン^{つかう}{使う}?
\ あんまり^{か}{変}わんないと^{おも}{思}うけど

\Alpha ありがと

\page[65]
\Nai いくよ

\Alpha うん!

\page[66]
\Alpha ぐ

\page[68]
\Alpha ほんとに^{と}{飛}んでる

\page[69]
\Nai あ、そうだ
\ アルファ、カメラのコード^{いま}{今}、^{も}{持}ってる?

\Alpha うん

\Nai それ、くわえて^{まえ}{前}のボードの^{あな}{穴}にさしてみな、おもしろいから

\Alpha え?
\ うん……
\ これかな

\Alpha うぇ!!

\Nai なんかベロが^{かぜ}{風}に^{お}{押}される^{かん}{感}じしないか

\Nai これ、^{な}{慣}れるとベロで^{そくど}{速度}とかわかるんだ

\Alpha ベロ?

\page[71]
\Alpha ^{ひこう}{飛行}^{き}{機}がないよ!!

\Nai え?

\page[72]
\Alpha いきなり、はだかで^{そと}{外}に^{ほう}{放}り^{だ}{出}されたのかと^{おも}{思}った

\Alpha いろんな^{ちから}{力}が^{からだ}{体}じゅうにドツとおしよせる

\Alpha ナイも^{わたし}{私}も、しばらくわけがわからなかったけど

\Alpha ^{すこ}{少}しして^{ふたり}{2人}ともどういうことなのか、わかってきた

\page[73]
\Nai アルファ、あんた

\page[74]
\Alpha あー

\Alpha ^{みず}{水}の^{なか}{中}みたい

\Nai ^{みず}{水}?

\Alpha うん

\page[75]
\Sign ノースアメリカンATー6「てきさん」

\Sign ^{きち}{基地}のある^{まち}{町}にはよくころかこてますね

\Sign これはナイのKNー021「あっぎ2^{ごう}{号}」

\page[76]
\Sign ^{みず}{水}とお^{ちゃ}{茶}
\ かめ
\ 16^{ごう}{号}^{どお}{通}り


\subsection{第71話\ ^{たに}{谷}の^{みち}{道}}

\page[78]
\Alpha ^{ひこう}{飛行}^{き}{機}は^{きょうれつ}{強烈}だった

\Alpha あの^{ひ}{日}、^{みち}{道}に^{まよ}{迷}って、^{くうこう}{空港}に^{で}{出}てなかったら

\Alpha おばちゃんにも、ナイにも、^{あ}{会}えなかっただろう

\page[79]
\Alpha あれからしばらく16^{ごう}{号}^{どお}{通}りにある^{みず}{水}^{や}{屋}さんの^{てつだ}{手伝}いをしていた

\Alpha 16^{ごう}{号}^{どお}{通}りは^{たび}{旅}の^{ひと}{人}が^{おお}{多}い

\Alpha ^{まいにち}{毎日}いろんな^{はなし}{話}が^{き}{聞}ける

\Alpha あったかくなって、また、^{うご}{動}くことにした

\Alpha ^{ほんらい}{本来}なら、^{はちおうじ}{八王子}あたりで^{つぎ}{次}の^{しごと}{仕事}をさがすところだけど

\Alpha ひとつ、^{き}{気}になる^{はなし}{話}を^{き}{聞}いたので、そっちに^{む}{向}かって^{い}{行}く

\page[80]
\Alpha ^{おおどお}{大通}りをはずれて^{たに}{谷}の^{みち}{道}にはいる

\Alpha ^{いわ}{岩}や^{とうぼく}{倒木}だらけの^{やま}{山}の^{みち}{道}は、
^{むかし}{昔}、ひとつの^{まち}{町}のメインストリートだった

\Alpha もう、^{くるま}{車}の^{とお}{通}れない、アスファルトの^{うえ}{上}を^{こけ}{苔}がおおっている

\page[81]
\Alpha かつての^{しゃどう}{車道}を、ぬうようにつくられたせまい^{みち}{道}

\Alpha この^{みち}{道}にあるという^{へん}{変}な^{しょくぶつ}{植物}のうわさ

\Alpha ^{たび}{旅}の^{ひと}{人}が^{はな}{話}していた、^{しょくぶつ}{植物}というのは、この^{き}{木}のことだ

\page[82]
\Alpha どう^{み}{見}ても、^{こうえん}{公園}の^{がいろ}{街路}^{とう}{灯}にしか^{み}{見}えない

\Alpha でも、^{おお}{大}きさはバラバラだし、^{ね}{根}っこもあるみたいだし

\Alpha ^{たし}{確}かに、^{しぜん}{自然}のものではあるようだ

\page[83]
\Alpha まわりをよく^{み}{見}ると、あっちにもこっちにも^{は}{生}えている

\Alpha ^{がけ}{崖}の^{とちゅう}{途中}とか、やぶの^{なか}{中}とか

\Alpha ^{みち}{道}を^{い}{行}く^{ひとびと}{人々}は^{き}{木}のことを^{き}{気}にもとめない

\Alpha ^{ほそ}{細}くくねる^{みち}{道}

\Alpha この^{みち}{道}は^{ふじ}{富士}^{さん}{山}への^{ちかみち}{近道}なので、
^{ひとどお}{人通}りはそれなりにある

\page[85]
\Alpha ^{ゆうがた}{夕方}

\Alpha ^{くろ}{黒}い^{やま}{山}の^{かげ}{影}にだいだい^{いろ}{色}に^{ひか}{光}る
^{たに}{谷}の^{みち}{道}が^{う}{浮}かびあがる

\page[86]
\Alpha あの^{き}{木}が^{ひか}{光}りはじめた

\page[87]
\Alpha ^{き}{木}の^{ひかり}{光}は、^{みち}{道}を^{て}{照}らすちょうちんの
^{ひかり}{光}を^{むし}{無視}して、^{べつ}{別}のラインをつくりはじめる

\Alpha ^{たび}{旅}の^{ひと}{人}に^{き}{聞}いた^{はなし}{話}

\page[88]
\Alpha ^{き}{木}の^{ひかり}{光}が^{しめ}{示}すのは、かつての^{まち}{町}のあとをなぞるライン

\Alpha ^{みち}{道}の^{きおく}{記憶}

\Alpha ^{ひと}{人}が^{わす}{忘}れてしまっても


\subsection{第72話\ ササゲ}

\page[95]
\Takahiro おめーよー
\ ^{おめ}{重}えよ!

\Makki ^{しつれい}{失礼}な!

\Takahiro でー?
\ ^{きょう}{今日}はどこつれてけって?

\Makki きぬがさまでつれてってよ

\Takahiro きぬがさ!
\ ちっと^{とお}{遠}いな

\Makki どーせ^{きょう}{今日}ヒマなんでしょが

\Takahiro ^{しつれい}{失礼}な!

\page[96]
\Makki おまたへ

\Takahiro ^{なに}{何}、^{か}{買}ったのよ

\Makki これ?
\ ササゲ

\Takahiro ^{なに}{何}、ササゲって

\Makki ^{まめ}{豆}だよ

\Takahiro なんだ
\ ほかには?
\ もういいのか?

\Makki うん

\page[97]
\Makki ねえ
\ すなはまの^{ほう}{方}^{とお}{通}ってこうよ

\Takahiro いいけどよ
\ ちっと^{じかん}{時間}おせえから、^{およ}{泳}げねえぞ

\Makki いいよ
\ ^{よ}{寄}ってくだけでも

\Takahiro そか

\page[98]
\Takahiro ^{すなはま}{砂浜}には、ひさしぶりに^{き}{来}た

\Takahiro ^{しお}{潮}は、もう^{あ}{上}げていて、^{なみ}{波}も^{で}{出}ている

\page[102]
\Makki ん?

\Takahiro んーん


\subsection{第73話\ チョコレートケーキ}

\page[104]
\Sign ハニー
\ とうもろこし

\Person ^{いっぽん}{一本}ちょうだい

\Alpha あっ
\ はーい!

\page[105]
\Alpha あのね
\ ちょっと^{ま}{待}ってもらえれば、^{や}{焼}きなおしますよ

\Person んーー
\ じゃ^{や}{焼}いて

\Alpha はい

\Alpha うちのは^{さんち}{産地}^{ちょくそう}{直送}だからね、
^{れいとう}{冷凍}なんかと^{ちが}{違}っておいしいですよ〜〜

\Person あー
\ ^{れいとう}{冷凍}のは^{く}{食}えたもんじゃねえもんな

\Person ふい〜

\Person ここよ!
\ あれだよなー

\Alpha はいー?

\page[106]
\Person ^{め}{目}が^{よ}{良}くなるよなー

\Alpha そうですねー

\page[108]
\Alpha いつも^{うみ}{海}のむこうに^{き}{切}り^{かみ}{紙}みたいに
^{う}{浮}かんでいた^{ふじ}{富士}^{さん}{山}が^{いま}{今}、
^{め}{目}の^{まえ}{前}にドカンとある

\Alpha はーーい
\ おまたせ

\Person うい

\Person あ、そうだ
\ なんか^{の}{飲}み^{もの}{物}ある?

\Alpha ええ

\page[109]
\Alpha お^{ちゃ}{茶}とコーヒーとゆずジュースがありますけど

\Person んー、じゃあ、お^{ちゃ}{茶}

\Alpha はい

\Sign カラス
\ ^{むぎ}{麦}

\Person なにそれ……
\ あー、^{むぎちゃ}{麦茶}ね

\Alpha ありあとあしたー

\page[110]
\Alpha ^{ちか}{近}くで^{み}{見}る^{ふじ}{富士}^{さん}{山}はなんか^{えんきん}{遠近}^{かん}{感}がおかしくて

\Alpha ^{め}{目}が^{お}{追}いつかないほど^{おお}{大}きいのに、ホイと^{て}{手}でさわれそうに、^{み}{見}える

\Alpha なんか、^{や}{柔}っこそうで^{いろ}{色}も^{かん}{感}じも、ちょうどココアパウダーがふってあるみたいで……

\Alpha うまそうっす……

\page[111]
\Alpha ^{ふじ}{富士}^{さん}{山}の^{みなみがわ}{南側}が^{とお}{通}れないので、
こんなせまい^{みち}{道}でも^{くるま}{車}がよく^{とお}{通}る

\Person あの〜〜
\ もっと^{ふじ}{富士}^{さん}{山}の^{ちか}{近}くに^{い}{行}ける^{みち}{道}ないですかねえ

\Alpha ああ〜〜
\ ないみたいですよ

\Person そーすか〜〜
\ ども

\Alpha ^{だいはんじょう}{大繁盛}だねー

\page[113]
\Alpha ^{きいろ}{黄色}
\ プロペラひとつ

\Alpha おーい!!
\ おーい!!

\page[114]
\Alpha お〜い!!

\page[118]
\Alpha ん


\subsection{第74話\ ^{もうまく}{網膜}}

\page[120]
\Sign アトリエ
\ ^{まる}{丸}^{こ}{子}

\page[123]
\Kokone ^{まる}{丸}^{こ}{子}さん
\ あの〜〜
\ もっと^{かる}{軽}〜くしてもらわないとー……

\Kokone こまります

\Maruko いや〜〜

\Maruko ひさしぶりだったからねえ

\Maruko もりあがっちゃったよ
\ おいちゃん

\Kokone おいちゃんなんですか

\page[124]
\Maruko じゃー
\ チェックね

\Kokone はい

\Maruko ひさしぶりに^{とお}{遠}くの^{ともだち}{友達}から^{けしき}{景色}が^{おく}{送}られてきた

\Maruko ^{おく}{送}られてくる^{けしき}{景色}はそいつ^{この}{好}みのコントラ
ストの^{つよ}{強}い^{いろ}{色}あいで

\Maruko ^{わたし}{私}は^{す}{好}きだ

\Maruko はじめて、ココネさんに、^{じか}{直}に^{わた}{渡}してもらった^{とき}{時}はびっくりした

\Maruko ^{かのじょ}{彼女}を^{とお}{通}して^{み}{見}る^{ふうけい}{風景}は、リアル^{かん}{感}がちがう

\page[127]
\Maruko ひさしぶりに^{み}{見}る^{ともだち}{友達}の^{ひこう}{飛行}^{き}{機}

\Maruko その^{まえ}{前}で、じっと、^{わたし}{私}を^{み}{見}ている、あずき^{いろ}{色}の^{ひとみ}{瞳}の^{ひと}{人}

\Maruko ただでさえリアルなまわりの^{ふうけい}{風景}より、^{なんばい}{何倍}もくっきりとそこにいる

\Maruko ^{みどりいろ}{緑色}の^{かみ}{髪}のこの^{ひと}{人}

\page[128]
\Kokone な……なにか^{へん}{変}な^{ところ}{所}ありました?

\Maruko んーん
\ そうじゃなくてね

\Maruko ココネさんやっぱ、すごいなって……

\Maruko はい!
\ サインするよ

\Kokone あっ
\ はい
\ どうも

\page[129]
\Maruko また、よろしくね

\Kokone はい

\Maruko じゃ
\ ^{こんど}{今度}、^{あそ}{遊}ぼうよ

\Kokone あ、はい
\ ありがとうございました

\page[130]
\Maruko アルファさんか

\Maruko ぱっと^{み}{見}て、なんとなくわかった

\Maruko ^{いま}{今}、^{たび}{旅}に^{で}{出}ていると^{き}{聞}いてたけど

\page[131]
\Maruko ^{ふたり}{二人}の^{ともだち}{友達}をいっぺんに^{と}{取}られたような

\Maruko ^{きも}{気持}ちにもなった
\ けどね

\page[132]
\Maruko ナイ
\ さあ
\ ^{てがみ}{手紙}くらいはそえてよね


\subsection{第75話\ ^{のび}{野火}}

\page[134]
\Alpha ^{おおたるみ}{大垂水}^{とうげ}{峠}をこえて、むさしのの^{くに}{国}にはいる

\Alpha ^{やま}{山}をおりるのは^{かんが}{考}えてみれば、ひさしぶりだ

\page[135]
\Sign やきそば
\ …ステラ

\Alpha ^{ひの}{日野}という^{だいち}{台地}にさしかかると、^{きゅう}{急}に、にぎやかになってきた

\Alpha ^{だいち}{台地}が^{とぎ}{途切}れる^{がけ}{崖}の^{ところ}{所}に、^{ひと}{人}が^{あつ}{集}まっている

\page[136]
\Alpha ^{め}{目}の^{まえ}{前}にひろがるムサシノの^{はら}{原}を^{おお}{大}きな^{くも}{雲}がおおっている

\page[137]
\Alpha ^{けむり}{煙}だ

\Alpha ムサシノのすすきの^{のはら}{野原}が、みんな、^{も}{燃}えている

\Alpha こういった^{かじ}{火事}は^{なんねん}{何年}かに^{いちど}{一度}はあるそうだ

\Alpha ^{ひ}{火}は、^{しぜん}{自然}に^{き}{消}えるまで^{ほう}{放}っておくしかないという

\Alpha それまで、ここから^{ひがし}{東}へは^{い}{行}けない

\page[139]
\Sign …でん
\ あま^{さけ}{酒}

\Alpha ^{かじ}{火事}は、あと^{みっか}{三日}は^{つづ}{続}くという

\Alpha あの^{ひ}{火}のむこう^{がわ}{側}に^{い}{行}きたかったけど……

\page[140]
\Alpha ^{あした}{明日}、^{みなみ}{南}の^{ほう}{方}に^{い}{行}くことにした

\Alpha ^{きょう}{今日}は、ここで、^{ひ}{火}を^{み}{見}ていようと^{おも}{思}う


\subsection{第76話\ ^{くり}{栗}}

\page[142]
\Alpha ^{みなみ}{南}へ^{む}{向}かう^{ちょくせん}{直線}の^{みち}{道}

\Alpha もう、ずうっとこんな^{ふうけい}{風景}が^{つづ}{続}いている

\Alpha たき^{ひ}{火}の^{けむり}{煙}のにおい^{いがい}{以外}に、^{ひと}{人}の^{けはい}{気配}はない

\page[143]
\Alpha どっちを^{むか}{向}いても^{やま}{山}の^{なか}{中}みたいな^{みち}{道}

\Alpha でも、ときどき、^{こ}{濃}い^{しお}{潮}のにおいがまじる

\Alpha す〜

\Alpha 「^{かえ}{帰}ってきている」と^{おも}{思}った

\Alpha ほんとに、^{いちねん}{一年}も^{である}{出歩}いていたんだろうか

\page[144]
\Alpha この^{しま}{島}は、^{ちず}{地図}を^{み}{見}て^{そうぞう}{想像}するのより、ずっとずっと^{ひろ}{広}い

\Alpha ^{いちねん}{一年}もあれば、もっと^{にし}{西}やもっと^{きた}{北}の「やまと」とか「みちのく」とかの
エリアにだって^{い}{行}けると^{おも}{思}ってた

\Alpha でも、まー
\ ^{よ}{良}しとしよう

\page[145]
\Alpha ^{みち}{道}はただただ^{みなみ}{南}へのびている

\page[146]
\Alpha そういえば、きのうからなにも^{た}{食}べてなかったっけ

\Alpha なんか^{ひと}{人}も、ひさしぶりに^{み}{見}た

\page[147]
\Alpha こ
\ こんにちは

\Alpha いい〜〜い^{てんき}{天気}ですね〜〜

\Person あ?
\ あーー
\ そうなー

\Alpha じゃ

\Person んー

\Alpha あの〜〜
\ この^{さき}{先}なんか、^{た}{食}べさせてくれるお^{みせ}{店}とか、
あります?

\Person あ?

\page[148]
\Person この^{さき}{先}?
\ ずーっと^{うみ}{海}までこんな^{はやし}{林}ん^{なか}{中}だよ

\Alpha ふっ

\Alpha そ
\ そですか
\ ども

\Person あー
\ ねえさんよ

\Person ^{くり}{栗}あんけど、^{く}{食}う?

\Alpha あああ
\ ありがとうございます

\page[149]
\Person ^{くり}{栗}^{さんこう}{山行}った^{かえ}{帰}りでよー
\ よかったな

\Person ほら
\ これにすんか?

\Person ^{なんこ}{何個}くう?

\Alpha あ
\ ^{いっこ}{1個}で……

\page[150]
\Person じゃ
\ おれも^{いっこ}{一個}いってみんかな

\Person 30^{っぷん}{分}くらい^{ま}{待}ちな

\Alpha はい

\Alpha ^{わたし}{私}、^{くり}{栗}^{や}{焼}くのはじめて^{み}{見}ます

\Alpha たいてい、^{に}{煮}ちゃいますよねー

\Person そうか?

\page[151]
\Person よっ

\Person ^{いちおう}{一応}ね

\page[154]
\Person どこ^{と}{飛}んでくかわかんねえからよ

\Person あぶねえよな

\Person あちちちちち

\Person ほい
\ あちいよ

\Alpha おー

\Alpha はお〜

\page[157]
\Sign たべきれないっス


\subsection{アルファの^{さくもつ}{作物}の^{あらし}{嵐}!}
\Alpha 「^{もも}{桃}^{くり}{栗}^{さんねん}{三年}、^{かき}{柿}^{はちねん}{八年}、
  ^{なし}{梨}のバカのが^{じゅうはちねん}{十八年}」そか^{い}{言}いますネ

\Sign ^{さんねん}{三年}^{くり}{栗}

\Alpha なでごこちがたまらん

\Sign ^{はちねん}{八年}^{かき}{柿}

\Sign ^{さんねん}{三年}^{もも}{桃}

\Alpha 4〜5^{こ}{個}がまとまってつく。^{たま}{玉}の^{おお}{大}きさはふつう

\Alpha いちばん^{うえ}{上}の。

\Alpha ひとつがとてつもなくうまい。

\Alpha まんなかの^{あじ}{味}は^{なみ}{並}。

\Alpha ^{さいこう}{最高}にまずい。

\Sign ^{じゅうはちねん}{十八年}^{なし}{梨}は・・

\Alpha くくくく・・
