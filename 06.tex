\section{Volume 6}

\subsection{^{だい}{第}43^{わ}{話}\ ひとりおどり}

\page[2]
\A ^{わたし}{私}のカメラ

\A いつも、^{わたし}{私}を^{そと}{外}に^{つ}{連}れ^{だ}{出}す

\page
\A でも、^{きょう}{今日}みたいにいかにも、なにか^{と}{撮}りそうに^{で}{出}てる^{ひ}{日}は、
たぶん、バッグからも^{だ}{出}さない

\page
\A ^{ある}{歩}く^{みち}{道}を^{えら}{選}ぶようになった
\ ^{わたし}{私}はどんな^{とこ}{所}を^{とお}{通}るのが^{す}{好}きなのか

\A そうやってのろのろと^{ある}{歩}いてる
\ ここは^{くうき}{空気}と^{じめん}{地面}の^{さかいめ}{境目}

\page
\A こういう^{とき}{時}、^{わたし}{私}は^{くうき}{空気}^{じゅう}{中}に
つきだした^{じめん}{地面}の^{で}{出}っぱりになる

\page
\A ^{で}{出}っぱりな^{わたし}{私}はまわりを^{み}{見}ながら^{ある}{歩}いてて

\A ^{くち}{口}からは^{で}{出}まかせの^{うた}{歌}

\A くりかえす^{ほはば}{歩幅}のリズム

\A ^{ふ}{振}りが^{おお}{大}きくなっていく

\page[8]
\A やっぱりカメラは^{で}{出}なかったな

\A ^{へん}{変}な^{うた}{歌}とおどりは^{うち}{家}まで^{あと}{後}ひいた


\subsection{第44話\ ^{ほし}{星}の^{め}{目}\ ^{ひと}{人}の^{め}{目}}

\page[11]
\A うちもお^{きゃく}{客}さん^{こ}{来}ないけど……

\O うちもまけてねえな

\A おじさん、ガソリンってさあ

\O くさるよ

\page[15]
\P どう?
\ ^{した}{下}は

\page
\AM ええ
\ あいかわらずっていうより

\AM ^{わる}{悪}くなってくみたいです

\AM ここから^{み}{見}る^{かぎ}{限}り

\P うーーん

\page
\AM いきなりなくなる^{まち}{町}ですとか……ちらほらと

\AM でも

\AM なんていうか
\ こう

\AM ^{まえ}{前}みたいなやな^{かん}{感}じがしないっていうか

\P なんとなくわかるよ

\page
\P まあ、でも、^{わたし}{私}らにゃー

\P なんでなのかたしかめようもないやね

\AM ええ

\AM あっちでなにがあっても^{とお}{遠}くから^{み}{見}てるだけ

\P ここからぜいたくな^{い}{言}えないけどね

\page[20]
\AM ^{わたし}{私}たち
\ なかまはずれなのかなー

\P さあ
\ どうかねえ

\page
\P さと
\ アルファー^{しつちょう}{室長}

\P ^{しんちゃ}{新茶}があるんだけど、^{あじ}{味}^{み}{見}してみない?

\AM あ、はい、^{わたし}{私}も^{い}{行}きます

\page[23]
\A あ!
\ ^{ね}{寝}に^{よ}{寄}ったんじゃないんだ

\O う?

\O あ〜〜
\ わりいな
\ こっちのペースにつきあわしちゃってよ

\A いやーー
\ ペースは^{に}{似}たようなもんなんですけどね

\A っん〜〜

\page
\A こりゃー
\ ガソリンもくさっちゃうねー

\O あーー
\ くさるくさる


\subsection{第45話\ みんなの^{ふね}{船}}

\page[26]
\T で
\ ^{か}{買}うのってこいだけだったっけか

\M うん
\ おつりはくれるってさ!

\page
\T そっか

\T で〜〜
\ マッキよう

\M ん?

\T ^{あした}{明日}あたりアルファんとこ^{い}{行}かねえか

\T サラッとよ

\M えっ

\M あ〜〜
\ あした〜〜?

\page
\T ま〜〜
\ ^{あした}{明日}じゃなくても
\ ^{ちか}{近}えうちによ

\M あ!
\ べつに^{い}{行}きたくないわけじゃないよ!

\T おう
\ わかってんよー

\T あそこは^{とお}{遠}いしなー

\page
\T でも、まー
\ なにしろ^{いっかい}{一回}^{あ}{会}わしてみてーわけよー

\M はあ
\ そーすかー

\page
\A ありゃ!
\ やっぱタカヒロ!!

\T えっ?

\T あ
\ アルファ!!

\page
\T アルファも^{か}{買}いもの?

\A うん

\A めずらしいね、^{みなみまち}{南町}で^{あ}{会}うのなんか

\A で
\ ひょっとしてマッキちゃん?!

\T そう!

\T マッキ、ほら!
\ アルファだよ!

\page
\M よっ!!

\A よ

\page
\A タカヒロからいつも^{き}{聞}いてるよ
\ はじめまして

\M は
\ はじめ〜〜

\T なにキンチョーしてんだよ!

\A ね
\ ふたりとも^{いま}{今}ヒマ?

\M えと
\ これからタカヒロと

\T うん
\ ヒマ!!

\A そっか
\ じゃ、かるーくなんか^{た}{食}べてこっか

\T うん

\page
\Sign ところてん

\A ^{ふたり}{二人}はみつまめでもなんでもよかったのに

\T ところてん^{す}{好}きだよ

\T でも、なんか、ひさしぶりだね、アルファと^{あ}{会}うの

\A そうね

\T このごろそっち^{い}{行}ってなかったしなーー

\T あはは

\M ^{こえ}{声}がちがうぞ、てめー

\page
\A マッキちゃんてさあ、けっこうおとなしいんだね

\M え?
\ そ
\ そんなことないよ

\T うん、いつもはすごくうるさいんだけどね
\ あがってんだよ

\M ことばづかいちがうぞ、てめー

\T ちょっとトイレ
\ ^{ふたり}{二人}で^{はな}{話}しててな

\M えっ

\M ばかやろ

\page
\M ア……

\M アルファさってさあ

\A うん

\M ^{としした}{年下}の^{おとこ}{男}の^{こ}{子}が^{す}{好}きなの?

\A はあ?

\page
\M タカヒロが^{す}{好}きなの?

\M タカヒロはアルファさんが^{だい}{大}^{す}{好}きなんだよ

\A マッキちゃんはタカヒロが^{す}{好}きなんだね

\M ま
\ まあね

\A ^{わたし}{私}もタカヒロ^{す}{好}きだよ

\page
\A うーーん
\ でも、タカヒロのことじゃ、^{わたし}{私}マッキちゃんがうらやましいよ

\A マッキちゃんとタカヒロは^{おな}{同}じ^{じだい}{時代}に^{の}{乗}ってるんだもん

\A ^{わたし}{私}も、^{いま}{今}はみんなといっしょにいるけど

\A これからも^{おな}{同}じ^{じだい}{時代}の^{ひと}{人}って^{い}{言}えるのか、わかんないし

\A マッキちゃんはタカヒロと^{じかん}{時間}も^{からだ}{体}もいっしょの^{ふね}{船}に^{の}{乗}ってる

\A ^{わたし}{私}は、みんなの^{ふね}{船}を^{きし}{岸}で^{み}{見}てるだけかもしれない

\page
\A マッキちゃんとタカヒロは、ずっといっしょなんだよね

\A それがうらやましいよ

\M なんか、よくわかんないけど

\M まあ
\ いいや

\A そっか

\M こんど、お^{みせ}{店}に^{い}{行}くよ

\M タカヒロぬきでも

\A ^{き}{来}て^{き}{来}て

\page
\T いや〜〜

\T あれ?

\A ……でね〜

\M か〜

\T なんか、^{かん}{感}じがちがう……

\A あ!

\T え?
\ なに?
\ どしたの?


\subsection{第46話\ レコード}

\page[44]
\K ^{ちょうない}{町内}めぐり

\K ^{としょ}{図書}^{かん}{館}があれば、とりあえず^{よ}{寄}ってみて

\K たいていボロットの^{ほん}{本}をさがします

\K アルファ^{かた}{型}ロボットのデータ^{しゅう}{集}みたいな^{ほん}{本}はわりとどこにでもあるけど

\K 「アルファシリーズにまつわる^{はなし}{話}」なんていう^{ほん}{本}は、
なかなか^{み}{見}つからなくて……

\page
\K だから、そういったことは

\Sign アルファ^{かた}{型}^{き}{機}^{の }{之}^{いしぶみ}{碑}

\K いまだによくわかりません

\K それでも、まだ「Aー7」についてはいい^{かた}{方}です

\K Aー6^{いぜん}{以前}の^{こと}{事}になると、もうなにもありません

\page
\Sign ^{せたがや}{世田谷}\ ^{ちょうりつ}{町立}
\ ^{きぬた}{砧}^{じどう}{児童}^{かん}{館}

\K あーー
\ ^{じどう}{児童}^{かん}{館}
\ いいよねー
\ ちょっとわびしくて

\K こんにちは!

\Sign ^{にゅうかん}{入館}のきまり

\page
\K わびしい

\K はいります\ よー

\K そーー
\ このボロボロの^{えほん}{絵本}とかね

\Sign よみおわった
\ ^{ほん}{本}は
\ もとにもどし
\ ましょう

\page
\K だれもいないのかな

\page
\Sign ^{きぞう}{寄贈}^{としょ}{図書}…

\K あ

\Sign アルファタイプ^{ぎ}{議}
\ Aー7^{かいはつ}{開発}

\K こんなとこにもおなじみの^{ほん}{本}

\page
\Sign Aー2
\ (M1ーM4)


\subsection{第47話\ レコード・Ⅱ}

\page[53]
\K 「Aー2」と^{か}{書}かれた^{ふる}{古}いボール^{かみ}{紙}のレコードジャケット

\Sign Aー2
\ (M1ーM4)

\K ^{なか}{中}には^{おおむかし}{大昔}の^{はり}{針}^{しき}{式}レコードが2^{まい}{枚}はいってました

\page
\K ジャケットには^{しゅうろく }{収録}^{ないよう}{内容}の^{せつめい}{説明}もなく

\K ^{み}{見}た^{め}{目}は^{う}{売}る^{け}{気}のない
\ ^{じしゅせいさく}{自主制作}^{はん}{版}という^{かん}{感}じです

\K レコードにはスタンプで^{うらおもて}{裏表}にMー1からMー4までふってあって……

\K これは、やっぱり、どうしても^{じぶん}{自分}^{かって}{勝手}な^{そうぞう}{想像}がひろがってしまう

\page
\K ただの^{ぐうぜん}{偶然}だとも^{おも}{思}うけど、でも……

\K そだ、プレイヤー、プレイヤーは……

\page
\K あっ

\P あっ

\K かっ
\ ^{かんり}{管理}^{にん}{人}さんですか?

\P えっ?
\ あ
\ いや
\ まあね

\K あっ
\ あの
\ これ^{き}{聴}き……
\ ^{か}{借}りたいんですけど!

\P えっ

\P あー
\ ごめん
\ ここのやつ^{も}{持}ち^{だ}{出}し^{きんし}{禁止}なんだよ

\page
\P ま〜〜
\ といって、^{いま}{今}、プレイヤーもないんだけどね、ここ
\ ^{わる}{悪}いけど

\K あーー
\ そうですか

\P あんたー
\ プレイヤーもってんの?

\K いえ……
\ ただ、これ^{き}{気}になっちゃって

\page
\P そーかー

\P んーー
\ ^{むかし}{昔}、^{いっかい}{1回}だけきいてみたことあったけどね、それ

\K どっ
\ どんなでした?!

\P え?!
\ いや……なんか、まぬけな^{おんがく}{音楽}だったよ

\P ^{とちゅう}{途中}でねちゃったし

\K はあ

\K そうですか

\K ^{おんがく}{音楽}……

\page
\P まあ、アンプやら、^{はり}{針}やらで、^{おと}{音}も^{か}{変}わるだろうけどね

\P ^{いま}{今}じゃ、どうしようもないなあ

\K いつかプレイヤーさがしてきますから
\ ずっと^{お}{置}いといてくださいね!

\P はいよー

\P こだわるコだなー
\ マニアだな

\page
\K やっぱり
\ なんとなく

\K あのレコードのAー2と^{わたし}{私}のAー7には^{かんけい}{関係}があるような^{き}{気}がする

\page
\K ひょっとしたら

\K 「Aなんとか」っていうのはボロットだけの^{なまえ}{名前}じゃないのかもしれない

\K たとえば、^{おんがく}{音楽}なんかも……

\page
\K ただの^{くうそう}{空想}の^{はなし}{話}になっちゃうけど

\K でも、もしかしたら

\K ^{わたし}{私}の^{ち}{血}の^{なか}{中}には^{おと}{音}が^{なが}{流}れてる

\page[64]
\K わからないこと

\K わかる^{とき}{時}まではこの^{こた}{答}えにしておきます


\subsection{第48話\ レコード・Ⅲ}

\page[68]
\P じゃあ、^{こんかい}{今回}は、こいつでやろうか

\P おめえも^{ものず}{物好}きっつうか、ヒマっつうか

\P ^{でんとう}{電灯}^{み}{見}て、^{さけ}{酒}のめるかふつう

\P あんたもなー

\page[70]
\P でも
\ ^{みょう}{妙}だよなー

\P こんなクズ^{み}{見}て、なんでこういう^{きぶん}{気分}になってくんのか

\P オレよう

\P ^{ひと}{人}って、なんか^{ね}{根}っこの^{ほう}{方}が、
^{ひかり}{光}とか^{おと}{音}とかで^{うご}{動}いてんじゃねえかって^{おも}{思}ってんだけど

\P そういった^{かんけい}{関係}かもなー
\ これも

\page
\P そりゃどうだかなー

\P ばはは

\P まー
\ もう^{いっぱい}{一杯}

\P んー


\subsection{第49話\ あついすな}

\page[78]
\A はい!
\ ここが“^{すなはま}{砂浜}“!

\T は〜〜

\T ほんとに^{すな}{砂}だけの^{かいがん}{海岸}だ

\A いいでしょーー

\page
\A でも、ほら^{み}{見}て

\A あの^{ぼう}{棒}のとこから、あっちは^{そこ}{底}なしになってるからー

\A あの^{す}{洲}よか、こっちっ^{かわちゅうしん}{側中心}に^{せ}{攻}めるってことでねー

\T わかった

\M そこなし…

\page
\M でわ〜

\page
\A ^{ふたり}{二人}ともー
\ ^{みずぎ}{水着}は?

\T え?

\T もってないよーー!
\ あんまし^{およ}{泳}がないもん

\T あっぢ〜〜
\ あぢゃぢゃぢゃ

\A あ
\ そうだよね

\page
\M ぬるいねえ

\T アーールファー

\A まっててー
\ すぐ^{い}{行}くー!

\page[84]
\T つめてー!!

\A ^{しお}{潮}、^{あ}{上}げてきたね

\A そろそろあがろうかー

\page
\T あっという^{あいだ}{間}だ

\A うん

\M ここ
\ もっと^{ちか}{近}ければいいのにね

\T あっぱ

\T ^{みずぎ}{水着}ほしいよなー

\A よし!
\ ^{つぎ}{次}は^{ぜんいん}{全員}^{みずぎ}{水着}だ!

\A ぶん

\page
\T あーー
\ すげえやけたー

\M きょうはおフロだめだー

\T アルファもやけたねえ

\A えへへ

\A ^{わたし}{私}のは、^{よる}{夜}にはもどっちゃうけどね


\subsection{第50話\ ^{みず}{水}の^{とけい}{時計}}

\page[90]
\T ^{みなみまち}{南町}で、おばさんに^{みずぎ}{水着}^{か}{買}ってもらった
\ マッキといっしょに

\T マッキは^{かえ}{帰}ってからすぐ、^{みずぎ}{水着}になって

\T ^{ひとり}{一人}で^{うみ}{海}に^{い}{行}ったんだって

\page
\T でも、^{ゆうがた}{夕方}ちかくのカナカナが^{な}{鳴}くような、^{いりえ}{入江}ってのは

\T けっこうおっかないでしょ

\T ^{じゅっぷん}{10分}もしないで^{かえ}{帰}ったんだけど

\T ^{すこ}{少}し^{あ}{開}けた^{かわ}{川}のある^{ところ}{所}

\T ^{はし}{橋}をわたるとき

\page
\T いきなり、ま^{よこ}{横}に^{た}{立}ってたんだって

\T マッキは、みさごに^{あ}{会}うの^{にかいめ}{2回目}だったけど

\page
\T ぜんぜんうごけなかったって

\P マッキちゃんじゃんかよ

\M びくっ!!

\page
\A ふ〜〜ん

\A ^{さいきん}{最近}マッキちゃんミサゴづいてるね

\T うん
\ そう

\page
\A それで
\ ^{いま}{今}、どうしてる?
\ ^{かのじょ}{彼女}

\T マッキ?
\ ^{か}{変}わんないよ
\ ぜんぜん、^{き}{気}にしてないみたいだ
\ うるさいし

\A そっか!

\A なにかとがんじょうだよねー
\ あの^{こ}{子}

\T みさごは……

\page
\T みさごには……
\ マッキが「^{あ}{会}いたい^{こども}{子供}」になったのかなー
\ おれじゃなくて

\A そうかもしれないね

\T ぴくっ

\page
\A マッキちゃんに^{あ}{会}ってから

\A ミサゴにはタカヒロが、もう^{こども}{子供}に^{み}{見}えなくなったのかもしれない……
\ ね

\page
\A タカヒロ
\ ^{すこ}{少}し^{おお}{大}きくなったもんね

\page
\T ん〜〜
\ そうかな

\T でも、やっぱ
\ このまんまじゃきついよなー

\T ^{かんが}{考}えてみれば
\ ^{いっかい}{1回}も、ちゃんと^{あ}{会}ってないんだ
\ みさごに

\page
\A でも
\ あんがい
\ みんなそうなのかもしれないよ

\T アヤセとかも

\A うん

\A ……で
\ きっと、マッキちゃんも……

\page
\T あと^{いっかい}{1回}だけでいいんだけどなあ

\T 「よう」とか^{い}{言}えるだけでもいいんだ

\T もう^{あ}{会}えないって

\T まだ、^{き}{決}まったわけじゃない

\page[104]
\A ^{じかん}{時間}の^{なが}{流}れはみんなに^{ひとつ}{1個}ずつあって

\A とまらない


\subsection{第51話\ やまのあな}

\page[107]
\A あづい〜〜〜
\ ^{かぜ}{風}もない〜〜

\A ^{せかい}{世界}^{じゅう}{中}こんななのかな〜〜

\page
\Y さみ〜〜

\Y にいがたからぐんまの^{くに}{国}への^{どうちゅう}{道中}

\Y メインイベントの^{やまご}{山越}えだったわけだが

\Y ^{ゆざわ}{湯沢}のあたりでオヤジにだまされた

\Y くそー

\page
\P 「^{はなし}{話}のタネに^{いちど}{一度}^{ある}{歩}いときな^{だい}{大}トンネル」
\ 「^{みくに}{三国}^{とうげ}{峠}の^{ほう}{方}はふつーに^{ご}{越}えたい
  ^{ひと}{人}だけが^{い}{行}く^{みち}{道}だよ!」

\Y 「あー、やっぱ、そうすか〜〜
  \ そりゃ^{ある}{歩}いとかねえと……」

\P 「このライトを^{つか}{使}いなさい!」
\ 「^{やす}{安}くしとくから!」

\page
\Y だれも^{とお}{通}りゃしねえじゃねえか!

\Y トンネルは10キロ^{いじょう}{以上}もあるという

\Y もう
\ ^{ひ}{引}き^{かえ}{返}せる^{きょり}{距離}でもなく……
\ ^{ぜんぽう}{前方}には^{えいえん}{永遠}のような^{やみ}{闇}と

\Y ^{えいえん}{永遠}のような^{けいこうとう}{蛍光灯}の^{てんてん}{点々}のライン

\page
\Y カマスのやろうは^{うみ}{海}から^{はな}{離}れて^{いらい}{以来}
ずっとカバンの^{なか}{中}でねている

\Y オレのことバスかなんかと^{おも}{思}ってんじゃねえのか?

\Y ^{しょうめい}{照明}
\ ^{くれ}{暗}えなあ

\page
\Y ライトも^{くれ}{暗}えぞ

\Y えっ!?
\ あれっ!?

\Y こら!!

\Y あんだえ〜
\ あのオヤジ〜〜

\page
\Y あとはあの^{くら}{暗}い^{けいこうとう}{蛍光灯}がたよりだ

\Y ^{てんてん}{点々}だけを^{み}{見}て^{ある}{歩}く

\Y だんだん^{へん}{変}な^{きぶん}{気分}になってくる

\Y ん〜〜
\ やべえかなこりゃー

\page[115]
\Sign MIDWAY
\ ^{とうげ}{峠}^{の}{之}^{ちゃ}{茶}

\P おつかれ^{にい}{兄}ちゃん

\Y ふ〜〜

\Y い……
\ ^{いのち}{命}びろいって^{かん}{感}じっすよ

\P まーー
\ そー
\ ^{し}{死}ぬこともねえけどな

\P ^{め}{目}まわして、^{はんたい}{反対}に^{ある}{歩}ってっちゃうヤツは
^{おお}{多}いんだよ、^{じっさい}{実際}

\P なんかのむ?

\Y じゃ、^{えだまめ}{枝豆}と^{むぎちゃ}{麦茶}…

\Y わかります

\page
\Y ひょっとして、ずっとここに?

\P ^{いま}{今}は^{かよ}{通}いだけどな

\P いずれは^{す}{住}むかもな

\P ほらよ
\ ^{でんち}{電池}^{ぎ}{切}れだな

\Y ああ、ありがとうございます

\page
\Y じゃ

\P おう
\ あと^{はんぶん}{半分}だ

\Y ^{みせ}{店}が^{ちい}{小}さくなって
\ また^{てんてん}{点々}の^{うちゅう}{宇宙}

\Y でも、^{いま}{今}は、この^{みち}{道}がおもしろい

\Y ライトを^{け}{消}して^{ある}{歩}いてみる

\Y ^{ゆざわ}{湯沢}のオヤジは^{ただ}{正}しかった

\page[119]
\Y ^{からだ}{体}に^{べつ}{別}の^{くうき}{空気}があたる
\ ^{たいへいよう}{太平洋}のにおいだ

\Y まだ^{うみ}{海}じゃねえよ

\page
\A やっと^{かぜ}{風}、^{で}{出}てきたなーー


\subsection{第52話\ NIGHTBIRD}

\page[122]
\A ん〜〜ん

\A ふーう

\page
\A お

\page
\A めずらしく^{ゆうびん}{郵便}がきてました、ココネからです

\A あいかわらずやたらたくさん^{か}{書}いてある^{てがみ}{手紙}

\A いっしょの^{こづつみ}{小包}のことなんかほとんど^{はし}{端}っこの^{ほう}{方}に^{い}{行}っちゃってます

\page
\K 「あ、そうだ、^{こづつみ}{小包}の^{ほう}{方}はですね、お^{さけ}{酒}なんですよ」

\A お^{さけ}{酒}かー

\K 「アルファさん、^{いぜん}{以前}にミルクは^{にがて}{苦手}だけど、
  ^{はや}{早}く^{な}{慣}れたいって^{い}{言}ってましたよね」

\K 「これはコーヒーのお^{さけ}{酒}です、ミルクで^{わ}{割}ってのむんですよ」

\K 「すごく^{あま}{甘}いので、アルファさんもきっと^{き}{気}に^{い}{入}ると^{おも}{思}います」

\A ^{あじ}{味}の^{かた}{方}はともかく……
\ ココネ

\A きみは^{ぎゅうにゅう}{牛乳}の^{はなし}{話}の^{じゅうよう}{重要}な^{ぶぶん}{部分}を^{わす}{忘}れている

\page
\A まー
\ お^{さけ}{酒}はキライじゃないけどね〜〜

\A ^{よわ}{弱}いけど

\page[128]
\A こりゃ〜〜なんだか、^{わたし}{私}にもいけそうな^{き}{気}がするわ!

\A ひょっとしたら、^{わたし}{私}も、このお^{さけ}{酒}で……

\A ^{ぎゅうにゅう}{牛乳}くらいは^{な}{慣}れていけるかもしれない!

\A カフェオレでしびれてちゃー
\ ^{なさ}{情}けないもんね

\page
\A く

\A うまい!!
\ って^{い}{言}うか
\ あまい!!

\A よ
\ よ〜〜し
\ ^{きょう}{今日}は^{いっぱい}{1杯}だけ^{ため}{試}してみよう

\page
\A ^{こんや}{今夜}はホットでね

\page
\A ん〜〜

\A ああ……わたしやっぱりほんとは^{ぎゅうにゅう}{牛乳}もお^{さけ}{酒}
も^{だい}{大}^{す}{好}きなのかも!!

\page
\A ^{からだ}{体}は、まだ^{だいじょうぶ}{大丈夫}……

\page
\A あうう

\A ダメだ〜〜
\ やっぱ〜〜

\page
\A あれ?

\page
\A この^{たか}{高}さ……
\ ああ……これいつもの^{ゆめ}{夢}だー

\A そっかー……
\ ^{わたし}{私}、あのまんまねちゃったのかー

\A お^{みせ}{店}……
\ ^{し}{閉}めといてよかったなー

\A ^{かぜ}{風}ーー
\ すずしー……

\A これからあのミルク^{わ}{割}りはねる^{まえ}{前}にのもー……

\page
\A ん

\Sign ^{こくさん}{国産}^{だいず}{大豆}

\A ^{あさ}{朝}、^{め}{目}がさめてひなたがあったかいです

\A う?
\ あり?

\A あ〜〜
\ お^{さけ}{酒}も^{ぎゅうにゅう}{牛乳}もまだまだ……
\ ぜんぜん……


\subsection{第53話\ ^{せんせい}{先生}のマーク}

\page[140]
\O ^{せんぱい}{先輩}!
\ ^{も}{持}って^{き}{来}ましたよ!

\O オレの^{も}{持}ってる^{あお}{青}の^{とりょう}{塗料}ったらこんぐれえなんすけどー

\page
\S あーー
\ わりいね

\O あに
\ やんですかきょうは

\S ちっとこいつにラクガキをね……

\O えっ?!

\O そりゃ、ちっともったいなくねえすか?!

\O せっかく^{びひん}{美品}^{み}{見}っけてきたのに

\page
\O それに、そうゆうのはふつう……

\O カッティングシートとか^{つか}{使}うんじゃねえかなーと……

\S いいんだよ!

\S ^{わたし}{私}が「やる」ってんだから!

\O はあ……
\ でも〜〜

\S あに?
\ なんか、もんくあんの?!

\O いえ
\ ないです

\page
\S あ……
\ これいいな

\S ウルトラマリン

\S この^{いろ}{色}がいいな
\ これにしよ

\O その^{とりょう}{塗料}だと^{いっぱつ}{一発}でキメないと

\O やりなおしききませんよ

\S ^{え}{絵}の^{ぐ}{具}のくいつきがいいってことだよね

\S ^{じょうとう}{上等}!

\page
\S おし!

\O なんすか?
\ これ

\S へへへ

\S ^{がっこう}{学校}でさ……
\ ^{びじゅつ}{美術}^{ぶ}{部}の^{ひと}{人}と^{しっぽうやき}{七宝焼}やってさ

\S そん^{とき}{時}
\ ^{じぶん}{自分}のマーク^{かんが}{考}えて
\ ペンダント^{つく}{作}ったんだよ

\S これ^{わたし}{私}のマーク……

\S ふふふ

\O はあ

\page
\O なんか、えらく^{ちゅうしょうてき}{抽象的}っすね〜〜

\O しぶいっつうかー

\O ん〜〜

\O ^{じょしこうせい}{女子高生}とも^{おも}{思}えねー
\ ^{か}{枯}れたデザインっすね〜〜

\S あに?

\S なんか、バカにしてる?

\O ほめてんですよ

\S なーんだ

\page
\S これはね……
\ 「あっちこっちを^{み}{見}て^{ある}{歩}いちゃ、よろこんでいる^{やつ}{奴}」なんだよ

\S だから、^{いま}{今}ここに^{えが}{描}いたのはバイクと^{わたし}{私}の……
\ なんていうか……
\ ^{きょうつう}{共通}の^{め}{目}なんだ

\O はあ

\page
\O オレ
\ ^{せんぱい}{先輩}の^{みょうじ}{名字}の……
\ ^{こ}{子}^{うみ}{海}^{いし}{石}の「^{こ}{子}」かなって^{おも}{思}いましたよ

\S ああ
\ ちっと^{に}{似}てんね

\S うん……
\ それもいいね

\page
\S バイクと^{はし}{走}り^{だ}{出}してから^{せかい}{世界}が10^{ばい}{倍}にもひろがった

\page
\S ^{わたし}{私}は、こいつとどこらへんまで^{み}{見}ていけるんだろう

\page[152]
\S どこらへんまで^{み}{見}ていけるんだろう


\subsection{第54話\ ^{むさしの}{武蔵野}^{つうしん}{通信}}

\page[155]
\Sign たばこ

\Sign ^{ゆうびんきょうりょくてん}{郵便協力店}

\page
\Sign 201
\ K\ タカツ

\page[158]
\K あ〜〜

\K ^{ね}{寝}すぎた

\K ん〜〜
\ ^{すいどう}{水道}……と
\ ピザ^{や}{屋}さん
\ ^{ひ}{引}っ^{こ}{越}し^{や}{屋}さん

\page[164]
\P おまちどおさまでした

\K あ
\ はい

\P きょっ
\ きょうはごきげんですね

\K は?

\K えっ
\ ええ……

\page
\K そうなんです

\P ご
\ ごゆっくり!

\page[167]
\P あっ、ハンコないっすか
\ じゃ、ここにサインを……

\P ……はい
\ じゃっ
\ ありあとあしたー!

\A どうもー


\subsection{そんなココネと^{まる}{丸}^{こ}{子}であった}

\K そしたらアルファさんが…

\K それからアルファさんが…

\K でも、アルファさんは

\R まったく、このムスメは

\R 「アルファさん」はもう、わかったからさー

\R もっと^{じぶん}{自分}のこととか^{はな}{話}してみなよー

\R あ…
\ ^{い}{言}い^{かた}{方}キツかったか?!

\R あ〜〜
\ いや…
\ だからさ…

\K ^{わたし}{私}がアルファさんを、ひ…
\ …に、^{だ}{出}しすぎるから

\K …のせいで
\ アルファさんの^{いんしょう}{印象}がわるくなっちゃったら…
\ ^{わたし}{私}、アルファさんに

\R そーゆう
\ ^{じき}{時期}なのかな
\ これわ…
