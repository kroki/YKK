\section{Volume 7}

\subsection{^{だい}{第}55^{わ}{話}\ ^{ちゅうくう}{中空}の^{しろ}{白}}

\page[2]
\A ^{そら}{空}を^{と}{飛}ぶ^{ゆめ}{夢}

\A いつもの^{たか}{高}さ
\ この^{ゆめ}{夢}を、^{わたし}{私}はよく^{み}{見}る

\page
\A ^{て}{手}がなんとなく^{つばさ}{翼}の^{かたち}{形}になってる

\A この^{つばさ}{翼}が、^{ゆめ}{夢}で^{と}{飛}ぶ^{とき}{時}に^{いみ}{意味}があるのかはわからないけど……

\A これは、たぶん、^{わたし}{私}が「^{そら}{空}を^{と}{飛}ぶにはハネが^{ひつよう}{必要}」と
^{おも}{思}いこんでるからだと^{おも}{思}う

\page
\A ^{しず}{静}かな^{そら}{空}
\ ^{かぜ}{風}が^{すこ}{少}しある

\page
\A ^{みち}{道}に^{ひと}{人}がいる

\page
\A タカヒロだ

\A こっち^{み}{見}てるね

\page
\A ほわあ

\page
\T アルファ……
\ あのさー

\A ん?

\T じつはこないだそのへんでね……
\ なんか^{しろ}{白}いモノがスー……

\A こっ
\ こわい^{はなし}{話}ならやめてね


\subsection{第56話\ ^{しろ}{白}い^{あさ}{朝}}

\page[10]
\T いつもの^{ちゅうないえんかい}{駐内宴会}が^{お}{終}わりゃしないからアルファのところへ^{ひなん}{避難}した

\T ものすごい^{は}{晴}れ^{かた}{方}の^{ひ}{日}だ

\T ^{そら}{空}は^{くろ}{黒}いくらい^{あお}{青}くて^{ふか}{深}い

\page[12]
\T ^{よる}{夜}になると^{きゅう}{急}にカキーンと^{ひ}{冷}えてきた

\T ^{とお}{遠}くのあかりがいつもの^{なんばい}{何倍}も^{あか}{明}るくピカピカしている

\page
\A さび〜

\T アルファ、さびい

\A タカヒロ、タカヒロ!!

\page[15]
\A うあーー!!
\ すごーー!!

\T ^{こな}{粉}をぶんまいたみたいな、すごい^{かず}{数}の^{ほし}{星}だ

\T きもちわりー!!

\A え〜〜〜?
\ きれいじゃんよーー

\T ん〜〜
\ まー
\ たしかに

\T さび〜〜

\page
\T なんか、あんましあったまんないね

\A うん……
\ このヒーター^{いっこ}{一個}じゃだめかな

\A ふとん、^{さき}{先}しいちゃおうよ

\A おフロでもわかす?

\T ^{きょう}{今日}はいいす

\page
\T あーだめだ
\ ふとんは
\ ねむくなる

\A ^{わたし}{私}も……
\ ^{ね}{寝}ちゃおうよもう

\A ^{あさ}{朝}、あそぼう

\page
\T もう^{ね}{寝}ちゃったよ
\ ^{はや}{早}いなー

\page
\A うん

\page
\A タカヒロ!!

\page
\A あ
\ ^{お}{起}きた?!
\ ^{き}{来}てみ!!
\ すごいよ!!

\A ^{はや}{早}く、^{はや}{早}く!!
\ ^{そと}{外}^{き}{来}て、そと!!

\A タカヒローー!

\page
\A ほら!
\ ^{しも}{霜}!!
\ まっしろ!!

\page
\A すごいねーー
\ きれいだねーー

\page
\T うん

\T ^{しも}{霜}がおりた^{あさ}{朝}


\subsection{第57話\ ^{しろ}{白}ペンキ}

\page[27]
\A ^{はる}{春}だ〜〜

\A ^{そら}{空}は^{しろ}{白}い
\ ^{かぜ}{風}はない

\A ^{はる}{春}とんぼと
\ なんだか、わかんない^{ちい}{小}さい^{はねむし}{羽虫}

\page
\A このまま30^{びょう}{秒}^{と}{止}められたらラッキーデー…

\page
\A そういえば

\A この^{まえ}{前}ペンキ^{ぬ}{塗}ったのいつだったっけ

\A うちの^{しろ}{白}ペンキは^{とおめ}{遠目}に^{み}{見}ると、
きれいに^{み}{見}えるけど

\A ^{ちか}{近}くで^{み}{見}るとボロボロ……

\A ペンキぬりかー
\ ペンキ^{たか}{高}いしなー

\page
\A お^{みせ}{店}はもともとひなたの^{いま}{居間}として
^{ぞうちく}{増築}したらしいけど

\A ずっと^{せんたく}{洗濯}^{もの}{物}とサボテン^{よう}{用}のへやに
なっていたとか

\A その^{とき}{時}には^{わたし}{私}はもうここにいたと

\A ^{むかし}{昔}、オーナーに^{き}{聞}いた

\page
\A ^{いま}{今}はもうなじんでるうちとお^{みせ}{店}

\A けっこうよれよれ

\A ^{かんきせん}{換気扇}つけたいな
\ ^{たか}{高}いかな、やっぱ……

\page
\A ^{わたし}{私}が、^{ものごころ}{物心}ついたときにはこのへやは、
もう^{いま}{今}の、お^{みせ}{店}みたいになってた

\A オーナーは

\A なんでここをお^{みせ}{店}にしようと^{おも}{思}ったんだろう

\A お^{きゃく}{客}さんこないよー

\page
\A ^{はじ}{初}めのころはほんとにだれも^{こ}{来}なかった

\A お^{みせ}{店}は、^{わたし}{私}とオーナーのためのコーヒー^{へや}{部屋}だった

\page
\A あの^{ふく}{服}で^{はじ}{初}めてお^{みせ}{店}に^{で}{出}た^{ひ}{日}のこと

\A よくおぼえてる

\A 「どうですか?……^{にあ}{似合}いますか?」

\page
\A あー
\ あのころなんか、^{わたし}{私}いつも、あたふたしてたなー

\A ^{はや}{早}く^{へいてん}{閉店}^{じかん}{時間}になんないかなって^{おも}{思}ってたよ

\page
\A ^{よる}{夜}はオーナーと

\A いろんな^{はなし}{話}をした

\A くだらないことでも、なんでも^{き}{聞}いた

\A そのへんから、むこうの^{きおく}{記憶}はだんだん
\ ^{ほんと}{本当}のことか、^{あと}{後}で^{きゃくしょく}{脚色}したことか、わからなくなって

\A かすんでいく

\page[38]
\A カメラの^{とき}{時}^{いらい}{以来}オーナーからの^{れんらく}{連絡}はない

\page
\A おじさんやタカヒロと^{し}{知}りあうまで、ずっと^{ひとり}{一人}でもさびしくなかった

\A オーナーがいなくても、^{す}{過}ごしていけた

\A ^{ひとり}{一人}は^{す}{好}きだ

\A でも、もし^{いま}{今}おじさんがいなかったら

\A ココネがいなかったら

\A ^{わたし}{私}はさびしくてがまんできない

\page
\A オーナーの^{こと}{事}を^{かんが}{考}えたのは^{ひさ}{久}しぶりだ

\A オーナーの^{きおく}{記憶}の^{うえ}{上}には、もう

\A おじさんやココネとの^{おも}{思}い^{で}{出}がある


\subsection{第58話\ ^{あい}{藍}の^{つぶ}{粒}ふたつ}

\page[43]
\A でもあれじつは、おいしくないんですよ〜〜

\S あ
\ ^{わたし}{私}もそう^{おも}{思}った

\page
\S なんかさ

\A はい

\S なんか^{いま}{今}、^{いっしゅん}{一瞬}

\S ^{むかし}{昔}の^{どうきゅうせい}{同級生}と^{はな}{話}してる^{とき}{時}みたいな
^{きぶん}{気分}になったわ

\S ^{ゆめ}{夢}で、^{がくせい}{学生}やってる^{とき}{時}みたいな^{かん}{感}じ

\A はーー

\page
\A ^{わたし}{私}は、なんか……
\ たとえるものはないんですけど

\A ^{きも}{気持}ち
\ いいです

\page
\S ^{かんが}{考}えてみれば、アルファさんと^{わたし}{私}ってさー
\ ^{しんせき}{親戚}みたいなもんなのよね

\A そっ
\ そうですよね!

\page
\A おかわり
\ もってきます

\S ありがと

\page[49]
\Sign ^{きしょう}{氣象}^{ぶ}{部}
\ ^{にきょく}{二局}

\page
\AM ^{にきょくちょう}{二局長}さん
\ ごぶさた

\P おお
\ アルさん

\AM おじゃまかしら

\P ^{かんげい}{歓迎}

\P ^{しつちょう}{室長}なんか^{とく}{特}に^{かんげい}{歓迎}

\page
\AM どんなですか?

\P ^{うみ}{海}の^{みず}{水}あったかいね

\P ^{こんねん}{今年}は、^{うみぞ}{海沿}いの^{あめ}{雨}すごいかもよ

\AM ^{うみ}{海}が^{とお}{遠}い^{ところ}{所}は?

\P ^{あめ}{雨}なしかなあ

\P はい
\ お^{ちゃ}{茶}しかないけどね

\AM いただきます

\page[53]
\P ^{か}{変}わったネックレスだよね

\P なんかのおまじない?

\AM は?

\AM ああ……
\ これですか

\page
\AM ^{むかし}{昔}
\ ^{した}{下}で
\ もらったものなんですよ

\AM たしか、「^{み}{見}て^{ある}{歩}く^{もの}{者}」って^{いみ}{意味}だったと^{おも}{思}います

\P ふ〜〜ん

\page
\P そか
\ ^{わたし}{私}らには、ぴったりかもね

\AM はい


\subsection{第59話\ ^{あいいろ}{藍色}の^{ひとみ}{瞳}}

\page[58]
\A うちのまわりで^{しゃしん}{写真}を^{と}{撮}ろうと^{おも}{思}った

\page
\A あ、そうだ

\A となりの^{いえ}{家}の^{どだい}{土台}が^{のこ}{残}ってたっけな

\A あそこから……

\page[61]
\A ^{どだい}{土台}のあたりの^{した}{下}には、もう^{つち}{土}がなかった

\A あたたたた

\page
\A カメラ!

\page
\A この^{じき}{時季}にはめずらしくうねりがある^{うみ}{海}

\A ^{なみ}{波}は^{きし}{岸}を^{けず}{削}り、^{みず}{水}はにごっている

\A ^{みず}{水}の^{なか}{中}はわからない

\page[65]
\A ^{のぼ}{登}れる^{ところ}{所}をさがして^{ある}{歩}いた


\subsection{第60話\ ^{あお}{青}い^{ふく}{服}}

\page[74]
\SH あれ
\ まだ^{かえ}{帰}んないの?

\K うん
\ ^{ちか}{近}くに^{ひと}{一}つだけ^{はいたつ}{配達}はいっちゃった

\SH そっか
\ じゃ、わるいけど
\ ^{きょう}{今日}は、お^{さき}{先}かな〜〜

\K うん

\SH ^{き}{気}いつけてねーー

\K うん

\page[76]
\R お

\page
\R ココネさん!

\R ^{しごと}{仕事}^{ちゅう}{中}かー
\ ^{めだ}{目立}つコだねー

\page
\R あのこは^{めだ}{目立}つ
\ ^{むらさき}{紫}がかった^{ちゃ}{茶}^{いろ}{色}の^{かみ}{髪}とあの^{あお}{青}い^{せいふく}{制服}は
^{いろみ}{色味}の^{たんじゅん}{単純}なこの^{がい}{街}の^{なか}{中}では^{とく}{特}に、よく^{は}{映}える

\R それに、なにより

\R ^{かのじょ}{彼女}は、^{はだ}{肌}が^{うつく}{美}しい
\ ^{おな}{同}じロボットの^{わたし}{私}が^{み}{見}ても、そう^{かん}{感}じる

\R なんで?
\ たべもの?
\ ^{き}{気}のせい?!

\page
\R わっ!!

\page[81]
\R お
\ おっかね

\K ま\ ま\ ま\ ま\ ま\ ま\ ま

\K ^{まる}{丸}^{こ}{子}さん!!

\R はあい

\page
\K うわ〜〜

\K ごめんなさい!
\ ごめんなさい!

\R そーかー

\R あなたんとこそんな^{れんしゅう}{練習}もするんだ……

\page
\R あんな^{じょうきょう}{状況}でいきなる、うしろからおどかされたら

\R そりゃ
\ ああなるって
\ なかないでよー

\R まー
\ わるいのは^{わたし}{私}の^{ほう}{方}なんだし

\R それよりさ
\ その^{にもつ}{荷物}^{とど}{届}けちゃってさ……ね!

\K はい

\R そのあとなんかおごるわ

\page
\Sign うなぎや
\ たまツ

\page
\R やっとおちついたね

\K はあ

\K ごめんなさい
\ ほんとに

\R だから、もういいってのに

\page
\R でもさ、^{あんがい}{案外}あれじゃない?

\R ^{たま}{弾}とか、はいってないんでしょ、^{じつ}{実}は

\K いえ……
\ ^{さん}{3}^{ぱつ}{発}……

\R あ
\ そうなの

\K 9ミリ^{でんき}{電気}^{だま}{弾}ってやつなんですけど……

\R ああ!

\R ^{し}{知}ってる^{し}{知}ってる
\ 「ビリッ」てくるやつでしょ?

\K ええ……でも、これけっこうシャレになんないんですよ

\K なんか、^{あと}{跡}が^{のこ}{残}るって^{い}{言}うし

\R へ〜〜

\page
\R あ、そうだ!
\ ココネさんさあ、^{なつやす}{夏休}みとるでしょ?
\ ^{ちば}{千葉}にさ、^{およ}{泳}ぎ、^{い}{行}かない?

\K あ〜〜

\K あのあの^{わたし}{私}
\ ^{なつ}{夏}は、その〜〜
\ たぶん^{みなみ}{南}の^{ほう}{方}に^{い}{行}っちゃうのでー

\R あーー
\ そうなんだ

\R ^{みなみ}{南}ね

\R そりゃ^{ざんねん}{残念}

\page
\R じゃ、また^{こんど}{今度}ね〜〜

\K ごめんなさい

\R いや
\ あやまんなくてもいいんだけどさーー

\K はあ


\subsection{第61話\ ^{べに}{紅}の^{やま}{山}}


\subsection{第62話\ ^{たいふう}{颱風}}

\page[98]
\P えーー

\P ただいま^{たいふう}{台風}「メイホワ」の^{ちゅうしん}{中心}^{ふきん}{付近}^{じょうくう}{上空}です!

\P これより^{わたしたち}{私達}「^{はままつ}{浜松}ウエザーアタック」がーー

\P どこよりも^{はや}{早}く^{たいふう}{台風}^{じょうほう}{情報}をお^{おく}{送}りします!

\page
\P えーー

\P ^{とうか}{投下}された^{けいそくき}{計測器}からの^{じょうほう}{情報}によりますと!

\P ^{げんざい}{現在}^{ちゅうしん}{中心}の^{きあつ}{気圧}は905hPa、
^{ちゅうしん}{中心}^{ふきん}{付近}の^{さいだい}{最大}^{ふうそく}{風速}は……

\page
\A ^{たいふう}{台風}が^{き}{来}ている

\A きのうは^{まいど}{毎度}のように^{いえ}{家}のまわりをかたづけて

\A ^{あまど}{雨戸}をはめて

\A わりとワクワクしながら^{よる}{夜}をむかえた

\A ^{よる}{夜}、おじさんが^{くるま}{車}で^{き}{来}て

\A 「^{ねん}{念}のため、とりあえず^{だいじ}{大事}なものだけ^{も}{持}って、
うちのスタンドで^{ね}{寝}な」と^{い}{言}う

\A そうすることにした

\page
\A おじさんは、「^{うち}{家}の^{ほう}{方}のことやってくる」と^{い}{言}って、^{かえ}{帰}っていった

\A しばらくして、^{だい}{大}あらしになる

\A ^{じな}{地鳴}りのような^{かぜ}{風}の^{おと}{音}と

\A ^{じゃり}{砂利}をまくような^{あめ}{雨}の^{おと}{音}が、^{なみ}{波}になってひびく

\page
\A ^{あさ}{朝}ーー

\P メイホワは……わっくりこ・・

\page
\A ^{たいふう}{台風}は、まだこの^{へん}{辺}をうろついているらしい

\A ずっと^{つづ}{続}く、この^{かぜおと}{風音}を^{き}{聞}いてると、
^{てっきん}{鉄筋}コンクリートづくりとはいえ

\A ^{すこ}{少}し^{ふあん}{不安}になる

\A おじさんは^{だいじょうぶ}{大丈夫}だろうか

\A ^{たいふう}{台風}をたくさん^{し}{知}ってるから、^{へいき}{平気}だと^{おも}{思}うけど

\page
\A そうだ
\ うち、どうなってるかなあ

\A ^{みせ}{店}のガラスはかなりあぶないな……

\A ああ!
\ ^{おもや}{母屋}の^{ほう}{方}も^{あまど}{雨戸}^{ぜんぶ}{全部}くぎづけすればよかったかも……

\A あっ!
\ ^{みせ}{店}の^{かんばん}{看板}!
\ ^{だ}{出}しっぱなしだ!

\A おじさんが^{き}{来}たとき^{も}{持}ち^{だ}{出}したのは、
^{ふく}{服}がすこしと、あとカメラと^{てっぽう}{鉄砲}と^{げっきん}{月琴}……

\A それと、このラジオぐらいだ

\page
\A おじさん^{まと}{的}には……

\A なんかやっぱこっちに^{ひなん}{避難}した^{ほう}{方}がいいと^{おも}{思}ったんだろうな

\A ここも^{たかだい}{高台}で、
^{ふ}{吹}きさらしだけど、^{うみ}{海}の^{ちか}{近}くはもっときついかもしんないもんな

\A あ〜〜〜〜〜〜〜〜
\ なんか、^{ふあん}{不安}になってきた

\A ^{はや}{早}くやまないかな

\A ぐ〜

\page[108]
\P ^{しょうなん}{湘南}^{こ}{湖}から^{じょうりく}{上陸}した^{たいふう}{台風}「メイホワ」は
そのまま^{へいや}{平野}^{ぶ}{部}を^{おうだん}{横断}し……

\P いばらきの^{くに}{国}より、^{かしま}{鹿島}^{なだ}{灘}へめけました

\P え〜〜
\ それにしても^{こんかい}{今回}わがチームはまっ^{さき}{先}に……


\subsection{第63話\ ^{わたし}{私}の^{ばしょ}{場所}}

\page[118]
\A なんか、おじさんのスタンドで、お^{みせ}{店}やってるみたい

\O ^{に}{似}てんちゃー
\ ^{に}{似}てんな

\O でもよー

\O ^{みせ}{店}まるごと^{と}{飛}ばされてまうとまでは

\O ^{おも}{思}わなかったな

\A ええ
\ でも、なんか^{よかん}{予感}はありました

\page
\A お^{みせ}{店}……
\ がんばった^{ほう}{方}かもしれません

\A もう、ガタガタでしたもん

\O あの^{ひ}{日}よー
\ あんた^{へいぜん}{平然}としてたべ

\O ^{かんしん}{感心}したんだよ

\A おじさんが^{く}{来}る^{まえ}{前}まで、パニックだったんですよ

\A しばらく……
\ かなり……

\A ^{な}{泣}いてたような^{き}{気}がします

\page
\O そっか

\A あの

\A タカヒロは

\A どうしてますか?
\ あれから

\page
\O あーーやっぱ
\ けっこうきつかったみてえだな

\O タカにすりゃー
\ なじみの^{けしき}{景色}が^{な}{無}くなってまうってのは

\O ^{はじ}{初}めてかもしんねーからな

\A そうですね

\page
\O ^{みせ}{店}よー
\ ^{いっしゅうかん}{一週間}で、よく^{ていさい }{体裁}^{ととの}{整}えたよなー

\A まだ、^{かり}{仮}の^{かり}{仮}^{ほしゅう}{補修}ですけどね
\ ^{おもや}{母屋}の^{あま}{雨}もりはなんとか……

\O まーしばらくは^{みせ}{店}やってけんべー
\ ^{は}{晴}れりゃー

\A ええ

\A でも、いつか、ぜったい^{もとどお}{元通}りにします

\page
\O は〜

\O なんか、アルファさんの「きっぱり」^{はじ}{初}めて^{み}{見}た

\A え?

\O そうか〜〜
\ でもまあ
\ ^{つか}{使}える^{ざい}{材}とかは、^{きょくりょく}{極力}、^{つか}{使}うとしてもよー

\O けっこう、かかんぞー
\ いろいろ……

\A はい

\page
\A ^{わたし}{私}
\ ^{すこ}{少}し、^{そと}{外}に^{で}{出}てみようと^{おも}{思}うんです

\O ^{でかせ}{出稼}ぎか!

\A ん〜〜
\ まあ……

\A ってゆうか、ちょっと^{なが}{長}めに^{そと}{外}^{ある}{歩}きしようかなって

\page
\A なんか、いい^{きかい}{機会}ですし

\A ^{わたし}{私}ね……おじさんやココネと^{し}{知}りあってから、^{おも}{思}ったことがあるんですよ

\A ずっと^{まえ}{前}は、うちしか^{し}{知}らなかったから
\ ここが、^{わたし}{私}の^{ばしょ}{場所}だったんだけど

\A ちょっと^{まえ}{前}は、スタンドの^{ところ}{所}、
^{ま}{曲}がると^{かえ}{帰}って^{き}{来}たなって^{き}{気}になって

\A ^{いま}{今}は、^{あさひな}{朝比奈}^{とうげ}{峠}あたりから
\ こっちは、^{じもと}{地元}って^{かん}{感}じがする

\page
\A ^{じぶん}{自分}の^{ばしょ}{場所}は^{ひろ}{広}くなるんだなあって

\A いろいろ^{み}{見}てみたいし

\A まー、^{でかせ}{出稼}ぎは、そのついでと^{い}{言}うかー

\O そっか

\A ええ……だから、^{こんど}{今度}はいつもより

\A ^{すこ}{少}し^{こ}{濃}いめにフラフラしようと^{おも}{思}います

\page
\O いつ^{で}{出}んのよ

\A ^{じかん}{時間}おくとだんだんめんどうに、なっちゃうから

\A 2
\ 3^{ひ}{日}うちに

\O そっか
\ ^{き}{気}いつけて、フラフラしてきなよ

\A はい

\page
\A おかわりどうです?

\O いる


\subsection{第64話\ ^{てがみ}{手紙}}

\page[130]
\Sign ^{あさひな}{朝比奈}

\A ふう

\Sign ^{かまくら}{鎌倉}
\ ^{よこはま}{横浜}

\page
\A よ

\page[135]
\O お

\page[137]
\O よう

\O ん〜〜

\page
\K はあ
\ そうだったんですか……

\O なんか……
\ いろいろタイミングわりかったよな

\K あの……
\ じゃ、^{わたし }{私}^{かえ}{帰}ります

\K あの……
\ ありがとうございました

\page
\O へえよ

\O アルファさんからカギあずかってんだけどよ

\O ^{きょう}{今日}、ここに^{と}{泊}まってったらどうよ

\O もう^{くら}{暗}えし、^{やまみち}{山道}ばっかだべ

\O カギはあしたスタンドで^{わた}{渡}してけーねーか

\K え
\ でも

\O そうしな

\page
\K あ

\K ^{く}{来}る^{まえ}{前}に、^{わたし}{私}が^{だ}{出}した^{てがみ}{手紙}

\K むりもないか

\page
\K おじゃましまず

\Sign 井浜みかん

\page
\Sign オーナーへ
\ アルファ

\page
\K ふう

\page
\K じゃ……また
\ お^{せわ}{世話}^{さま}{様}でした、ほんとに

\O ^{ざんねん}{残念}だったけどな

\O ^{ね}{寝}てねえなありゃ


\subsection{第65話\ ^{きし}{岸}}

\page[146]
\A 「タカヒロ」

\page
\A 「^{わたし}{私}がいないあいだ、タカヒロにスクーターあずかっててほしいんだ」

\page
\T えっ

\T スクーターで^{い}{行}くんじゃないの?

\A うん

\A ガソリンスタンドとか、どこでもあるとは^{かぎ}{限}んないしね

\T あー
\ そっか

\page
\A でね……そのあいだタカヒロに^{の}{乗}っててくれたらなあって

\A ^{うご}{動}かしてた^{ほう}{方}がバイクにもいいし

\A タカヒロがよければ

\T うん

\T でもおれにできるかな、^{うんてん}{運転}

\A ちょっと^{た}{立}ってみ

\page
\A ほら
\ タカヒロもう^{わたし}{私}と^{おな}{同}じくらいだもん

\T あ

\A バイクも^{らくしょう}{楽勝}だと^{おも}{思}うよ

\T そか

\page
\A 13^{さい}{歳}になったんだっけね

\T うん
\ そう

\A ほんと、どんどん^{おお}{大}きくなるんだね

\T ^{よる}{夜}ヒザとか^{いた}{痛}いんだよ

\T じいちゃんに^{き}{聞}いたら、なんか、
^{からだ}{体}が^{いま}{今}バキバキ^{せいちょう}{成長}してんだってさ

\page
\A ふ〜〜ん
\ バキバキ
\ すごいねえ

\A こんど^{あ}{会}ったらさ、^{ぬ}{抜}かれてるね、きっと

\T かなー

\A じゃあ
\ スクーターたのむね……

\T うん
\ あーあとで、^{つか}{使}いかた^{おそ}{教}わりに^{い}{行}くよ

\page
\A ^{ざこう}{座高}はいっしょなのにね

\T あはは

\page
\T どっ
\ ア……?

\page
\A ごめん
\ なんでもないんだけど……

\A ちょっとだけ
\ ごめん……

\page
\M タカヒローー!!

\page
\M おーー!

\T おう
\ あに、^{か}{買}いもんか

\M うん
\ ちょ〜〜どよかった!

\T あん?

\M うちまでのっけてってよ

\page
\T いいけどよー
\ ケツいたくなんぞ、また

\M へへへ

\T なにおめえ、ザブトン^{も}{持}ち^{ある}{歩}いてんの?
\ ふだん

\M こっちの^{みち}{道}はさ

\M よく^{し}{知}ってるバイク^{とお}{通}るからね
\ さっき^{か}{買}ったんだよ

\page
\T あそー

\M そー

\T コーヒーのむか?

\M うん!

\page
\T ほい

\M なんだ、のみかけか


\subsection{ミサゴ、カマスのごぶさた!}

\Saying{ミザゴ} いちど^{いりえ}{入江}で^{あ}{会}ったっけね

\Saying{カマス} そーだっけ?

\Saying{カマス} まー、でも、あんましこーゆう^{きかい}{機会}ねえよな

\Saying{ミザゴ} ^{でばん}{出番}もないしね

\Saying{ミザゴ} わたしらわりそ^{むくち}{無口}だから…

\Saying{カマス} たまには^{い}{言}いてえことあんべ

\Saying{ミザゴ} そう!\ ^{い}{言}いたいこと!!\ も〜〜!\ ^{やま}{山}もりってゆうかこう…

\Saying{カマス} そんなにはねえべ
