\section{Volume 7}

\subsection{^{だい}{第}55^{わ}{話}\ ^{ちゅうくう}{中空}の^{しろ}{白}}

\page[2]
\Alpha ^{そら}{空}を^{と}{飛}ぶ^{ゆめ}{夢}

\Alpha いつもの^{たか}{高}さ
\ この^{ゆめ}{夢}を、^{わたし}{私}はよく^{み}{見}る

\page[3]
\Alpha ^{て}{手}がなんとなく^{つばさ}{翼}の^{かたち}{形}になってる

\Alpha この^{つばさ}{翼}が、^{ゆめ}{夢}で^{と}{飛}ぶ^{とき}{時}に^{いみ}{意味}があるのかはわからないけど……

\Alpha これは、たぶん、^{わたし}{私}が「^{そら}{空}を^{と}{飛}ぶにはハネが^{ひつよう}{必要}」と
^{おも}{思}いこんでるからだと^{おも}{思}う

\page[4]
\Alpha ^{しず}{静}かな^{そら}{空}
\ ^{かぜ}{風}が^{すこ}{少}しある

\page[5]
\Alpha ^{みち}{道}に^{ひと}{人}がいる

\page[6]
\Alpha タカヒロだ

\Alpha こっち^{み}{見}てるね

\page[7]
\Alpha ほわあ

\page[8]
\Takahiro アルファ……
\ あのさー

\Alpha ん?

\Takahiro じつはこないだそのへんでね……
\ なんか^{しろ}{白}いモノがスー……

\Alpha こっ
\ こわい^{はなし}{話}ならやめてね


\subsection{第56話\ ^{しろ}{白}い^{あさ}{朝}}

\page[10]
\Takahiro いつもの^{ちゅうないえんかい}{駐内宴会}が^{お}{終}わりゃしないからアルファのところへ^{ひなん}{避難}した

\Takahiro ものすごい^{は}{晴}れ^{かた}{方}の^{ひ}{日}だ

\Takahiro ^{そら}{空}は^{くろ}{黒}いくらい^{あお}{青}くて^{ふか}{深}い

\page[12]
\Takahiro ^{よる}{夜}になると^{きゅう}{急}にカキーンと^{ひ}{冷}えてきた

\Takahiro ^{とお}{遠}くのあかりがいつもの^{なんばい}{何倍}も^{あか}{明}るくピカピカしている

\page[13]
\Alpha さび〜

\Takahiro アルファ、さびい

\Alpha タカヒロ、タカヒロ!!

\page[15]
\Alpha うあーー!!
\ すごーー!!

\Takahiro ^{こな}{粉}をぶんまいたみたいな、すごい^{かず}{数}の^{ほし}{星}だ

\Takahiro きもちわりー!!

\Alpha え〜〜〜?
\ きれいじゃんよーー

\Takahiro ん〜〜
\ まー
\ たしかに

\Takahiro さび〜〜

\page[16]
\Takahiro なんか、あんましあったまんないね

\Alpha うん……
\ このヒーター^{いっこ}{一個}じゃだめかな

\Alpha ふとん、^{さき}{先}しいちゃおうよ

\Alpha おフロでもわかす?

\Takahiro ^{きょう}{今日}はいいす

\page[17]
\Takahiro あーだめだ
\ ふとんは
\ ねむくなる

\Alpha ^{わたし}{私}も……
\ ^{ね}{寝}ちゃおうよもう

\Alpha ^{あさ}{朝}、あそぼう

\page[18]
\Takahiro もう^{ね}{寝}ちゃったよ
\ ^{はや}{早}いなー

\page[19]
\Alpha うん

\page[20]
\Alpha タカヒロ!!

\page[21]
\Alpha あ
\ ^{お}{起}きた?!
\ ^{き}{来}てみ!!
\ すごいよ!!

\Alpha ^{はや}{早}く、^{はや}{早}く!!
\ ^{そと}{外}^{き}{来}て、そと!!

\Alpha タカヒローー!

\page[22]
\Alpha ほら!
\ ^{しも}{霜}!!
\ まっしろ!!

\page[23]
\Alpha すごいねーー
\ きれいだねーー

\page[24]
\Takahiro うん

\Takahiro ^{しも}{霜}がおりた^{あさ}{朝}


\subsection{第57話\ ^{しろ}{白}ペンキ}

\page[27]
\Alpha ^{はる}{春}だ〜〜

\Alpha ^{そら}{空}は^{しろ}{白}い
\ ^{かぜ}{風}はない

\Alpha ^{はる}{春}とんぼと
\ なんだか、わかんない^{ちい}{小}さい^{はねむし}{羽虫}

\page[28]
\Alpha このまま30^{びょう}{秒}^{と}{止}められたらラッキーデー…

\page[29]
\Alpha そういえば

\Alpha この^{まえ}{前}ペンキ^{ぬ}{塗}ったのいつだったっけ

\Alpha うちの^{しろ}{白}ペンキは^{とおめ}{遠目}に^{み}{見}ると、
きれいに^{み}{見}えるけど

\Alpha ^{ちか}{近}くで^{み}{見}るとボロボロ……

\Alpha ペンキぬりかー
\ ペンキ^{たか}{高}いしなー

\page[30]
\Alpha お^{みせ}{店}はもともとひなたの^{いま}{居間}として
^{ぞうちく}{増築}したらしいけど

\Alpha ずっと^{せんたく}{洗濯}^{もの}{物}とサボテン^{よう}{用}のへやに
なっていたとか

\Alpha その^{とき}{時}には^{わたし}{私}はもうここにいたと

\Alpha ^{むかし}{昔}、オーナーに^{き}{聞}いた

\page[31]
\Alpha ^{いま}{今}はもうなじんでるうちとお^{みせ}{店}

\Alpha けっこうよれよれ

\Alpha ^{かんきせん}{換気扇}つけたいな
\ ^{たか}{高}いかな、やっぱ……

\page[32]
\Alpha ^{わたし}{私}が、^{ものごころ}{物心}ついたときにはこのへやは、
もう^{いま}{今}の、お^{みせ}{店}みたいになってた

\Alpha オーナーは

\Alpha なんでここをお^{みせ}{店}にしようと^{おも}{思}ったんだろう

\Alpha お^{きゃく}{客}さんこないよー

\page[33]
\Alpha ^{はじ}{初}めのころはほんとにだれも^{こ}{来}なかった

\Alpha お^{みせ}{店}は、^{わたし}{私}とオーナーのためのコーヒー^{へや}{部屋}だった

\page[34]
\Alpha あの^{ふく}{服}で^{はじ}{初}めてお^{みせ}{店}に^{で}{出}た^{ひ}{日}のこと

\Alpha よくおぼえてる

\Alpha 「どうですか?……^{にあ}{似合}いますか?」

\page[35]
\Alpha あー
\ あのころなんか、^{わたし}{私}いつも、あたふたしてたなー

\Alpha ^{はや}{早}く^{へいてん}{閉店}^{じかん}{時間}になんないかなって^{おも}{思}ってたよ

\page[36]
\Alpha ^{よる}{夜}はオーナーと

\Alpha いろんな^{はなし}{話}をした

\Alpha くだらないことでも、なんでも^{き}{聞}いた

\Alpha そのへんから、むこうの^{きおく}{記憶}はだんだん
\ ^{ほんとう}{本当}のことか、^{あと}{後}で^{きゃくしょく}{脚色}したことか、わからなくなって

\Alpha かすんでいく

\page[38]
\Alpha カメラの^{とき}{時}^{いらい}{以来}オーナーからの^{れんらく}{連絡}はない

\page[39]
\Alpha おじさんやタカヒロと^{し}{知}りあうまで、ずっと^{ひとり}{一人}でもさびしくなかった

\Alpha オーナーがいなくても、^{す}{過}ごしていけた

\Alpha ^{ひとり}{一人}は^{す}{好}きだ

\Alpha でも、もし^{いま}{今}おじさんがいなかったら

\Alpha ココネがいなかったら

\Alpha ^{わたし}{私}はさびしくてがまんできない

\page[40]
\Alpha オーナーの^{こと}{事}を^{かんが}{考}えたのは^{ひさ}{久}しぶりだ

\Alpha オーナーの^{きおく}{記憶}の^{うえ}{上}には、もう

\Alpha おじさんやココネとの^{おも}{思}い^{で}{出}がある


\subsection{第58話\ ^{あい}{藍}の^{つぶ}{粒}ふたつ}

\page[43]
\Alpha でもあれじつは、おいしくないんですよ〜〜

\Sensei あ
\ ^{わたし}{私}もそう^{おも}{思}った

\page[44]
\Sensei なんかさ

\Alpha はい

\Sensei なんか^{いま}{今}、^{いっしゅん}{一瞬}

\Sensei ^{むかし}{昔}の^{どうきゅうせい}{同級生}と^{はな}{話}してる^{とき}{時}みたいな
^{きぶん}{気分}になったわ

\Sensei ^{ゆめ}{夢}で、^{がくせい}{学生}やってる^{とき}{時}みたいな^{かん}{感}じ

\Alpha はーー

\page[45]
\Alpha ^{わたし}{私}は、なんか……
\ たとえるものはないんですけど

\Alpha ^{きも}{気持}ち
\ いいです

\page[46]
\Sensei ^{かんが}{考}えてみれば、アルファさんと^{わたし}{私}ってさー
\ ^{しんせき}{親戚}みたいなもんなのよね

\Alpha そっ
\ そうですよね!

\page[47]
\Alpha おかわり
\ もってきます

\Sensei ありがと

\page[49]
\Sign ^{きしょう}{氣象}^{ぶ}{部}
\ ^{にきょく}{二局}

\page[50]
\ASevenMOne ^{にきょくちょう}{二局長}さん
\ ごぶさた

\Person おお
\ アルさん

\ASevenMOne おじゃまかしら

\Person ^{かんげい}{歓迎}

\Person ^{しつちょう}{室長}なんか^{とく}{特}に^{かんげい}{歓迎}

\page[51]
\ASevenMOne どんなですか?

\Person ^{うみ}{海}の^{みず}{水}あったかいね

\Person ^{こんねん}{今年}は、^{うみぞ}{海沿}いの^{あめ}{雨}すごいかもよ

\ASevenMOne ^{うみ}{海}が^{とお}{遠}い^{ところ}{所}は?

\Person ^{あめ}{雨}なしかなあ

\Person はい
\ お^{ちゃ}{茶}しかないけどね

\ASevenMOne いただきます

\page[53]
\Person ^{か}{変}わったネックレスだよね

\Person なんかのおまじない?

\ASevenMOne は?

\ASevenMOne ああ……
\ これですか

\page[54]
\ASevenMOne ^{むかし}{昔}
\ ^{した}{下}で
\ もらったものなんですよ

\ASevenMOne たしか、「^{み}{見}て^{ある}{歩}く^{もの}{者}」って^{いみ}{意味}だったと^{おも}{思}います

\Person ふ〜〜ん

\page[55]
\Person そか
\ ^{わたし}{私}らには、ぴったりかもね

\ASevenMOne はい


\subsection{第59話\ ^{あいいろ}{藍色}の^{ひとみ}{瞳}}

\page[58]
\Alpha うちのまわりで^{しゃしん}{写真}を^{と}{撮}ろうと^{おも}{思}った

\page[59]
\Alpha あ、そうだ

\Alpha となりの^{いえ}{家}の^{どだい}{土台}が^{のこ}{残}ってたっけな

\Alpha あそこから……

\page[61]
\Alpha ^{どだい}{土台}のあたりの^{した}{下}には、もう^{つち}{土}がなかった

\Alpha あたたたた

\page[62]
\Alpha カメラ!

\page[63]
\Alpha この^{じき}{時季}にはめずらしくうねりがある^{うみ}{海}

\Alpha ^{なみ}{波}は^{きし}{岸}を^{けず}{削}り、^{みず}{水}はにごっている

\Alpha ^{みず}{水}の^{なか}{中}はわからない

\page[65]
\Alpha ^{のぼ}{登}れる^{ところ}{所}をさがして^{ある}{歩}いた


\subsection{第60話\ ^{あお}{青}い^{ふく}{服}}

\page[74]
\Shiba あれ
\ まだ^{かえ}{帰}んないの?

\Kokone うん
\ ^{ちか}{近}くに^{ひと}{一}つだけ^{はいたつ}{配達}はいっちゃった

\Shiba そっか
\ じゃ、わるいけど
\ ^{きょう}{今日}は、お^{さき}{先}かな〜〜

\Kokone うん

\Shiba ^{き}{気}いつけてねーー

\Kokone うん

\page[76]
\Maruko お

\page[77]
\Maruko ココネさん!

\Maruko ^{しごと}{仕事}^{ちゅう}{中}かー
\ ^{めだ}{目立}つコだねー

\page[78]
\Maruko あのこは^{めだ}{目立}つ
\ ^{むらさき}{紫}がかった^{ちゃ}{茶}^{いろ}{色}の^{かみ}{髪}とあの^{あお}{青}い^{せいふく}{制服}は
^{いろみ}{色味}の^{たんじゅん}{単純}なこの^{がい}{街}の^{なか}{中}では^{とく}{特}に、よく^{は}{映}える

\Maruko それに、なにより

\Maruko ^{かのじょ}{彼女}は、^{はだ}{肌}が^{うつく}{美}しい
\ ^{おな}{同}じロボットの^{わたし}{私}が^{み}{見}ても、そう^{かん}{感}じる

\Maruko なんで?
\ たべもの?
\ ^{き}{気}のせい?!

\page[79]
\Maruko わっ!!

\page[81]
\Maruko お
\ おっかね

\Kokone ま\ ま\ ま\ ま\ ま\ ま\ ま

\Kokone ^{まる}{丸}^{こ}{子}さん!!

\Maruko はあい

\page[82]
\Kokone うわ〜〜

\Kokone ごめんなさい!
\ ごめんなさい!

\Maruko そーかー

\Maruko あなたんとこそんな^{れんしゅう}{練習}もするんだ……

\page[83]
\Maruko あんな^{じょうきょう}{状況}でいきなる、うしろからおどかされたら

\Maruko そりゃ
\ ああなるって
\ なかないでよー

\Maruko まー
\ わるいのは^{わたし}{私}の^{ほう}{方}なんだし

\Maruko それよりさ
\ その^{にもつ}{荷物}^{とど}{届}けちゃってさ……ね!

\Kokone はい

\Maruko そのあとなんかおごるわ

\page[84]
\Sign うなぎや
\ たまツ

\page[85]
\Maruko やっとおちついたね

\Kokone はあ

\Kokone ごめんなさい
\ ほんとに

\Maruko だから、もういいってのに

\page[86]
\Maruko でもさ、^{あんがい}{案外}あれじゃない?

\Maruko ^{たま}{弾}とか、はいってないんでしょ、^{じつ}{実}は

\Kokone いえ……
\ ^{さん}{3}^{ぱつ}{発}……

\Maruko あ
\ そうなの

\Kokone 9ミリ^{でんき}{電気}^{だま}{弾}ってやつなんですけど……

\Maruko ああ!

\Maruko ^{し}{知}ってる^{し}{知}ってる
\ 「ビリッ」てくるやつでしょ?

\Kokone ええ……でも、これけっこうシャレになんないんですよ

\Kokone なんか、^{あと}{跡}が^{のこ}{残}るって^{い}{言}うし

\Maruko へ〜〜

\page[87]
\Maruko あ、そうだ!
\ ココネさんさあ、^{なつやす}{夏休}みとるでしょ?
\ ^{ちば}{千葉}にさ、^{およ}{泳}ぎ、^{い}{行}かない?

\Kokone あ〜〜

\Kokone あのあの^{わたし}{私}
\ ^{なつ}{夏}は、その〜〜
\ たぶん^{みなみ}{南}の^{ほう}{方}に^{い}{行}っちゃうのでー

\Maruko あーー
\ そうなんだ

\Maruko ^{みなみ}{南}ね

\Maruko そりゃ^{ざんねん}{残念}

\page[88]
\Maruko じゃ、また^{こんど}{今度}ね〜〜

\Kokone ごめんなさい

\Maruko いや
\ あやまんなくてもいいんだけどさーー

\Kokone はあ


\subsection{第61話\ ^{べに}{紅}の^{やま}{山}}


\subsection{第62話\ ^{たいふう}{颱風}}

\page[98]
\Person えーー

\Person ただいま^{たいふう}{台風}「メイホワ」の^{ちゅうしん}{中心}^{ふきん}{付近}^{じょうくう}{上空}です!

\Person これより^{わたしたち}{私達}「^{はままつ}{浜松}ウエザーアタック」がーー

\Person どこよりも^{はや}{早}く^{たいふう}{台風}^{じょうほう}{情報}をお^{おく}{送}りします!

\page[99]
\Person えーー

\Person ^{とうか}{投下}された^{けいそくき}{計測器}からの^{じょうほう}{情報}によりますと!

\Person ^{げんざい}{現在}^{ちゅうしん}{中心}の^{きあつ}{気圧}は905hPa、
^{ちゅうしん}{中心}^{ふきん}{付近}の^{さいだい}{最大}^{ふうそく}{風速}は……

\page[100]
\Alpha ^{たいふう}{台風}が^{き}{来}ている

\Alpha きのうは^{まいど}{毎度}のように^{いえ}{家}のまわりをかたづけて

\Alpha ^{あまど}{雨戸}をはめて

\Alpha わりとワクワクしながら^{よる}{夜}をむかえた

\Alpha ^{よる}{夜}、おじさんが^{くるま}{車}で^{き}{来}て

\Alpha 「^{ねん}{念}のため、とりあえず^{だいじ}{大事}なものだけ^{も}{持}って、
うちのスタンドで^{ね}{寝}な」と^{い}{言}う

\Alpha そうすることにした

\page[101]
\Alpha おじさんは、「^{うち}{家}の^{ほう}{方}のことやってくる」と^{い}{言}って、^{かえ}{帰}っていった

\Alpha しばらくして、^{だい}{大}あらしになる

\Alpha ^{じな}{地鳴}りのような^{かぜ}{風}の^{おと}{音}と

\Alpha ^{じゃり}{砂利}をまくような^{あめ}{雨}の^{おと}{音}が、^{なみ}{波}になってひびく

\page[102]
\Alpha ^{あさ}{朝}ーー

\Person メイホワは……わっくりこ・・

\page[103]
\Alpha ^{たいふう}{台風}は、まだこの^{へん}{辺}をうろついているらしい

\Alpha ずっと^{つづ}{続}く、この^{かぜおと}{風音}を^{き}{聞}いてると、
^{てっきん}{鉄筋}コンクリートづくりとはいえ

\Alpha ^{すこ}{少}し^{ふあん}{不安}になる

\Alpha おじさんは^{だいじょうぶ}{大丈夫}だろうか

\Alpha ^{たいふう}{台風}をたくさん^{し}{知}ってるから、^{へいき}{平気}だと^{おも}{思}うけど

\page[104]
\Alpha そうだ
\ うち、どうなってるかなあ

\Alpha ^{みせ}{店}のガラスはかなりあぶないな……

\Alpha ああ!
\ ^{おもや}{母屋}の^{ほう}{方}も^{あまど}{雨戸}^{ぜんぶ}{全部}くぎづけすればよかったかも……

\Alpha あっ!
\ ^{みせ}{店}の^{かんばん}{看板}!
\ ^{だ}{出}しっぱなしだ!

\Alpha おじさんが^{き}{来}たとき^{も}{持}ち^{だ}{出}したのは、
^{ふく}{服}がすこしと、あとカメラと^{てっぽう}{鉄砲}と^{げっきん}{月琴}……

\Alpha それと、このラジオぐらいだ

\page[105]
\Alpha おじさん^{まと}{的}には……

\Alpha なんかやっぱこっちに^{ひなん}{避難}した^{ほう}{方}がいいと^{おも}{思}ったんだろうな

\Alpha ここも^{たかだい}{高台}で、
^{ふ}{吹}きさらしだけど、^{うみ}{海}の^{ちか}{近}くはもっときついかもしんないもんな

\Alpha あ〜〜〜〜〜〜〜〜
\ なんか、^{ふあん}{不安}になってきた

\Alpha ^{はや}{早}くやまないかな

\Alpha ぐ〜

\page[108]
\Person ^{しょうなん}{湘南}^{こ}{湖}から^{じょうりく}{上陸}した^{たいふう}{台風}「メイホワ」は
そのまま^{へいや}{平野}^{ぶ}{部}を^{おうだん}{横断}し……

\Person いばらきの^{くに}{国}より、^{かしま}{鹿島}^{なだ}{灘}へめけました

\Person え〜〜
\ それにしても^{こんかい}{今回}わがチームはまっ^{さき}{先}に……


\subsection{第63話\ ^{わたし}{私}の^{ばしょ}{場所}}

\page[118]
\Alpha なんか、おじさんのスタンドで、お^{みせ}{店}やってるみたい

\Ojisan ^{に}{似}てんちゃー
\ ^{に}{似}てんな

\Ojisan でもよー

\Ojisan ^{みせ}{店}まるごと^{と}{飛}ばされてまうとまでは

\Ojisan ^{おも}{思}わなかったな

\Alpha ええ
\ でも、なんか^{よかん}{予感}はありました

\page[119]
\Alpha お^{みせ}{店}……
\ がんばった^{ほう}{方}かもしれません

\Alpha もう、ガタガタでしたもん

\Ojisan あの^{ひ}{日}よー
\ あんた^{へいぜん}{平然}としてたべ

\Ojisan ^{かんしん}{感心}したんだよ

\Alpha おじさんが^{く}{来}る^{まえ}{前}まで、パニックだったんですよ

\Alpha しばらく……
\ かなり……

\Alpha ^{な}{泣}いてたような^{き}{気}がします

\page[120]
\Ojisan そっか

\Alpha あの

\Alpha タカヒロは

\Alpha どうしてますか?
\ あれから

\page[121]
\Ojisan あーーやっぱ
\ けっこうきつかったみてえだな

\Ojisan タカにすりゃー
\ なじみの^{けしき}{景色}が^{な}{無}くなってまうってのは

\Ojisan ^{はじ}{初}めてかもしんねーからな

\Alpha そうですね

\page[122]
\Ojisan ^{みせ}{店}よー
\ ^{いっしゅうかん}{一週間}で、よく^{ていさい }{体裁}^{ととの}{整}えたよなー

\Alpha まだ、^{かり}{仮}の^{かり}{仮}^{ほしゅう}{補修}ですけどね
\ ^{おもや}{母屋}の^{あま}{雨}もりはなんとか……

\Ojisan まーしばらくは^{みせ}{店}やってけんべー
\ ^{は}{晴}れりゃー

\Alpha ええ

\Alpha でも、いつか、ぜったい^{もとどお}{元通}りにします

\page[123]
\Ojisan は〜

\Ojisan なんか、アルファさんの「きっぱり」^{はじ}{初}めて^{み}{見}た

\Alpha え?

\Ojisan そうか〜〜
\ でもまあ
\ ^{つか}{使}える^{ざい}{材}とかは、^{きょくりょく}{極力}、^{つか}{使}うとしてもよー

\Ojisan けっこう、かかんぞー
\ いろいろ……

\Alpha はい

\page[124]
\Alpha ^{わたし}{私}
\ ^{すこ}{少}し、^{そと}{外}に^{で}{出}てみようと^{おも}{思}うんです

\Ojisan ^{でかせ}{出稼}ぎか!

\Alpha ん〜〜
\ まあ……

\Alpha ってゆうか、ちょっと^{なが}{長}めに^{そと}{外}^{ある}{歩}きしようかなって

\page[125]
\Alpha なんか、いい^{きかい}{機会}ですし

\Alpha ^{わたし}{私}ね……おじさんやココネと^{し}{知}りあってから、^{おも}{思}ったことがあるんですよ

\Alpha ずっと^{まえ}{前}は、うちしか^{し}{知}らなかったから
\ ここが、^{わたし}{私}の^{ばしょ}{場所}だったんだけど

\Alpha ちょっと^{まえ}{前}は、スタンドの^{ところ}{所}、
^{ま}{曲}がると^{かえ}{帰}って^{き}{来}たなって^{き}{気}になって

\Alpha ^{いま}{今}は、^{あさひな}{朝比奈}^{とうげ}{峠}あたりから
\ こっちは、^{じもと}{地元}って^{かん}{感}じがする

\page[126]
\Alpha ^{じぶん}{自分}の^{ばしょ}{場所}は^{ひろ}{広}くなるんだなあって

\Alpha いろいろ^{み}{見}てみたいし

\Alpha まー、^{でかせ}{出稼}ぎは、そのついでと^{い}{言}うかー

\Ojisan そっか

\Alpha ええ……だから、^{こんど}{今度}はいつもより

\Alpha ^{すこ}{少}し^{こ}{濃}いめにフラフラしようと^{おも}{思}います

\page[127]
\Ojisan いつ^{で}{出}んのよ

\Alpha ^{じかん}{時間}おくとだんだんめんどうに、なっちゃうから

\Alpha 2
\ 3^{ひ}{日}うちに

\Ojisan そっか
\ ^{き}{気}いつけて、フラフラしてきなよ

\Alpha はい

\page[128]
\Alpha おかわりどうです?

\Ojisan いる


\subsection{第64話\ ^{てがみ}{手紙}}

\page[130]
\Sign ^{あさひな}{朝比奈}

\Alpha ふう

\Sign ^{かまくら}{鎌倉}
\ ^{よこはま}{横浜}

\page[131]
\Alpha よ

\page[135]
\Ojisan お

\page[137]
\Ojisan よう

\Ojisan ん〜〜

\page[138]
\Kokone はあ
\ そうだったんですか……

\Ojisan なんか……
\ いろいろタイミングわりかったよな

\Kokone あの……
\ じゃ、^{わたし }{私}^{かえ}{帰}ります

\Kokone あの……
\ ありがとうございました

\page[139]
\Ojisan へえよ

\Ojisan アルファさんからカギあずかってんだけどよ

\Ojisan ^{きょう}{今日}、ここに^{と}{泊}まってったらどうよ

\Ojisan もう^{くら}{暗}えし、^{やまみち}{山道}ばっかだべ

\Ojisan カギはあしたスタンドで^{わた}{渡}してけーねーか

\Kokone え
\ でも

\Ojisan そうしな

\page[140]
\Kokone あ

\Kokone ^{く}{来}る^{まえ}{前}に、^{わたし}{私}が^{だ}{出}した^{てがみ}{手紙}

\Kokone むりもないか

\page[141]
\Kokone おじゃましまず

\Sign 井浜みかん

\page[142]
\Sign オーナーへ
\ アルファ

\page[143]
\Kokone ふう

\page[144]
\Kokone じゃ……また
\ お^{せわ}{世話}^{さま}{様}でした、ほんとに

\Ojisan ^{ざんねん}{残念}だったけどな

\Ojisan ^{ね}{寝}てねえなありゃ


\subsection{第65話\ ^{きし}{岸}}

\page[146]
\Alpha 「タカヒロ」

\page[147]
\Alpha 「^{わたし}{私}がいないあいだ、タカヒロにスクーターあずかっててほしいんだ」

\page[148]
\Takahiro えっ

\Takahiro スクーターで^{い}{行}くんじゃないの?

\Alpha うん

\Alpha ガソリンスタンドとか、どこでもあるとは^{かぎ}{限}んないしね

\Takahiro あー
\ そっか

\page[149]
\Alpha でね……そのあいだタカヒロに^{の}{乗}っててくれたらなあって

\Alpha ^{うご}{動}かしてた^{ほう}{方}がバイクにもいいし

\Alpha タカヒロがよければ

\Takahiro うん

\Takahiro でもおれにできるかな、^{うんてん}{運転}

\Alpha ちょっと^{た}{立}ってみ

\page[150]
\Alpha ほら
\ タカヒロもう^{わたし}{私}と^{おな}{同}じくらいだもん

\Takahiro あ

\Alpha バイクも^{らくしょう}{楽勝}だと^{おも}{思}うよ

\Takahiro そか

\page[151]
\Alpha 13^{さい}{歳}になったんだっけね

\Takahiro うん
\ そう

\Alpha ほんと、どんどん^{おお}{大}きくなるんだね

\Takahiro ^{よる}{夜}ヒザとか^{いた}{痛}いんだよ

\Takahiro じいちゃんに^{き}{聞}いたら、なんか、
^{からだ}{体}が^{いま}{今}バキバキ^{せいちょう}{成長}してんだってさ

\page[152]
\Alpha ふ〜〜ん
\ バキバキ
\ すごいねえ

\Alpha こんど^{あ}{会}ったらさ、^{ぬ}{抜}かれてるね、きっと

\Takahiro かなー

\Alpha じゃあ
\ スクーターたのむね……

\Takahiro うん
\ あーあとで、^{つか}{使}いかた^{おそ}{教}わりに^{い}{行}くよ

\page[153]
\Alpha ^{ざこう}{座高}はいっしょなのにね

\Takahiro あはは

\page[154]
\Takahiro どっ
\ ア……?

\page[155]
\Alpha ごめん
\ なんでもないんだけど……

\Alpha ちょっとだけ
\ ごめん……

\page[156]
\Makki タカヒローー!!

\page[157]
\Makki おーー!

\Takahiro おう
\ あに、^{か}{買}いもんか

\Makki うん
\ ちょ〜〜どよかった!

\Takahiro あん?

\Makki うちまでのっけてってよ

\page[158]
\Takahiro いいけどよー
\ ケツいたくなんぞ、また

\Makki へへへ

\Takahiro なにおめえ、ザブトン^{も}{持}ち^{ある}{歩}いてんの?
\ ふだん

\Makki こっちの^{みち}{道}はさ

\Makki よく^{し}{知}ってるバイク^{とお}{通}るからね
\ さっき^{か}{買}ったんだよ

\page[159]
\Takahiro あそー

\Makki そー

\Takahiro コーヒーのむか?

\Makki うん!

\page[160]
\Takahiro ほい

\Makki なんだ、のみかけか


\subsection{ミサゴ、カマスのごぶさた!}

\Saying{ミザゴ} いちど^{いりえ}{入江}で^{あ}{会}ったっけね

\Saying{カマス} そーだっけ?

\Saying{カマス} まー、でも、あんましこーゆう^{きかい}{機会}ねえよな

\Saying{ミザゴ} ^{でばん}{出番}もないしね

\Saying{ミザゴ} わたしらわりそ^{むくち}{無口}だから…

\Saying{カマス} たまには^{い}{言}いてえことあんべ

\Saying{ミザゴ} そう!\ ^{い}{言}いたいこと!!\ も〜〜!\ ^{やま}{山}もりってゆうかこう…

\Saying{カマス} そんなにはねえべ
