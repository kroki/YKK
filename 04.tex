\section{Volume 4}

\subsection{^{だい}{第}24^{わ}{話}\ ^{にちにち}{日々}のお^{こたち}{子達}}

\page[4]
\Takahiro あと
\ この^{じ}{字}は?

\Person 「^{えんちょう}{延長}^{じく}{軸}」な^{えんちょう}{延長}と^{じく}{軸}で
\ ノートに^{か}{書}いてね

\Person おらよ
\ スイカ^{き}{切}ったからよ

\Takahiro あっ

\page[5]
\Person じゃ
\ キリいいから
\ ^{きょう}{今日}んとこ^{お}{終}わしな

\Takahiro うん

\Person ^{しお}{塩}いんかよ

\Person ^{あした}{明日}^{しゃかい}{社会}な^{じしん}{地震}の^{はなし}{話}んとっから

\Person へーー
\ ^{しお}{塩}よ
\ しお

\Takahiro うん

\page[6]
\Makki タカヒロ〜〜!!

\Makki あっ!!
\ ^{きょう}{今日}もうおわったね!!

\Makki ちょっと^{うみ}{海}いこ^{うみ}{海}!!
\ ほらこれ!!

\page[7]
\Makki お〜〜ら
\ ^{やま}{山}もりのフナムシ!!

\Makki いっぺんに^{うみ}{海}いぶちまけるよ^{み}{見}たいでしょ!!
\ ほら
\ ^{はや}{早}く^{はや}{早}く!!

\Takahiro マッキよ〜〜

\Takahiro あんななー
\ ^{いっぱつ}{一発}^{げい}{芸}だべが

\Takahiro ヘロヘロおよぐだけの

\Makki え〜〜〜〜
\ でっかいのいっぱいとってきたのに

\Makki ^{きいろ}{黄色}いやつ
\ ^{み}{見}る?

\page[8]
\Takahiro わーっ!!

\Makki あらら

\page[9]
\Makki あ〜〜
\ また^{と}{取}んなきゃ
\ ^{うみ}{海}いこタカヒロ

\Takahiro フナムシはもういいからよ〜〜

\Person スイカくえ
\ スイカよー

\Takahiro マッキは^{くく}{九九}おぼえたのかよ

\Makki ^{くく}{九九}なんかもー
\ とっくだよ!
\ ねーー

\Person うん

\page[10]
\Makki よし!
\ ^{うみ}{海}だ
\ うみ!!

\Makki フナムシだ!

\Takahiro ん〜〜^{わる}{悪}いが
\ ^{さき}{先}に^{やくそく}{約束}があってだな
\ ^{きょう}{今日}はー

\Takahiro ガキみたくフナムシで^{あそ}{遊}んでらんねえんだな

\Makki ^{おし}{教}えてくれたのタカヒロじゃんかー

\Makki フナムシのヘロヘロ

\page[11]
\Takahiro じゃ

\Takahiro じゃ
\ ^{あした}{明日}こそおめえと^{あそ}{遊}ぶ^{ひ}{日}にすんべー

\Takahiro ^{きょう}{今日}はダメな!

\Makki あの^{ひと}{女}の^{ところ}{所}に^{い}{行}くのね

\Takahiro ふつうにしゃべれよ

\Takahiro そうなんだけどね

\page[12]
\Takahiro ^{きょう}{今日}あたり^{き}{来}てごらんとアルファが^{い}{言}ってた

\Takahiro まーー
\ きちっと^{やくそく}{約束}したわけじゃないし

\Takahiro マッキともいっしょに^{い}{行}ければいいんだけどね

\Takahiro あいつにはまだちょっと^{とお}{遠}いだろうな

\page[13]
\Takahiro お^{きゃく}{客}さんだ
\ めずらしー

\Takahiro やほーー

\page[14]
\Alpha あら
\ いらっしゃい!

\Kokone あ
\ タカヒロくん?

\page[15]
\Takahiro ココネ?

\Alpha あれ?
\ ^{しょたいめん}{初対面}だったよね

\Kokone ええ

\Kokone やっぱり

\Kokone いつもアルファさんに^{き}{聞}いてるから

\Kokone 「あ\ この^{こ}{子}ね」って

\Takahiro ぼくも

\page[16]
\Kokone じゃ
\ ^{あらた}{改}めて
\ ココネです
\ よろしくね

\Takahiro よろしく

\Alpha か〜

\Alpha な〜〜に
\ ^{き}{気}どってんのタカヒロ
\ ^{みみ}{耳}^{あか}{赤}いよ

\page[17]
\Kokone ほんと^{はじ}{初}めてって^{かん}{感}じしないね

\Takahiro うん

\Takahiro ひさしぶり^{かえ}{帰}ってきた^{ねえ}{姉}ちゃんって^{かん}{感}じだ

\Takahiro ココネはしばらくいんの?

\Kokone うん

\Kokone んーー
\ ^{に}{2}\ ^{さんにち}{3日}だけどね

\Takahiro そっか〜

\Alpha あーー
\ きみたち

\Alpha ^{わたし}{私}のまざっていいかね

\page[18]
\Takahiro ^{こんど}{今度}マッキもつれてこよう

\Makki タカヒロのかってぼー!


\subsection{第25話\ ^{とお}{遠}い^{なつやす}{夏休}み}

\page[20]
\Alpha じゃ
\ ^{ゆうがた}{夕方}までおかりします

\Ojisan おう

\page[21]
\Ojisan こんだ^{みずぎ}{水着}ぐれえ^{か}{買}いな

\Alpha ココネに^{か}{買}ってきてもらっちゃった

\Alpha えへへ

\Ojisan あ〜〜
\ ふうん

\Ojisan あんだかちっとなつかしい^{かん}{感}じ

\Ojisan ムスメさんたちよー

\page[22]
\Kokone タカヒロくんもくればよかったのに

\Alpha なんか^{ともだち}{友達}と^{やくそく}{約束}あるんだってー

\Alpha ^{くるま}{車}かりたのにねー

\page[23]
\Kokone わあ!

\Kokone ^{すなはま}{砂浜}!
\ まだあったんですね!

\Alpha うん
\ もうここだけ

\page[24]
\Kokone じ〜ん

\Kokone ^{わたし}{私}^{うみ}{海}でおよぐの^{はじ}{初}めてなんですよ〜〜
\ って
\ あれ?

\Kokone およがないんですか?

\Alpha いや

\Alpha ん〜〜
\ なんかはずかしいねえ

\Kokone は?

\Kokone だれもいませんよ

\Alpha そうね
\ うん

\page[25]
\Kokone なんか^{たの}{楽}しんで^{えら}{選}んじゃったんですけど

\Kokone ^{き}{着}てみてどうです?

\Alpha どうって
\ やっぱはずかしいっす

\Alpha あははは

\Kokone う〜〜ん
\ でも^{おも}{思}った^{とお}{通}り
\ バッチリですよ

\Alpha そっ
\ そう?

\Alpha なんかしっぽあるけど

\page[26]
\Kokone ^{きも}{気持}ちいい

\page[27]
\Alpha どう?
\ ^{うみ}{海}

\Kokone えっ?
\ ええ

\Kokone なんだか
\ すごい

\Alpha こっから^{ふか}{深}いから
\ ^{き}{気}つけてね

\Kokone はい

\page[30]
\Kokone すごいおよぎ^{かた}{方}ですね!

\Alpha べへへ

\Alpha そかな

\page[31]
\Kokone ^{なみ}{波}^{で}{出}てきましたね

\Alpha うん
\ もうあがりかな!

\Kokone ほんとにだーれもいない

\Kokone ^{いぜん}{以前}のにぎやかだった^{なつ}{夏}の^{きおく}{記憶}

\Kokone わかってはいたけど
\ あっさりと^{あじけ}{味気}ないほどの^{うみ}{海}は^{すこ}{少}し^{いがい}{意外}でした

\page[32]
\Kokone ちょっと
\ さみしいですね

\Alpha えっ
\ なにが?

\Kokone え
\ いえ
\ なんか^{うみ}{海}が

\Alpha そうかなー
\ なんか^{わたし}{私}うかれちゃっててね

\Alpha えへへ

\Alpha こんなふうに^{うみ}{海}きたの^{はじ}{初}めてだから

\page[33]
\Narrator ーーそう
\ なつかしい「^{なつ}{夏}^{やす}{休}みの^{かいすい}{海水}^{よくじょう}{浴場}」はもうありません

\Narrator この^{うみ}{海}はこんなに^{きも}{気持}ちいいのに

\Narrator もうもどらないもの
\ まで\ ほしがるのはぜいたくでしょうね

\Kokone ^{かんが}{考}えてみればここ^{ぜんぶ}{全部}^{ふたり}{2人}^{じ}{占}めですね!

\Alpha うん!

\page[34]
\Ojisan まー
\ 「オレも^{い}{行}く」とは^{い}{言}えねえわな


\subsection{『^{げっきん}{月琴}ってなにすか?』 (なにそれコーナー)}

^{びわ}{琵琶}って^{がっき}{楽器}ありますよね!
^{げっきん}{月琴}はそれの^{ちゅうごく}{中国}のいとこです。
なんでも「まるい^{かたち}{形}が^{つき}{月}みたいだから」というお^{はなし}{話}。
ふつうは4^{つる}{弦}でネックがもっと^{みじか}{短}くて、^{なか}{中}には
^{ごうか}{豪華}なかざりがついた^{たか}{高}そうなのもあるみたいですけど、
^{わたし}{私}の^{げっきん}{月琴}は^{じかた}{地方}^{つく}{作}られた^{てせい}{手製}のものじゃないか
と^{おも}{思}います。

^{おと}{音}の^{ほう}{方}はというと、ウクレレとマンドリンにバンジョーを^{た}{足}したって^{かん}{感}じで。
ほんとはトレモロで^{えんそう}{演奏}するんですけど、^{わたし}{私}はそんな
^{こうど}{高度}なワザもってないんで……。
ポロポロと^{じこ}{自己}^{りゅう}{流}でやってます!

←のページでは、^{わたし}{私}の^{げっきん}{月琴}を^{つく}{作}った^{ひと}{人}たちかも?
の、^{ちゅうごく}{中国}は^{ゆんなん?}{雲南}
・サニ^{ぞく}{族}の^{おんな}{女}の^{こ}{子}の^{いしょう}{衣装}を^{き}{着}させてもらいました。
^{にあ}{似合}ってます?

こちらゴンチチさんのもってる^{げっきん}{月琴}

たしかこんな^{かん}{感}じ……
だったと^{おも}{思}う……

かわいいですネ


\subsection{第26話\ ^{あお}{青}のM1}

\page[39]
\Ojisan カミナリ^{へいき}{平気}なのかよ

\Alpha え?
\ うん
\ ^{み}{見}るのは^{す}{好}き

\Kokone え〜

\page[40]
\Alpha やっぱ^{きょう}{今日}^{とま}{泊}ってった^{ほう}{方}がいいよ

\Alpha ^{じかん}{時間}もあれだしこんなんじゃー

\Kokone うう
\ ^{たす}{助}かります
\ いつもいつも

\page[42]
\ASevenMOne ^{くも}{雲}が^{おお}{多}い

\ASevenMOne ^{きょう}{今日}は^{ちじょう}{地上}まで^{み}{見}えない

\page[43]
\ASevenMOne ここを^{とお}{通}るとき
\ ^{わたし}{私}は^{かんかく}{感覚}を^{そと}{外}にさらす

\page[44]
\ASevenMOne ^{おも}{思}い^{で}{出}そうとしてるのだが

\ASevenMOne ^{みず}{水}の^{なか}{中}というのは
\ こんな^{かん}{感}じだっただろうか

\page[45]
\Person ^{しつちょう}{室長}

\Person アルファー^{しつちょう}{室長}

\ASevenMOne はい

\Person アルファーさん

\ASevenMOne え?

\Person もしもし?

\Person あら

\Person ^{した}{下}^{み}{見}てたの?
\ ごめんね

\page[46]
\ASevenMOne ああ
\ いや
\ いいんですよ

\Person そうか
\ あなたこの^{した}{下}の^{しま}{島}のうまれだっけ

\ASevenMOne はい

\ASevenMOne ^{いっかい}{1回}だけ^{うみ}{海}で^{およ}{泳}いだ^{こと}{事}があって

\ASevenMOne なんとなく^{おも}{思}い^{だ}{出}すんですよ

\page[47]
\ASevenMOne もう^{いっかい}{1回}^{およ}{泳}いでみたかったな

\Person ^{した}{下}は^{たいへん}{大変}だからそれどころじゃないよ
\ きっと

\ASevenMOne ええ

\Person あ
\ そうだ
\ ^{じちょう}{次長}さんが^{あと}{後}で^{き}{来}てくれってさ

\ASevenMOne ああ
\ ありがとう

\page[48]
\ASevenMOne ^{くもま}{雲間}から^{すこ}{少}し^{ちじょう}{地上}が^{み}{見}えた

\ASevenMOne ^{した}{下}では^{わたし}{私}の^{のち}{後}にたくさん
^{いもうと}{妹}や^{おとうと}{弟}がつくられたと^{き}{聞}く

\page[49]
\ASevenMOne この^{さき}{先}あなた^{たち}{達}と^{あ}{会}うことはたぶんないのだろうが

\ASevenMOne できればしあわせでいてほしいと^{おも}{思}う

\page[52]
\Alpha ^{は}{晴}れちゃったね

\Kokone でも^{と}{泊}めてもらっちゃお


\subsection{第27話\ ^{あさひな}{朝比奈}^{とうげ}{峠}}

\page[55]
\Alpha じゃ
\ ぼちぼち

\Kokone はい

\Kokone すみません
\ わざわざ

\Alpha ココネを^{おく}{送}る

\Alpha ^{かんが}{考}えてみれば^{ふたり}{2人}で^{はし}{走}るのははじめて

\page[56]
\Alpha おじさんいないや

\page[57]
\Kokone ありがとうございました
\ こんな^{とお}{遠}くまで

\Alpha ^{よこはま}{横浜}にでも^{い}{行}けたらいいんだけど

\Alpha ココネあした^{はやばん}{早番}じゃなきゃねー

\Kokone あはは
\ ^{つぎ}{次}はぜひ

\Kokone じゃ
\ お^{せわ}{世話}さまでした

\Alpha ううん

\page[58]
\Kokone ^{こんど}{今度}^{わたし}{私}のとこにも^{き}{来}てくださいね

\Alpha うん

\page[60]
\Saying{both} またね


\subsection{第28話\ ^{ゆかり}{縁}}

\page[62]
\Alpha あはは

\Alpha 「^{せんせい}{先生}
\ あの〜〜
\ こんなの^{つく}{作}ってみたんですけど
\ よかったらもらってください」

\Sensei あらま
\ へえ
\ アルファさはなんか^{つく}{作}るの^{す}{好}きね

\Alpha 「えへへ
\ デザインちょっとワンパターンですか?」

\page[63]
\Sensei ^{かのじょ}{彼女}らしい^{しゅみ}{趣味}
\ アルファさんのくらし^{かた}{方}に^{いちばん}{一番}おどろいているのは
たぶん^{わたし}{私}だろう

\Sensei ^{じぶん}{自分}なりの^{たの}{楽}しみ^{かた}{方}を
ほじくり^{だ}{出}す^{かのじょ}{彼女}の^{しんじょう}{心情}はすでに
\ ^{ほんらい}{本来}のスペックからは^{せつめい}{説明}できない

\Sensei かつてロボットに「^{じりつ}{自律}^{しん}{心}」はおろか^{まんぞく}{満
  足}な^{だいよう}{代用}^{ひん}{品}すらなく

\Sensei せめてそのきっかけでも^{て}{手}に^{はい}{入}れようとしていた^{ころ}{頃}の^{こと}{事}を
^{わたし}{私}は^{おも}{思}い^{だ}{出}す

\Sensei アルファさんーーーー
\ A7^{りょうさん}{量産}^{しさく}{試作}^{き}{機}M2の3^{たい}{体}の
うちのひとり

\Sensei ^{かのじょ}{彼女}の^{そんざい}{存在}がまだ^{とお}{遠}い^{ゆめ}{夢}
だった^{むかし}{昔}のことだ

\page[64]
\Sensei ^{かながわ}{神奈川}・^{よこすか}{横須賀}^{ま}{馬}^{ほり}{堀}^{かいがん}{海岸}

\Sensei ^{いちどめ}{1度目}の^{だい }{大}^{ こうちょう}{高潮}がようやくいさまり
\ ^{まち}{町}や^{みなと}{港}の^{た}{立}て^{なお}{直}しが^{はじ}{始}まった^{ころ}{頃}

\page[65]
\Sensei ^{しゅよう}{主要}^{こうわん}{港湾}の^{ざんてい}{暫定}^{てき}{的}
^{さいかい}{再開}^{いしゅうかん}{1週間}^{まえ}{前}だ

\Sensei ^{わたし}{私}たちは^{たいがん}{対岸}の^{ちば}{千葉}・^{ふなばし}{船橋}までの
50キロを^{ちょくせん}{直線}で^{むす}{結}ぶ^{こうそく }{高速}^{ じゅうだん}{縦断}に^{いど}{挑}む

\page[66]
\Sensei すべてが^{つなわた}{綱渡}りだった
\ ^{どうりょく}{動力}は^{くるま}{車}のターボチャージャーに
^{て}{手}のはえたような
\ ^{てせい}{手製}のジェットエンジンで

\Sensei そして^{いがく}{医学}^{ぶ}{部}^{しょくいん}{職員}の^{わたし}{私}が^{そうじゅう}{操縦}する

\Sensei ^{せんたい}{船体}のまっ^{きいろ}{黄色}なカラーリングもかなりダサめだ

\Sensei でも^{ふね}{船}の^{せいさく}{製作}スタッフのひとりがつけたという
^{なまえ}{名前}は^{き}{気}に^{い}{入}った

\Sensei 「ミサゴ」^{すいじょう}{水上}の^{たか}{鷹}

\page[67]
\Sensei エンジン^{しどう}{始動}

\Sensei ^{なみ}{波}の^{しず}{静}かなうちにすべてを^{お}{終}わらせなければならない

\page[68]
\Sensei スタート

\page[69]
\Sensei すぐに^{ぜんかい}{全開}にする

\Sensei あぶなっかしいタービン^{おと}{音}

\Sensei ^{はら}{腹}に^{くうき}{空気}をためて
\ ミサゴが^{みなも}{水面}を^{と}{飛}ぶ

\Sensei フルスロットルでないと^{あんてい}{安定}しない

\Sensei だから^{ねんりょう}{燃料}は^{ごふん}{5分}しかもたない

\Sensei そしてエンジンも^{ごふん}{5分}しかもたない

\page[70]
\Sensei グラグラ^{うご}{動}く^{そくど}{速度}^{けい}{計}が
^{じゅんこう}{巡航}^{そく}{速}の450キロあたりを^{しめ}{示}すが

\Sensei ^{わたし}{私}は^{げんかい}{限界}^{そくど}{速度}の600キロ^{ちょう}{超}までひっぱる

\Sensei ^{じつ}{実}はその^{ひ}{日}

\Sensei 「^{そくど}{速度}」と「^{うみ}{海}を^{わた}{渡}る^{こと}{事}」^{いがい}{以外}に
もう^{ひとつ}{1つ}^{もくてき}{目的}があった

\page[71]
\Sensei 「^{きょくげん}{極限}の^{じょうきょう}{状況}である^{め}{目}^{じるし}{標}に
^{ちょうせん}{挑戦}する^{とき}{時}の^{にんげん}{人間}の^{かんかく}{感覚}
\ さらには^{げんかい}{限界}をこえる^{とき}{時}の^{たっせい}{達成}^{かん}{感}」

\Sensei そのデータを^{え}{得}ること

\Sensei だから^{わたし}{私}はそのデータの^{はっせい}{発生}^{げん}{源}^{じたい}{自体}でもあるわけだ

\page[72]
\Sensei ^{ふなばし}{船橋}のスキー^{ば}{場}が^{み}{見}えてくる

\Sensei ^{ねんりょう}{燃料}^{けい}{計}はすでにEだ

\Sensei ^{じそく}{時速}
\ 620ーー

\page[74]
\Sensei この^{とき}{時}のデータがいずれ^{きかい}{機械}に^{じりつ}{自律}^{しょう}{性}を
^{も}{持}たせるきっかけくらいにはなると^{おも}{思}われたが

\Sensei ^{けっきょく}{結局}たいして^{やく}{役}にはたたなかった

\Sensei その^{あと}{後}^{まった}{全}く^{べつ}{別}の^{してん}{視点}から
アルファタイプが^{う}{生}まれることになる

\page[75]
\Sensei あの^{ひ}{日}のことはデータとしてはあまり^{い}{生}きなかったけど

\Sensei ^{あと}{後}のアルファタイプ^{かいほつ}{開発}にとっては
\ その^{とき}{時}
\ ^{わたし}{私}が^{からだ}{体}で^{う}{得}た^{けいけん}{経験}
\ そのものの^{ほう}{方}がずっと^{じゅうよう}{重要}だった

\page[76]
\Alpha 「^{いちおう}{一応}お^{れい}{礼}のつまりです!」

\Sensei アルファさんは^{いま}{今}もう^{じぶん}{自分}の^{みち}{道}を^{ある}{歩}いている


\subsection{第29話\ ひなた}

\page[80]
\Kokone 「こんにちは」

\Takahiro こ
\ こんにちは

\Alpha おーい

\page[82]
\Makki んーー
\ タカヒロねてたからさーー

\Makki となりにねたはずだけどな

\Takahiro おこせよ

\page[83]
\Takahiro あれ
\ ^{きょう}{今日}おばさんと^{みなみ }{南}^{ まち}{町}^{い}{行}ってんじゃねえの?

\Makki ^{い}{行}かないよ

\Makki タカヒロのほうがいい

\Takahiro ほお

\Takahiro きのう^{あそ}{遊}んだんだから^{きょう}{今日}は^{で}{出}ないぞ

\Makki いいよ^{べつ}{別}に

\Makki ^{ほん}{本}こんなに^{か}{借}りてきたのによーー
\ ^{ずかん}{図鑑}だけど
\ ^{あした}{明日}^{かえ}{返}すんだぞこれ

\Takahiro いいじゃん^{よ}{読}んでれば?

\page[84]
\Takahiro なんか^{へん}{変}なもん
\ うめえなーー
\ おめえ

\Takahiro ^{が}{蛾}?

\Makki ちょう!

\page[85]
\Takahiro ^{はら}{腹}へったか?

\Makki うん!

\Takahiro ちょっとなんか^{つく}{作}るか

\Takahiro ここでまってな

\Makki うん!

\page[86]
\Takahiro だからまってろってえばよ!!

\Takahiro あ〜〜
\ イモむける?

\Makki うん

\Takahiro じゃ
\ むけててね
\ 3コくらい

\page[87]
\Narrator じゃがいもとたまねぎのみそしる
\ トッピングにはばのり

\Takahiro おかわりは?

\Makki いる

\page[88]
\Takahiro マッキよー

\Makki ん〜〜?

\Takahiro こんだ
\ いっしょにアルファんとこ^{い}{行}こうか

\Makki うん

\page[89]
\Takahiro ほわ〜

\Takahiro あ〜〜〜
\ わりい
\ オレちっとねるわ

\Makki わたしもー

\page[90]
\Alpha いらっしゃい

\page[91]
\Takahiro マッキだよ

\Alpha ありゃま!

\Alpha よっ

\Makki よっ

\Alpha タカヒロ!
\ うちにも^{あ}{会}わせたい^{ひと}{人}がいるの

\Alpha ^{あ}{会}ってくれる?

\Takahiro うん!

\page[92]
\Takahiro ん〜

\Takahiro うーん
\ うー


\subsection{第30話\ カフェ\ アルファ}

\page[94]
\Narrator もともと、この^{みち}{道}は^{さき}{先}の^{ほう}{方}にあった
\ ^{べっそう}{別荘}^{がい}{街}のためにつくられたらしい

\page[95]
\Narrator アルファルトの^{ほそう}{舗装}は^{いま}{今}となっては、
^{なお}{直}しようがないと^{み}{見}えて^{く}{来}るたびに、ボロくなっていく

\page[96]
\Narrator ちょっと^{しん}{信}じられないが、この^{さき}{先}に^{きっさ}{喫茶}^{てん}{店}がある

\Narrator カフェ\ アルファ

\Narrator ^{むかし}{昔}の^{べいぐん}{米軍}^{じゅうたく}{住宅}^{ふう}{風}、
^{しら}{白}ペンキの^{いえ}{家}を^{かいぞう}{改造}した^{みせ}{店}だ

\page[97]
\Narrator ^{いきお}{勢}いの^{つよ}{強}いススキの^{のはら}{野原}に
かろうじて^{う}{埋}もれずにある

\page[98]
\Alpha あっ
\ いらっしゃいませ!

\Narrator ごぶさた

\page[99]
\Alpha ありゃ!

\Alpha ひさしぶりですね!
\ あ
\ どうぞ

\Narrator まー
\ ちょっとブラブラとね

\Alpha あいかわらずヒマですねー
\ うちみたいですねー

\Alpha あはは

\page[100]
\Alpha えーーと
\ アフェオレ?でしたっけ?

\Alpha うん
\ ^{ぎゅうにゅう}{牛乳}あったらね

\Narrator カフェオレ
\ のめるようになった?

\Alpha ^{わたし}{私}?
\ だめ!ぜんぜん

\page[101]
\Alpha えへへ
\ ごいっしょしてもいいですか?

\Narrator え?
\ あ
\ いいよ

\Alpha 5・6と

\Narrator アルファさんはいつも^{じぶん}{自分}から^{らい}{来}ながら、
やたら^{て}{照}れくさそうな^{かお}{顔}をする

\Narrator そして^{あまとう}{甘党}

\page[102]
\Narrator じわじわと^{かげ}{影}がうごいていく

\Narrator アルファさんはめったに^{こ}{来}ない^{きゃく}{客}の^{わたし}{私}に
^{ひ}{日}ごろのことを^{きき}{喜々}として^{はな}{話}し

\page[103]
\Narrator かと^{おも}{思}うと

\Alpha ^{たいよう}{太陽}きついですか

\Narrator あ?
\ いや
\ ^{へいき}{平気}だよ

\page[104]
\Narrator アルファさんは^{じぶん}{自分}ではなにも^{も}{持}ってないと
^{おも}{思}ってるかもしれない

\Narrator でも、^{かのじょ}{彼女}に^{あ}{会}った^{ひと}{人}はきっと^{き}{気}づくことがある

\Narrator お^{きゃく}{客}さんは、たぶん、それが^{み}{見}たくてここに^{く}{来}るんだと^{おも}{思}う

\page[105]
\Alpha おかわりなんてどうですか?

\Narrator おお
\ いただき

\Narrator ^{きょう}{今日}も^{わたし}{私}^{いがい}{以外}の^{きゃく}{客}は^{み}{見}なかった

\page[106]
\Narrator ごちそうさま

\Alpha あ
\ お^{かえ}{帰}りですか

\Alpha ーーまたいらして^{くだ}{下}さいね

\Narrator うん

\page[107]
\Narrator この^{みせ}{店}はえらく^{ふべん}{不便}な^{とお}{遠}い^{ところ}{所}だ

\Narrator でも、どの^{くらい }{位}^{ さき}{先}のことになっても、たぶん、また^{く}{来}る

\page[108]
\Narrator どれだけ^{あいだ}{間}があいても^{じょうれん}{常連}になれる^{みせ}{店}だ


\subsection{第31話\ ^{あか}{赤}い^{みず}{水}}

\page[110]
\Alpha う

\page[111]
\Alpha このごろ、^{ときどき}{時々}^{いどみず}{井戸水}がしょっぱい

\Alpha ^{きた}{北}の^{まち}{町}^{けいゆ}{経由}で^{き}{来}てる
^{すいどう}{水道}のコックがあったはずだけど

\page[112]
\Alpha あ〜〜
\ もう
\ この^{へん}{辺}なんだけどなあ

\Alpha ^{きょう}{今日}^{やす}{休}もかな

\page[113]
\Alpha あった!

\Alpha これをあけて^{いえ}{家}のきりかえコックをまわせば

\Alpha よっ

\Alpha あれ

\Alpha いよっ!!

\Alpha あ
\ ^{かた}{固}い

\page[114]
\Takahiro ^{てつだ}{手伝}う?

\Alpha タカヒロ!

\Takahiro なんかガサガサしてたから

\Alpha ^{しんぞう}{心臓}とまるかと^{おも}{思}った

\Alpha ^{しんぞう}{心臓}?

\Takahiro あーー
\ ごめん

\page[115]
\Takahiro こう?

\Alpha そうそう

\Alpha せーの

\Alpha よっ!!

\Alpha やったーー!!

\page[117]
\Alpha なんか
\ カゼひいちゃうってかんじ?

\Takahiro やばいね

\Alpha ^{いま}{今}おフロにお^{ゆ}{湯}いれてるから

\Takahiro うん

\Alpha じゃ
\ ま〜〜
\ そういうわけだ

\page[118]
\Alpha あーー
\ まあ、ハダカのつまあいってことで!

\Takahiro うん

\Alpha あり?

\page[119]
\Takahiro お^{ゆ}{湯}
\ まっかだね

\Alpha そうね

\Alpha しばらく^{みず}{水}^{だ}{出}しっぱなしにしなきゃ

\Alpha ^{かんが}{考}えてみたら

\Alpha ^{わたし}{私}の^{ばあい}{場合}カゼはひかないかもしれない
\ なーー

\page[120]
\Alpha ^{ふく}{服}、たぶん、^{きょう}{今日}かわかないから
\ それでガマンしてね

\Takahiro うん

\Alpha あ?

\page[121]
\Alpha はい?

\Ojisan よ

\Alpha あっ

\Ojisan あに
\ ^{みせ}{店}まだかよ

\Alpha いえ、ちょっと^{みず}{水}まわりいじってて

\Alpha あはは

\Ojisan おー
\ きのうから^{みず}{水}しょっぺえしなーー

\Ojisan まー
\ ^{ながつづ}{長続}きゃー
\ しねえと^{おも}{思}うけんどよーー

\Ojisan タカ^{き}{来}てねえ?

\page[122]
\Alpha ^{き}{来}てますよ

\Ojisan あんだ
\ ^{ふく}{服}かりてんのか

\Ojisan やっぱ^{ふ}{降}らいたか

\Ojisan こいつ^{で}{出}てから^{あめ}{雨}んなったからよーー

\Alpha おじさんあがっていきません?

\Ojisan でも
\ へえよー

\Ojisan アルファさんこいからいそがしいから

\Ojisan タカいったん^{けえ}{帰}んか

\Takahiro うん

\page[123]
\Alpha じゃあ
\ タカヒロ、あとでね!

\Takahiro うん

\Alpha あ

\Alpha まずかったのだろうか

\page[124]
\Takahiro おう

\Makki おう

\page[125]
\Takahiro あによ
\ ^{き}{来}てたんならアルファに^{あ}{会}ってけばよかったのによ

\Makki こんどタカヒロと^{い}{行}く

\Takahiro あそう

\page[126]
\Makki タカヒロ

\Takahiro あ?

\Makki ^{かえ}{帰}ったらおフロはいろう

\Takahiro あんでよ

\Alpha ふう


\subsection{Short Essay\ ^{そっきょうきょく}{即興曲}}

\page[127]
\Alpha ^{うみ}{海}から、しょっぱい^{かぜ}{風}が
とろとろと^{く}{来}る
\ ^{じこく}{時刻}になりました

\Alpha ^{きょう}{今日}は、これを
あなたのために^{ひ}{弾}きましょう

\page[128]
\Alpha この^{じかん}{時間}は^{つる}{弦}の^{おと}{音}が
^{えきたい}{液体}のような^{くうき}{空気}とよく^{と}{溶}けて
^{みょう}{妙}に^{ごかん}{五感}にひびきます

\page[129]
\Alpha ^{つる}{弦}の^{おと}{音}は
どんな^{かお}{香}りが^{み}{見}えるでしょうか

\Alpha ^{こんじょう}{今生}まれているこの^{きょく}{曲}にとって
あなたの^{そんざい}{存在}は、とても^{たいせつ}{大切}

\page[130]
\Alpha よろしかったら、
^{いっしょ}{一緒}にうたってくださいね
\ ^{くら}{暗}くなるまでには
まだしばらくあります

\Alpha いちばんおいしい^{じかん}{時間}です


\subsection{マッキのコマセでゴー!}
\Ayase フナムシとあそぶ^{しょうじょ}{少女}

\Ayase メルヒェンとやつ?

\Makki あははは

\Ayase ガキってやつぁーよー
