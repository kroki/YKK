\section{Volume 4}

\subsection{^{だい}{第}24^{わ}{話}\ ^{にちにち}{日々}のお^{こたち}{子達}}

\page[4]
\T あと
\ この^{じ}{字}は?

\P 「^{えんちょう}{延長}^{じく}{軸}」な^{えんちょう}{延長}と^{じく}{軸}で
\ ノートに^{か}{書}いてね

\P おらよ
\ スイカ^{き}{切}ったからよ

\T あっ

\page
\P じゃ
\ キリいいから
\ ^{きょう}{今日}んとこ^{お}{終}わしな

\T うん

\P ^{しお}{塩}いんかよ

\P ^{あした}{明日}^{しゃかい}{社会}な^{じしん}{地震}の^{はなし}{話}んとっから

\P へーー
\ ^{しお}{塩}よ
\ しお

\T うん

\page
\M タカヒロ〜〜!!

\M あっ!!
\ ^{きょう}{今日}もうおわったね!!

\M ちょっと^{うみ}{海}いこ^{うみ}{海}!!
\ ほらこれ!!

\page
\M お〜〜ら
\ ^{やま}{山}もりのフナムシ!!

\M いっぺんに^{うみ}{海}いぶちまけるよ^{み}{見}たいでしょ!!
\ ほら
\ ^{はや}{早}く^{はや}{早}く!!

\T マッキよ〜〜

\T あんななー
\ ^{いっぱつ}{一発}^{げい}{芸}だべが

\T ヘロヘロおよぐだけの

\M え〜〜〜〜
\ でっかいのいっぱいとってきたのに

\M ^{きいろ}{黄色}いやつ
\ ^{み}{見}る?

\page
\T わーっ!!

\M あらら

\page
\M あ〜〜
\ また^{と}{取}んなきゃ
\ ^{うみ}{海}いこタカヒロ

\T フナムシはもういいからよ〜〜

\P スイカくえ
\ スイカよー

\T マッキは^{くく}{九九}おぼえたのかよ

\M ^{くく}{九九}なんかもー
\ とっくだよ!
\ ねーー

\P うん

\page
\M よし!
\ ^{うみ}{海}だ
\ うみ!!

\M フナムシだ!

\T ん〜〜^{わる}{悪}いが
\ ^{さき}{先}に^{やくそく}{約束}があってだな
\ ^{きょう}{今日}はー

\T ガキみたくフナムシで^{あそ}{遊}んでらんねえんだな

\M ^{おし}{教}えてくれたのタカヒロじゃんかー

\M フナムシのヘロヘロ

\page
\T じゃ

\T じゃ
\ ^{あした}{明日}こそおめえと^{あそ}{遊}ぶ^{ひ}{日}にすんべー

\T ^{きょう}{今日}はダメな!

\M あの^{ひと}{女}の^{ところ}{所}に^{い}{行}くのね

\T ふつうにしゃべれよ

\T そうなんだけどね

\page
\T ^{きょう}{今日}あたり^{き}{来}てごらんとアルファが^{い}{言}ってた

\T まーー
\ きちっと^{やくそく}{約束}したわけじゃないし

\T マッキともいっしょに^{い}{行}ければいいんだけどね

\T あいつにはまだちょっと^{とお}{遠}いだろうな

\page
\T お^{きゃく}{客}さんだ
\ めずらしー

\T やほーー

\page
\A あら
\ いらっしゃい!

\K あ
\ タカヒロくん?

\page
\T ココネ?

\A あれ?
\ ^{しょたいめん}{初対面}だったよね

\K ええ

\K やっぱり

\K いつもアルファさんに^{き}{聞}いてるから

\K 「あ\ この^{こ}{子}ね」って

\T ぼくも

\page
\K じゃ
\ ^{あらた}{改}めて
\ ココネです
\ よろしくね

\T よろしく

\A か〜

\A な〜〜に
\ ^{き}{気}どってんのタカヒロ
\ ^{みみ}{耳}^{あか}{赤}いよ

\page
\K ほんと^{はじ}{初}めてって^{かん}{感}じしないね

\T うん

\T ひさしぶり^{かえ}{帰}ってきた^{ねえ}{姉}ちゃんって^{かん}{感}じだ

\T ココネはしばらくいんの?

\K うん

\K んーー
\ ^{に}{2}\ ^{さんにち}{3日}だけどね

\T そっか〜

\A あーー
\ きみたち

\A ^{わたし}{私}のまざっていいかね

\page
\T ^{こんど}{今度}マッキもつれてこよう

\M タカヒロのかってぼー!


\subsection{第25話\ ^{とお}{遠}い^{なつやす}{夏休}み}

\page[20]
\A じゃ
\ ^{ゆうがた}{夕方}までおかりします

\O おう

\page
\O こんだ^{みずぎ}{水着}ぐれえ^{か}{買}いな

\A ココネに^{か}{買}ってきてもらっちゃった

\A えへへ

\O あ〜〜
\ ふうん

\O あんだかちっとなつかしい^{かん}{感}じ

\O ムスメさんたちよー

\page
\K タカヒロくんもくればよかったのに

\A なんか^{ともだち}{友達}と^{やくそく}{約束}あるんだってー

\A ^{くるま}{車}かりたのにねー

\page
\K わあ!

\K ^{すなはま}{砂浜}!
\ まだあったんですね!

\A うん
\ もうここだけ

\page
\K じ〜ん

\K ^{わたし}{私}^{うみ}{海}でおよぐの^{はじ}{初}めてなんですよ〜〜
\ って
\ あれ?

\K およがないんですか?

\A いや

\A ん〜〜
\ なんかはずかしいねえ

\K は?

\K だれもいませんよ

\A そうね
\ うん

\page
\K なんか^{たの}{楽}しんで^{えら}{選}んじゃったんですけど

\K ^{き}{着}てみてどうです?

\A どうって
\ やっぱはずかしいっす

\A あははは

\K う〜〜ん
\ でも^{おも}{思}った^{どお}{通}り
\ バッチリですよ

\A そっ
\ そう?

\A なんかしっぽあるけど

\page
\K ^{きも}{気持}ちいい

\page
\A どう?
\ ^{うみ}{海}

\K えっ?
\ ええ

\K なんだか
\ すごい

\A こっから^{ふか}{深}いから
\ ^{き}{気}つけてね

\K はい

\page[30]
\K すごいおよぎ^{かた}{方}ですね!

\A べへへ

\A そかな

\page
\K ^{なみ}{波}^{で}{出}てきましたね

\A うん
\ もうあがりかな!

\K ほんとにだーれもいない

\K ^{いぜん}{以前}のにぎやかだった^{なつ}{夏}の^{きおく}{記憶}

\K わかってはいたけど
\ あっさりと^{あじけ}{味気}ないほどの^{うみ}{海}は^{すこ}{少}し^{いがい}{意外}でした

\page
\K ちょっと
\ さみしいですね

\A えっ
\ なにが?

\K え
\ いえ
\ なんか^{うみ}{海}が

\A そうかなー
\ なんか^{わたし}{私}うかれちゃっててね

\A えへへ

\A こんなふうに^{うみ}{海}きたの^{はじ}{初}めてだから

\page
\N ーーそう
\ なつかしい「^{なつ}{夏}^{やす}{休}みの^{かいすい}{海水}^{よくじょう}{浴場}」はもうありません

\N この^{うみ}{海}はこんなに^{きも}{気持}ちいいのに

\N もうもどらないもの
\ まで\ ほしがるのはぜいたくでしょうね

\K ^{かんが}{考}えてみればここ^{ぜんぶ}{全部}^{ふたり}{2人}^{じ}{占}めですね!

\A うん!

\page
\O まー
\ 「オレも^{い}{行}く」とは^{い}{言}えねえわな


\subsection{『^{げっきん}{月琴}ってなにすか?』 (なにそれコーナー)}

^{びわ}{琵琶}って^{がっき}{楽器}ありますよね!
^{げっきん}{月琴}はそれの^{ちゅうごく}{中国}のいとこです。
なんでも「まるい^{かたち}{形}が^{つき}{月}みたいだから」というお^{はなし}{話}。
ふつうは4^{つる}{弦}でネックがもっと^{みじか}{短}くて、^{なか}{中}には
^{ごうか}{豪華}なかざりがついた^{たか}{高}そうなのもあるみたいですけど、
^{わたし}{私}の^{げっきん}{月琴}は^{じかた}{地方}^{つく}{作}られた^{てせい}{手製}のものじゃないか
と^{おも}{思}います。

^{おと}{音}の^{ほう}{方}はというと、ウクレレとマンドリンにバンジョーを^{た}{足}したって^{かん}{感}じで。
ほんとはトレモロで^{えんそう}{演奏}するんですけど、^{わたし}{私}はそんな
^{こうど}{高度}なワザもってないんで……。
ポロポロと^{じこ}{自己}^{りゅう}{流}でやってます!

←のページでは、^{わたし}{私}の^{げっきん}{月琴}を^{つく}{作}った^{ひと}{人}たちかも?
の、^{ちゅうごく}{中国}は^{ゆんなん?}{雲南}
・サニ^{ぞく}{族}の^{おんな}{女}の^{こ}{子}の^{いしょう}{衣装}を^{き}{着}させてもらいました。
^{にあ}{似合}ってます?

こちらゴンチチさんのもってる^{げっきん}{月琴}

たしかこんな^{かん}{感}じ……
だったと^{おも}{思}う……

かわいいですネ


\subsection{第26話\ ^{あお}{青}のM1}

\page[39]
\O カミナリ^{へいき}{平気}なのかよ

\A え?
\ うん
\ ^{み}{見}るのは^{す}{好}き

\K え〜

\page
\A やっぱ^{きょう}{今日}^{とま}{泊}ってった^{ほう}{方}がいいよ

\A ^{じかん}{時間}もあれだしこんなんじゃー

\K うう
\ ^{たす}{助}かります
\ いつもいつも

\page[42]
\AM ^{くも}{雲}が^{おお}{多}い

\AM ^{きょう}{今日}は^{ちじょう}{地上}まで^{み}{見}えない

\page
\AM ここを^{とお}{通}るとき
\ ^{わたし}{私}は^{かんかく}{感覚}を^{そと}{外}にさらす

\page
\AM ^{おも}{思}い^{で}{出}そうとしてるのだが

\AM ^{みず}{水}の^{なか}{中}というのは
\ こんな^{かん}{感}じだっただろうか

\page
\P ^{しつちょう}{室長}

\P アルファー^{しつちょう}{室長}

\AM はい

\P アルファーさん

\AM え?

\P もしもし?

\P あら

\P ^{した}{下}^{み}{見}てたの?
\ ごめんね

\page
\AM ああ
\ いや
\ いいんですよ

\P そうか
\ あなたこの^{した}{下}の^{しま}{島}のうまれだっけ

\AM はい

\AM ^{いっかい}{1回}だけ^{うみ}{海}で^{およ}{泳}いだ^{こと}{事}があって

\AM なんとなく^{おも}{思}い^{だ}{出}すんですよ

\page
\AM もう^{いっかい}{1回}^{およ}{泳}いでみたかったな

\P ^{した}{下}は^{たいへん}{大変}だからそれどころじゃないよ
\ きっと

\AM ええ

\P あ
\ そうだ
\ ^{じちょう}{次長}さんが^{あと}{後}で^{き}{来}てくれってさ

\AM ああ
\ ありがとう

\page
\AM ^{くもま}{雲間}から^{すこ}{少}し^{ちじょう}{地上}が^{み}{見}えた

\AM ^{した}{下}では^{わたし}{私}の^{のち}{後}にたくさん
^{いもうと}{妹}や^{おとうと}{弟}がつくられたと^{き}{聞}く

\page
\AM この^{さき}{先}あなた^{たち}{達}と^{あ}{会}うことはたぶんないのだろうが

\AM できればしあわせでいてほしいと^{おも}{思}う

\page[52]
\A ^{は}{晴}れちゃったね

\K でも^{と}{泊}めてもらっちゃお


\subsection{第27話\ ^{あさひな}{朝比奈}^{とうげ}{峠}}

\page[55]
\A じゃ
\ ぼちぼち

\K はい

\K すみません
\ わざわざ

\A ココネを^{おく}{送}る

\A ^{かんが}{考}えてみれば^{ふたり}{2人}で^{はし}{走}るのははじめて

\page
\A おじさんいないや

\page
\K ありがとうございました
\ こんな^{とお}{遠}くまで

\A ^{よこはま}{横浜}にでも^{い}{行}けたらいいんだけど

\A ココネあした^{はやばん}{早番}じゃなきゃねー

\K あはは
\ ^{つぎ}{次}はぜひ

\K じゃ
\ お^{せわ}{世話}さまでした

\A ううん

\page
\K ^{こんど}{今度}^{わたし}{私}のとこにも^{き}{来}てくださいね

\A うん

\page[60]
\Saying{both} またね


\subsection{第28話\ ^{ゆかり}{縁}}

\page[62]
\A あはは

\A 「^{せんせい}{先生}
\ あの〜〜
\ こんなの^{つく}{作}ってみたんですけど
\ よかったらもらってください」

\S あらま
\ へえ
\ アルファさはなんか^{つく}{作}るの^{す}{好}きね

\A 「えへへ
\ デザインちょっとワンパターンですか?」

\page
\S ^{かのじょ}{彼女}らしい^{しゅみ}{趣味}
\ アルファさんのくらし^{かた}{方}に^{いちばん}{一番}おどろいているのは
たぶん^{わたし}{私}だろう

\S ^{じぶん}{自分}なりの^{たの}{楽}しみ^{かた}{方}を
ほじくり^{だ}{出}す^{かのじょ}{彼女}の^{しんじょう}{心情}はすでに
\ ^{ほんらい}{本来}のスペックからは^{せつめい}{説明}できない

\S かつてロボットに「^{じりつ}{自律}^{しん}{心}」はおろか^{まんぞく}{満
  足}な^{だいよう}{代用}^{ひん}{品}すらなく

\S せめてそのきっかけでも^{て}{手}に^{はい}{入}れようとしていた^{ころ}{頃}の^{こと}{事}を
^{わたし}{私}は^{おも}{思}い^{だ}{出}す

\S アルファさんーーーー
\ A7^{りょうさん}{量産}^{しさく}{試作}^{き}{機}M2の3^{たい}{体}の
うちのひとり

\S ^{かのじょ}{彼女}の^{そんざい}{存在}がまだ^{とお}{遠}い^{ゆめ}{夢}
だった^{むかし}{昔}のことだ

\page
\S ^{かながわ}{神奈川}・^{よこすか}{横須賀}^{ま}{馬}^{ほり}{堀}^{かいがん}{海岸}

\S ^{いちどめ}{1度目}の^{だい }{大}^{ こうちょう}{高潮}がようやくいさまり
\ ^{まち}{町}や^{みなと}{港}の^{た}{立}て^{なお}{直}しが^{はじ}{始}まった^{ころ}{頃}

\page
\S ^{しゅよう}{主要}^{こうわん}{港湾}の^{ざんてい}{暫定}^{てき}{的}
^{さいかい}{再開}^{いしゅうかん}{1週間}^{まえ}{前}だ

\S ^{わたし}{私}たちは^{たいがん}{対岸}の^{ちば}{千葉}・^{ふなばし}{船橋}までの
50キロを^{ちょくせん}{直線}で^{むす}{結}ぶ^{こうそく }{高速}^{ じゅうだん}{縦断}に^{いど}{挑}む

\page
\S すべてが^{つなわた}{綱渡}りだった
\ ^{どうりょく}{動力}は^{くるま}{車}のターボチャージャーに
^{て}{手}のはえたような
\ ^{てせい}{手製}のジェットエンジンで

\S そして^{いがく}{医学}^{ぶ}{部}^{しょくいん}{職員}の^{わたし}{私}が^{そうじゅう}{操縦}する

\S ^{せんたい}{船体}のまっ^{きいろ}{黄色}なカラーリングもかなりダサめだ

\S でも^{ふね}{船}の^{せいさく}{製作}スタッフのひとりがつけたという
^{なまえ}{名前}は^{き}{気}に^{い}{入}った

\S 「ミサゴ」^{すいじょう}{水上}の^{たか}{鷹}

\page
\S エンジン^{しどう}{始動}

\S ^{なみ}{波}の^{しず}{静}かなうちにすべてを^{お}{終}わらせなければならない

\page
\S スタート

\page
\S すぐに^{ぜんかい}{全開}にする

\S あぶなっかしいタービン^{おと}{音}

\S ^{はら}{腹}に^{くうき}{空気}をためて
\ ミサゴが^{みなも}{水面}を^{と}{飛}ぶ

\S フルスロットルでないと^{あんてい}{安定}しない

\S だから^{ねんりょう}{燃料}は^{ごふん}{5分}しかもたない

\S そしてエンジンも^{ごふん}{5分}しかもたない

\page
\S グラグラ^{うご}{動}く^{そくど}{速度}^{けい}{計}が
^{じゅんこう}{巡航}^{そく}{速}の450キロあたりを^{しめ}{示}すが

\S ^{わたし}{私}は^{げんかい}{限界}^{そくど}{速度}の600キロ^{ちょう}{超}までひっぱる

\S ^{じつ}{実}はその^{ひ}{日}

\S 「^{そくど}{速度}」と「^{うみ}{海}を^{わた}{渡}る^{こと}{事}」^{いがい}{以外}に
もう^{ひとつ}{1つ}^{もくてき}{目的}があった

\page
\S 「^{きょくげん}{極限}の^{じょうきょう}{状況}である^{め}{目}^{じるし}{標}に
^{ちょうせん}{挑戦}する^{とき}{時}の^{にんげん}{人間}の^{かんかく}{感覚}
\ さらには^{げんかい}{限界}をこえる^{とき}{時}の^{たっせい}{達成}^{かん}{感}」

\S そのデータを^{え}{得}ること

\S だから^{わたし}{私}はそのデータの^{はっせい}{発生}^{げん}{源}^{じたい}{自体}でもあるわけだ

\page
\S ^{ふなばし}{船橋}のスキー^{ば}{場}が^{み}{見}えてくる

\S ^{ねんりょう}{燃料}^{けい}{計}はすでにEだ

\S ^{じそく}{時速}
\ 620ーー

\page[74]
\S この^{とき}{時}のデータがいずれ^{きかい}{機械}に^{じりつ}{自律}^{しょう}{性}を
^{も}{持}たせるきっかけくらいにはなると^{おも}{思}われたが

\S ^{けっきょく}{結局}たいして^{やく}{役}にはたたなかった

\S その^{あと}{後}^{まった}{全}く^{べつ}{別}の^{してん}{視点}から
アルファタイプが^{う}{生}まれることになる

\page
\S あの^{ひ}{日}のことはデータとしてはあまり^{い}{生}きなかったけど

\S ^{あと}{後}のアルファタイプ^{かいほつ}{開発}にとっては
\ その^{とき}{時}
\ ^{わたし}{私}が^{からだ}{体}で^{う}{得}た^{けいけん}{経験}
\ そのものの^{ほう}{方}がずっと^{じゅうよう}{重要}だった

\page
\A 「^{いちおう}{一応}お^{れい}{礼}のつまりです!」

\S アルファさんは^{いま}{今}もう^{じぶん}{自分}の^{みち}{道}を^{ある}{歩}いている


\subsection{第29話\ ひなた}

\page[80]
\K 「こんにちは」

\T こ
\ こんにちは

\A おーい

\page[82]
\M んーー
\ タカヒロねてたからさーー

\M となりにねたはずだけどな

\T おこせよ

\page
\T あれ
\ ^{きょう}{今日}おばさんと^{みなみ }{南}^{ まち}{町}^{い}{行}ってんじゃねえの?

\M ^{い}{行}かないよ

\M タカヒロのほうがいい

\T ほお

\T きのう^{あそ}{遊}んだんだから^{きょう}{今日}は^{で}{出}ないぞ

\M いいよ^{べつ}{別}に

\M ^{ほん}{本}こんなに^{か}{借}りてきたのによーー
\ ^{ずかん}{図鑑}だけど
\ ^{あした}{明日}^{かえ}{返}すんだぞこれ

\T いいじゃん^{よ}{読}んでれば?

\page
\T なんか^{へん}{変}なもん
\ うめえなーー
\ おめえ

\T ^{が}{蛾}?

\M ちょう!

\page
\T ^{はら}{腹}へったか?

\M うん!

\T ちょっとなんか^{つく}{作}るか

\T ここでまってな

\M うん!

\page
\T だからまってろってえばよ!!

\T あ〜〜
\ イモむける?

\M うん

\T じゃ
\ むけててね
\ 3コくらい

\page
\N じゃがいもとたまねぎのみそしる
\ トッピングにはばのり

\T おかわりは?

\M いる

\page
\T マッキよー

\M ん〜〜?

\T こんだ
\ いっしょにアルファんとこ^{い}{行}こうか

\M うん

\page
\T ほわ〜

\T あ〜〜〜
\ わりい
\ オレちっとねるわ

\M わたしもー

\page
\A いらっしゃい

\page
\T マッキだよ

\A ありゃま!

\A よっ

\M よっ

\A タカヒロ!
\ うちにも^{あ}{会}わせたい^{ひと}{人}がいるの

\A ^{あ}{会}ってくれる?

\T うん!

\page
\T ん〜

\T うーん
\ うー


\subsection{第30話\ カフェ\ アルファ}

\page[94]
\N もともと、この^{みち}{道}は^{さき}{先}の^{ほう}{方}にあった
\ ^{べっそう}{別荘}^{がい}{街}のためにつくられたらしい

\page
\N アルファルトの^{ほそう}{舗装}は^{いま}{今}となっては、
^{なお}{直}しようがないと^{み}{見}えて^{く}{来}るたびに、ボロくなっていく

\page
\N ちょっと^{しん}{信}じられないが、この^{さき}{先}に^{きっさ}{喫茶}^{てん}{店}がある

\N カフェ\ アルファ

\N ^{むかし}{昔}の^{べいぐん}{米軍}^{じゅうたく}{住宅}^{ふう}{風}、
^{しら}{白}ペンキの^{いえ}{家}を^{かいぞう}{改造}した^{みせ}{店}だ

\page
\N ^{いきお}{勢}いの^{つよ}{強}いススキの^{のはら}{野原}に
かろうじて^{う}{埋}もれずにある

\page
\A あっ
\ いらっしゃいませ!

\N ごぶさた

\page
\A ありゃ!

\A ひさしぶりですね!
\ あ
\ どうぞ

\N まー
\ ちょっとブラブラとね

\A あいかわらずヒマですねー
\ うちみたいですねー

\A あはは

\page
\A えーーと
\ アフェオレ?でしたっけ?

\A うん
\ ^{ぎゅうにゅう}{牛乳}あったらね

\N カフェオレ
\ のめるようになった?

\A ^{わたし}{私}?
\ だめ!ぜんぜん

\page
\A えへへ
\ ごいっしょしてもいいですか?

\N え?
\ あ
\ いいよ

\A 5・6と

\N アルファさんはいつも^{じぶん}{自分}から^{らい}{来}ながら、
やたら^{て}{照}れくさそうな^{かお}{顔}をする

\N そして^{あまとう}{甘党}

\page
\N じわじわと^{かげ}{影}がうごいていく

\N アルファさんはめったに^{こ}{来}ない^{きゃく}{客}の^{わたし}{私}に
^{ひ}{日}ごろのことを^{きき}{喜々}として^{はな}{話}し

\page
\N かと^{おも}{思}うと

\A ^{たいよう}{太陽}きついですか

\N あ?
\ いや
\ ^{へいき}{平気}だよ

\page
\N アルファさんは^{じぶん}{自分}ではなにも^{も}{持}ってないと
^{おも}{思}ってるかもしれない

\N でも、^{かのじょ}{彼女}に^{あ}{会}った^{ひと}{人}はきっと^{き}{気}づくことがある

\N お^{きゃく}{客}さんは、たぶん、それが^{み}{見}たくてここに^{く}{来}るんだと^{おも}{思}う

\page
\A おかわりなんてどうですか?

\N おお
\ いただき

\N ^{きょう}{今日}も^{わたし}{私}^{いがい}{以外}の^{きゃく}{客}は^{み}{見}なかった

\page
\N ごちそうさま

\A あ
\ お^{かえ}{帰}りですか

\A ーーまたいらして^{くだ}{下}さいね

\N うん

\page
\N この^{みせ}{店}はえらく^{ふべん}{不便}な^{とお}{遠}い^{ところ}{所}だ

\N でも、どの^{くらい }{位}^{ さき}{先}のことになっても、たぶん、また^{く}{来}る

\page
\N どれだけ^{あいだ}{間}があいても^{じょうれん}{常連}になれる^{みせ}{店}だ


\subsection{第31話\ ^{あか}{赤}い^{みず}{水}}

\page[110]
\A う

\page
\A このごろ、^{ときどき}{時々}^{いどみず}{井戸水}がしょっぱい

\A ^{きた}{北}の^{まち}{町}^{けいゆ}{経由}で^{き}{来}てる
^{すいどう}{水道}のコックがあったはずだけど

\page
\A あ〜〜
\ もう
\ この^{へん}{辺}なんだけどなあ

\A ^{きょう}{今日}^{やす}{休}もかな

\page
\A あった!

\A これをあけて^{いえ}{家}のきりかえコックをまわせば

\A よっ

\A あれ

\A いよっ!!

\A あ
\ ^{かた}{固}い

\page
\T ^{てつだ}{手伝}う?

\A タカヒロ!

\T なんかガサガサしてたから

\A ^{しんぞう}{心臓}とまるかと^{おも}{思}った

\A ^{しんぞう}{心臓}?

\T あーー
\ ごめん

\page
\T こう?

\A そうそう

\A せーの

\A よっ!!

\A やったーー!!

\page[117]
\A なんか
\ カゼひいちゃうってかんじ?

\T やばいね

\A ^{いま}{今}おフロにお^{ゆ}{湯}いれてるから

\T うん

\A じゃ
\ ま〜〜
\ そういうわけだ

\page
\A あーー
\ まあ、ハダカのつまあいってことで!

\T うん

\A あり?

\page
\T お^{ゆ}{湯}
\ まっかだね

\A そうね

\A しばらく^{みず}{水}^{だ}{出}しっぱなしにしなきゃ

\A ^{かんが}{考}えてみたら

\A ^{わたし}{私}の^{ばあい}{場合}カゼはひかないかもしれない
\ なーー

\page
\A ^{ふく}{服}、たぶん、^{きょう}{今日}かわかないから
\ それでガマンしてね

\T うん

\A あ?

\page
\A はい?

\O よ

\A あっ

\O あに
\ ^{みせ}{店}まだかよ

\A いえ、ちょっと^{みず}{水}まわりいじってて

\A あはは

\O おー
\ きのうから^{みず}{水}しょっぺえしなーー

\O まー
\ ^{ながつづ}{永続}きゃー
\ しねえと^{おも}{思}うけんどよーー

\O タカ^{き}{来}てねえ?

\page
\A ^{き}{来}てますよ

\O あんだ
\ ^{ふく}{服}かりてんのか

\O やっぱ^{ふ}{降}らいたか

\O こいつ^{で}{出}てから^{あめ}{雨}んなったからよーー

\A おじさんあがっていきません?

\O でも
\ へえよー

\O アルファさんこいからいそがしいから

\O タカいったん^{けえ}{帰}んか

\T うん

\page
\A じゃあ
\ タカヒロ、あとでね!

\T うん

\A あ

\A まずかったのだろうか

\page
\T おう

\M おう

\page
\T あによ
\ ^{き}{来}てたんならアルファに^{あ}{会}ってけばよかったのによ

\M こんどタカヒロと^{い}{行}く

\T あそう

\page
\M タカヒロ

\T あ?

\M ^{かえ}{帰}ったらおフロはいろう

\T あんでよ

\A ふう


\subsection{Short Essay\ ^{そっきょうきょく}{即興曲}}

\page[127]
\A ^{うみ}{海}から、しょっぱい^{かぜ}{風}が
とろとろと^{く}{来}る
\ ^{じこく}{時刻}になりました

\A ^{きょう}{今日}は、これを
あなたのために^{ひ}{弾}きましょう

\page
\A この^{じかん}{時間}は^{つる}{弦}の^{おと}{音}が
^{えきたい}{液体}のような^{くうき}{空気}とよく^{と}{溶}けて
^{みょう}{妙}に^{ごかん}{五感}にひびきます

\page
\A ^{つる}{弦}の^{おと}{音}は
どんな^{かお}{香}りが^{み}{見}えるでしょうか

\A ^{こんじょう}{今生}まれているこの^{きょく}{曲}にとって
あなたの^{そんざい}{存在}は、とても^{たいせつ}{大切}

\page
\A よろしかったら、
^{いっしょ}{一緒}にうたってくださいね
\ ^{くら}{暗}くなるまでには
まだしばらくあります

\A いちばんおいしい^{じかん}{時間}です


\subsection{マッキのコマセでゴー!}
\Y フナムシとあそぶ^{しょうじょ}{少女}

\Y メルヒェンとやつ?

\M あははは

\Y ガキってやつぁーよー
