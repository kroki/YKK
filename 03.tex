\section{Volume 3}

\subsection{^{だい}{第}16^{わ}{話}\ エプロン}

\page[5]
\Alpha まだ^{よこはま}{横浜}にでも^{い}{行}ってみようかなー

\Alpha いらっしゃいませ!

\page[6]
\Kokone こんにちは

\Alpha わーーっ!!

\Alpha うわーーっ!!

\page[7]
\Kokone ^{しごと}{仕事}^{ちゅう}{中}おじゃまします
\ ^{こんかい}{今回}は^{あそ}{遊}びなんですけど

\Alpha ^{ぜんぜん}{全然}じゃまじゃないよ!

\Alpha うわ〜〜
\ ありがとう
\ ^{き}{来}てくれて!
\ びっくりしちゃったよ

\Kokone あはは
\ あのー

\Kokone 10^{にち}{日}^{くらい}{位}^{まえ}{前}に^{てがみ}{手紙}^{おく}{送}ったんですけど

\Alpha えっ

\Alpha あ

\Alpha ^{ゆうびんばこ}{郵便箱}^{さいきん}{最近}ぜんぜん^{み}{見}てなかった

\Kokone はあ

\page[8]
\Alpha ひょっとしてまた^{ある}{歩}いてきたとか

\Kokone いえ

\Kokone ^{せんじつ}{先日}スクーター^{か}{買}っちゃって
\ それで

\Alpha えっ!
\ ^{み}{見}せて^{み}{見}せて!

\page[9]
\Alpha ありゃま!

\Alpha かわいいネ!
\ ^{なん}{何}㏄?

\Kokone あ

\Kokone これ^{でんどう}{電動}なんです
\ パワー90㏄^{くらい}{位}かな?

\Alpha ^{でんどう}{電動}

\Alpha ちょ
\ ちょっと^{の}{乗}っていいかな?

\Kokone ええ
\ はい
\ キー

\page[10]
\Alpha く

\Alpha これって
\ キーONにするとき

\Kokone あ
\ きましたか

\Kokone すぐ^{な}{慣}れますよ
\ ^{わたし}{私}もありました

\Alpha はあ

\page[11]
\Alpha ん

\Alpha ちょっと^{きも}{気持}ちいいね

\Kokone そうなんですよ

\Alpha これ^{しず}{静}か!

\Kokone ^{き}{気}に^{い}{入}ってます
\ アルファさんのにも^{に}{似}てますし

\Alpha ^{こんど}{今度}どっか^{い}{行}こうか!

\Kokone いいですね

\page[12]
\Kokone あの
\ このへんにいい^{やど}{宿}ありませんか

\Alpha え
\ ^{きょう}{今日}^{と}{泊}まりなの?

\Kokone ええ
\ せっかくこっち^{き}{来}たんだし
\ ゆっくりしてこうかと

\Alpha なーんだ

\Alpha うちにとまってよ!
\ このへん^{やど}{宿}ないし

\Kokone え

\Kokone でも
\ そんな
\ じゃ

\page[13]
\Kokone あの
\ ^{なに}{何}かお^{てつだ}{手伝}いします

\Alpha んーん

\Kokone でも
\ なんかおちつかなくて

\Alpha いそがしいわけじゃなし

\Alpha エプロン^{つ}{着}けてみる?
\ ^{たいき}{待機}^{ちゅう}{中}のスタッフって^{かん}{感}じで

\Kokone はい

\page[14]
\Kokone ^{へん}{変}ですか?

\Alpha あっ
\ いやいや

\Alpha なんか「^{きっさ}{喫茶}^{てん}{店}」って^{かん}{感}じだなーって

\Kokone は?

\page[15]
\Kokone あの
\ もしよかったら

\Kokone コーヒーのいれ^{かた}{方}^{おし}{教}えていただけますか?

\Alpha え〜〜
\ でも^{わたし}{私}のやき^{かた}{方}たぶん^{ただ}{正}しくないよ

\Kokone そのやり^{かた}{方}でぜひ

\Alpha うん

\page[16]
\Ojisan よ

\page[17]
\Kokone いらっしゃいませ!

\Alpha どうしました?

\Ojisan あ〜〜
\ うや

\Ojisan 「^{きっさ}{喫茶}^{てん}{店}」みてえだなって^{おも}{思}ってよ

\Alpha はあ

\page[18]
\Kokone あの
\ さきほどはどうも

\Ojisan おう

\Alpha あ
\ ^{よ}{寄}ったんだ

\Alpha いきなり
\ にぎやかな^{ひ}{日}


\subsection{第17話\ ^{なみ}{波}}

\page[20]
\Kokone アルファさんの^{へや}{部屋}はらしいといえばらしいです

\page[21]
\Alpha ん?

\Kokone お^{さかな}{魚}^{す}{好}きなんですね

\Alpha うん

\Alpha なんとなくね
\ ^{むかし}{昔}から

\page[22]
\Alpha ちょこちょこ^{つく}{作}ったり^{ひろ}{拾}ったりした^{もの}{物}だけど

\Alpha ずいぶんたまったわね

\Kokone はー

\Kokone ^{とく}{特}に^{め}{目}を^{ひ}{引}いたこの^{がっき}{楽器}

\Kokone オーナーさんのもので
\ ^{げっきん}{月琴}と^{い}{言}うそうです

\page[23]
\Alpha ちょっと^{まえ}{前}までね

\Alpha ^{ひとまえ}{人前}でなんて^{ひ}{弾}かなかったのよ

\Alpha ^{さいきん}{最近}だなー

\page[24]
\Kokone ^{はじ}{初}めて^{み}{見}る^{ひょうじょう}{表情}

\page[25]
\Kokone ^{むかし}{昔}の^{どうよう}{童謡}のような
\ でもちょっとブラジル^{ふう}{風}の

\Kokone はじまりはそんな^{きょく}{曲}でした

\page[26]
\Kokone なにかコトツとはまった^{かん}{感}じがして

\Kokone ^{し}{知}ってる^{うた}{歌}みたいにハミングできます

\page[27]
\Kokone ^{きょく}{曲}は^{か}{変}わってゆき

\Kokone アルファさんもハミング

\Kokone ^{ふたり}{2人}の^{こえ}{声}
\ ^{げっきん}{月琴}の^{おと}{音}

\Kokone ^{みつ}{3つ}^{と}{溶}けて
\ なにも^{き}{聞}こえません

\page[30]
\Kokone ^{ふしぎ}{不思議}な^{じかん}{時間}でした

\Kokone おたがいが^{つぎ}{次}に^{くち}{口}にする^{こえ}{声}も
\ ^{いき}{息}つぎをする^{ところ}{所}もわかります

\page[31]
\Kokone ^{さいご}{最後}の^{つる}{弦}の^{しんどう}{振動}が
\ ^{くうちゅう}{空中}から^{き}{消}えるまでの^{あいだ}{間}

\page[33]
\Alpha こういうことって
\ あるのね

\Kokone どうやって^{うた}{歌}ったんでしょう

\page[34]
\Kokone もう^{おも}{思}い^{で}{出}せません

\Alpha お
\ お^{ちゃ}{茶}かな?

\Kokone あ
\ そうですね


\subsection{第18話\ ^{こと}{古都}ムサシノ}

\page[37]
\Kokone こんにちは!

\Person あら

\Person ココちゃん
\ ^{ちこく}{遅刻}じゃねえの?

\Kokone ^{きょう}{今日}は^{おそばん}{遅番}なんです

\Kokone あ

\Kokone お^{やちん}{家賃}は^{こんげつ}{今月}^{ちゅう}{中}に

\Person おお

\page[39]
\Kokone アルファさんへ

\Kokone ^{せんじつ}{先日}はいきなりでごめんなさい

\page[41]
\Kokone あの^{ひ}{日}は^{わたし}{私}にとってすごく^{たいせつ}{大切}な^{ひ}{日}になってます

\page[42]
\Kokone あのあと^{じぶん}{自分}のこといろいろ^{かんが}{考}えちゃいました

\Sign かんぱち
\ ^{つじ}{辻}の^{ちゃ}{茶}

\Kokone ^{いぜん}{以前}なんでも「もっと^{にんげん}{人間}っぽく」から^{はじ}{始}めてた^{こと}{事}とか

\page[43]
\Kokone ^{さいしょ}{最初}にアルファさんに^{あ}{会}ったとき

\Kokone ^{じぶん}{自分}が^{きよう}{器用}に「ロボットっぽくなく」ふるまうのが
\ ^{み}{見}えてしまいまして

\Kokone それでだいぶ^{かた}{肩}の^{ちから}{力}が^{ぬ}{抜}けて

\Kokone ^{きお}{気負}わない^{い}{生}き^{かた}{方}のお^{てほん}{手本}を
\ アルファさんに^{もと}{求}めました

\Kokone ^{にどめ}{2度目}に^{あ}{会}うまで

\page[44]
\Kokone ^{せんじつ}{先日}のことは^{しょうげき}{衝撃}でした

\Kokone ^{こころ}{心}を^{あそ}{遊}ばせるのに
\ なにも
\ ^{ひと}{人}のやり^{かた}{方}である
\ ^{ひつよう}{必要}はないんだと

\Kokone ^{わたし}{私}はまだ^{じょうず}{上手}に^{ひと}{人}のまねをしようとしてたみたいです

\page[45]
\Kokone ^{わたし}{私}たちがどれほどのものなのか

\Sign アルファタイプ^{しょうさい}{詳細}

\Sign A-7 ^{かいはつ}{開発}^{し}{史}

\Sign ALPHA7 series data

\Sign ^{ちゅうごく}{中国}の^{おんがく}{音楽}^{くに}{国}^{つる}{弦}^{がっき}{楽器}

\page[46]
\Kokone このごろ^{きょうみ}{興味}がでてきました

\Kokone でも
\ そういうののってない

\Kokone ^{いま}{今}はロボットって^{こと}{事}は^{こせい}{個性}のひとつなんだなって^{おも}{思}います

\page[47]
\Kokone ^{さいきん}{最近}^{すこ}{少}し^{きらく}{気楽}になる
\ ^{ほうほう}{方法}を^{み}{見}つけました

\Kokone あまり^{きらく}{気楽}になろうとしないことです

\page[48]
\Kokone ^{わたし}{私}
\ アルファさんに^{あ}{会}えてほんとによかった

\Sign の^{ゆうびん}{郵便}^{しゃ}{社}

\Sign ^{こう}{高}^{いど}{井戸}^{てん}{店}

\Kokone おおげさですか?
\ でもいいです!

\page[49]
\Kokone また^{あそ}{遊}びに^{い}{行}きますね
\ ^{ぜったい}{絶対}!

\page[50]
\Kokone それでは
\ また
\ ^{たか}{鷹}^{つ}{津}ココネ

\Alpha なんか
\ ふかーく^{かんが}{考}えてるし

\Alpha らしいというか


\subsection{第19話\ ^{きた}{北}の^{おお}{大}^{くず}{崩}れ}

\page[52]
\Alpha ^{きた}{北}の^{おお}{大}^{くず}{崩}れ

\Alpha ^{かいめん}{海面}^{じょう}{上}を^{き}{来}た^{かぜ}{風}がぶちあたっていく^{ところ}{所}です

\page[53]
\Alpha ^{へん}{変}な^{ばめん}{場面}を^{み}{見}ました

\page[55]
\Alpha すごい!!

\Ayase わっ!!

\Ayase びっくりしたーー!

\Alpha ああっ!

\page[57]
\Alpha あ〜〜

\Alpha いっちゃった

\Ayase おー
\ また^{く}{来}んよ

\Alpha え?

\Alpha あっ

\page[58]
\Ayase へえよ

\Alpha すっ
\ すごい!

\Alpha し
\ ^{し}{知}り^{あ}{合}いですか?

\Ayase ^{し}{知}り^{あ}{合}い?
\ つうか
\ まあそうな

\page[59]
\Alpha へー

\page[60]
\Alpha すごくじょうずですね
\ ^{と}{飛}ぶの

\Ayase ^{さかな}{魚}のくせになー

\Ayase ^{かわ}{皮}とかワックスっぽくてよー

\Ayase あんか^{す}{好}きみてえよ
\ ^{そと}{外}
\ ^{め}{目}だきゃーゴーグルいんけど

\page[61]
\Alpha あの
\ この^{へん}{辺}の^{かた}{方}でしょ

\Ayase おお
\ ^{で}{出}はなあんま^{く}{来}ねえけんどよ

\Ayase ^{きょう}{今日}はちっとばっか^{こじんてき}{個人的}イベントでよー
\ あによ

\Alpha ごめんなさい
\ ちょっとね

\Alpha くくく

\Alpha ^{し}{知}ってるおじさん^{おも}{思}いたしちゃった
\ しゃべり^{ほう}{方}で

\Ayase あ〜ん?

\page[62]
\Alpha おじゃましました

\Ayase おー
\ あんだえー

\Alpha ああ
\ ^{わたし}{私}この^{みなみ}{南}の「^{にし}{西}の^{みさき}{岬}」でコーヒー^{や}{屋}やってんですよ

\Alpha ^{ちか}{近}くに^{き}{来}たらよってくださいね

\page[63]
\Ayase ^{にし}{西}の^{みさき}{岬}

\Ayase おれの^{もと}{元}フィールドじゃんかよ
\ ^{みせ}{店}なんかあったっけ?

\Alpha ^{きんじょ}{近所}じゃ^{ゆうめい}{有名}なんですよ!

\Alpha って^{き}{聞}いた

\Ayase へー

\Ayase じゃ
\ ^{い}{行}くようならよるわ

\Alpha そうしてください

\Ayase ああ

\page[64]
\Ayase タカヒロって^{し}{知}ってんか?

\Ayase あによ

\Ayase ^{へいき}{平気}かよ

\Alpha こっち〜

\page[65]
\Alpha タっ
\ タカヒロ!?

\Alpha あ
\ どうも

\Ayase ^{し}{知}ってんの

\Alpha ^{し}{知}ってるもなにも

\Ayase そーかー
\ じゃーよろしくってゆっといてけーねー?

\page[66]
\Ayase アヤセが^{せ}{言}ってたってよ

\Ayase いや
\ ^{まえ}{前}ちょっとミサゴのことでよー

\Alpha ミっ
\ ミサゴー!?

\Ayase おう

\Ayase うるせえやつだな


\subsection{第20話\ ^{ほう}{鵬}(HOH)}


\page[68]
\Alpha そういえば^{い}{言}ってましたよ
\ タカヒロ

\Alpha ^{さかな}{魚}とばすおっさんがいたって

\Ayase で
\ こいつがタカヒロの^{い}{言}う^{おんな}{女}の^{こ}{子}な

\page[69]
\Alpha ぜひ
\ ^{みせ}{店}に^{き}{来}てくださいよ

\Alpha タカヒロも^{よ}{呼}んで
\ ん〜〜ちょっと^{とお}{遠}いですけどね

\Alpha 20^{キロ}{km}くらいあるから

\Alpha ああ
\ ^{ふたり}{二人}^{の}{乗}りでもよければ^{いっしょ}{一緒}に

\Ayase ^{きょう}{今日}はパスするわ
\ あっち^{やど}{宿}ねえし
\ ^{のじゅく}{野宿}にゃ^{はや}{早}えし

\page[70]
\Alpha タカヒロんちで^{と}{泊}めてくれますよ

\Alpha ん〜〜なんならうちでもいいですし

\Ayase おめえ
\ ちっとは^{けいかい}{警戒}しろよ

\Alpha へ?

\Ayase へ?じゃねえよ

\Ayase いや
\ いかにも「^{さいかい}{再会}」っつうのはあんかヤだしよーー

\Ayase ^{かまくら}{鎌倉}の^{ほう}{方}^{い}{行}くよ

\Alpha なんだ〜〜
\ そうか〜〜

\page[71]
\Ayase けえんじゃねえの?

\Alpha いえ
\ せっかくだから

\Alpha アヤセさん^{かえ}{帰}るまでいますよ

\Ayase あそう

\Ayase さっきおれ^{こじんてき}{個人的}イベントっつったっけ?

\Ayase そいでここにいんだよ

\page[72]
\Ayase ^{ひこう}{飛行}^{き}{機}^{し}{知}ってんべ?

\Alpha は?

\Alpha ヒコウキ

\Alpha そりゃもう

\Alpha あの
\ ブーンとかゴワーとかいってとぶやつ!

\Alpha たま〜〜に^{とお}{通}りますよ

\Ayase ブーンとかゴワー

\Ayase これからそれの^{めずら}{珍}しいのが^{とお}{通}んからよ

\Ayase ^{み}{見}てな

\Alpha ほー

\Ayase あと^{じゅっぷん}{10分}ぐれえだ

\page[73]
\Ayase ^{き}{来}た

\Alpha トンビ?

\Ayase もっと^{うえ}{上}!
\ ^{あお}{青}いとこ

\page[76]
\Alpha ^{おと}{音}がしない

\Ayase ^{こうど}{高度}が^{なみ}{並}じゃねえからなあ

\Alpha きれい

\Ayase このラインを^{きょう}{今日}^{とお}{通}るって^{まえ}{前}に^{き}{聞}いてよ

\page[77]
\Ayase 「ターポン」っつんだあれ

\Alpha ターポン?

\Alpha ^{へん}{変}な^{なまえ}{名前}

\Alpha ^{ひと}{人}のってんのかな

\Ayase おお
\ もうおりらんねえらしいや

\Ayase ずーっと^{と}{飛}んでる

\page[80]
\Alpha あの
\ ありがとうございました

\Ayase あん?
\ あにが?

\Alpha いえ

\Ayase こういうやつか

\page[81]
\Ayase じゃな

\Alpha ^{こんど}{今度}は^{みせ}{店}に^{き}{来}てくださいね

\page[82]
\Alpha いろんな^{ひと}{人}たち

\Alpha ずっと^{そら}{空}にいる^{ひと}{人}
\ ずっと^{ある}{歩}いてる^{ひと}{人}

\Alpha ゴワーってね


\subsection{第21話\ ^{みず}{水}^{がみ}{神}さま}

\page[84]
\Narrator アルファさんに^{あ}{会}ったことをタカヒロに^{はな}{話}したんだけど

\page[85]
\Alpha でね

\Alpha タカヒロによろしくだってさ!

\Takahiro へ〜〜

\Takahiro そっか〜〜

\Alpha あり?

\page[86]
\Person ^{にい}{兄}ちゃん
\ ^{いわつき}{岩槻}かい

\Person ^{ふね}{舟}が^{ちか}{近}いよ

\Ayase いやーー

\Ayase ^{おおみや}{大宮}の^{ほう}{方}^{まわ}{回}るんすよ

\Ayase ^{みず}{水}^{がみ}{神}さん^{おが}{拝}んでみてえんで

\page[87]
\Person あれなー

\Person たぶんよそ^{さま}{様}にゃ^{み}{見}せねえと^{おも}{思}うよ

\Person この^{さき}{先}の^{うみぞ}{海沿}いだけどね

\Ayase どうも

\Ayase さいたまの^{くに}{国}では^{み}{見}^{ぬま}{沼}^{いりえ}{入江}の
 「^{みず}{水}^{がみ}{神}さん」を^{おが}{拝}めと^{むかし}{昔}すすめられた

\Ayase ^{しょうかいじょう}{紹介状}もある

\page[88]
\Ayase あ
\ どうも

\Sign ^{さんぱい}{参拝}^{うけつけ}{受付}

\Ayase あのー
\ ^{みず}{水}^{がみ}{神}さん^{おが}{拝}みてえんすけど

\page[89]
\Person すいません
\ よそ^{さま}{様}にはお^{み}{見}せできなおんで

\Person ここから^{おが}{拝}んでいってくれませんかね

\Ayase やっぱ?

\Ayase ^{しょうかいじょう}{紹介状}があるんですが

\Person ほお

\Person おお

\Person ^{はつせの}{初瀬野}^{せんせい}{先生}のお^{し}{知}り^{あ}{合}いなら
\ どうぞ

\Ayase ああ
\ どっ
\ どうもすいません

\page[90]
\Person すぐそこですよ

\Ayase お^{やしろ}{社}はねえみてえすね

\Person はあ
\ ^{つく}{作}らん^{ほう}{方}がよかんべーと
\ ^{せんせい}{先生}もおっしゃったんで

\Ayase ははあ

\page[91]
\Person あそこです

\Ayase おお

\Ayase ^{しろ}{白}いすね
\ ^{だいり}{大理}^{せき}{石}
\ ^{せきえい}{石英}^{せい}{製}かな

\Person どうぞ^{ちか}{近}くで
\ さわらないでくださいね

\Ayase はい
\ じゃ

\Ayase ちっと^{しつれい}{失礼}して

\page[93]
\Ayase こっ

\Person ^{みず}{水}^{がみ}{神}さんです

\page[94]
\Ayase ^{にんげん}{人間}のように^{み}{見}えた

\Ayase しかし
\ ^{からだ}{体}は^{しろ}{白}いわたのようなものでおおわれ

\Ayase ^{かお}{顔}はビロードっぽい
\ ^{こま}{細}かい^{きもう}{起毛}で^{しろ}{白}く^{ひか}{光}っている

\Ayase ^{い}{生}きてるみてえだ

\Person はあ
\ ^{い}{生}きてます

\Ayase え

\Person ^{のうは}{脳波}があります

\page[95]
\Person たぶん
\ ^{ひと}{人}の^{こども}{子供}かと
\ ^{み}{見}つかってからずっと^{え}{絵}みたいに^{うご}{動}きませんが

\Person さわらん^{ほう}{方}がいいだろうって
\ ことなんで
\ ^{とち}{土地}の^{もん}{者}で
\ ^{ばん}{番}してるんですわ

\Person 「^{いりえ}{入江}の^{かみ}{神}さん」ってわけです

\page[96]
\Person なんせ
\ ^{ぜんぜん}{全然}^{とし}{年}をとらんもんで

\page[97]
\Ayase いや
\ どうも

\Person いいえ

\Ayase じゃ
\ やっぱ^{き}{来}て^{せいかい}{正解}でした

\Person はあ
\ ご^{くろう}{苦労}さんでした

\Ayase ん〜

\page[98]
\Alpha アヤセさん^{しょうかい}{紹介}したかったのはわかるけどさーー

\Alpha もう^{きげん}{機嫌}なおしてよ〜〜

\Takahiro あんまりムネおしつけないでよ

\Takahiro あだだ
\ あだだだ


\subsection{第22話\ ヨコスカ^{じゅんこう}{巡航}}

\page[102]
\Alpha カメラのサラサラしたボディをなでていて

\Alpha いつのまにか^{ごご}{午後}になっちゃった^{ひ}{日}

\page[104]
\Alpha おもいだした!

\Alpha ^{いちばん}{1番}はじめのころオーナーにつれてってもらったところ

\Alpha なんで^{わす}{忘}れてたかなー

\Alpha おじさーん

\page[105]
\Alpha やほーーーー

\Ojisan おう

\Ojisan あに
\ どこに^{い}{行}くのヨ

\Alpha ^{こじんてき}{個人的}イベントー!

\Ojisan あんだそりゃー

\page[106]
\Alpha あそこはたしか^{きた}{北}の^{まち}{町}に^{い}{行}ったときに

\Alpha ^{いま}{今}はもうなつかしさとイメージだけだけど

\Alpha ^{あか}{明}るいうちに^{つ}{着}けるかな

\page[107]
\Alpha ここだ!
\ やっぱり

\Alpha なんか^{かん}{感}じがちがうけど

\Alpha ここだ

\Alpha ん〜

\Sensei アルファさん?

\Alpha えっ

\page[108]
\Alpha ^{せんせい}{先生}!

\Sensei あ
\ やっぱり

\Sensei へーー
\ ここにはよく^{く}{来}るの?

\Alpha あはは

\Alpha あ
\ いえ
\ ^{まえ}{前}に^{いっかい}{1回}だけ^{き}{来}たような

\Alpha ^{せんせい}{先生}はきょうは

\Sensei ^{きた}{北}の^{まち}{町}にね
\ ^{ようじ}{用事}のついて

\Alpha はー
\ ここにはよく?

\Sensei たまにね

\page[109]
\Alpha ^{ひ}{日}が^{しず}{沈}む^{ごろ}{頃}から

\Alpha ^{きも}{気持}ちが^{たか}{高}ぶってきてとまらなくなる

\Alpha ^{へん}{変}に^{むね}{胸}かぐ〜〜っとして

\Sensei どしたの?

\Alpha えへへ
\ なんだかもりあがってきちゃって

\Sensei そう

\Sensei ^{わたし}{私}もつきあおうかな

\Alpha やがて^{あお}{青}が^{こ}{濃}くなり^{よる}{夜}になる
\ ^{ひとり}{1人}でいたら^{たぶん}{多分}^{かえ}{帰}る^{こと}{事}を^{かんが}{考}えていたころ

\Alpha ^{じょじょ}{徐々}に^{はじ}{始}まった

\page[112]
\Alpha こっ
\ こんなに

\page[113]
\Sensei そうね

\Sensei ここは^{むかし}{昔}の^{がいとう}{街灯}がたくさんのこってるわ

\Sensei たぶんあなた^{むかし}{昔}ここで
\ この^{やけい}{夜景}を^{み}{見}てるのね

\Sensei ^{いま}{今}はただの^{はやし}{林}になってる^{ところ}{所}も
^{うみ}{海}の^{うえ}{上}の^{がいとう}{街灯}も^{ひか}{光}る

\Sensei かつて^{じつよう}{実用}のために^{ひか}{光}っていた^{がいとう}{街灯}

\page[114]
\Sensei ^{いま}{今}はただ^{ひか}{光}るためだけに^{ひか}{光}る

\Sensei ^{み}{見}る^{ひと}{人}いないけど
\ こんな^{はな}{華}もあっていいわね

\page[116]
\Sensei ^{むかし}{昔}の^{ひと}{人}がのこしてくれた^{ひかり}{光}の^{はな}{花}


\subsection{第23話\ ^{いま}{今}の^{ひと}{人}}

\page[130]
\Alpha はい?

\Takahiro だれ?

\Alpha ^{いし}{石}がね

\Alpha だれもいないよねえ

\Takahiro うん

\page[131]
\Takahiro う?

\Alpha ん?

\Takahiro なんかおちてきた

\Alpha なっ
\ なに?

\Takahiro ^{かに}{蟹}


\subsection{^{じゃくねん}{若年}じじいのそこがいいっす!}

\Sensei じゃ
\ ^{い}{行}こうか!

\Ojisan ういす

\Ojisan ^{きょう}{今日}も^{ふたり}{2人}っきりで^{ひとけ}{人気}のない^{ところ}{所}に

\Ojisan センパイ
\ ^{じつ}{実}はオレのこと

\Ojisan ん〜

\Ojisan お?

\Sensei あーー
\ ^{だいじょうぶ}{大丈夫}ね?

\Ojisan ういす

\Sensei ^{せんこう}{先行}ってんよー

\Ojisan ちがう
\ これはちがう
