\section{Volume 13}

\subsection{^{だい}{第}121^{わ}{話}\ 50km、6^{じ}{時}}

\page[5]
\M ^{はかい}{破壊}?

\A う

\A いや、こんぐらいやんないと、ムレるんだ、^{けっきょく}{結局}

\M ほ〜〜

\page[7]
\SH おっ

\K あっ

\SH ^{わたし}{私}の^{ほう}{方}が^{さき}{先}かと^{おも}{思}ったよ

\K へへ

\P いらっしゃいませ

\SH あ、これもひとつ

\P はい

\page
\K おつかれさまー
\ ^{はいそう}{配送}ありがと

\K おかげで、こっちも^{はや}{早}くおわったよ

\SH んー

\SH あー
\ あっちの^{えいぎょうしょ}{営業所}でさあ……
\ ^{い}{言}うんだよ

\SH あそこって^{どくしん}{独身}の^{おとこ}{男}ばっかじゃん

\K あ
\ うん…

\SH 「あれ?なんでココちゃんじゃないの?」だってさ
\ ストレートに^{しつれい}{失礼}だよねー

\page
\K え……でも、^{かみきた}{上北}の^{えいぎょうしょ}{営業所}って
^{はな}{話}しかけてくれる^{ひと}{人}あんまりいないよ

\SH みんなあんたを^{み}{見}ている……
\ いろいろ^{き}{聞}かれたよ〜〜

\K えっ……
\ どどどんな…

\SH ^{わたし}{私}はさー
\ 「^{ほんにん}{本人}に^{き}{聞}け!」って^{い}{言}ったんだけどねー
\ たとえばーーー

\page
\SH ファン^{おお}{多}いよー
\ うらやましい

\SH あ〜〜あ
\ ^{わたし}{私}に^{よ}{寄}ってきたヤローなんて、^{けっきょく}{結局}、^{ひとり}{一人}だけだったなあ

\K 「だった」だなんて……

\K あの^{ひと}{人}とだってまだ、いくらでも^{なか}{仲}なおり、できるのに……

\SH やだ!
\ まだやだ

\page
\SH ^{まえ}{前}にも、^{き}{聞}いたんだけどさあ

\K うん

\SH ココネはこう……
\ いいなって^{おも}{思}うひととかいないの?

\K シバちゃんとー

\SH アルファさんとかじゃなくてー

\page
\SH あんたのこと、^{み}{見}てる^{おとこ}{男}、^{まえ}{前}から^{おお}{多}かったけど

\SH ^{き}{気}づいてはいたでしょ

\SH あのさ

\K ^{わたし}{私}、はっきり^{い}{言}って^{おんな}{女}と^{おとこ}{男}って、ピンとこないの

\K ^{あたま}{頭}ではわかってるんだけど……

\K なんてゆうか……
\ ^{からだ}{体}の^{さいぼう}{細胞}が、^{もと}{求}めてる^{かん}{感}じが、いまいちってゆうか…

\K ^{せだい}{世代}もすぐずれてっちゃうし……

\page
\K やっぱ

\K ロボットだからかな〜〜

\SH また〜〜

\SH すーぐ
\ そっちに^{に}{逃}げる……

\SH ココネ^{じぶん}{自分}の^{からだ}{体}のせいにしてすーぐ^{ぼうえ
  い}{防衛}ライン^{は}{張}っちゃう

\SH ^{おとこともだち}{男友達}とか^{とく}{特}にそう

\K そうだよ、きっと……
\ いいんですよ、それで……

\SH お
\ ^{ひら}{開}きなおった

\page
\SH まあ、いいけどね

\SH ^{おとこ}{男}にメロメロのココネか、^{み}{見}てえなー

\K やだなあ

\K ^{わたし}{私}は……
\ ^{いま}{今}は……

\K ^{いま}{今}、^{す}{好}きな^{ひとたち}{人達}ともっと^{なかよ}{中良}くしたいだけなんだ

\SH ^{しょうきょくてき}{消極的}だけど……

\page
\K シバちゃん^{す}{好}きだよ

\SH つまらんなー

\K つまんなくないよ

\page[17]
\A ^{さいきん}{最近}、わかったのはね

\A ここは、^{うみべ}{海辺}の^{たかだい}{高台}じゃなくて、^{なみう}{波打}ち^{ぎわ}{際}なんだってこと

\M わかってました

\A ^{ほんと}{本当}は、^{じゅう}{住}みづらい^{ところ}{所}

\M いいとこだけだね

\page
\M わっ!
\ ^{か}{蚊}っ!

\A ^{わたし}{私}、^{へいき}{平気}ー


\subsection{第122話\ スイカの^{ひ}{日}}

\page[21]
\A こんにちは

\O おぅ

\A すずしげで

\O あちいわ

\A はい

\A うちのナスとオクラです

\O あらよー

\page
\O スイカ^{も}{持}ってくか

\A ^{きょう}{今日}は……
\ ^{も}{持}ってけないかな〜〜

\O そっか、じゃ^{くるま}{車}で^{よる}{夜}^{も}{持}ってくわ

\A あ〜〜〜
\ ありがとございます

\O ああ
\ ^{いま}{今}、^{とうゆ}{灯油}しきゃねえや

\A ありゃま

\O ガソリンあさってぐれえ^{く}{来}んからよ

\O そいまで、ガスけちって^{はし}{走}ってけえな

\A わかりました

\page[24]
\O あに、マッキがるすばんか

\A あ、いえ
\ マッキちゃんは^{なつやす}{夏休}みなんです

\A こう^{あつ}{暑}いと^{にっちゅう}{日中}だれも^{き}{来}ませんし

\A ^{ひるま}{昼間}うちまで^{く}{来}るのは、マッキちゃんにもキツいです

\O あー

\page
\A タカヒロから
\ ^{てがみ}{手紙}とか、なんか、^{き}{来}ましたか?

\O いやー
\ ^{こ}{来}ねえわな

\O ^{こ}{来}ねえもんよ、あんぐれえの^{こ}{子}は

\A はあ

\page
\O いずれマッキも^{そと}{外}に^{で}{出}んわなー

\A はい

\O したら、10キロ^{しほう}{四方}ジジババばっかだわ

\O あんた^{いがい}{以外}

\page
\A マッキちゃんやタカヒロを^{み}{見}てると

\A ^{じかん}{時間}のスピードがわかりますね

\O んー

\O タカのやつよ

\O なんか、^{で}{出}てく^{まえ}{前}、^{きゅう}{急}に^{おとな}{大人}になりやんの

\page
\A はい

\Sign スイカ

\page
\O アルファさんよ

\A はい

\O おそかれ、^{はや}{早}かれよ

\O ここいら、もっと^{しず}{静}かになんべ

\page
\O アルファさん^{きづ}{気付}いてんだろうけんどよ

\O あんたんとこも、^{なみ}{波}にさらわれんと^{おも}{思}うわ

\A はい

\page
\O アルファさんよ

\A はい

\O いずれ、ここに^{す}{住}んで^{みせ}{店}やりゃいい

\O ここなら^{たけ}{高}えから

\O まあ^{あ}{飽}きんまで^{す}{住}んでられんべ

\page
\O アルファさんがやじゃなきゃよ

\A ありがとうござ……

\page[34]
\O まあ
\ まだ^{さき}{先}のこんだ

\A はい


\subsection{第123話\ ^{なつ}{夏}の^{お}{終}わりに}

\page[37]
\Y ^{なんねん}{何年}かに^{いちど}{一度}は、ここにもどって^{く}{来}る

\Y この^{たに}{谷}には、^{う}{生}まれた^{いえ}{家}があった

\Y ^{いま}{今}は、^{くさき}{草木}に^{う}{埋}まって、あとかたもないが

\page
\Y セミだらけの^{へん}{変}な^{あま}{甘}いにおいと

\Y きっと^{みみ}{耳}に^{よ}{良}くないセミ^{あらし}{嵐}の^{なか}{中}

\page
\Y ^{えもの}{獲物}はだいたいボラだ

\Y ボラという^{さかな}{魚}はそれほどすごくうれしい^{さかな}{魚}でもないが

\Y なにしろ、よくとれる

\page[42]
\M よ

\Y か〜〜!!

\M スキだらけー
\ ニブすぎー

\page
\Y ヒマそうだなあ

\M まあね

\M ^{まえ}{前}ほどじゃないけど

\page
\Y ほー

\M あのさ

\M この^{まえ}{前}の^{はなし}{話}だけど……

\Y ああ、「ついて^{く}{来}るか」って^{はなし}{話}な

\M うん
\ やっぱ、^{い}{行}かない

\Y そうか

\M なんかこう
\ やりたいことが^{で}{出}てきそうな、^{き}{気}がしてる

\Y そうか

\page
\Y たまに^{く}{来}んし……

\Y そいつと、また、^{あそ}{遊}んでけえな

\M うん

\page[47]
\M ^{かえ}{帰}る

\M アヤセはどうすんの?

\Y もうちっとしたら、^{やど}{宿}に^{かえ}{帰}んわ

\M ^{つぎ}{次}、いつ^{く}{来}んの?

\Y さー……
\ また^{なつ}{夏}ごろかなあ

\page[49]
\M じゃあね

\Y おう

\page
\Y ^{むかし}{昔}のことを^{おも}{思}い^{だ}{出}した

\Y ミサゴに、^{さいご}{最後}に^{あ}{会}った^{とき}{時}の^{きぶん}{気分}だ


\subsection{第124話\ ^{こどう}{鼓動}}

\page[52]
\Sign Kカー^{とくしゅう}{特集}
\ ’98

\A は

\A ^{なに}{何}さがしてたんだっけ

\page
\A いいか

\A ^{ひさ}{久}しぶりに^{ものおき}{物置}をあさった

\A もう^{なが}{長}いこと^{ぜんたい}{全体}を^{み}{見}ていないタルガトップのクルマがあって

\A その^{おく}{奥}に^{ふる}{古}い^{はこ}{箱}がある

\page
\A ^{なか}{中}にはサビサビの^{かん}{缶}とボロボロの^{ぬの}{布}

\A ^{ぬの}{布}の^{なか}{中}には^{きんぞく}{金属}のかたまりがあった

\A パッと^{み}{見}…

\A ^{いじょう}{異常}に^{ちい}{小}さいエンジンに^{み}{見}える

\page
\A ^{あぶら}{油}で^{へんしょく}{変色}しきった^{せつめい}{説明}^{しょ}{書}をどうにか^{よ}{読}むと、
^{もけい}{模型}^{ひこう}{飛行}^{き}{機}のエンジンとわかった

\A ^{し}{知}らなかったオーナーの^{しゅみ}{趣味}

\A ^{わたし}{私}が^{うご}{動}かしてあげる

\page
\A ^{たんじゅん}{単純}なつくりなので

\A そうじして、^{ねんりょう}{燃料}あげれば、^{うご}{動}きそう

\A プラグも^{い}{生}きてる……

\A で、この^{かん}{缶}が^{ねんりょう}{燃料}らしい

\A でも、この「グロー^{ねんりょう}{燃料}」ってなんだろう……

\A 3^{こ}{個}のうち2^{こ}{個}はほとんどカラで、なんかドロドロしてる

\A ひとつだけ、どうにか^{つか}{使}えそうな^{りょう}{量}があった

\A ^{ねんりょう}{燃料}の^{しょうたい}{正体}がわからない^{いま}{今}は

\A この^{えきたい}{液体}に^{か}{懸}けるしかない

\A ^{へん}{変}なにおい

\A ^{せきゆ}{石油}じゃないなあ

\page
\A ^{つぎ}{次}の^{ひ}{日}

\A おし

\A オーバーチョーク

\A 「エンジン^{うち }{内}^{ねんりょう}{燃料}でビショビショやりなおし」

\A 「プロペル^{ちゅうい}{注意}」

\A ふ〜…

\page
\A おお!

\A かかった!!

\A すごい^{おと}{音}!

\A コードをはずす…
\ ^{かいてん}{回転}をあげて、^{あんてい}{安定}させる

\page[60]
\A エンジンは
^{なんじゅうねん}{何十年}ぶりかの^{かいてん}{回転}の^{よろこ}{喜}びを^{ぜんりょく}{全力}で^{しめ}{示}す

\A ^{さいご}{最後}かもしれない^{ねんりょう}{燃料}を、
^{いってき}{一滴}も^{のこ}{残}すものかとあせるように、^{はし}{走}るように、
^{な}{泣}きながら^{まわ}{回}る

\page
\A ^{まえ}{前}にも、こんなことがあった

\A ある^{ひ}{日}、^{からだ}{体}に^{わ}{湧}いた^{いのち}{命}

\A それを^{ぜんぶ}{全部}^{も}{燃}やして、
^{な}{無}かったかもしれない^{じぶん}{自分}を^{うた}{歌}う^{もの}{者}

\page[63]
\A ^{きゅう}{急}に^{おと}{音}が^{たか}{高}くなる

\page[65]
\A ものすごく^{しず}{静}かになった

\A ^{ねんりょう}{燃料}は、もう、ない

\A たぶん、もう^{まわ}{回}ることはない

\page
\A でも、^{あぶら}{油}と^{ねつ}{熱}でやけにナマナマしい^{いろ}{色}になった

\A もう、ただの^{きんぞく}{金属}のかたまりには^{み}{見}えない

\A じっと、^{だま}{黙}っているように^{み}{見}える


\subsection{第125話\ ^{かお}{顔}にあたる^{くうき}{空気}}

\page[69]
\A 「^{だい}{大}^{さんさ}{三叉}^{ろ}{路}」の^{さき}{先}を^{ひがし}{東}へ、
^{だいす}{大好}きな^{のうどう}{農道}の^{こうち}{高地}に^{き}{来}た

\page
\A ^{うみ}{海}のむこうには^{ちば}{千葉}が^{おお}{大}きく、すぐ^{ちか}{近}くに^{み}{見}える

\A スクーターのエンジンを^{や}{止}めた

\A ももの^{あいだ}{間}からエンジンの^{ねつ}{熱}がのぼってきた

\page
\A マフラーの^{ひ}{冷}えていく^{おと}{音}だけが^{き}{聞}こえる

\A ^{にしび}{西日}をあびて、^{かし}{樫}の^{き}{木}がもこもこ^{ちゃいろ}{茶色}に^{ひか}{光}っている

\A ^{はたけ}{畑}には、^{しろ}{白}い^{けい}{軽}トラと^{あお}{青}いトタンの^{こや}{小屋}

\A ^{ひと}{人}はいない

\page[73]
\A また、エンジンをかける

\A スクーターの^{ちい}{小}さな^{はいき}{排気}^{おん}{音}が^{せかいじゅう}{世界中}に^{ひび}{響}きわたる

\page
\A ちょっと^{かいてん}{回転}を^{あ}{上}げてみて

\A またもどす

\page
\A こうしていると、スクーターと^{わたし}{私}は^{みちばた}{道端}に^{た}{立}つ、
^{おな}{同}じようなものに^{おも}{思}えてくる

\A そうなった^{とき}{時}

\A ^{き}{気}の^{も}{持}ちようにもよるけど

\page
\A ^{わたし}{私}は、^{め}{目}の^{まえ}{前}の^{けしき}{景色}の^{なか}{中}を^{と}{飛}ぶことができる

\page
\A ^{からだ}{体}ごと、^{かたち}{形}を^{か}{変}えて、どんな^{はや}{速}さででも

\page
\A もちろん

\A イメージの^{わざ}{業}だと^{おも}{思}ってはいるけど
\ でも、^{きょう}{今日}は

\page
\A ^{め}{目}の^{まえ}{前}の^{たし}{確}かな^{たいしょう}{対象}に

\A ^{すこ}{少}しあせった

\page[81]
\A ^{ほんと}{本当}に^{いま}{今}、
^{と}{飛}んできたような^{かんかく}{感覚}が、^{からだ}{体}に^{のこ}{残}っている

\page
\A ^{べつ}{別}に^{だれ}{誰}にも^{い}{言}わないけど

\A ^{わたし}{私}は、けっこう、よく^{そら}{空}を^{と}{飛}んでいる


\subsection{第126話\ おねえさん}

\page[84]
\Sign ^{かっちゅうさかな}{甲冑魚}

\page
\M お^{みせ}{店}さー

\M やっぱ、^{だいく}{大工}さんにはいってもらってよかったね

\A え

\A まーねー
\ ^{じさく}{自作}だと、^{あつ}{暑}いか^{さむ}{寒}いかだったもんね

\A どうしても

\M ^{なつ}{夏}にだいたいぶっこわしといて^{せいかい}{正解}だったねー

\A ぶっこわしてはいない

\page
\Sign くまむしマニア

\A マッキちゃん
\ ^{あし}{足}!

\A お^{きゃく}{客}さん^{く}{来}るって!

\M ^{こ}{来}ないよ

\page[88]
\A わあ

\K こんにちは

\K ごぶさたです

\A おわー!!
\ ひさしぶりー!

\K ほんとに!

\page
\K はー
\ ^{むかし}{昔}のお^{みせ}{店}と^{おな}{同}じに、なったんですね

\A やっとね
\ ほぼね

\page
\M いらっしゃいませ

\K マッキちゃんね?

\K はじめまして

\page
\M メニュー
\ です

\K ありがと

\page
\K オリジナルブレンド……
\ ください

\M はい……
\ ブレンド

\page[95]
\M ココネさんに^{あ}{会}った

\M アルファさんの^{かお}{顔}を^{はじ}{初}めて、ちゃんと^{み}{見}た

\K マッキちゃんこっち^{こ}{来}ない

\M あ
\ はい

\A なんか^{きょう}{今日}、おとなしいねえ

\page
\M そいじゃ…

\A また、あさって

\K ^{き}{気}をつけてね

\M あれ、ココネさんは

\A ココネはいつもお^{と}{泊}まり

\A マッキちゃんも^{と}{泊}まってく?

\M え

\page
\M う〜〜と

\M まっ……
\ また^{こんど}{今度}……

\A ありゃま

\K またね


\subsection{第127話\ ^{しずく}{滴}}

\page[101]
\A うわっ
\ さむっ!

\K あっ
\ すいません

\A なんか、^{み}{見}えた?

\K いえ…
\ ^{ほし}{星}は^{で}{出}てますけど…

\A ^{へや}{部屋}からじゃ^{むり}{無理}だよ、やっぱ

\K そうですね

\page
\A いちおう、あったかいの^{も}{持}ってきたから

\A ^{そと}{外}^{い}{行}こうよ

\K あ…
\ はあ

\A でも、^{さき}{先}に^{からだ}{体}あっためてからにしよう

\K あ

\K はあ…

\K あの…なにか^{てつだ}{手伝}えますか?

\A んーん、すぐすぐ

\page
\A しょうが^{ゆ}{湯}

\K あっ、いいですね

\A ……ちょっと、ブランデー^{い}{入}れてみようかー

\K なんで、そんなのがあるんですか

\page
\A ちょっと^{まえ}{前}にラジオで、^{こんや}{今夜}、^{りゅうせいぐん}{流星群}が^{で}{出}ると^{い}{言}っていた

\A ^{なが}{流}れ^{ぼし}{星}はよく^{み}{見}るけど、「^{りゅうせいぐん}{流星群}」は^{み}{見}たことがない

\A マッキちゃんもさそったんだけどねえ

\A 「^{よる}{夜}は^{ね}{寝}る!」だって

\K あはは

\A 「^{なが}{流}れ^{ぼし}{星}なんて、くさるほど^{み}{見}た!」だって

\K あはは

\page[106]
\A あ〜〜
\ ^{ゆ}{湯}タンポ、^{さいこう}{最高}

\K ほんとに

\page
\A あれ?

\A もう、^{で}{出}てんじゃないの?!

\K え?

\A あ

\K あ

\page
\A ^{ほんき}{本気}になって、^{ちゅうい}{注意}してみると、スッスッと、
10^{びょう}{秒}に^{いっかい}{一回}くらい、もう^{なが}{流}れ^{はじ}{始}めていた

\A ^{ぜんぜん}{全然}、^{き}{気}がつかなかった

\A ^{かんが}{考}えてみれば、^{あいず}{合図}があるわけないもんねえ

\K ^{おと}{音}がしないのが^{ふしぎ}{不思議}ですね

\page
\A ^{ときどき}{時々}ものすごく^{あか}{明}るいのが^{はし}{走}る

\A そんなのは^{すこ}{少}し、^{ひかり}{光}が^{のこ}{残}る

\K ^{みどり}{緑}とオレンジが^{おお}{多}いですね

\K こんなにたくさん^{で}{出}るなんて……

\A ほんと…

\A ^{そうぞう}{想像}^{いじょう}{以上}

\page[112]
\A ^{りゅうせい }{流星}^{あめ}{雨}!!

\A すごい

\K すごい

\page
\A ^{からだ}{体}が^{うえ}{上}に^{すす}{進}んでく……

\K ほんと

\page
\A ^{なが}{流}れ^{ぼし}{星}は、^{しろ}{白}み^{はじ}{始}めた、^{あ}{明}け^{がた}{方}の^{そら}{空}にも

\A まだ^{ふ}{降}り^{つづ}{続}けていた


\subsection{第128話\ ^{しせん}{視線}の^{ほし}{星}}

\page[116]
\A なんだか、^{みょう}{妙}な^{いわ}{違和}^{かん}{感}で^{め}{目}が^{さ}{覚}めた

\page
\A ^{いえ}{家}の^{うら}{裏}のガケがゴソッと^{くず}{崩}れていた

\A お〜

\A ^{あかつち}{赤土}の^{ところ}{所}が、^{なみ}{波}に、もっていかれて、
^{がんばん}{岩盤}が、むき^{だ}{出}しに、なっていた

\A そっか…
\ それで^{なみ}{波}の^{おと}{音}が^{か}{変}わったんだ

\A う〜ん

\page
\Y ふ〜…

\page[121]
\P あんだ、おめえ

\P まだ、いたのか

\Y あー
\ ども

\page
\Y ^{きょう}{今日}の^{ひる}{昼}すぎにでも、^{かえ}{帰}ろうと^{おも}{思}います

\P よくおめー
\ ここで^{よる}{夜}、^{あ}{明}かしたなあ

\P てえした^{どきょう}{度胸}だ

\Y ものずきなんすよ

\P そうかよ

\page
\P じゃあなー

\Y ども

\page
\Y きのうから^{いちにち}{一日}この^{しろ}{白}いもののとなりにいる

\Y ^{ちば}{千葉}の^{くに}{国}の^{ぎょうぶさき}{刑部岬}は
^{むかしみなと}{昔港}や^{へいや}{平野}を^{み}{見}おろしていた

\Y ^{いま}{今}は、ひたすら^{うみ}{海}だけが^{ひろ}{広}がり、^{りく}{陸}は^{とお}{遠}くに^{さ}{去}った

\page
\Y まったく^{うご}{動}かないこの^{しろ}{白}いもの

\Y ほぼ^{れいがい}{例外}なく、^{みず}{水}、というか、^{けしき}{景色}の^{み}{見}える^{ところ}{所}にいる

\Y ^{うる}{潤}んだ^{め}{目}をうすく^{ひら}{開}きはるか^{とお}{遠}くを^{み}{見}ている

\Y うかつに^{さわ}{触}ることをためらわせる、^{どくとく}{独特}な^{しつかん}{質感}の^{ひょうめん}{表面}

\page
\Y ^{ゆうべ}{昨夜}ーーー

\Y ここからは、かつての「^{くじゅうく}{九十九}^{さとはま}{里浜}」が^{み}{見}てとれた

\Y ^{れい}{例}の^{がいとう}{街灯}の^{れつ}{列}だ

\page
\Y あちこちで^{み}{見}る、その^{ひかり}{光}は、^{ひと}{人}の^{きおく}{記憶}を、これでもかとなぞり、
^{み}{見}る^{もの}{者}の^{きも}{気持}ちを、かき^{みだ}{乱}す

\Y ^{もと}{元}の「^{がいとう}{街灯}」そのままの^{ばしょ}{場所}もあれば、
^{しぜん}{自然}^{もの}{物}と^{じんこう}{人工}^{もの}{物}の^{ちゅうかん}{中間}のような、
^{ふしぎ}{不思議}な^{こうげん}{光源}の^{ところ}{所}もある

\Y ^{ひと}{人}の^{しせん}{視線}だけを^{いしき}{意識}して、そこにあるかのような、^{ひかり}{光}の^{れつ}{列}

\page
\Y なぜ、^{にんげん}{人間}の^{せいかつ}{生活}の^{おも}{思}い^{で}{出}が
こんなに^{とくべつ}{特別}あつかいされているんだろう

\Y それとも、
^{わたしたち}{私達}だけが^{あたま}{頭}の^{なか}{中}で^{かって}{勝手}に^{み}{見}ているんだろうか

\page
\Y ^{はつ}{初}^{せ}{瀬}^{の}{野}^{せんせい}{先生}は…

\Y どのへんまで^{い}{行}ったんだか

\page
\A う〜ん

\A まー
\ ^{うみ}{海}におりやすくはなったかなあ


\subsection{第129話\ solo}

\page[132]
\K お^{せわ}{世話}さまでした

\Sign ^{おぎ}{荻}モーター

\P いやー

\P でも、あれだ

\P ^{こんど}{今度}モーターちょっと、ドコドコするよ

\P ^{まえ}{前}のモーターは^{きんにく}{筋肉}が^{にほん}{二本}で、スムーズだったんだけど

\P これ、^{いっぽん}{一本}^{きんにく}{筋肉}だからな

\K はい、いいんです

\page
\P そうか

\P まあ、こっちの^{ほう}{方}がいいって^{ひと}{人}も^{おお}{多}いんだよ

\P ドコドコなとこがな、^{ぶひん}{部品}も^{ほうふ}{豊富}だし

\K それじゃ

\K ありがとうございました

\P うん

\page
\K う!

\P あに

\P ^{だいじょうぶ}{大丈夫}かよ

\K あ……
\ ^{だいじょうぶ}{大丈夫}です

\K キーのパチッはいつもなんですけど

\K ちょっと^{かん}{感}じが^{ちが}{違}ったんで、びっくり……

\P へえ

\page
\P ああ

\P ^{きんにく}{筋肉}モーターはココネさんと^{ちか}{近}い^{ところ}{所}あるわけだしなあ

\K コク

\P あ…

\P なんか、^{しつれい}{失礼}なこと

\K いいえ!
\ ^{ぜんぜん}{全然}!!

\K じゃ、また

\P うん

\page
\K わあ!

\K ほんとだーー!

\P な

\page
\Sign ^{おうめ}{青梅}^{かいどう}{街道}

\K ずっと、スクーターのモーターの^{かいてん}{回転}が^{みだ}{乱}れていたので

\K ^{おも}{思}い^{き}{切}って、^{べつ}{別}のモーターにとりかえてもらった

\K ^{おぎ}{荻}モーターのご^{しゅじん}{主人}が^{い}{言}うように

\K いままでと^{ちが}{違}って、^{きんにく}{筋肉}の^{うご}{動}く^{かん}{感}じがコトコト^{つた}{伝}わってくる

\K あ
\ こっちの^{ほう}{方}が^{す}{好}きかも

\page
\K モーターの^{いっぽん}{一本}お^{にく}{肉}があったまってくる

\K アルファさんのスクーターのエンジンのような、^{つよ}{強}い^{しゅちょう}{主張}はしないけど

\K ^{なに}{何}かつぶやくように、コトコトと、ドキドキとモーターはまわる

\page
\K ああ、なんて

\K ^{きも}{気持}ちいいんだろう

\page
\K いつも^{とお}{遠}くに^{み}{見}ている^{いんしょうてき}{印象的}な^{とう}{塔}を^{まぢか}{間近}に^{み}{見}て

\page
\K それを^{とお}{通}り^{す}{過}ぎると、^{みち}{道}のまわりは、
^{ひく}{低}い^{き}{木}が^{お}{生}い^{しげ}{繁}るばかりの^{ばしょ}{場所}になる

\K ^{みち}{道}はなお、まっすぐ^{にし}{西}へ^{む}{向}かう

\page[143]
\K なんだか、^{ふあん}{不安}になってきて

\K ^{と}{止}まった

\K もうちょっと^{い}{行}ってみたいけど

\page
\K きっと、アルファさんなら、もっともっと^{さき}{先}まで^{い}{行}って……

\K まっ^{くら}{暗}になっちゃうんだろうな

\K モーターのあたりが^{ひとはだ}{人肌}にあたたかい

\page
\K この^{さき}{先}はまた^{こんど}{今度}

\K ^{かえ}{帰}ろ

\page
\K いろんな^{ところ}{所}へ

\K ^{い}{行}こうね


\subsection{第130話\ ^{つき}{月}の^{わ}{輪}}

\page[148]
\O よう

\A あ

\A いらっしゃいませ

\page
\O へー
\ ^{そと}{外}、いいかよ

\A あ
\ ^{けっこう}{結構}ですよ

\page
\A おまたせ……
\ ^{そと}{外}、ぬるいですねえ

\O んー

\page
\O ^{みせ}{店}ぇ

\O ^{なお}{直}ったな

\A おかげさまで

\A 6^{ねん}{年}かかりましたね

\O あんだかんだでへー、^{けっきょく}{結局}、^{まえ}{前}のまんまの^{かたち}{形}な

\A ええ、やっぱ、これが^{いちばん}{一番}です

\page
\O ここ^{き}{来}てよー

\O ^{きょう}{今日}、^{はじ}{初}めてしゃべったわ

\page
\A ^{わたし}{私}

\A もともとはタバコの^{けむり}{煙}^{にがて}{苦手}だったんですよ

\O あ゛?

\O おお!
\ わりい!

\A あっ!
\ あっ!
\ そうじゃなくて

\A ^{そと}{外}にいる^{とき}{時}にですね…

\A ちょっと^{はな}{離}れた^{ところ}{所}でも、
おじさんがタバコに^{ひ}{火}つけた^{しゅんかん}{瞬間}がわかるんですよ

\A タバコの^{けはい}{気配}とか…
\ ^{す}{好}きなんです

\page
\O そうか

\O そうなー

\O おれも^{こども}{子供}んとき^{おやじ}{親父}が^{そと}{外}でタバコ^{す}{吸}ってん、
におい^{す}{好}きだったわ

\O ^{にちようび}{日曜日}とかな

\page
\O とか^{い}{言}われて

\O また^{す}{吸}う

\A おじさん

\A ^{わたし}{私}、^{まえ}{前}に、
^{ひとり}{一人}の^{とき}{時}が^{す}{好}きって^{い}{言}った^{こと}{事}があります

\O あー

\A ^{いま}{今}は、^{ひとり}{一人}じゃない^{とき}{時}の^{ほう}{方}が^{す}{好}き

\page
\O そうか

\A なんて^{い}{言}うか

\A ^{じぶん}{自分}^{ひとり}{一人}じゃ^{うご}{動}かせない^{ところ }{所}^{のう}{脳}ミソにありますよね

\A ^{あ}{開}かない^{ところ}{所}ってゆうか

\A タカヒロがいなくなって、マッキちゃんもたぶん……

\A みんな^{とお}{遠}くに^{い}{行}っちゃって

\page
\A そしたら

\O アルファさんもでかくなってんだよなあ

\page
\O いや…

\O あんたは^{むかし}{昔}からできた^{おとな}{大人}だったけんどよ

\O ^{むかし}{昔}の^{うつわ}{器}じゃ、^{おさ}{収}まりきんねえで、あっちこっちハミ^{だ}{出}してきたんだべな

\O アルファさんはタカやマッキといっしょに

\O ^{せいちょう}{成長}してんだ

\O 「^{ひとり}{一人}^{す}{好}き」を^{かた}{語}るにゃー
\ アルファさんは、まだ^{わか}{若}すぎんかなー

\page
\A わか…
\ ^{わか}{若}い?
\ ですかね…

\O ガキんちょ

\A うおう!!

\O もう、ふたまわりもでかくなったら

\O また^{ひとり}{一人}が^{す}{好}きになんかもしんねえし

\O その^{まえ}{前}に、タカのやつが^{な}{泣}いて^{かえ}{帰}って^{く}{来}んかもしんねえ

\page
\A おじさん

\A ^{つき}{月}の^{わ}{輪}っかっていくつ^{み}{見}えます?

\O ん?
\ ん〜〜
\ ひとつだな
\ ^{だいたい}{大体}

\A ^{わたし}{私}

\A ^{み}{三}つ^{み}{見}えます

\page
\O ^{め}{目}ぇ、いいな

\A はい

\A おじさん、^{わたし}{私}が^{み}{見}ていることって…

\A ^{ほんと}{本当}のことですよね

\O あ?

\O あー

\O お^{たが}{互}い^{ちが}{違}うもん^{み}{見}えてんかもしんねえけどな

\O ^{しん}{信}じな

\page
\A ありがとう、おじさん


\subsection{マッキのお^{しごと}{仕事}です!}

\M ヒマすぎる!!

\A ん〜
\ まあね

\M ウエイトレスらしいことしたいー!

\A あっ

\M いらっ・・
\ いらっしゃ・・
\ ませっ

\A ^{ほんばん}{本番}に^{よわ}{弱}いネ

\A あるイミオッケー
