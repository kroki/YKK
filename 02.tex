\section{Volume 2}

\subsection{^{だい}{第}8^{わ}{話}\ ^{ごご}{午後}1/1}

\page[4]
\K あの

\A まー
\ まー

\A ちょっと^{やす}{休}んでってよ

\K いえ
\ あの
\ まだ

\A このあと^{しごと}{仕事}?
\ ^{よてい}{予定}あんの?

\K いえ

\K あとは^{かえ}{帰}りの^{くるま}{車}との^{ま}{待}ちあわせまでは

\A じゃあ
\ なにかのんでかない?

\A もちろんセービス!

\A はい
\ これ
\ ^{しなかず}{品数}^{すく}{少}ないけど

\K あ
\ どうも
\ じゃ

\page
\K あの
\ このメイポロってなんですか?

\A そういう^{き}{木}の^{しる}{汁}のお^{ゆ}{湯}^{わ}{割}りよ
\ それにする?

\K あ
\ いえ
\ コーヒーで

\A ん!

\A ごめんね
\ つきあわせちゃって

\K いえ

\K それにまだ

\A や〜〜
\ なんだろ
\ なんでもいいんだけどねー

\page
\A お?!

\A これ
\ カメラかな?

\K そうですね

\A わ〜〜〜!
\ なんでカメラなのかな

\A ^{へん}{変}な^{ひと}{人}だよねー

\A えへへ

\A なんか
\ コメントぐらいほしいよね

\K あっ

\page
\K あの
\ ご^{いらい}{依頼}^{ぬし}{主}^{さま}{様}からのメッセージもおあずかりしてますので

\K まず
\ そちらから

\A あっ
\ なんだ〜〜
\ そうなの!

\K いえ

\K こちらから^{じか}{直}に

\A え?

\A ジカ?

\K はい?

\A ジカって?

\page
\K あの
\ ^{しつれい}{失礼}ですが
\ A7M2^{かた}{型}^{き}{機}でいらっしゃいますよね?

\A A7
\ あーー
\ そんな^{なまえ}{名前}もあったよ
\ そういえば

\A ああ!
\ ^{じか}{直}に!

\K はい

\A て
\ ことは
\ ひょっとしてあなたも

\K いえ
\ ^{わたし}{私}は^{ふきゅう}{普及}^{かた}{型}のA7M3ですけど

\page
\A え〜〜〜〜っ!?
\ ロボットなの!?

\K は?!

\K まさか
\ わからなかったとか?

\A そんなの^{い}{言}われなきゃわかんないよ

\K わかりますよ!
\ ^{ふんいき}{雰囲気}とか
\ ^{かみ}{髪}が^{みどり}{緑}や^{むらさき}{紫}とか

\A そっ
\ そうなの?

\page
\K じゃ
\ ^{ちょくせつ}{直接}で^{おく}{送}りますね

\A ん〜〜

\A やっぱり^{ことば}{言葉}にしない?

\K あの
\ ^{わたし}{私}は^{ないよう}{内容}まではわからないんですよ

\K ね
\ さっとやっちゃいましょうよ

\A う〜〜ん

\page
\K じゃ
\ ^{て}{手}つなぎましょうか

\K はい
\ ^{した}{舌}^{み}{見}せてください

\K では

\K いきますね

\K あの
\ ちぢこまらないでください
\ では
\ ^{しつれい}{失礼}して

\page
\K ^{つた}{伝}わりました?

\A ちょっとまって
\ ^{お}{落}ちつくまで

\page
\H 「しばらくは^{かえ}{帰}らないと^{おも}{思}う
\ だから^{き}{気}にせずに^{そと}{外}へ^{で}{出}て
\ まわりを^{み}{見}て^{ある}{歩}くことをすすめる」

\H 「^{きみ}{君}にとっては^{じゅうねん}{十年}も^{いちにち}{一日}も
\ さして^{ちが}{違}うことはないかもしれないが
\ いつか
\ なつかしく^{おも}{思}う^{ことがら}{事柄}もできるだろう」

\A オーナーのメッセージはこんな^{かん}{感}じだった

\page
\H 「その^{とき}{時}
\ ^{きおく}{記憶}の^{たす}{助}けになると^{おも}{思}うのでこれを^{おく}{送}る」
\ と

\H 「^{からだ}{体}に^{き}{気}をつけるように」
\ と

\A も〜〜

\A なつかしい^{こと}{事}ぐらいあるわよ

\page
\A ありがとう

\K いえ

\K またお^{とど}{届}けものがあったら
\ ^{わたし}{私}が^{たんとう}{担当}させていただきますので

\A あらら

\A ^{たか}{鷹}^{つ}{津}ココネさん!?

\K はい!

\page
\K すいません
\ ^{おく}{送}っていただいちゃって

\A ^{ま}{待}ちあわせ^{ばしょ}{場所}まで10キロも^{ある}{歩}くつもりだったなんて

\A ^{こんど}{今度}はうちまで^{き}{来}てもらいなね

\K はい

\page
\K ^{わたし}{私}
\ これでも^{じもと}{地元}じゃ「^{にんげん}{人間}っぽいね」って^{い}{言}われてて

\K ^{じぶん}{自分}でもそうしなきゃって^{おも}{思}ってたんです

\A んー?

\K でも
\ ^{いま}{今}はなんだか

\K もっと^{き}{気}を^{らく}{楽}にしてみようかなって

\A えーー?

\A なに?

\page
\K ^{しごと}{仕事}^{いがい}{以外}であそびに^{き}{来}てもいいですか?!

\A もちろん!
\ いつでも!

\A やほー
\ おじさーん

\O おお
\ あんだえ〜〜


\subsection{第9話\ ひと^{つぶ}{粒}300^{まい}{枚}}

\page[20]
\A ^{くるま}{車}の^{じかん}{時間}まではまだ^{あいだ}{間}がある

\page
\A そうそう
\ これのねー

\K あ
\ もってきちゃったんですね

\A うん
\ ^{こん}{今}^{つか}{使}い^{かた}{方}わかる?

\K ええ
\ このタイプなら^{だいたい}{大体}

\page
\K ^{ふぞく}{付属}のこのキャラメルみたいなのを

\K ^{うし}{後}ろのフタあけて
\ くぼみにはめて

\K フタしめて

\K で
\ ここでマブタをあけてこれを^{お}{押}せば

\K このピースなら1^{つぶ}{粒}300^{かい}{回}はうつせますよ

\A ふ〜〜ん

\page
\A このお〜〜
\ ^{め}{目}のとこってさあ〜〜

\K ^{わたし}{私}たちの^{め}{目}に^{ちか}{近}いモノ^{つか}{使}ってますね

\A やっぱり
\ ^{めだま}{目玉}だったのね

\page
\K コードあります?

\A おお!

\A これ?

\K ええ

\K その^{ほそ}{細}い^{かた}{方}をフタの^{あな}{穴}にさして

\A さして

\K はじっこの^{たま}{玉}をくわえれば

\page
\K カメラの^{み}{見}てる^{え}{絵}も^{み}{見}えますし

\A うえ!

\K はじめはこれでカメラの^{してん}{視点}をおぼえるといいですよ

\A なんか
\ きもちわ^{る}{う}いね

\K ^{め}{目}つぶってても
\ ^{じぶん}{自分}が^{み}{見}えたりしておもしろいですよネ

\A く〜〜

\page
\A せーの

\A んーーん
\ なんか
\ わかってきたよ

\A こりゃ^{だいじ}{大事}につかわないと

\A 300^{まい}{枚}なんてすぐ^{と}{撮}っちゃいそうね

\page
\A ^{くるま}{車}
\ こないね

\K いつものことですよ
\ こちらも^{はや}{早}かったですし

\page
\A ムサシノかー
\ ^{とお}{遠}いなあ

\K そんなことないですよ

\A 「ココネさん」ってかわいい^{なまえ}{名前}ね!

\K あはは

\K なんか
\ ^{さいしょ}{最初}に^{けんしゅう}{研修}^{じょ}{所}^{い}{行}くときに
\ ^{じぶん}{自分}でつけた^{なまえ}{名前}らしいんですけど

\K なんで
\ そうなのか^{わす}{忘}れちゃいました

\A ふ〜〜ん

\page
\K アルファさんは
\ あの
\ ^{こゆう}{固有}^{めい}{名}はなんと

\A 「アルファ」が^{こゆう}{固有}^{めい}{名}なのよ

\A ^{わたし}{私}たちみたいなアルファ^{かた}{型}がまだめずらしくてねー

\A 「アルファさん」って^{よ}{呼}ばれてなの

\A それで^{なまえ}{名前}になっちゃった
\ まあ
\ ^{き}{気}に^{い}{入}ってるし

\K はーー
\ なるほど

\page
\A だいぶたって^{くるま}{車}がきた

\A また^{あ}{会}えると^{おも}{思}う

\A ^{し}{知}らない^{くに}{国}へ^{かえ}{帰}る^{かのじょ}{彼女}を^{み}{見}てると
\ さびしくなってくるけどね

\page
\A ムサシノの^{くに}{国}
\ ^{ひがし}{東}の^{みやこ}{都}と^{よ}{呼}ばれたこともある^{ところ}{所}

\A となりの^{くに}{国}なのに
\ すごく^{とお}{遠}く^{かん}{感}じるよ

\page
\A おじさーん

\O お

\O あ〜〜によ〜〜

\page
\O ガスかよ

\A ううん
\ ゴメンね
\ おこしちゃって

\O あーー
\ さっきのコ^{し}{知}りあいかよ
\ ^{ひるま}{昼間}^{みち}{道}ィ^{き}{聞}いてってよー

\O ^{そ}{し}^{し}{た}^{た}{っ}^{ら}{け}
\ あんたに^{に}{似}てんじゃんかー
\ へー
\ びっくりよちっとばっか

\A ^{に}{似}てた!?

\O おお

\page
\A うん

\A ^{いもうと}{妹}なの!

\A いらないシーンはひとつもない

\A 300じゃとてもたりないよ

\Sign みかん^{ちゃ}{茶}


\subsection{第10話\ カマスのアヤセ}

\page[36]
\A このごろタカヒロ^{み}{見}ないけど
\ どうしたの?

\O なんかよー
\ あったかくなってから

\O ^{いりえ}{入江}に^{い}{行}ってんよいつも

\A あーりゃま

\page
\T うーん

\T ^{いちど}{一度}はちゃんと^{あ}{会}ってみたいよなー

\page
\T あんなに^{きょうれつ}{強烈}な^{いんしょう}{印象}なのに
\ いまいち^{げんじつ}{現実}^{かん}{感}がない

\T あーー
\ こんなかんじのあとだっけな
\ いまなり

\page
\T ^{ちい}{小}さいな
\ でも
\ ひょっとして

\T みっ
\ みさご!?

\page
\Y カマスだ

\Y おれのカマスだ

\T みざごってのはー

\Y ^{し}{知}ってんよ

\Y ミサゴはおれも^{むかし}{昔}^{み}{見}たことあんよ

\T え!

\page
\Y ふうん
\ ミザゴ^{み}{見}たかまだいんのか

\T うん

\Y おめえ^{なまえ}{名前}は

\T タカヒロ

\Y おれはアヤセ

\Y ^{み}{見}てな

\page[44]
\Y ^{やどだい}{宿代}の^{た}{足}しによ

\T すごい

\Y ミサゴはよ
\ いつもいきなしじゃんか

\page
\Y ^{き}{気}んなって
\ ^{なんど}{何度}か^{き}{来}たけどよ
\ そういうときや^{で}{出}ねえ

\Y ^{き}{気}んなってる^{やろう}{野郎}が
\ ^{ふたり}{二人}もいちゃ^{く}{来}んもんもこねえか

\T しかも^{ひとり}{一人}はおっさん

\Y それが^{いちばん}{一番}まじいな

\N ^{とし}{年}の^{はな}{離}れた^{おとこ}{男}が
\ ^{おな}{同}じ^{おんな}{女}の^{こと}{事}を^{かんが}{考}える

\N ^{あ}{会}ったのは^{べつ}{別}の^{じだい}{時代}^{し}{知}っているのは
\ ^{おな}{同}じ^{すがた}{姿}だ

\page[48]
\Y あれっきりだったね

\Y あっちからはずっとこっち^{み}{見}てたんだべな

\Y じゃあな

\T うん

\T ^{こんど}{今度}いつくんの?

\Y わかんねえな

\page
\T ^{こんど}{今度}^{あ}{会}わせたい^{ともだち}{友達}がいるんだよ

\Y あんだよ
\ そりゃいいよ^{べつ}{別}に

\T ^{み}{見}たかんじあやせぶらいの^{とし}{年}の^{おんな}{女}の^{こ}{子}だよ

\Y そりゃちょっと^{き}{気}になんかな!

\page
\A うちのお^{きゃくさま}{客様}って8^{わり}{割}がたタカヒロかなー

\A ううん
\ そうね〜〜


\subsection{第11話\ プロテイン}

\page[52]
\A ちょっとハデ?

\K かわいいよ

\page
\K でも
\ こっちのがおいしそう

\K ^{た}{食}べる?

\A いい

\K そうね
\ アルファちゃん^{さかな}{魚}だめなのよね

\A うん
\ ココネ^{ほんと}{本当}になんともない?

\K 「^{しょうか}{消化}^{ほう}{法}」って^{つた}{伝}わるのかな?

\A えっ

\A わーーっ!

\page[55]
\A そういえば
\ ココネって

\A ^{にく}{肉}もサカナも^{へいき}{平気}なのよねー

\A この^{まえ}{前}のココネの^{はなし}{話}だと
\ ロボットでも^{どうぶつ}{動物}タンパクの^{しょうか}{消化}にムリはないって

\A どうやら^{わたし}{私}だけのクセみたい

\page
\A ^{ほんき}{本気}で^{な}{慣}らしていこうかな

\A おさしみくらい
\ みんなにつきあえなきゃだめよね
\ やっぱ

\A こんな^{ひ}{日}はお^{きゃく}{客}さんも^{こ}{来}ないし

\A よし!

\page
\Sign コーヒー^{ぎゅうにゅう}{牛乳}

\A ^{あじ}{味}は^{す}{好}きなんだし
\ ^{き}{気}にしなきゃ
\ なんでもないかもね

\page
\A なんか
\ ^{だいじょうぶ}{大丈夫}みたい

\page
\A やっぱ
\ ダメみたい

\A でも
\ これに^{な}{慣}れなきゃ

\A まずい

\A ^{きょう}{今日}はもう^{へいてん}{閉店}

\page
\T あっ
\ ^{ひさ}{久}しぶり!

\A そっ
\ そうね!
\ はいって

\T ^{きょう}{今日}はおわり?

\A うん
\ タカヒロはいいよ

\page
\A こりゃ
\ ^{いじ}{意地}でもふつうにしなきゃ

\A なな
\ なに^{の}{飲}む?

\T あっ
\ じゃ
\ メイポロ

\A はい

\T ありがと

\page
\T あっ
\ こないだねー

\A ああ
\ なにかしゃべってる

\A ちゃんと^{き}{聞}かなきゃ
\ ちゃんと

\page
\T そいでねー

\A う

\A ^{こえ}{声}
\ でちゃった

\T なに?

\A え?
\ なに?

\page
\A もっとゆっくりしてけば

\T こんどね

\T ^{はや}{早}くなおしなー
\ すぐねた^{ほう}{方}がいいよ

\A カゼに^{み}{見}えたのかな

\page
\A その^{あと}{後}すぐ^{あめ}{雨}はやんだ
\ この^{じき}{時期}にはめずらしく
\ ^{ふじ}{富士}^{さん}{山}がよく^{み}{見}える

\A ^{たいない}{体内}の^{あらし}{嵐}もけろっとなおった

\page
\A おさしみはまだ^{とお}{遠}いです

\A タカヒロ
\ ごめんね
\ ^{きょう}{今日}は


\subsection{第12話\ ナビ}

\page[68]
\A オーナーにもらったカメラ

\A あれからしまいこみがちになっちゃってる

\A バンバン^{つか}{使}った^{ほう}{方}がいいとは^{おも}{思}うけど
\ つい

\A ^{たからもの}{宝物}としては^{も}{持}ち^{ある}{歩}くのに
\ ピッタリなんだよね
\ ^{かる}{軽}いし

\page
\A そうだ
\ ^{かわ}{皮}のバッグ

\A ^{てっぽう}{鉄砲}^{よう}{用}に^{つく}{作}ったやつ!
\ ちょっとブカブカだけど

\A ばっちりじゃないの!

\A よし

\A ^{きょう}{今日}は^{さつえい}{撮影}かな!

\page
\A きみ
\ かっこいいじゃないか

\page
\A よく^{み}{見}てみると

\A とっておきたい^{ふうけい}{風景}は^{おお}{多}すぎる

\A まだ
\ いいか

\page
\A おじさーん!

\O お
\ あにした

\A うん
\ ちょっとモデルになってほしいかなー
\ なんて

\O あん?

\A ふつうに
\ ^{しぜん}{自然}にしててね!

\page
\A ごっ
\ ごめんなさい

\A またくるね

\O か
\ あによー

\A なんか
\ うまくいかないなあ

\page
\A よし!

\A でも
\ もうちょい^{さき}{先}かな

\page
\A まだ^{せん}{千}^{なんびゃく}{何百}^{まい}{枚}もあるんだし

\A 1^{まい}{枚}くらい^{と}{撮}ってもいいのに

\A ^{わたし}{私}ってそんなにケチだったのかなあ

\page
\A いい^{え}{絵}はいっぱいある

\A でも

\A もう^{わたし}{私}は1^{まい}{枚}にしぼった

\A ^{きょう}{今日}はそれが^{と}{撮}れればいい

\page
\A ^{きた}{北}の^{だい}{大}^{くず}{崩}れ

\A ここからの^{ゆうけい}{夕景}にきめた

\A シルエットの^{え}{江}^{の}{ノ}^{しま}{島}と
\ ^{かいじょう}{海上}に^{なら}{並}ぶ^{がいとう}{街灯}

\A ^{つぎ}{次}に^{く}{来}る^{いっしゅん}{一瞬}
\ ^{ふじいろ}{藤色}のフィルター

\page[80]
\A ^{と}{撮}らなかった

\A なんだか
\ おもいっきり^{みい}{見入}っちゃった

\A ^{とうぶん}{当分}は^{あじ}{味}わえるわ

\page
\A ^{きょう}{今日}は^{けっきょく}{結局}
\ カメラにひっぱりまわしてもらったみたい

\A ^{うつ}{写}してやる!
\ って^{とき}{時}^{と}{撮}れなかったのは

\A ^{せいかい}{正解}だったかもしれない

\page
\A ^{で}{出}がけに^{と}{撮}った
\ なんてことないスクーターが
\ ^{きょう}{今日}
\ ^{ゆいいつ}{唯一}の^{しゃしん}{写真}

\A でも
\ ^{わたし}{私}の^{しゃしん}{写真}はそれでいいと^{おも}{思}う

\A ^{のこ}{残}りの^{まいすう}{枚数}は
\ そんなに^{すく}{少}ないわけじゃないって^{き}{気}がした

\A それに
\ ^{きょう}{今日}の^{ふうけい}{風景}は
\ いつもより^{せんめい}{鮮明}に^{おも}{思}い^{だ}{出}せる



\subsection{第13話\ ^{かまくら}{鎌倉}^{はなび}{花火}}

\page[84]
\O あんとかまいあったか

\O けっこ^{き}{来}てんなー

\page
\A ^{きょう}{今日}は10^{ねん}{年}ぶりの^{はなび}{花火}^{たいかい}{大会}!
\ ^{カマクラ}{鎌倉}の^{いりえ}{入江}あたりは^{ひと}{人}だらけです

\O やつらみゃ^{かい}{会}えねえかもなー

\O このへんだけんどよ

\P おーう

\page
\P へえヨ!
\ こっちだよ!

\A あっ

\O ちっ

\P おお
\ アルちゃんよかたか!?
\ よし!
\ まずひとつ

\A ^{きょう}{今日}はパス!

\P か〜〜
\ それで
\ おどってくれりゃ
\ う〜〜ん

\A だからパス!

\page
\P そこ^{く}{来}んかじじい

\O あん?

\A はい
\ これムギ^{ちゃ}{茶}です
\ あと

\A うちでとった^{えだまめ}{枝豆}なんだけど

\P あの^{ふね}{船}からあがんの?

\P うん
\ もうすぐ
\ なんか
\ まだ^{あか}{明}るいけど

\page
\A あっ
\ ^{じかん}{時間}かな?

\T あっちいってくる!

\A ^{わたし}{私}も

\P あんだえー

\A ^{きょう}{今日}のメインは^{さいしょ}{最初}の1^{はつ}{発}^{め}{目}

\A この^{へん}{辺}では^{こんかい}{今回}で^{さいご}{最後}になる
\ ^{おおだま}{大玉}です

\page
\A ^{わたし}{私}も^{おおだま}{大玉}って^{はじ}{初}めて^{み}{見}る

\page[92]
\A こっ
\ こうかーー!!

\page[96]
\A あの
\ ^{わたし}{私}

\A タカヒロと^{さき}{先}に^{せん}{先}^{れい}{礼}します

\page
\O あ〜〜
\ わりいけんど
\ たのむわ

\O てーげー
\ こいじゃ
\ ^{あさ}{朝}んなんべなーー

\O なる

\O おれは^{あした}{明日}こいつらと^{かえ}{帰}んよ

\A すみません

\A じゃ
\ みなさん
\ お^{さき}{先}に

\P おーー

\page
\A あんなにすごい^{み}{見}たばっかなのに

\page
\A タカヒロ^{きょう}{今日}のことちゃんとおぼえてる?

\page
\A こんな^{かん}{感}じかなあ^{おとうと}{弟}って

\A ぐーー


\subsection{第14話\ ^{すな}{砂}の^{はま}{浜}}

\page[102]
\A ^{きょう}{今日}は^{せんせい}{先生}の^{ところ}{所}に^{き}{来}ています

\S あらま
\ アルファさんがつくったの?

\A ^{いちおう}{一応}トマト
\ ^{おお}{大}きさバラバラですけど

\S ごちそうさま
\ あれから^{からだ}{体}の^{ほう}{方}は?

\A はい

\A ^{いぜん}{以前}よりいい^{くらい}{位}です

\A じゃ
\ また

\page
\A そういえば

\A ここって
\ ^{すな}{砂}^{はま}{浜}なんですね

\O あーー
\ もうここぐれえだべ
\ おおかた^{しず}{沈}んで
\ まったっけ

\A ちょーっと
\ ^{みず}{水}あびてきます!

\O あん?
\ ^{みずぎ}{水着}はよ

\A ^{わたし}{私}^{みずぎ}{水着}もってないんです

\O あ〜〜
\ ふうん
\ ^{きゅう}{急}に^{ふか}{深}えから
\ ^{き}{気}いつけな

\page[106]
\S なんか
\ いいわね
\ ^{かのじょ}{彼女}

\O あ?
\ あーー
\ そうすね

\S ここも^{か}{変}わっちゃったわね

\O そうすね

\page
\O ーーあんだ
\ もう^{あ}{上}がりか

\A いいものひろっちゃいましたよ!

\A ほら!

\O ほらって
\ おめえ
\ こりゃ

\A きれいですよね!

\A ^{うみ}{海}の^{みず}{水}の^{いろ}{色}ですね

\page
\O そうな

\O ^{むかし}{昔}はゴロゴロしてたけどな

\S ^{ひさ}{久}しぶりに^{み}{見}たわね

\A ^{すな}{砂}の^{はま}{浜}と^{うみ}{海}の^{いろ}{色}のビン

\A ^{きょう}{今日}の^{しゅうかく}{収穫}です


\subsection{第15話\ ^{すな}{砂}の^{みち}{道}}

\page[110]
\A あのー

\A ^{ふく}{服}かわくまでいいですか?

\O おう
\ いいよ

\page
\O あのビン^{した}{下}の^{みせ}{店}のやつですかね

\S ああ
\ そうかもね

\S ^{み}{見}たかんじは^{べつ}{別}の^{ばしょ}{場所}みたいだけど

\S やっぱり^{おな}{同}じ^{とこ}{所}なのよね

\page
\S ^{おな}{同}じ^{とこ}{所}なのよね

\page[114]
\S おそい!

\O すんません

\O あー
\ ^{せんぱい}{先輩}

\O やっぱ
\ イッちゃんも
\ ^{しごと}{仕事}で^{く}{来}でねえそうです

\S あそー
\ まあいいや^{ふたり}{2人}で^{い}{行}くかたまにゃー

\O はあ

\page
\S じゃあ

\S ついて^{く}{来}るように

\O ういーす

\O いきなし^{よ}{呼}び^{だ}{出}された

\O ^{へいさ}{閉鎖}された^{かいがん}{海岸}^{どうろ}{道路}の
\ ^{みおさ}{見納}め^{たいかい}{大会}だというが

\O ^{へいじつ}{平日}ヒマなのは^{だいがく}{大学}^{づと}{勤}めの^{せんぱい}{先輩}と
\ プータローの^{おれ}{俺}くらいだ

\page
\S あーん
\ ^{しよう}{使用}^{ちゅう}{中}の^{みち}{道}でこんなかあ

\O ^{みち}{道}ってよか
\ ^{ていぼう}{堤防}ですね

\page
\S じゃあ
\ ^{なみ}{波}の^{あいだ}{間}ねらって

\O やっぱ^{い}{行}くんすか

\S あゴー!

\O あうー!

\page
\Sign ^{つうこうどめ}{通行止め}
\ ^{?}{う}^{まわ}{回}して^{くだ}{下}さい

\O こっからですか

\S うん

\S どう^{つうこうどめ}{通行止め}なのか
\ ^{たし}{確}かめなきゃね

\O そうなんすか?

\page
\O じゅうぶん^{たし}{確}かめました!

\S そうね
\ ^{なっとく}{納得}!

\page
\S これが
\ あの^{じゅうたい}{渋滞}^{どうろ}{道路}

\S ^{なみ}{波}かぶりそう

\O その^{みち}{道}1^{ほん}{本}じゃ
\ すまないでしょうね
\ この^{せん}{先}

\S ^{いっしんいったい}{一進一退}
\ ^{なん}{何}10^{ねん}{年}かしならあもう

\page
\O ^{せんぱい}{先輩}

\O ^{おれたち}{俺達}くらいじゃないすかね

\O いまどきブラブラしてんのこんなとこで

\S ^{いま}{今}しか^{み}{見}らんない^{けしき}{景色}だよ

\O ^{せんぱい}{先輩}がよく^{おれ}{俺}に^{み}{見}せたのは
\ そんな「^{しゅん}{旬}のもの」の^{ふうけい}{風景}だった

\O ^{じだい}{時代}をよく^{あらわ}{表}し
\ ^{べつ}{別}に^{ちゅうもく}{注目}されず
\ 2^{ど}{度}と^{み}{見}られないもの

\page
\S ^{ほん}{本}とかでわかることでも
\ ^{げんば}{現場}で^{かん}{感}じるのと^{ぜんぜん}{全然}ちがう

\S ^{め}{目}の^{まえ}{前}のモノちゃんと^{み}{見}て

\page
\O はあ

\page
\O おおっ!
\ びっくりしたーー!

\A お^{ま}{待}たせしました

\A ^{いま}{今}
\ ^{べつ}{別}の^{せかい}{世界}^{み}{見}てたでしょ
\ ^{ふたり}{2人}で

\O まあな!

\page
\A いーなー
\ ^{ふたり}{2人}の^{せかい}{世界}
\ ^{み}{見}てみたい

\S ^{ぞくへん}{続編}なら^{み}{見}られるわよ

\S ^{きょう}{今日}の^{こと}{事}ちゃんとおぼえとけばね

\A はあ

\page
\A この^{とき}{時}の^{きおく}{記憶}
\ 3^{にん}{人}の^{こうけい}{光景}が「^{ぞくへん}{続編}」でした

\A ^{き}{気}がつくのは^{のち}{後}になってからです


\subsection{ココネのなんかよさそう!}

\SH えっくし!

\K なんかいいなー
\ くしゃみって
\ ^{おし}{教}えてよ
\ やり^{かた}{方}

\SH やり^{かた}{方}って
\ あんた

\SH よーし

\SH こうすんのよ!

\SH ほれ

\K わくわく

\K そっ
\ それから!?

\SH きかねーし

\SH ムリにやるほどのもんじゃないって
