\section{Volume 2}

\subsection{^{だい}{第}8^{わ}{話}\ ^{ごご}{午後}1/1}

\page[4]
\Kokone あの

\Alpha まー
\ まー

\Alpha ちょっと^{やす}{休}んでってよ

\Kokone いえ
\ あの
\ まだ

\Alpha このあと^{しごと}{仕事}?
\ ^{よてい}{予定}あんの?

\Kokone いえ

\Kokone あとは^{かえ}{帰}りの^{くるま}{車}との^{ま}{待}ちあわせまでは

\Alpha じゃあ
\ なにかのんでかない?

\Alpha もちろんセービス!

\Alpha はい
\ これ
\ ^{しなかず}{品数}^{すく}{少}ないけど

\Kokone あ
\ どうも
\ じゃ

\page
\Kokone あの
\ このメイポロってなんですか?

\Alpha そういう^{き}{木}の^{しる}{汁}のお^{ゆ}{湯}^{わ}{割}りよ
\ それにする?

\Kokone あ
\ いえ
\ コーヒーで

\Alpha ん!

\Alpha ごめんね
\ つきあわせちゃって

\Kokone いえ

\Kokone それにまだ

\Alpha や〜〜
\ なんだろ
\ なんでもいいんだけどねー

\page
\Alpha お?!

\Alpha これ
\ カメラかな?

\Kokone そうですね

\Alpha わ〜〜〜!
\ なんでカメラなのかな

\Alpha ^{へん}{変}な^{ひと}{人}だよねー

\Alpha えへへ

\Alpha なんか
\ コメントぐらいほしいよね

\Kokone あっ

\page
\Kokone あの
\ ご^{いらい}{依頼}^{ぬし}{主}^{さま}{様}からのメッセージもおあずかりしてますので

\Kokone まず
\ そちらから

\Alpha あっ
\ なんだ〜〜
\ そうなの!

\Kokone いえ

\Kokone こちらから^{じか}{直}に

\Alpha え?

\Alpha ジカ?

\Kokone はい?

\Alpha ジカって?

\page
\Kokone あの
\ ^{しつれい}{失礼}ですが
\ A7M2^{かた}{型}^{き}{機}でいらっしゃいますよね?

\Alpha A7
\ あーー
\ そんな^{なまえ}{名前}もあったよ
\ そういえば

\Alpha ああ!
\ ^{じか}{直}に!

\Kokone はい

\Alpha て
\ ことは
\ ひょっとしてあなたも

\Kokone いえ
\ ^{わたし}{私}は^{ふきゅう}{普及}^{かた}{型}のA7M3ですけど

\page
\Alpha え〜〜〜〜っ!?
\ ロボットなの!?

\Kokone は?!

\Kokone まさか
\ わからなかったとか?

\Alpha そんなの^{い}{言}われなきゃわかんないよ

\Kokone わかりますよ!
\ ^{ふんいき}{雰囲気}とか
\ ^{かみ}{髪}が^{みどり}{緑}や^{むらさき}{紫}とか

\Alpha そっ
\ そうなの?

\page
\Kokone じゃ
\ ^{ちょくせつ}{直接}で^{おく}{送}りますね

\Alpha ん〜〜

\Alpha やっぱり^{ことば}{言葉}にしない?

\Kokone あの
\ ^{わたし}{私}は^{ないよう}{内容}まではわからないんですよ

\Kokone ね
\ さっとやっちゃいましょうよ

\Alpha う〜〜ん

\page
\Kokone じゃ
\ ^{て}{手}つなぎましょうか

\Kokone はい
\ ^{した}{舌}^{み}{見}せてください

\Kokone では

\Kokone いきますね

\Kokone あの
\ ちぢこまらないでください
\ では
\ ^{しつれい}{失礼}して

\page
\Kokone ^{つた}{伝}わりました?

\Alpha ちょっとまって
\ ^{お}{落}ちつくまで

\page
\Hatsuseno 「しばらくは^{かえ}{帰}らないと^{おも}{思}う
\ だから^{き}{気}にせずに^{そと}{外}へ^{で}{出}て
\ まわりを^{み}{見}て^{ある}{歩}くことをすすめる」

\Hatsuseno 「^{きみ}{君}にとっては^{じゅうねん}{十年}も^{いちにち}{一日}も
\ さして^{ちが}{違}うことはないかもしれないが
\ いつか
\ なつかしく^{おも}{思}う^{ことがら}{事柄}もできるだろう」

\Alpha オーナーのメッセージはこんな^{かん}{感}じだった

\page
\Hatsuseno 「その^{とき}{時}
\ ^{きおく}{記憶}の^{たす}{助}けになると^{おも}{思}うのでこれを^{おく}{送}る」
\ と

\Hatsuseno 「^{からだ}{体}に^{き}{気}をつけるように」
\ と

\Alpha も〜〜

\Alpha なつかしい^{こと}{事}ぐらいあるわよ

\page
\Alpha ありがとう

\Kokone いえ

\Kokone またお^{とど}{届}けものがあったら
\ ^{わたし}{私}が^{たんとう}{担当}させていただきますので

\Alpha あらら

\Alpha ^{たか}{鷹}^{つ}{津}ココネさん!?

\Kokone はい!

\page
\Kokone すいません
\ ^{おく}{送}っていただいちゃって

\Alpha ^{ま}{待}ちあわせ^{ばしょ}{場所}まで10キロも^{ある}{歩}くつもりだったなんて

\Alpha ^{こんど}{今度}はうちまで^{き}{来}てもらいなね

\Kokone はい

\page
\Kokone ^{わたし}{私}
\ これでも^{じもと}{地元}じゃ「^{にんげん}{人間}っぽいね」って^{い}{言}われてて

\Kokone ^{じぶん}{自分}でもそうしなきゃって^{おも}{思}ってたんです

\Alpha んー?

\Kokone でも
\ ^{いま}{今}はなんだか

\Kokone もっと^{き}{気}を^{らく}{楽}にしてみようかなって

\Alpha えーー?

\Alpha なに?

\page
\Kokone ^{しごと}{仕事}^{いがい}{以外}であそびに^{き}{来}てもいいですか?!

\Alpha もちろん!
\ いつでも!

\Alpha やほー
\ おじさーん

\Ojisan おお
\ あんだえ〜〜


\subsection{第9話\ ひと^{つぶ}{粒}300^{まい}{枚}}

\page[20]
\Alpha ^{くるま}{車}の^{じかん}{時間}まではまだ^{あいだ}{間}がある

\page
\Alpha そうそう
\ これのねー

\Kokone あ
\ もってきちゃったんですね

\Alpha うん
\ ^{こん}{今}^{つか}{使}い^{かた}{方}わかる?

\Kokone ええ
\ このタイプなら^{だいたい}{大体}

\page
\Kokone ^{ふぞく}{付属}のこのキャラメルみたいなのを

\Kokone ^{うし}{後}ろのフタあけて
\ くぼみにはめて

\Kokone フタしめて

\Kokone で
\ ここでマブタをあけてこれを^{お}{押}せば

\Kokone このピースなら1^{つぶ}{粒}300^{かい}{回}はうつせますよ

\Alpha ふ〜〜ん

\page
\Alpha このお〜〜
\ ^{め}{目}のとこってさあ〜〜

\Kokone ^{わたし}{私}たちの^{め}{目}に^{ちか}{近}いモノ^{つか}{使}ってますね

\Alpha やっぱり
\ ^{めだま}{目玉}だったのね

\page
\Kokone コードあります?

\Alpha おお!

\Alpha これ?

\Kokone ええ

\Kokone その^{ほそ}{細}い^{かた}{方}をフタの^{あな}{穴}にさして

\Alpha さして

\Kokone はじっこの^{たま}{玉}をくわえれば

\page
\Kokone カメラの^{み}{見}てる^{え}{絵}も^{み}{見}えますし

\Alpha うえ!

\Kokone はじめはこれでカメラの^{してん}{視点}をおぼえるといいですよ

\Alpha なんか
\ きもちわ^{る}{う}いね

\Kokone ^{め}{目}つぶってても
\ ^{じぶん}{自分}が^{み}{見}えたりしておもしろいですよネ

\Alpha く〜〜

\page
\Alpha せーの

\Alpha んーーん
\ なんか
\ わかってきたよ

\Alpha こりゃ^{だいじ}{大事}につかわないと

\Alpha 300^{まい}{枚}なんてすぐ^{と}{撮}っちゃいそうね

\page
\Alpha ^{くるま}{車}
\ こないね

\Kokone いつものことですよ
\ こちらも^{はや}{早}かったですし

\page
\Alpha ムサシノかー
\ ^{とお}{遠}いなあ

\Kokone そんなことないですよ

\Alpha 「ココネさん」ってかわいい^{なまえ}{名前}ね!

\Kokone あはは

\Kokone なんか
\ ^{さいしょ}{最初}に^{けんしゅう}{研修}^{じょ}{所}^{い}{行}くときに
\ ^{じぶん}{自分}でつけた^{なまえ}{名前}らしいんですけど

\Kokone なんで
\ そうなのか^{わす}{忘}れちゃいました

\Alpha ふ〜〜ん

\page
\Kokone アルファさんは
\ あの
\ ^{こゆう}{固有}^{めい}{名}はなんと

\Alpha 「アルファ」が^{こゆう}{固有}^{めい}{名}なのよ

\Alpha ^{わたし}{私}たちみたいなアルファ^{かた}{型}がまだめずらしくてねー

\Alpha 「アルファさん」って^{よ}{呼}ばれてなの

\Alpha それで^{なまえ}{名前}になっちゃった
\ まあ
\ ^{き}{気}に^{い}{入}ってるし

\Kokone はーー
\ なるほど

\page
\Alpha だいぶたって^{くるま}{車}がきた

\Alpha また^{あ}{会}えると^{おも}{思}う

\Alpha ^{し}{知}らない^{くに}{国}へ^{かえ}{帰}る^{かのじょ}{彼女}を^{み}{見}てると
\ さびしくなってくるけどね

\page
\Alpha ムサシノの^{くに}{国}
\ ^{ひがし}{東}の^{みやこ}{都}と^{よ}{呼}ばれたこともある^{ところ}{所}

\Alpha となりの^{くに}{国}なのに
\ すごく^{とお}{遠}く^{かん}{感}じるよ

\page
\Alpha おじさーん

\Ojisan お

\Ojisan あ〜〜によ〜〜

\page
\Ojisan ガスかよ

\Alpha ううん
\ ゴメンね
\ おこしちゃって

\Ojisan あーー
\ さっきのコ^{し}{知}りあいかよ
\ ^{ひるま}{昼間}^{みち}{道}ィ^{き}{聞}いてってよー

\Ojisan ^{そ}{し}^{し}{た}^{た}{っ}^{ら}{け}
\ あんたに^{に}{似}てんじゃんかー
\ へー
\ びっくりよちっとばっか

\Alpha ^{に}{似}てた!?

\Ojisan おお

\page
\Alpha うん

\Alpha ^{いもうと}{妹}なの!

\Alpha いらないシーンはひとつもない

\Alpha 300じゃとてもたりないよ

\Sign みかん^{ちゃ}{茶}


\subsection{第10話\ カマスのアヤセ}

\page[36]
\Alpha このごろタカヒロ^{み}{見}ないけど
\ どうしたの?

\Ojisan なんかよー
\ あったかくなってから

\Ojisan ^{いりえ}{入江}に^{い}{行}ってんよいつも

\Alpha あーりゃま

\page
\Takahiro うーん

\Takahiro ^{いちど}{一度}はちゃんと^{あ}{会}ってみたいよなー

\page
\Takahiro あんなに^{きょうれつ}{強烈}な^{いんしょう}{印象}なのに
\ いまいち^{げんじつ}{現実}^{かん}{感}がない

\Takahiro あーー
\ こんなかんじのあとだっけな
\ いまなり

\page
\Takahiro ^{ちい}{小}さいな
\ でも
\ ひょっとして

\Takahiro みっ
\ みさご!?

\page
\Ayase カマスだ

\Ayase おれのカマスだ

\Takahiro みざごってのはー

\Ayase ^{し}{知}ってんよ

\Ayase ミサゴはおれも^{むかし}{昔}^{み}{見}たことあんよ

\Takahiro え!

\page
\Ayase ふうん
\ ミザゴ^{み}{見}たかまだいんのか

\Takahiro うん

\Ayase おめえ^{なまえ}{名前}は

\Takahiro タカヒロ

\Ayase おれはアヤセ

\Ayase ^{み}{見}てな

\page[44]
\Ayase ^{やどだい}{宿代}の^{た}{足}しによ

\Takahiro すごい

\Ayase ミサゴはよ
\ いつもいきなしじゃんか

\page
\Ayase ^{き}{気}んなって
\ ^{なんど}{何度}か^{き}{来}たけどよ
\ そういうときや^{で}{出}ねえ

\Ayase ^{き}{気}んなってる^{やろう}{野郎}が
\ ^{ふたり}{二人}もいちゃ^{く}{来}んもんもこねえか

\Takahiro しかも^{ひとり}{一人}はおっさん

\Ayase それが^{いちばん}{一番}まじいな

\Narrator ^{とし}{年}の^{はな}{離}れた^{おとこ}{男}が
\ ^{おな}{同}じ^{おんな}{女}の^{こと}{事}を^{かんが}{考}える

\Narrator ^{あ}{会}ったのは^{べつ}{別}の^{じだい}{時代}^{し}{知}っているのは
\ ^{おな}{同}じ^{すがた}{姿}だ

\page[48]
\Ayase あれっきりだったね

\Ayase あっちからはずっとこっち^{み}{見}てたんだべな

\Ayase じゃあな

\Takahiro うん

\Takahiro ^{こんど}{今度}いつくんの?

\Ayase わかんねえな

\page
\Takahiro ^{こんど}{今度}^{あ}{会}わせたい^{ともだち}{友達}がいるんだよ

\Ayase あんだよ
\ そりゃいいよ^{べつ}{別}に

\Takahiro ^{み}{見}たかんじあやせぶらいの^{とし}{年}の^{おんな}{女}の^{こ}{子}だよ

\Ayase そりゃちょっと^{き}{気}になんかな!

\page
\Alpha うちのお^{きゃくさま}{客様}って8^{わり}{割}がたタカヒロかなー

\Alpha ううん
\ そうね〜〜


\subsection{第11話\ プロテイン}

\page[52]
\Alpha ちょっとハデ?

\Kokone かわいいよ

\page
\Kokone でも
\ こっちのがおいしそう

\Kokone ^{た}{食}べる?

\Alpha いい

\Kokone そうね
\ アルファちゃん^{さかな}{魚}だめなのよね

\Alpha うん
\ ココネ^{ほんとう}{本当}になんともない?

\Kokone 「^{しょうか}{消化}^{ほう}{法}」って^{つた}{伝}わるのかな?

\Alpha えっ

\Alpha わーーっ!

\page[55]
\Alpha そういえば
\ ココネって

\Alpha ^{にく}{肉}もサカナも^{へいき}{平気}なのよねー

\Alpha この^{まえ}{前}のココネの^{はなし}{話}だと
\ ロボットでも^{どうぶつ}{動物}タンパクの^{しょうか}{消化}にムリはないって

\Alpha どうやら^{わたし}{私}だけのクセみたい

\page
\Alpha ^{ほんき}{本気}で^{な}{慣}らしていこうかな

\Alpha おさしみくらい
\ みんなにつきあえなきゃだめよね
\ やっぱ

\Alpha こんな^{ひ}{日}はお^{きゃく}{客}さんも^{こ}{来}ないし

\Alpha よし!

\page
\Sign コーヒー^{ぎゅうにゅう}{牛乳}

\Alpha ^{あじ}{味}は^{す}{好}きなんだし
\ ^{き}{気}にしなきゃ
\ なんでもないかもね

\page
\Alpha なんか
\ ^{だいじょうぶ}{大丈夫}みたい

\page
\Alpha やっぱ
\ ダメみたい

\Alpha でも
\ これに^{な}{慣}れなきゃ

\Alpha まずい

\Alpha ^{きょう}{今日}はもう^{へいてん}{閉店}

\page
\Takahiro あっ
\ ^{ひさ}{久}しぶり!

\Alpha そっ
\ そうね!
\ はいって

\Takahiro ^{きょう}{今日}はおわり?

\Alpha うん
\ タカヒロはいいよ

\page
\Alpha こりゃ
\ ^{いじ}{意地}でもふつうにしなきゃ

\Alpha なな
\ なに^{の}{飲}む?

\Takahiro あっ
\ じゃ
\ メイポロ

\Alpha はい

\Takahiro ありがと

\page
\Takahiro あっ
\ こないだねー

\Alpha ああ
\ なにかしゃべってる

\Alpha ちゃんと^{き}{聞}かなきゃ
\ ちゃんと

\page
\Takahiro そいでねー

\Alpha う

\Alpha ^{こえ}{声}
\ でちゃった

\Takahiro なに?

\Alpha え?
\ なに?

\page
\Alpha もっとゆっくりしてけば

\Takahiro こんどね

\Takahiro ^{はや}{早}くなおしなー
\ すぐねた^{ほう}{方}がいいよ

\Alpha カゼに^{み}{見}えたのかな

\page
\Alpha その^{あと}{後}すぐ^{あめ}{雨}はやんだ
\ この^{じき}{時期}にはめずらしく
\ ^{ふじ}{富士}^{さん}{山}がよく^{み}{見}える

\Alpha ^{たいない}{体内}の^{あらし}{嵐}もけろっとなおった

\page
\Alpha おさしみはまだ^{とお}{遠}いです

\Alpha タカヒロ
\ ごめんね
\ ^{きょう}{今日}は


\subsection{第12話\ ナビ}

\page[68]
\Alpha オーナーにもらったカメラ

\Alpha あれからしまいこみがちになっちゃってる

\Alpha バンバン^{つか}{使}った^{ほう}{方}がいいとは^{おも}{思}うけど
\ つい

\Alpha ^{たからもの}{宝物}としては^{も}{持}ち^{ある}{歩}くのに
\ ピッタリなんだよね
\ ^{かる}{軽}いし

\page
\Alpha そうだ
\ ^{かわ}{皮}のバッグ

\Alpha ^{てっぽう}{鉄砲}^{よう}{用}に^{つく}{作}ったやつ!
\ ちょっとブカブカだけど

\Alpha ばっちりじゃないの!

\Alpha よし

\Alpha ^{きょう}{今日}は^{さつえい}{撮影}かな!

\page
\Alpha きみ
\ かっこいいじゃないか

\page
\Alpha よく^{み}{見}てみると

\Alpha とっておきたい^{ふうけい}{風景}は^{おお}{多}すぎる

\Alpha まだ
\ いいか

\page
\Alpha おじさーん!

\Ojisan お
\ あにした

\Alpha うん
\ ちょっとモデルになってほしいかなー
\ なんて

\Ojisan あん?

\Alpha ふつうに
\ ^{しぜん}{自然}にしててね!

\page
\Alpha ごっ
\ ごめんなさい

\Alpha またくるね

\Ojisan か
\ あによー

\Alpha なんか
\ うまくいかないなあ

\page
\Alpha よし!

\Alpha でも
\ もうちょい^{さき}{先}かな

\page
\Alpha まだ^{せん}{千}^{なんびゃく}{何百}^{まい}{枚}もあるんだし

\Alpha 1^{まい}{枚}くらい^{と}{撮}ってもいいのに

\Alpha ^{わたし}{私}ってそんなにケチだったのかなあ

\page
\Alpha いい^{え}{絵}はいっぱいある

\Alpha でも

\Alpha もう^{わたし}{私}は1^{まい}{枚}にしぼった

\Alpha ^{きょう}{今日}はそれが^{と}{撮}れればいい

\page
\Alpha ^{きた}{北}の^{おお}{大}^{くず}{崩}れ

\Alpha ここからの^{ゆうけい}{夕景}にきめた

\Alpha シルエットの^{え}{江}^{の}{ノ}^{しま}{島}と
\ ^{かいじょう}{海上}に^{なら}{並}ぶ^{がいとう}{街灯}

\Alpha ^{つぎ}{次}に^{く}{来}る^{いっしゅん}{一瞬}
\ ^{ふじいろ}{藤色}のフィルター

\page[80]
\Alpha ^{と}{撮}らなかった

\Alpha なんだか
\ おもいっきり^{みい}{見入}っちゃった

\Alpha ^{とうぶん}{当分}は^{あじ}{味}わえるわ

\page
\Alpha ^{きょう}{今日}は^{けっきょく}{結局}
\ カメラにひっぱりまわしてもらったみたい

\Alpha ^{うつ}{写}してやる!
\ って^{とき}{時}^{と}{撮}れなかったのは

\Alpha ^{せいかい}{正解}だったかもしれない

\page
\Alpha ^{で}{出}がけに^{と}{撮}った
\ なんてことないスクーターが
\ ^{きょう}{今日}
\ ^{ゆいいつ}{唯一}の^{しゃしん}{写真}

\Alpha でも
\ ^{わたし}{私}の^{しゃしん}{写真}はそれでいいと^{おも}{思}う

\Alpha ^{のこ}{残}りの^{まいすう}{枚数}は
\ そんなに^{すく}{少}ないわけじゃないって^{き}{気}がした

\Alpha それに
\ ^{きょう}{今日}の^{ふうけい}{風景}は
\ いつもより^{せんめい}{鮮明}に^{おも}{思}い^{だ}{出}せる



\subsection{第13話\ ^{かまくら}{鎌倉}^{はなび}{花火}}

\page[84]
\Ojisan あんとかまいあったか

\Ojisan けっこ^{き}{来}てんなー

\page
\Alpha ^{きょう}{今日}は10^{ねん}{年}ぶりの^{はなび}{花火}^{たいかい}{大会}!
\ ^{カマクラ}{鎌倉}の^{いりえ}{入江}あたりは^{ひと}{人}だらけです

\Ojisan やつらみゃ^{かい}{会}えねえかもなー

\Ojisan このへんだけんどよ

\Person おーう

\page
\Person へえヨ!
\ こっちだよ!

\Alpha あっ

\Ojisan ちっ

\Person おお
\ アルちゃんよかたか!?
\ よし!
\ まずひとつ

\Alpha ^{きょう}{今日}はパス!

\Person か〜〜
\ それで
\ おどってくれりゃ
\ う〜〜ん

\Alpha だからパス!

\page
\Person そこ^{く}{来}んかじじい

\Ojisan あん?

\Alpha はい
\ これムギ^{ちゃ}{茶}です
\ あと

\Alpha うちでとった^{えだまめ}{枝豆}なんだけど

\Person あの^{ふね}{船}からあがんの?

\Person うん
\ もうすぐ
\ なんか
\ まだ^{あか}{明}るいけど

\page
\Alpha あっ
\ ^{じかん}{時間}かな?

\Takahiro あっちいってくる!

\Alpha ^{わたし}{私}も

\Person あんだえー

\Alpha ^{きょう}{今日}のメインは^{さいしょ}{最初}の1^{はつ}{発}^{め}{目}

\Alpha この^{へん}{辺}では^{こんかい}{今回}で^{さいご}{最後}になる
\ ^{おおだま}{大玉}です

\page
\Alpha ^{わたし}{私}も^{おおだま}{大玉}って^{はじ}{初}めて^{み}{見}る

\page[92]
\Alpha こっ
\ こうかーー!!

\page[96]
\Alpha あの
\ ^{わたし}{私}

\Alpha タカヒロと^{さき}{先}に^{せん}{先}^{れい}{礼}します

\page
\Ojisan あ〜〜
\ わりいけんど
\ たのむわ

\Ojisan てーげー
\ こいじゃ
\ ^{あさ}{朝}んなんべなーー

\Ojisan なる

\Ojisan おれは^{あした}{明日}こいつらと^{かえ}{帰}んよ

\Alpha すみません

\Alpha じゃ
\ みなさん
\ お^{さき}{先}に

\Person おーー

\page
\Alpha あんなにすごい^{み}{見}たばっかなのに

\page
\Alpha タカヒロ^{きょう}{今日}のことちゃんとおぼえてる?

\page
\Alpha こんな^{かん}{感}じかなあ^{おとうと}{弟}って

\Alpha ぐーー


\subsection{第14話\ ^{すな}{砂}の^{はま}{浜}}

\page[102]
\Alpha ^{きょう}{今日}は^{せんせい}{先生}の^{ところ}{所}に^{き}{来}ています

\Sensei あらま
\ アルファさんがつくったの?

\Alpha ^{いちおう}{一応}トマト
\ ^{おお}{大}きさバラバラですけど

\Sensei ごちそうさま
\ あれから^{からだ}{体}の^{ほう}{方}は?

\Alpha はい

\Alpha ^{いぜん}{以前}よりいい^{くらい}{位}です

\Alpha じゃ
\ また

\page
\Alpha そういえば

\Alpha ここって
\ ^{すな}{砂}^{はま}{浜}なんですね

\Ojisan あーー
\ もうここぐれえだべ
\ おおかた^{しず}{沈}んで
\ まったっけ

\Alpha ちょーっと
\ ^{みず}{水}あびてきます!

\Ojisan あん?
\ ^{みずぎ}{水着}はよ

\Alpha ^{わたし}{私}^{みずぎ}{水着}もってないんです

\Ojisan あ〜〜
\ ふうん
\ ^{きゅう}{急}に^{ふか}{深}えから
\ ^{き}{気}いつけな

\page[106]
\Sensei なんか
\ いいわね
\ ^{かのじょ}{彼女}

\Ojisan あ?
\ あーー
\ そうすね

\Sensei ここも^{か}{変}わっちゃったわね

\Ojisan そうすね

\page
\Ojisan ーーあんだ
\ もう^{あ}{上}がりか

\Alpha いいものひろっちゃいましたよ!

\Alpha ほら!

\Ojisan ほらって
\ おめえ
\ こりゃ

\Alpha きれいですよね!

\Alpha ^{うみ}{海}の^{みず}{水}の^{いろ}{色}ですね

\page
\Ojisan そうな

\Ojisan ^{むかし}{昔}はゴロゴロしてたけどな

\Sensei ^{ひさ}{久}しぶりに^{み}{見}たわね

\Alpha ^{すな}{砂}の^{はま}{浜}と^{うみ}{海}の^{いろ}{色}のビン

\Alpha ^{きょう}{今日}の^{しゅうかく}{収穫}です


\subsection{第15話\ ^{すな}{砂}の^{みち}{道}}

\page[110]
\Alpha あのー

\Alpha ^{ふく}{服}かわくまでいいですか?

\Ojisan おう
\ いいよ

\page
\Ojisan あのビン^{した}{下}の^{みせ}{店}のやつですかね

\Sensei ああ
\ そうかもね

\Sensei ^{み}{見}たかんじは^{べつ}{別}の^{ばしょ}{場所}みたいだけど

\Sensei やっぱり^{おな}{同}じ^{とこ}{所}なのよね

\page
\Sensei ^{おな}{同}じ^{とこ}{所}なのよね

\page[114]
\Sensei おそい!

\Ojisan すんません

\Ojisan あー
\ ^{せんぱい}{先輩}

\Ojisan やっぱ
\ イッちゃんも
\ ^{しごと}{仕事}で^{く}{来}でねえそうです

\Sensei あそー
\ まあいいや^{ふたり}{2人}で^{い}{行}くかたまにゃー

\Ojisan はあ

\page
\Sensei じゃあ

\Sensei ついて^{く}{来}るように

\Ojisan ういーす

\Ojisan いきなし^{よ}{呼}び^{だ}{出}された

\Ojisan ^{へいさ}{閉鎖}された^{かいがん}{海岸}^{どうろ}{道路}の
\ ^{みおさ}{見納}め^{たいかい}{大会}だというが

\Ojisan ^{へいじつ}{平日}ヒマなのは^{だいがく}{大学}^{づと}{勤}めの^{せんぱい}{先輩}と
\ プータローの^{おれ}{俺}くらいだ

\page
\Sensei あーん
\ ^{しよう}{使用}^{ちゅう}{中}の^{みち}{道}でこんなかあ

\Ojisan ^{みち}{道}ってよか
\ ^{ていぼう}{堤防}ですね

\page
\Sensei じゃあ
\ ^{なみ}{波}の^{あいだ}{間}ねらって

\Ojisan やっぱ^{い}{行}くんすか

\Sensei あゴー!

\Ojisan あうー!

\page
\Sign ^{つうこうどめ}{通行止め}
\ ^{?}{う}^{まわ}{回}して^{くだ}{下}さい

\Ojisan こっからですか

\Sensei うん

\Sensei どう^{つうこうどめ}{通行止め}なのか
\ ^{たし}{確}かめなきゃね

\Ojisan そうなんすか?

\page
\Ojisan じゅうぶん^{たし}{確}かめました!

\Sensei そうね
\ ^{なっとく}{納得}!

\page
\Sensei これが
\ あの^{じゅうたい}{渋滞}^{どうろ}{道路}

\Sensei ^{なみ}{波}かぶりそう

\Ojisan その^{みち}{道}1^{ほん}{本}じゃ
\ すまないでしょうね
\ この^{せん}{先}

\Sensei ^{いっしんいったい}{一進一退}
\ ^{なん}{何}10^{ねん}{年}かしならあもう

\page
\Ojisan ^{せんぱい}{先輩}

\Ojisan ^{おれたち}{俺達}くらいじゃないすかね

\Ojisan いまどきブラブラしてんのこんなとこで

\Sensei ^{いま}{今}しか^{み}{見}らんない^{けしき}{景色}だよ

\Ojisan ^{せんぱい}{先輩}がよく^{おれ}{俺}に^{み}{見}せたのは
\ そんな「^{しゅん}{旬}のもの」の^{ふうけい}{風景}だった

\Ojisan ^{じだい}{時代}をよく^{あらわ}{表}し
\ ^{べつ}{別}に^{ちゅうもく}{注目}されず
\ 2^{ど}{度}と^{み}{見}られないもの

\page
\Sensei ^{ほん}{本}とかでわかることでも
\ ^{げんば}{現場}で^{かん}{感}じるのと^{ぜんぜん}{全然}ちがう

\Sensei ^{め}{目}の^{まえ}{前}のモノちゃんと^{み}{見}て

\page
\Ojisan はあ

\page
\Ojisan おおっ!
\ びっくりしたーー!

\Alpha お^{ま}{待}たせしました

\Alpha ^{いま}{今}
\ ^{べつ}{別}の^{せかい}{世界}^{み}{見}てたでしょ
\ ^{ふたり}{2人}で

\Ojisan まあな!

\page
\Alpha いーなー
\ ^{ふたり}{2人}の^{せかい}{世界}
\ ^{み}{見}てみたい

\Sensei ^{ぞくへん}{続編}なら^{み}{見}られるわよ

\Sensei ^{きょう}{今日}の^{こと}{事}ちゃんとおぼえとけばね

\Alpha はあ

\page
\Alpha この^{とき}{時}の^{きおく}{記憶}
\ 3^{にん}{人}の^{こうけい}{光景}が「^{ぞくへん}{続編}」でした

\Alpha ^{き}{気}がつくのは^{のち}{後}になってからです


\subsection{ココネのなんかよさそう!}

\Shiba えっくし!

\Kokone なんかいいなー
\ くしゃみって
\ ^{おし}{教}えてよ
\ やり^{かた}{方}

\Shiba やり^{かた}{方}って
\ あんた

\Shiba よーし

\Shiba こうすんのよ!

\Shiba ほれ

\Kokone わくわく

\Kokone そっ
\ それから!?

\Shiba きかねーし

\Shiba ムリにやるほどのもんじゃないって
