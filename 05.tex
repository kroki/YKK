\section{Volume 5}

\subsection{^{だい}{第}32^{わ}{話}\ ^{ふじ}{富士}^{さん}{山}}

\page[2]
\Alpha ひさしぶりにカメラいじってます

\Alpha また、^{みなみまち}{南町}にコーヒー^{まめ}{豆}が^{はい}{入}らなくなってきて

\Alpha ^{きょう}{今日}は、うちも^{きゅうじつ}{休日}

\page
\Alpha こういう^{ひ}{日}でもやることは^{かんが}{考}えれば、いっぱいあるけど

\Alpha 「なんにもしない^{ひ}{日}」にしました

\page
\Alpha ^{きょう}{今日}は^{はる}{春}にしてはモヤってないから

\Alpha ^{ゆうがた}{夕方}になれば^{ふじ}{富士}^{さん}{山}まで^{み}{見}えるかも

\page
\Alpha よっ

\page
\Alpha ^{ひ}{日}がしずんで、^{うし}{後}ろから^{ひかり}{光}を^{う}{受}けた
^{いず}{伊豆}が^{う}{浮}かんでくる

\page
\Alpha ふう

\Alpha ^{ふじ}{富士}^{さん}{山}

\Alpha ^{いぜん}{以前}のーーー
\ ^{かん}{完}ぺきだった、^{むかし}{昔}の^{すがた}{姿}もよかったけど

\page
\Alpha やっぱり^{わたし}{私}は、リラックスしてる^{いま}{今}の^{ふじ}{富士}^{さん}{山}が^{す}{好}き


\subsection{第33話\ ^{よこはま}{横浜}^{か}{買}い^{だ}{出}し}

\page[10]
\Sign Closed
\ ^{らいしゅう}{来週}まで…
\ ^{か}{買}い^{だ}{出}しです
\ Cafe\ アルファ

\page[12]
\Alpha ^{とうめん}{当面}のコーヒー^{まめ}{豆}を^{か}{買}いに^{おも}{思}いきって
^{よこはま}{横浜}に^{い}{行}くことにしました

\page
\Alpha ^{こんかい}{今回}は^{すこ}{少}し^{よゆう}{余裕}とってます

\page[15]
\Alpha ^{まち}{町}への^{えだみち}{枝道}に^{はい}{入}ると、やっと^{じんか}{人家}がふえてきて

\Alpha ^{いよう}{異様}にでっかくランドマークタワーが^{しょうめん}{正面}に^{で}{出}る

\page
\Alpha あのてっぺんにもいつか^{い}{行}ってみたいけど

\Alpha ^{かいだん}{階段}つらそうなんだよね

\page
\Alpha 「^{きょう}{今日}の^{もくてき}{目的}は^{まめ}{豆}!!」

\Alpha よし!

\page
\Alpha ああん

\Sign もなか

\Sign まんじゅう

\Alpha コーヒー^{まめ}{豆}……

\Alpha なんとか^{まめ}{豆}も^{か}{買}えました

\Alpha ほう

\page
\Alpha まだ^{じかん}{時間}もいっぱいあるからーー

\Sign ところてん\ あります

\Sign ^{いも}{芋}せんべい

\Sign ^{いも}{芋}けんぴ

\Sign ^{いも}{芋}よ??ん

\Sign あんこ^{たま}{玉}

\Alpha たぶん^{やど}{宿}もさがせるし、おそらくお^{かね}{金}も^{た}{足}りるし

\page
\Alpha やっぱり、ランドマークにものぼってみようかなーー

\Alpha ^{ほか}{他}に^{か}{買}っとくものは〜〜

\page
\Sign ココネ
\ ……江戸郡世田谷町

\page[23]
\Alpha おばちゃん、ごっそさん!

\Alpha ごめん!お^{わん}{椀}、ここ^{お}{置}いとくね!

\Person う〜〜い

\page
\Alpha いきなりだけど

\Alpha でも

\Alpha ^{い}{行}ってみよう!


\subsection{第34話\ お^{きゃく}{客}さま}

\page[26]
\Kokone ふう

\Kokone アルファさんの^{とき}{時}^{いらい}{以来}ひさしぶりに「メッセージ^{はいたつ}{配達}」の^{しごと}{仕事}です

\page
\Kokone あの^{ひ}{日}のあとこの^{たんとう}{担当}が、
なぜか^{すこ}{少}し^{おも}{重}く^{かん}{感}じるようになりました

\Kokone えーと

\Kokone ^{おお}{大}^{た}{田}^{むら}{村}、^{あざ}{字}^{でん}{田}^{えん}{園}^{ちょう}{調}^{ふ}{布}の

\Kokone このへんよね

\page
\Kokone あった

\Sign アトリエ\ ^{まる}{丸}^{こ}{子}

\Kokone ふう

\page
\Kokone ごめんください!

\Kokone ムサシノ^{うんそう}{運送}ですーー!

\Kokone ごめ……

\Maruko ああ
\ まってたのよ!

\Maruko まあ
\ はいって

\Kokone は……
\ はい

\page
\Maruko はい
\ お^{ちゃ}{茶}でも

\Kokone あ
\ どうも

\Kokone えーと
\ じゃここにサインを

\Maruko まー
\ それはメッセージ^{う}{受}けとってからね

\Maruko ひと^{いき}{息}^{い}{入}れた^{ほう}{方}が、
^{ないよう}{内容}の^{つた}{伝}わり^{かた}{方}もいいし

\Kokone そっ
\ そうですね

\page
\Kokone ああ……やっぱり、こんなに^{きんちょう}{緊張}してる

\Kokone ^{まえ}{前}は、なんでもなかったのに

\Kokone でも

\Kokone どうしてなんでもなかったんだろう

\Kokone ふー

\page
\Kokone はい!
\ では!

\Maruko ん!

\Maruko でも、ちょっとやけくそみたい
\ だから
\ ゆっくりめに

\Kokone はい

\page
\Maruko ふーー

\Maruko そんなリアルに^{つた}{伝}えられるものなのね

\Kokone は?

\Maruko いえね
\ ^{わたし}{私}、^{え}{絵}なんか^{えが}{描}いてるでしょ

\Maruko ^{ともだち}{友達}がね、^{み}{見}た^{ふうけい}{風景}の^{いんしょう}{印象}とか、
ときどき^{おく}{送}ってくれるのよ

\page
\Maruko でも、こんなに^{こま}{細}やかに^{つた}{伝}えてくれる^{ひと}{人}がいるなんて

\Maruko はい

\Kokone そっ
\ そうですか?

\Maruko あなたすごいわ

\Kokone はあ
\ どうも

\page
\Maruko ^{つぎ}{次}も、あなたに^{き}{来}てもらうように^{たの}{頼}めるかな?

\Kokone は……

\Maruko ^{き}{気}に^{い}{入}っちゃったから

\Kokone あ……はい
\ ありがとうございました!

\page[37]
\Sign ムサシノ^{うんそう}{運送}
\ ^{せたがや}{世田谷}^{してん}{支店}

\Sign ^{ごよう}{御用}は^{しょうめん}{正面}へ

\Shiba ありゃ

\Shiba どしたの
\ ^{そと}{外}^{ぎ}{着}のまんまで

\Kokone あ〜〜
\ シバちゃ〜〜ん

\Kokone なんだかつかれた

\Shiba あ?

\page
\Kokone も〜〜やたらあがるし、
\ ^{みょう}{妙}に^{う}{受}けがいいしで

\Shiba ふ〜〜ん

\Kokone ^{まえ}{前}は^{ぜんぜん}{全然}こんなことなかったのにな

\Shiba ん〜〜
\ まあ
\ でも
\ ほめられたんだし

\Kokone うん
\ まあ
\ そうだね

\Shiba ^{あした}{明日}^{やす}{休}みで、よかったじゃん

\Shiba ぼーっとしてなよ

\Shiba おつかれ

\Kokone うん
\ じゃあね

\page
\Kokone ただいまっと

\Kokone ふう

\page
\Alpha やほーー
\ ^{き}{来}ちゃったよ

\Person おかえりよーー


\subsection{第35話\ ^{よる}{夜}のにおい}

\page[42]
\Kokone どっ
\ どうぞ

\Alpha おじゃましますー

\page
\Kokone せまいとこですけど

\Alpha ううん
\ ごめんね
\ いきなり^{き}{来}ちゃってここが

\Alpha コッ
\ 「ココネの^{いえ}{家}ってのはココね!」

\Alpha あ〜〜
\ いや……

\Alpha なし!

\page
\Kokone おもしろくないでしょ、^{わたし}{私}のへや

\Alpha ううん

\Kokone はい……
\ そば^{ちゃ}{茶}です
\ これ

\Alpha ありがと

\page
\Alpha あーー
\ よかった

\Kokone え?

\Alpha いやー、ココネ^{かえ}{帰}ってきたときさあ

\Alpha リアクションなかったから……

\Alpha なんか、まずい^{とき}{時}に^{き}{来}ちゃったのかなって^{おも}{思}っちゃった

\Kokone あ……

\Kokone いえ……ちゃっと
\ ^{ひるま}{昼間}の^{しごと}{仕事}のこと^{かんが}{考}えてたんで

\Kokone びっくりして、つい……

\Alpha そっか

\page
\Kokone そうか
\ アルファさんと^{はじ}{初}めて^{あ}{会}った^{とき}{時}も

\Kokone うわ〜〜
\ ^{いま}{今}だったらできないな

\Kokone あ
\ でも
\ う〜〜ん

\page
\Alpha あのー
\ さー

\Kokone えっ

\Alpha いきなりおしかけといてなんなんだけど

\Alpha ^{きょう}{今日}^{と}{泊}まっていいかな

\Alpha だっ
\ だめ?

\Kokone いえ!
\ ^{ぜんぜん}{全然}!
\ ぜひ^{と}{泊}まってってください

\Kokone ^{わたし}{私}、あした^{やす}{休}みですし

\Alpha あ
\ よかった〜〜

\page
\Alpha いやー、ここさがした、さがした

\Kokone あ
\ やっぱり

\Alpha ^{わたし}{私}、^{がいこく}{外国}に^{き}{来}たの^{はじ}{初}めてだから
\ ^{ま}{舞}いあがっちゃって

\Alpha ムサシノって^{たい}{平}らな^{くに}{国}なのね

\Kokone この^{へん}{辺}はそうですね

\Kokone ^{ふゆ}{冬}になると^{ちへいせん}{地平線}までススキの^{うみ}{海}なんですよ

\Alpha ふーん

\page
\Kokone バイクでここまでってけっこうきつかったんじゃないですか

\Alpha そう!

\Alpha ココネいきなりうちまでよく^{き}{来}たよねー

\Alpha ^{わたし}{私}は^{よこはま}{横浜}でここのこと^{おも}{思}いついたから、
^{きぶん}{気分}^{てき}{的}に^{らく}{楽}だったけど

\Alpha あ

\Alpha ^{げっぺい}{月餅}^{た}{食}べる?

\Kokone わっ

\page[51]
\Alpha なんか、ちらかしに^{き}{来}たみたい

\Kokone いいえ

\Kokone そのままにしといてください

\Kokone ふとん……はもういいですね

\Alpha そうだね

\page[53]
\Alpha ^{しず}{静}か……

\Kokone ええ

\Kokone アルファさんのうちは^{なみ}{波}の^{おと}{音}が^{き}{聞}こえますもんね

\Alpha そうだったんだねー

\page
\Alpha ココネ

\Alpha あした、^{よてい}{予定}ある?

\Kokone ないですよ

\Alpha どっか^{い}{行}きたいね

\Kokone どこ^{い}{行}きましょうか……

\page[56]
\Alpha ムサシノの^{よる}{夜}は、^{とお}{遠}くから^{く}{来}る、^{やま}{山}のにおいがします


\subsection{第36話\ ^{まち}{町}のにおい}

\page[58]
\Alpha お^{ひる}{昼}も^{ちか}{近}くなってから^{まち}{町}に^{で}{出}ました

\page
\Alpha ^{まち}{町}とは^{い}{言}っても、^{よこはま}{横浜}みたいにかたまってなくて

\Alpha ずーっと^{とお}{遠}くまでパラパラと^{いえ}{家}や^{みせ}{店}が
ちりばめてあるって^{かん}{感}じです

\Alpha おわーー!

\page
\Alpha ムサシノは^{き}{木}も^{みち}{道}もまっすぐでやたらと^{おお}{大}きくて

\page
\Alpha さらに、^{わたし}{私}の^{くに}{国}よりもずっとさらさらした^{くうき}{空気}の^{かお}{香}り

\Alpha ココネの^{かみ}{髪}や^{ふく}{服}の^{かお}{香}りと^{すこ}{少}し^{に}{似}ています

\Alpha ^{じぶん}{自分}じゃわからないけど、^{わたし}{私}にも、そんな^{かお}{香}りがあるんでしょうか

\page
\Sign かんぱち
\ ^{つじ}{辻}の^{ちゃ}{茶}

\Kokone じつは、このへんあんまり^{みどころ}{見所}ないんですよ

\Alpha んーん
\ めずらしい^{もの}{物}ばかりだよ

\Alpha でも、ほんとの^{とこ}{所}それは^{に}{二}の^{つぎ}{次}です

\page
\Alpha あの、すいません、シャッター^{お}{押}していただけます?

\Person あ
\ はい

\Person あの
\ これ、ファインダーないんですが

\Alpha あ〜〜
\ ^{だいたい}{大体}でいいです

\Person はい
\ では
\ せえの

\page
\Alpha ^{けっきょく}{結局}、^{こんや}{今夜}も^{と}{泊}めてもらっちゃいます

\Alpha ^{に}{二}^{はく}{泊}^{みっか}{三日}の^{たび}{旅}でした


\subsection{第37話\ ^{あさ}{朝}^{はや}{早}く}

\page[68]
\Sensei 「^{あした}{明日}の^{あさ}{朝}、もしヒマで、^{はや}{早}く^{お}{起}きられたら、
  ^{わたし}{私}の^{ところ}{所}へ^{き}{来}てみてね」

\Alpha きのう^{せんせい}{先生}が^{ひさ}{久}しぶりに^{みせ}{店}にきてくれて

\Alpha なんだかそんな^{はなし}{話}が^{で}{出}た

\page
\Alpha ^{おも}{思}いっきり、さそわれたわけじゃないけど

\Alpha せっかくだから^{い}{行}ってみることにしました

\page
\Alpha あ
\ おはようございます

\Sensei あら

\page
\Sensei きてくれたのね

\Alpha えへへ

\Alpha なんだろって^{おも}{思}いまして
\ おじさんも^{き}{来}てるんですね

\Sensei うん
\ ^{かれ}{彼}にもつきあってもらうのよ

\Sensei ^{いま}{今}、^{げんち}{現地}で^{ま}{待}ってるの

\Alpha ^{げんち}{現地}?
\ なんなんですかきょうは!

\page
\Sensei ^{ほんとう}{本当}^{い}{言}うとね、アルファさんにはぜひ^{き}{来}てもらいたかったのよ

\Sensei ホッとしちゃった

\Sensei ^{すこ}{少}し^{ちから}{力}をかしてもらいたいことがあってね

\Sensei ……ていうか^{た}{立}ち^{あ}{会}ってほしかったもんだから

\page
\Sensei ^{げんち}{現地}っていってもすぐそこだけどね

\Alpha はあ

\Sensei ここよ

\page
\Alpha あっ
\ おじさん

\Ojisan よう

\page
\Ojisan ^{き}{来}たか

\Ojisan すげえもん^{み}{見}られんぞ、^{きょう}{今日}

\Alpha えっ!?
\ そーー!?
\ ^{せんせい}{先生}なんにも^{い}{言}ってくんないんですよーー!

\Sensei ごめん

\Alpha あはは

\Ojisan まあ
\ ^{き}{来}てみな

\Ojisan ^{み}{見}てな

\page
\Alpha おわーー!!

\page
\Alpha すごい

\Alpha なんですかこれ……ふね?

\Sensei ^{ふね}{船}……なのかね

\Sensei ^{がくせい}{学生}の^{とき}{時}、こういうの^{つく}{作}って^{あそ}{遊}んでたのよ

\Sensei この^{きたい}{機体}はまだ^{いちど}{1度}も^{はし}{走}ってないやつなんだけど

\Sensei もうボロボロね

\Sensei こいつを^{はし}{走}らせる^{さいご}{最後}の^{きかい}{機会}だと^{おも}{思}うの

\page
\Sensei ずっと^{おか}{陸}に^{お}{置}いといてもよかったんだけど

\Sensei それだと、こいつ^{う}{生}まれてから^{いっかい}{1回}も^{い}{生}きないで^{か}{枯}れてっちゃう

\Sensei ^{わたし}{私}もいつまでも^{げんき}{元気}なわけじゃないし

\Sensei ^{むかし}{昔}の^{なかま}{仲間}はいないから

\Sensei それに、こいつが^{はし}{走}るとこアルファさんにも^{み}{見}てほしかったしね

\page
\Alpha ありがとうごさいます

\Sensei ひとつ

\Sensei ^{てつだ}{手伝}ってほしい^{こと}{事}があるんだけど

\Alpha あ
\ はい

\Sensei アルファさんに^{そうじゅう}{操縦}してほしいの

\page
\Alpha そんな!
\ わ
\ わたし、こんなすごいの^{うんてん}{運転}したことないですよ!!

\Alpha なんか、^{の}{乗}るとこもないみたい……

\Sensei え?

\Sensei あら
\ ^{しんぱい}{心配}しないで

\Sensei ^{だいじょうぶ}{大丈夫}よ


\subsection{第38話\ ^{うみ}{海}の^{かわ}{河}}

\page[83]
\Alpha ^{せんせい}{先生}の^{ふね}{船}

\Alpha ^{きょう}{今日}はじめて^{けいけん}{経験}する^{いのち}{命}、
それを^{ぜんぶ}{全部}^{ぜんしん}{前進}に^{つか}{使}って

\Alpha ^{そとうみ}{外海}の^{ほう}{方}に^{はし}{走}っていきます

\Alpha ^{せんせい}{先生}は、^{ふね}{船}に^{くろしお}{黒潮}まで^{い}{行}って、
そこで^{やす}{休}んでほしいと^{おも}{思}っていたとか

\page
\Alpha ^{わたし}{私}は^{ふね}{船}をまっすぐ^{い}{行}かせる
^{やく}{役}を^{たんとう}{担当}することになりました

\Alpha ^{わたし}{私}には、^{くち}{口}のコレで^{ふね}{船}の^{ちい}{小}さな^{かじ}{舵}を
^{うご}{動}かすことができるそうなのです

\page
\Alpha どんどん^{かそく}{加速}していく^{ふね}{船}

\page
\Alpha なんか^{くち}{口}に

\Alpha ^{ふね}{船}の^{かんしょく}{感触}があります

\Sensei え?
\ そうなの?
\ ^{へん}{変}ね、むこうからは^{なに}{何}もきてないはずだけど

\Sensei でも
\ あ
\ もう
\ ^{み}{見}えなくなっちゃうわね

\page
\Sensei アルファさんのおかげで、うまく^{つるぎざき}{剱崎}と^{すのさき}{洲崎}の^{あいだ}{間}^{ぬ}{抜}けられそう

\Sensei アルファさん?

\page[94]
\Alpha あ……

\Sensei あ!

\Sensei ^{だいじょうぶ}{大丈夫}?

\Sensei ごめんなさいね……
\ まさか、あんなに^{ふか}{深}くつながっちゃうなんて

\page
\Sensei あらら

\page
\Sensei さっきはごめんね

\Alpha いやーー

\Alpha なんだか、よくおぼえてないんですよ……

\Sensei ^{きょう}{今日}アルファさんが^{き}{来}てくれてうれしかったわ

\Sensei あのね
\ ^{むかし}{昔}、あの^{ふね}{船}ぶっとばす^{こと}{事}とロボットの
^{けんきゅう}{研究}がつながったことがあるの

\Sensei まあ、でも、^{けっきょく}{結局}なんの^{やく}{役}にも^{た}{立}たなかったんだけど

\Sensei そんなこともあってね

\page
\Sensei きょうはありがとう

\Alpha いえ
\ そんな
\ ^{わたし}{私}の^{ほう}{方}こそ

\Alpha ^{きょう}{今日}は^{せんせい}{先生}の^{ふね}{船}の^{たんじょうび}{誕生日}

\Alpha それは^{どうじ}{同時}にお^{わか}{別}れの^{ひ}{日}でもありました

\page
\Alpha あの^{とき}{時}、なにか^{み}{見}たような^{き}{気}もします


\subsection{第39話\ ^{ごご}{午後}の^{むぎちゃ}{麦茶}}

\page[100]
\Alpha ここんとこ^{まめ}{豆}がなかなか^{へ}{減}りません

\page
\Alpha タカヒロ^{さいきん}{最近}、あんまり^{こ}{来}ないもんなー

\page[103]
\Takahiro あに
\ タラ〜〜って^{ある}{歩}いてんだよ

\Takahiro ついてくるっつったの、マッキだべー?

\Makki だあって、「^{うみ}{海}^{い}{行}く」ってゆうからさ!

\Makki とうもろこしだってゆでてさ!^{むぎちゃ}{麦茶}だって……

\Takahiro あ〜〜
\ はい

\Takahiro ありがとね

\Makki もっと、^{みず}{水}であそべるとこって^{おも}{思}ったのにさ!
\ こんな^{ぬま}{沼}みたいなとこ^{き}{来}て……

\page
\Takahiro だから、みさごいんのはここいらだって^{い}{言}ったべ

\Takahiro マッキだって^{いっかい}{1回}^{み}{見}りゃすげえって^{おも}{思}うよ

\Makki わたしミサゴなんかいいよ、^{べつ}{別}に

\Makki わざわざこわい^{おも}{思}いしにこんなとこまで^{き}{来}てさ

\Takahiro だいじょうぶ、こわくねえって

\Makki ほんと?

\Takahiro いや
\ けっこうこええかな

\page
\Takahiro ^{むぎちゃ}{麦茶}は^{せいかい}{正解}だったなあ

\Makki でしょ!

\page
\Takahiro ^{かえ}{帰}るか

\Makki うん!

\Takahiro めったに^{で}{出}ねえんだよ

\Makki うん

\page
\Takahiro あにした?

\Makki おしっこ

\Takahiro あそう
\ ここで^{ま}{待}ってら

\Makki ちっとかっこわりかったな〜〜

\page[111]
\Takahiro ^{い}{行}くか

\Makki あ
\ う
\ うん

\Makki お
\ おしっこひっこんじゃった

\Takahiro あんだえー

\page
\Makki タカヒロ

\Takahiro ん〜〜?

\Makki ミサゴってさ
\ なにしに^{で}{出}てくんのかな

\Takahiro あ?
\ さ〜〜
\ ウケねらいかな

\Makki でも、なかなか^{で}{出}ないんでしょ

\Takahiro ん〜〜

\Takahiro ^{て}{照}れてんのかな

\Makki そっか
\ ま
\ いいや

\page
\Makki ^{きょう}{今日}さ
\ タカヒロんち^{い}{行}くよ!

\Takahiro ^{いま}{今}から?

\Makki うん

\Makki すんごいことおしえてやるよ!

\Takahiro あによ

\Makki はっはっは


\subsection{第40話\ 月夜見(つくよみ)}

\page[116]
\Sign 霞 浦バス
\ ^{もと}{元}^{さんばし}{桟橋}
\ 土浦ゆき

\page[118]
\Person よーー

\Ayase ^{はや}{早}えな

\page
\Person きのう、^{お}{終}わったばっかなのによー

\Ayase あに、もう、^{つぎ}{次}い^{い}{行}くとか?

\Person おう
\ ^{いま}{今}ちっと^{ても}{手持}ちに^{よゆう}{余裕}がほしい

\Person アヤセよー
\ ^{みなみ}{南}に^{ほう}{方}^{い}{行}きゃまた^{はたけ}{畑}のバイトあるってよ

\Person ^{い}{行}ってみねえ?

\Ayase おれは^{はたけ}{畑}はしばらくいいや

\page
\Ayase あにしろあいつがまだよう

\Person あ?

\Person あー
\ 「カマス」っつったっけか

\Ayase おう
\ なんか、まだここ^{うご}{動}きたくねえみてえでよ

\page
\Person ^{せわ}{世話}かかんな〜〜

\Ayase ま、そゆことなんで

\Person そか

\Person じゃ、ま
\ いずれどっかでまた

\Ayase うーーす

\Person あ
\ そうだ

\page
\Person コメが^{すこ}{少}し^{て}{手}に^{はい}{入}った
\ ちょっとやるよ

\Ayase おお

\Ayase もう、^{あじ}{味}、おぼえてねえや

\Person まー
\ たまにはな

\Ayase じゃ
\ お^{かえ}{返}しと^{い}{言}うのもなんだか

\Ayase しょう^{あぶら}{油}でもあげよう

\Ayase この^{とうき}{陶器}のやつな

\Person えっ!こりゃ、おめえ

\page
\Person こっちの^{ほう}{方}がすげえよ!

\Ayase まー
\ たまにはな

\page[130]
\Ayase ^{あした}{明日}から、また、^{うご}{動}く


\subsection{第41話\ ^{いちがん}{一眼}}

\page[132]
\Alpha カメラを^{も}{持}って^{である}{出歩}いた^{ひ}{日}

\Alpha ^{かえ}{帰}りの^{とちゅう}{途中}おじさんに^{おく}{送}ってもらった

\page
\Ojisan あによ

\Ojisan じじいばっか^{と}{撮}っちゃっちゃもったいねえべー

\Alpha いーえー

\Alpha ^{いっしゅん}{一瞬}、すごくかっこよかったりしてますよ!

\Ojisan そうかー
\ ^{いっしゅん}{一瞬}かー

\page
\Ojisan ^{かんが}{考}えてみりゃ、このごろ^{しゃしん}{写真}とか^{み}{見}てねえよなー

\Ojisan こんだアルファさんのとったのとか^{み}{見}してけーねーか

\Alpha あ〜〜
\ でもこれ

\Alpha ロボットの^{ひと}{人}じゃないと、^{み}{見}らんないんじゃないかなあ

\Ojisan あんだ
\ こう、ペラッとしたのがあんじゃねえんだ

\Alpha うん

\Alpha ^{て}{手}はなさないでね

\Alpha もし、モニターとかプリンター^{つか}{使}っても

\Alpha たぶん^{み}{見}え^{かた}{方}は^{つた}{伝}わらないと^{おも}{思}います

\Ojisan へ〜〜

\page
\Ojisan ひとことで^{ゆ}{言}うとすんとどんな^{ふう}{風}?

\Alpha えーとね

\Alpha こう
\ いや
\ ん〜〜
\ あれ?

\Ojisan ま
\ なにしろ、まずプリンターからねえしなー

\Alpha はあ

\Ojisan テレビでよけりゃ、どっかにあったかな?

\Alpha あはは

\page[138]
\Alpha おじさんに^{しゃしん}{写真}のことをうまく^{せつめい}{説明}できないのがもどかしかった

\Alpha たぶん、このカメラで^{と}{撮}るのは、^{しゃしん}{写真}とは、
また^{べつ}{別}のものじゃないかと^{おも}{思}う

\page
\Alpha さっきのおじさんの^{しゃしん}{写真}

\page
\Alpha ^{ゆうがた}{夕方}の、あの^{とき}{時}からの^{じかん}{時間}^{けいか}{経過}が、まるでうそのように
\ ^{いま}{今}、^{たし}{確}かに^{め}{目}の^{まえ}{前}にひろがる^{こうけい}{光景}

\Alpha ^{うご}{動}きこそしないけど、^{すこ}{少}しなら^{してん}{視点}をずらすことさえできる

\page
\Alpha はじめのころは^{き}{気}づかなかった

\Alpha このカメラは、^{うつ}{写}した^{とき}{時}からの^{じかん}{時間}が
^{なが}{長}ければ^{なが}{長}いほど、^{ようしゃ}{容赦}なくリアルな^{むかし}{昔}へ
^{わたし}{私}をひきもどす

\Alpha ^{み}{見}ている^{わたし}{私}も、さっきのかっこをしていると^{かん}{感}じる

\Alpha ^{たし}{確}かめることはできないけど

\Alpha ずっと^{まえ}{前}の^{しゃしん}{写真}

\page[143]
\Alpha ^{わたし}{私}は、このカメラのことを^{ほんき}{本気}で、^{じかん}{時
  間}^{りょこう}{旅行}^{き}{機}のように^{かん}{感}じることがある

\page
\Alpha ^{め}{目}をあけるときは、^{いっしゅん}{一瞬}、ほんとうにどちらが
^{いま}{今}で、どちらが^{むかし}{昔}なのかわからないくらい

\page
\Alpha ふう

\Alpha んぅ〜

\page
\Alpha ^{きょう}{今日}のおじさんの^{しゃしん}{写真}もなんだか、もうなつかしく^{かん}{感}じた


\subsection{第42話\ ハルトンボ}

\page[148]
\Alpha そうじをほっぽりだして^{べつ}{別}の^{こと}{事}を^{はじ}{始}めてしまう

\page
\Alpha そのそうじも「^{かいてん}{開店}^{じゅんび}{準備}の^{まえ}{前}に」と、
やりだした^{こと}{事}だったのに

\page[154]
\Alpha よし!

\Alpha あーー
\ いかん!

\page[156]
\Alpha ふう

\page[158]
\Alpha ふっ

\Alpha にや

\page[160]
\Alpha あ〜〜


\subsection{おまけのページ}


\subsection{ふりむけばミサゴ}
\Takahiro みさごは、あからさまにさがすと、^{で}{出}てこないんだ。
そこんとこはちゃんとおさえてあるよ。

\Takahiro みさごが^{で}{出}そうな^{ところ}{所}は^{し}{知}ってる。
あとは^{き}{気}づかれないようにすればいいんだ。
