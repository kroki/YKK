\section{Volume 5}

\subsection{^{だい}{第}32^{わ}{話}\ ^{ふじ}{富士}^{さん}{山}}

\page[2]
\A ひさしぶりにカメラいじってます

\A また、^{みなみまち}{南町}にコーヒー^{まめ}{豆}が^{はい}{入}らなくなってきて

\A ^{きょう}{今日}は、うちも^{きゅうじつ}{休日}

\page
\A こういう^{ひ}{日}でもやることは^{かんが}{考}えれば、いっぱいあるけど

\A 「なんにもしない^{ひ}{日}」にしました

\page
\A ^{きょう}{今日}は^{はる}{春}にしてはモヤってないから

\A ^{ゆうがた}{夕方}になれば^{ふじ}{富士}^{さん}{山}まで^{み}{見}えるかも

\page
\A よっ

\page
\A ^{ひ}{日}がしずんで、^{うし}{後}ろから^{ひかり}{光}を^{う}{受}けた
^{いず}{伊豆}が^{う}{浮}かんでくる

\page
\A ふう

\A ^{ふじ}{富士}^{さん}{山}

\A ^{いぜん}{以前}のーーー
\ ^{かん}{完}ぺきだった、^{むかし}{昔}の^{すがた}{姿}もよかったけど

\page
\A やっぱり^{わたし}{私}は、リラックスしてる^{いま}{今}の^{ふじ}{富士}^{さん}{山}が^{す}{好}き


\subsection{第33話\ ^{よこはま}{横浜}^{か}{買}い^{だ}{出}し}

\page[10]
\Sign Closed
\ ^{らいしゅう}{来週}まで…
\ ^{か}{買}い^{だ}{出}しです
\ Cafe\ アルファ

\page[12]
\A ^{とうめん}{当面}のコーヒー^{まめ}{豆}を^{か}{買}いに^{おも}{思}いきって
^{よこはま}{横浜}に^{い}{行}くことにしました

\page
\A ^{こんかい}{今回}は^{すこ}{少}し^{よゆう}{余裕}とってます

\page[15]
\A ^{まち}{町}への^{えだみち}{枝道}に^{はい}{入}ると、やっと^{じんか}{人家}がふえてきて

\A ^{いよう}{異様}にでっかくランドマークタワーが^{しょうめん}{正面}に^{で}{出}る

\page
\A あのてっぺんにもいつか^{い}{行}ってみたいけど

\A ^{かいだん}{階段}つらそうなんだよね

\page
\A 「^{きょう}{今日}の^{もくてき}{目的}は^{まめ}{豆}!!」

\A よし!

\page
\A ああん

\Sign もなか

\Sign まんじゅう

\A コーヒー^{まめ}{豆}……

\A なんとか^{まめ}{豆}も^{か}{買}えました

\A ほう

\page
\A まだ^{じかん}{時間}もいっぱいあるからーー

\Sign ところてん\ あります

\Sign ^{いも}{芋}せんべい

\Sign ^{いも}{芋}けんぴ

\Sign ^{いも}{芋}よ??ん

\Sign あんこ^{たま}{玉}

\A たぶん^{やど}{宿}もさがせるし、おそらくお^{かね}{金}も^{た}{足}りるし

\page
\A やっぱり、ランドマークにものぼってみようかなーー

\A ^{ほか}{他}に^{か}{買}っとくものは〜〜

\page
\Sign ココネ
\ ……江戸郡世田谷町

\page[23]
\A おばちゃん、ごっそさん!

\A ごめん!お^{わん}{椀}、ここ^{お}{置}いとくね!

\P う〜〜い

\page
\A いきなりだけど

\A でも

\A ^{い}{行}ってみよう!


\subsection{第34話\ お^{きゃく}{客}さま}

\page[26]
\K ふう

\K アルファさんの^{とき}{時}^{いらい}{以来}ひさしぶりに「メッセージ^{はいたつ}{配達}」の^{しごと}{仕事}です

\page
\K あの^{ひ}{日}のあとこの^{たんとう}{担当}が、
なぜか^{すこ}{少}し^{おも}{重}く^{かん}{感}じるようになりました

\K えーと

\K ^{おお}{大}^{た}{田}^{むら}{村}、^{あざ}{字}^{でん}{田}^{えん}{園}^{ちょう}{調}^{ふ}{布}の

\K このへんよね

\page
\K あった

\Sign アトリエ\ ^{まる}{丸}^{こ}{子}

\K ふう

\page
\K ごめんください!

\K ムサシノ^{うんそう}{運送}ですーー!

\K ごめ……

\R ああ
\ まってたのよ!

\R まあ
\ はいって

\K は……
\ はい

\page
\R はい
\ お^{ちゃ}{茶}でも

\K あ
\ どうも

\K えーと
\ じゃここにサインを

\R まー
\ それはメッセージ^{う}{受}けとってからね

\R ひと^{いき}{息}^{い}{入}れた^{ほう}{方}が、
^{ないよう}{内容}の^{つた}{伝}わり^{かた}{方}もいいし

\K そっ
\ そうですね

\page
\K ああ……やっぱり、こんなに^{きんちょう}{緊張}してる

\K ^{まえ}{前}は、なんでもなかったのに

\K でも

\K どうしてなんでもなかったんだろう

\K ふー

\page
\K はい!
\ では!

\R ん!

\R でも、ちょっとやけくそみたい
\ だから
\ ゆっくりめに

\K はい

\page
\R ふーー

\R そんなリアルに^{つた}{伝}えられるものなのね

\K は?

\R いえね
\ ^{わたし}{私}、^{え}{絵}なんか^{えが}{描}いてるでしょ

\R ^{ともだち}{友達}がね、^{み}{見}た^{ふうけい}{風景}の^{いんしょう}{印象}とか、
ときどき^{おく}{送}ってくれるのよ

\page
\R でも、こんなに^{こま}{細}やかに^{つた}{伝}えてくれる^{ひと}{人}がいるなんて

\R はい

\K そっ
\ そうですか?

\R あなたすごいわ

\K はあ
\ どうも

\page
\R ^{つぎ}{次}も、あなたに^{き}{来}てもらうように^{たの}{頼}めるかな?

\K は……

\R ^{き}{気}に^{い}{入}っちゃったから

\K あ……はい
\ ありがとうございました!

\page[37]
\Sign ムサシノ^{うんそう}{運送}
\ ^{せたがや}{世田谷}^{してん}{支店}

\Sign ^{ごよう}{御用}は^{しょうめん}{正面}へ

\SH ありゃ

\SH どしたの
\ ^{そと}{外}^{ぎ}{着}のまんまで

\K あ〜〜
\ シバちゃ〜〜ん

\K なんだかつかれた

\SH あ?

\page
\K も〜〜やたらあがるし、
\ ^{みょう}{妙}に^{う}{受}けがいいしで

\SH ふ〜〜ん

\K ^{まえ}{前}は^{ぜんぜん}{全然}こんなことなかったのにな

\SH ん〜〜
\ まあ
\ でも
\ ほめられたんだし

\K うん
\ まあ
\ そうだね

\SH ^{あした}{明日}^{やす}{休}みで、よかったじゃん

\SH ぼーっとしてなよ

\SH おつかれ

\K うん
\ じゃあね

\page
\K ただいまっと

\K ふう

\page
\A やほーー
\ ^{き}{来}ちゃったよ

\P おかえりよーー


\subsection{第35話\ ^{よる}{夜}のにおい}

\page[42]
\K どっ
\ どうぞ

\A おじゃましますー

\page
\K せまいとこですけど

\A ううん
\ ごめんね
\ いきなり^{き}{来}ちゃってここが

\A コッ
\ 「ココネの^{いえ}{家}ってのはココね!」

\A あ〜〜
\ いや……

\A なし!

\page
\K おもしろくないでしょ、^{わたし}{私}のへや

\A ううん

\K はい……
\ そば^{ちゃ}{茶}です
\ これ

\A ありがと

\page
\A あーー
\ よかった

\K え?

\A いやー、ココネ^{かえ}{帰}ってきたときさあ

\A リアクションなかったから……

\A なんか、まずい^{とき}{時}に^{き}{来}ちゃったのかなって^{おも}{思}っちゃった

\K あ……

\K いえ……ちゃっと
\ ^{ひるま}{昼間}の^{しごと}{仕事}のこと^{かんが}{考}えてたんで

\K びっくりして、つい……

\A そっか

\page
\K そうか
\ アルファさんと^{はじ}{初}めて^{あ}{会}った^{とき}{時}も

\K うわ〜〜
\ ^{いま}{今}だったらできないな

\K あ
\ でも
\ う〜〜ん

\page
\A あのー
\ さー

\K えっ

\A いきなりおしかけといてなんなんだけど

\A ^{きょう}{今日}^{と}{泊}まっていいかな

\A だっ
\ だめ?

\K いえ!
\ ^{ぜんぜん}{全然}!
\ ぜひ^{と}{泊}まってってください

\K ^{わたし}{私}、あした^{やす}{休}みですし

\A あ
\ よかった〜〜

\page
\A いやー、ここさがした、さがした

\K あ
\ やっぱり

\A ^{わたし}{私}、^{がいこく}{外国}に^{き}{来}たの^{はじ}{初}めてだから
\ ^{ま}{舞}いあがっちゃって

\A ムサシノって^{たい}{平}らな^{くに}{国}なのね

\K この^{へん}{辺}はそうですね

\K ^{ふゆ}{冬}になると^{ちへいせん}{地平線}までススキの^{うみ}{海}なんですよ

\A ふーん

\page
\K バイクでここまでってけっこうきつかったんじゃないですか

\A そう!

\A ココネいきなりうちまでよく^{き}{来}たよねー

\A ^{わたし}{私}は^{よこはま}{横浜}でここのこと^{おも}{思}いついたから、
^{きぶん}{気分}^{てき}{的}に^{らく}{楽}だったけど

\A あ

\A ^{げっぺい}{月餅}^{た}{食}べる?

\K わっ

\page[51]
\A なんか、ちらかしに^{き}{来}たみたい

\K いいえ

\K そのままにしといてください

\K ふとん……はもういいですね

\A そうだね

\page[53]
\A ^{しず}{静}か……

\K ええ

\K アルファさんのうちは^{なみ}{波}の^{おと}{音}が^{き}{聞}こえますもんね

\A そうだったんだねー

\page
\A ココネ

\A あした、^{よてい}{予定}ある?

\K ないですよ

\A どっか^{い}{行}きたいね

\K どこ^{い}{行}きましょうか……

\page[56]
\A ムサシノの^{よる}{夜}は、^{とお}{遠}くから^{く}{来}る、^{やま}{山}のにおいがします


\subsection{第36話\ ^{まち}{町}のにおい}

\page[58]
\A お^{ひる}{昼}も^{ちか}{近}くなってから^{まち}{町}に^{で}{出}ました

\page
\A ^{まち}{町}とは^{い}{言}っても、^{よこはま}{横浜}みたいにかたまってなくて

\A ずーっと^{とお}{遠}くまでパラパラと^{いえ}{家}や^{みせ}{店}が
ちりばめてあるって^{かん}{感}じです

\A おわーー!

\page
\A ムサシノは^{き}{木}も^{みち}{道}もまっすぐでやたらと^{おお}{大}きくて

\page
\A さらに、^{わたし}{私}の^{くに}{国}よりもずっとさらさらした^{くうき}{空気}の^{かお}{香}り

\A ココネの^{かみ}{髪}や^{ふく}{服}の^{かお}{香}りと^{すこ}{少}し^{に}{似}ています

\A ^{じぶん}{自分}じゃわからないけど、^{わたし}{私}にも、そんな^{かお}{香}りがあるんでしょうか

\page
\Sign かんぱち
\ ^{つじ}{辻}の^{ちゃ}{茶}

\K じつは、このへんあんまり^{みどころ}{見所}ないんですよ

\A んーん
\ めずらしい^{もの}{物}ばかりだよ

\A でも、ほんとの^{とこ}{所}それは^{に}{二}の^{つぎ}{次}です

\page
\A あの、すいません、シャッター^{お}{押}していただけます?

\P あ
\ はい

\P あの
\ これ、ファインダーないんですが

\A あ〜〜
\ ^{だいたい}{大体}でいいです

\P はい
\ では
\ せえの

\page
\A ^{けっきょく}{結局}、^{こんや}{今夜}も^{と}{泊}めてもらっちゃいます

\A ^{に}{二}^{はく}{泊}^{みっか}{三日}の^{たび}{旅}でした


\subsection{第37話\ ^{あさ}{朝}^{はや}{早}く}

\page[68]
\S 「^{あした}{明日}の^{あさ}{朝}、もしヒマで、^{はや}{早}く^{お}{起}きられたら、
  ^{わたし}{私}の^{ところ}{所}へ^{き}{来}てみてね」

\A きのう^{せんせい}{先生}が^{ひさ}{久}しぶりに^{みせ}{店}にきてくれて

\A なんだかそんな^{はなし}{話}が^{で}{出}た

\page
\A ^{おも}{思}いっきり、さそわれたわけじゃないけど

\A せっかくだから^{い}{行}ってみることにしました

\page
\A あ
\ おはようございます

\S あら

\page
\S きてくれたのね

\A えへへ

\A なんだろって^{おも}{思}いまして
\ おじさんも^{き}{来}てるんですね

\S うん
\ ^{かれ}{彼}にもつきあってもらうのよ

\S ^{いま}{今}、^{げんち}{現地}で^{ま}{待}ってるの

\A ^{げんち}{現地}?
\ なんなんですかきょうは!

\page
\S ^{ほんと}{本当}^{い}{言}うとね、アルファさんにはぜひ^{き}{来}てもらいたかったのよ

\S ホッとしちゃった

\S ^{すこ}{少}し^{ちから}{力}をかしてもらいたいことがあってね

\S ……ていうか^{た}{立}ち^{あ}{会}ってほしかったもんだから

\page
\S ^{げんち}{現地}っていってもすぐそこだけどね

\A はあ

\S ここよ

\page
\A あっ
\ おじさん

\O よう

\page
\O ^{き}{来}たか

\O すげえもん^{み}{見}られんぞ、^{きょう}{今日}

\A えっ!?
\ そーー!?
\ ^{せんせい}{先生}なんにも^{い}{言}ってくんないんですよーー!

\S ごめん

\A あはは

\O まあ
\ ^{き}{来}てみな

\O ^{み}{見}てな

\page
\A おわーー!!

\page
\A すごい

\A なんですかこれ……ふね?

\S ^{ふね}{船}……なのかね

\S ^{がくせい}{学生}の^{とき}{時}、こういうの^{つく}{作}って^{あそ}{遊}んでたのよ

\S この^{きたい}{機体}はまだ^{いちど}{1度}も^{はし}{走}ってないやつなんだけど

\S もうボロボロね

\S こいつを^{はし}{走}らせる^{さいご}{最後}の^{きかい}{機会}だと^{おも}{思}うの

\page
\S ずっと^{おか}{陸}に^{お}{置}いといてもよかったんだけど

\S それだと、こいつ^{う}{生}まれてから^{いっかい}{1回}も^{い}{生}きないで^{か}{枯}れてっちゃう

\S ^{わたし}{私}もいつまでも^{げんき}{元気}なわけじゃないし

\S ^{むかし}{昔}の^{なかま}{仲間}はいないから

\S それに、こいつが^{はし}{走}るとこアルファさんにも^{み}{見}てほしかったしね

\page
\A ありがとうごさいます

\S ひとつ

\S ^{てつだ}{手伝}ってほしい^{こと}{事}があるんだけど

\A あ
\ はい

\S アルファさんに^{そうじゅう}{操縦}してほしいの

\page
\A そんな!
\ わ
\ わたし、こんなすごいの^{うんてん}{運転}したことないですよ!!

\A なんか、^{の}{乗}るとこもないみたい……

\S え?

\S あら
\ ^{しんぱい}{心配}しないで

\S ^{だいじょうぶ}{大丈夫}よ


\subsection{第38話\ ^{うみ}{海}の^{かわ}{河}}

\page[83]
\A ^{せんせい}{先生}の^{ふね}{船}

\A ^{きょう}{今日}はじめて^{けいけん}{経験}する^{いのち}{命}、
それを^{ぜんぶ}{全部}^{ぜんしん}{前進}に^{つか}{使}って

\A ^{そとうみ}{外海}の^{ほう}{方}に^{はし}{走}っていきます

\A ^{せんせい}{先生}は、^{ふね}{船}に^{くろしお}{黒潮}まで^{い}{行}って、
そこで^{やす}{休}んでほしいと^{おも}{思}っていたとか

\page
\A ^{わたし}{私}は^{ふね}{船}をまっすぐ^{い}{行}かせる
^{やく}{役}を^{たんとう}{担当}することになりました

\A ^{わたし}{私}には、^{くち}{口}のコレで^{ふね}{船}の^{ちい}{小}さな^{かじ}{舵}を
^{うご}{動}かすことができるそうなのです

\page
\A どんどん^{かそく}{加速}していく^{ふね}{船}

\page
\A なんか^{くち}{口}に

\A ^{ふね}{船}の^{かんしょく}{感触}があります

\S え?
\ そうなの?
\ ^{へん}{変}ね、むこうからは^{なに}{何}もきてないはずだけど

\S でも
\ あ
\ もう
\ ^{み}{見}えなくなっちゃうわね

\page
\S アルファさんのおかげで、うまく^{つるぎざき}{剱崎}と^{すのさき}{洲崎}の^{あいだ}{間}^{ぬ}{抜}けられそう

\S アルファさん?

\page[94]
\A あ……

\S あ!

\S ^{だいじょうぶ}{大丈夫}?

\S ごめんなさいね……
\ まさか、あんなに^{ふか}{深}くつながっちゃうなんて

\page
\S あらら

\page
\S さっきはごめんね

\A いやーー

\A なんだか、よくおぼえてないんですよ……

\S ^{きょう}{今日}アルファさんが^{き}{来}てくれてうれしかったわ

\S あのね
\ ^{むかし}{昔}、あの^{ふね}{船}ぶっとばす^{こと}{事}とロボットの
^{けんきゅう}{研究}がつながったことがあるの

\S まあ、でも、^{けっきょく}{結局}なんの^{やく}{役}にも^{た}{立}たなかったんだけど

\S そんなこともあってね

\page
\S きょうはありがとう

\A いえ
\ そんな
\ ^{わたし}{私}の^{ほう}{方}こそ

\A ^{きょう}{今日}は^{せんせい}{先生}の^{ふね}{船}の^{たんじょうび}{誕生日}

\A それは^{どうじ}{同時}にお^{わか}{別}れの^{ひ}{日}でもありました

\page
\A あの^{とき}{時}、なにか^{み}{見}たような^{き}{気}もします


\subsection{第39話\ ^{ごご}{午後}の^{むぎちゃ}{麦茶}}

\page[100]
\A ここんとこ^{まめ}{豆}がなかなか^{へ}{減}りません

\page
\A タカヒロ^{さいきん}{最近}、あんまり^{こ}{来}ないもんなー

\page[103]
\T あに
\ タラ〜〜って^{ある}{歩}いてんだよ

\T ついてくるっつったの、マッキだべー?

\M だあって、「^{うみ}{海}^{い}{行}く」ってゆうからさ!

\M とうもろこしだってゆでてさ!^{むぎちゃ}{麦茶}だって……

\T あ〜〜
\ はい

\T ありがとね

\M もっと、^{みず}{水}であそべるとこって^{おも}{思}ったのにさ!
\ こんな^{ぬま}{沼}みたいなとこ^{き}{来}て……

\page
\T だから、みさごいんのはここいらだって^{い}{言}ったべ

\T マッキだって^{いっかい}{1回}^{み}{見}りゃすげえって^{おも}{思}うよ

\M わたしミサゴなんかいいよ、^{べつ}{別}に

\M わざわざこわい^{おも}{思}いしにこんなとこまで^{き}{来}てさ

\T だいじょうぶ、こわくねえって

\M ほんと?

\T いや
\ けっこうこええかな

\page
\T ^{むぎちゃ}{麦茶}は^{せいかい}{正解}だったなあ

\M でしょ!

\page
\T ^{かえ}{帰}るか

\M うん!

\T めったに^{で}{出}ねえんだよ

\M うん

\page
\T あにした?

\M おしっこ

\T あそう
\ ここで^{ま}{待}ってら

\M ちっとかっこわりかったな〜〜

\page[111]
\T ^{い}{行}くか

\M あ
\ う
\ うん

\M お
\ おしっこひっこんじゃった

\T あんだえー

\page
\M タカヒロ

\T ん〜〜?

\M ミサゴってさ
\ なにしに^{で}{出}てくんのかな

\T あ?
\ さ〜〜
\ ウケねらいかな

\M でも、なかなか^{で}{出}ないんでしょ

\T ん〜〜

\T ^{て}{照}れてんのかな

\M そっか
\ ま
\ いいや

\page
\M ^{きょう}{今日}さ
\ タカヒロんち^{い}{行}くよ!

\T ^{いま}{今}から?

\M うん

\M すんごいことおしえてやるよ!

\T あによ

\M はっはっは


\subsection{第40話\ 月夜見(つくよみ)}

\page[116]
\Sign 霞 浦バス
\ ^{もと}{元}^{さんばし}{桟橋}
\ 土浦ゆき

\page[118]
\P よーー

\Y ^{はや}{早}えな

\page
\P きのう、^{お}{終}わったばっかなのによー

\Y あに、もう、^{つぎ}{次}い^{い}{行}くとか?

\P おう
\ ^{いま}{今}ちっと^{ても}{手持}ちに^{よゆう}{余裕}がほしい

\P アヤセよー
\ ^{みなみ}{南}に^{ほう}{方}^{い}{行}きゃまた^{はたけ}{畑}のバイトあるってよ

\P ^{い}{行}ってみねえ?

\Y おれは^{はたけ}{畑}はしばらくいいや

\page
\Y あにしろあいつがまだよう

\P あ?

\P あー
\ 「カマス」っつったっけか

\Y おう
\ なんか、まだここ^{うご}{動}きたくねえみてえでよ

\page
\P ^{せわ}{世話}かかんな〜〜

\Y ま、そゆことなんで

\P そか

\P じゃ、ま
\ いずれどっかでまた

\Y うーーす

\P あ
\ そうだ

\page
\P コメが^{すこ}{少}し^{て}{手}に^{はい}{入}った
\ ちょっとやるよ

\Y おお

\Y もう、^{あじ}{味}、おぼえてねえや

\P まー
\ たまにはな

\Y じゃ
\ お^{かえ}{返}しと^{い}{言}うのもなんだか

\Y しょう^{あぶら}{油}でもあげよう

\Y この^{とうき}{陶器}のやつな

\P えっ!こりゃ、おめえ

\page
\P こっちの^{ほう}{方}がすげえよ!

\Y まー
\ たまにはな

\page[130]
\Y ^{あした}{明日}から、また、^{うご}{動}く


\subsection{第41話\ ^{いちがん}{一眼}}

\page[132]
\A カメラを^{も}{持}って^{である}{出歩}いた^{ひ}{日}

\A ^{かえ}{帰}りの^{とちゅう}{途中}おじさんに^{おく}{送}ってもらった

\page
\O あによ

\O じじいばっか^{と}{撮}っちゃっちゃもったいねえべー

\A いーえー

\A ^{いっしゅん}{一瞬}、すごくかっこよかったりしてますよ!

\O そうかー
\ ^{いっしゅん}{一瞬}かー

\page
\O ^{かんが}{考}えてみりゃ、このごろ^{しゃしん}{写真}とか^{み}{見}てねえよなー

\O こんだアルファさんのとったのとか^{み}{見}してけーねーか

\A あ〜〜
\ でもこれ

\A ロボットの^{ひと}{人}じゃないと、^{み}{見}らんないんじゃないかなあ

\O あんだ
\ こう、ペラッとしたのがあんじゃねえんだ

\A うん

\A ^{て}{手}はなさないでね

\A もし、モニターとかプリンター^{つか}{使}っても

\A たぶん^{み}{見}え^{かた}{方}は^{つた}{伝}わらないと^{おも}{思}います

\O へ〜〜

\page
\O ひとことで^{ゆ}{言}うとすんとどんな^{ふう}{風}?

\A えーとね

\A こう
\ いや
\ ん〜〜
\ あれ?

\O ま
\ なにしろ、まずプリンターからねえしなー

\A はあ

\O テレビでよけりゃ、どっかにあったかな?

\A あはは

\page[138]
\A おじさんに^{しゃしん}{写真}のことをうまく^{せつめい}{説明}できないのがもどかしかった

\A たぶん、このカメラで^{と}{撮}るのは、^{しゃしん}{写真}とは、
また^{べつ}{別}のものじゃないかと^{おも}{思}う

\page
\A さっきのおじさんの^{しゃしん}{写真}

\page
\A ^{ゆうがた}{夕方}の、あの^{とき}{時}からの^{じかん}{時間}^{けいか}{経過}が、まるでうそのように
\ ^{いま}{今}、^{たし}{確}かに^{め}{目}の^{まえ}{前}にひろがる^{こうけい}{光景}

\A ^{うご}{動}きこそしないけど、^{すこ}{少}しなら^{してん}{視点}をずらすことさえできる

\page
\A はじめのころは^{き}{気}づかなかった

\A このカメラは、^{うつ}{写}した^{とき}{時}からの^{じかん}{時間}が
^{なが}{長}ければ^{なが}{長}いほど、^{ようしゃ}{容赦}なくリアルな^{むかし}{昔}へ
^{わたし}{私}をひきもどす

\A ^{み}{見}ている^{わたし}{私}も、さっきのかっこをしていると^{かん}{感}じる

\A ^{たし}{確}かめることはできないけど

\A ずっと^{まえ}{前}の^{しゃしん}{写真}

\page[143]
\A ^{わたし}{私}は、このカメラのことを^{ほんき}{本気}で、^{じかん}{時
  間}^{りょこう}{旅行}^{き}{機}のように^{かん}{感}じることがある

\page
\A ^{め}{目}をあけるときは、^{いっしゅん}{一瞬}、ほんとうにどちらが
^{いま}{今}で、どちらが^{むかし}{昔}なのかわからないくらい

\page
\A ふう

\A んぅ〜

\page
\A ^{きょう}{今日}のおじさんの^{しゃしん}{写真}もなんだか、もうなつかしく^{かん}{感}じた


\subsection{第42話\ ハルトンボ}

\page[148]
\A そうじをほっぽりだして^{べつ}{別}の^{こと}{事}を^{はじ}{始}めてしまう

\page
\A そのそうじも「^{かいてん}{開店}^{じゅんび}{準備}の^{まえ}{前}に」と、
やりだした^{こと}{事}だったのに

\page[154]
\A よし!

\A あーー
\ いかん!

\page[156]
\A ふう

\page[158]
\A ふっ

\A にや

\page[160]
\A あ〜〜


\subsection{おまけのページ}


\subsection{ふりむけばミサゴ}
\T みさごは、あからさまにさがすと、^{で}{出}てこないんだ。
そこんとこはちゃんとおさえてあるよ。

\T みさごが^{で}{出}そうな^{ところ}{所}は^{し}{知}ってる。
あとは^{き}{気}づかれないようにすればいいんだ。
